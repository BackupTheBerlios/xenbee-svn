\documentclass
[
  a4paper,
  11pt,
  oneside,
%  openright,
  chapterprefix,
%  draft,
  bibtotoc,
  idxtotoc,
  BCOR12.0mm,
  DIV11,
  liststotoc,
%  bigheadings,
] {scrbook} %scrreprt}


\usepackage{amsmath}
\usepackage{amssymb}
\usepackage{enumerate}
% DO NOT use amsthm. Most of its functionality is in ntheorem anyway.

\usepackage{parskip} %Supposed to tidy up space if using space between pars
  % tex-archive/macros/latex/contrib/other/misc/parskip.sty

\usepackage[amsmath,thmmarks]{ntheorem}
  % macros/latex/contrib/supported/ntheorem/
  % To install this package, download ntheorem.ins and ntheorem.dtx
  % These can be found from www.tex.ac.uk
  % run latex ntheorem.dtx
  % Copy the files ntheorem.sty and ntheorem.std to the same
  % folder as your latex source.

\usepackage{setspace}
  % tex-archive/macros/latex/contrib/supported/setspace/

%\usepackage{multicol}

\usepackage[compact]{titlesec}
  % tex-archive/macros/latex/contrib/supported/titlesec/


  % There's probably loads of packages needed by various people.
\usepackage[perpage,symbol*,hang]{footmisc}
\usepackage{url}
\usepackage{pdfsync}
\usepackage[T1]{fontenc}
\usepackage[utf8]{inputenc}
\usepackage[english]{babel}
\usepackage[short,nodayofweek]{datetime}
\usepackage{subfigure}
\usepackage{graphicx}
\usepackage[pdftex,usenames,dvipsnames]{color}
\usepackage{capt-of}
\usepackage[hang, small, bf, margin=20pt, tableposition=top]{caption}
\setlength{\abovecaptionskip}{0pt}
\usepackage{fancyhdr}
\usepackage{palatino}
\usepackage{a4}
\usepackage{listings}
\usepackage{booktabs}
\usepackage{makeidx}
\usepackage[square, comma, numbers, sort&compress]{natbib}
\usepackage[light,first,conditional]{draftcopy}
\usepackage[plainpages=false,pdfpagelabels,pdftex]{hyperref}
\usepackage[figure,table]{hypcap} % Correct a problem with hyperref
\hypersetup{
   bookmarksnumbered,
   pdfstartview={FitH},
   citecolor={black},
   linkcolor={black},
   urlcolor={black},
   pdfpagemode={UseOutlines}
}
\usepackage[style=super,cols=2,toc=true,hyper=true,number=none]{glossary} % hyper=true for hyperrefs
\graphicspath{{graphics/}}

% include some often used commands
%% make a superscript (R) registered sign
\newcommand{\trademark}{${}^{\mbox{\scriptsize\textregistered}}$}

%Extra things that are included for simplicity.

\newcommand{\ie}{\emph{i.e.\@} }
\newcommand{\eg}{\emph{e.g.\@} }
\newcommand{\vs}{\emph{vs.\@} }
\newcommand{\etc}{\emph{etc.\@} }

\storeglosentry{glo:API}{
  name={API},
  description={\emph{Application Programming Interface}}
}

\storeglosentry{glo:DHCP}{
  name={DHCP},
  description={\emph{Dynamic Host Configuration Protocol}}
}

\storeglosentry{glo:MAC}{
  name={MAC},
  description={\emph{Media Access Control}. MAC addresses are uniquely
    assigned identifiers for network interface cards (NICs). No two
    network adapters must have the same MAC when communicating
    within the same physical network.}
}

\storeglosentry{glo:XEN}{
  name={Xen},
  description={\emph{Xen virtual machine monitor} \cite{xen}}
}

\storeglosentry{glo:VM}{
  name={VM},
  description={\emph{Virtual Machine}},
}

\storeglosentry{glo:VMM}{
  name={VMM},
  description={\emph{Virtual Machine Monitor}}
}

\storeglosentry{glo:MLS}{
  name={MLS},
  description={\emph{Message Layer Security}}
}

\storeglosentry{glo:TLS}{
  name={TLS},
  description={\emph{Transport Layer Security}. The protocol allows client/server
    applications to communicate in a way that is designed to prevent
    eavesdropping, tampering, or message forgery (taken from abstract of
    the RFC 4346).
  }
}

\storeglosentry{glo:virtmem}{
  name={virtual memory},
  description={Virtual memory is an addressing scheme implemented in
    hardware and software that allows non-contiguous memory to be
    addressed as if it is contiguous.},
  sort={M}
}

\storeglosentry{glo:multi-programming}{
  name={multi--programming},
  description={
    In multiprogramming systems, the running task keeps running until it
    performs an operation that requires waiting for an external event
    (e.g. reading from a tape) or until the computer's scheduler forcibly
    swaps the running task out of the CPU. Multiprogramming systems are
    designed to maximize CPU usage.},
}

\storeglosentry{glo:time-sharing}{
  name={time--sharing},
  description={The term \emph{time-sharing} refers to sharing a given computing resource
    among different users by multitasking.}
}

\storeglosentry{glo:multi-tasking}{
  name={multitasking},
  description={\emph{Multitasking} is a method by which multiple tasks share
    common processing resources.},
}

\storeglosentry{glo:extracode}{
  name={extracode},
  description={
    The term \emph{extracode} was used  in the Atlas computer system to allow
    new  instructions being  added in  software (that  what is  now called
    firmware).
  },
}

\storeglosentry{glo:supervisor}{
  name={supervisor},
  description={
    A supervisory program or supervisor is a computer program, usually
    part of an operating system, that controls the execution of other
    routines and regulates work scheduling, input/output operations, error
    actions, and similar functions and regulates the flow of work in a
    data processing system.
  }
}

\storeglosentry{glo:virtualization}{
  name={virtualization},
  description={
    Virtualization in  computing describes a technique for hiding the
    physical characteristics of computing resources.
  }
}

\storeglosentry{glo:BLAST}{
  name={BLAST},
  description={\emph{\textbf{B}asic       \textbf{L}ocal      \textbf{A}lignment
    \textbf{S}earch \textbf{T}ool}
  }
}

\storeglosentry{glo:OS}{
  name={OS},
  description={\emph{Operating System}. Refers to the conglomerate of
    \emph{kernel}, as the central layer between hardware and software,
    and \emph{system-software} that render the whole system usable in
    the first place.}
}

\storeglosentry{glo:image}{
  name={image},
  description={The  term  \emph{image}  relates  in  this  work  to  a
    file-system image  (\ie a regular file, that for  example contains an
    \texttt{ext2} partition).  Those image-files  must be mountable by the
    standard UNIX-command \texttt{mount}.
 }
}

\storeglosentry{glo:kernel}{
  name={kernel},
  description={A \emph{kernel} refers to the most central component of an
    operating system. It mediates between user applications and the
    hardware on which the operating systems run, \ie it assigns the CPU,
    available memory and other resources to processes running on top of
    the \gls{glo:OS}.}
}

\storeglosentry{glo:JSDL}{
  name={JSDL},
  description={The    \emph{Job     Submission    Description    Language}
    \cite{jsdl-spec},  an   upcoming  standard  for   the  description  of
    computational jobs.  It is designed to  be used in but  not limited to
    grid environments.
  }
}

\storeglosentry{glo:POSIX}{
  name={POSIX},
  description={The \emph{Portable Operating System Interface} \cite{posix}
    is  a   collective  name  for  a  family   of  specifications.   These
    specifications  for  instance  describe  standard  library  functions,
    system behaviour and many more,  so that applications, which make only
    use of  library functions defined in the  specifications, are supposed
    to run on all POSIX-compliant systems.
  }
}

\storeglosentry{glo:BES}{    name={BES},    description={The   \emph{Basic
      Execution  Service}  \cite{ogsa-bes} describes  the  semantics of  a
    Web-Service,  which   is  responsible   for  the  execution   of  some
    \emph{activity}.   The  specification  does  not consider  the  actual
    execution  of any  activity but  it defines  the operations  and their
    semantics that such a service must implement.} }

\storeglosentry{glo:OGSA}{
  name={OGSA},
  description={The \emph{Open Grid Service Architecture}, see \cite{ogsa}
    for more information.}
}

\storeglosentry{glo:TCP}{
  name={TCP},
  description={The \emph{Transmission Control Protocol} is a
    connection-oriented stream-based protocol which guarantees an ordered,
    loss-free and correct (\ie checksummed) transport of packets in
    packet-switched computer networks.
    For detailed information have a look at RFC 793.
  }
}

\storeglosentry{glo:MQS}{
  name={MQS},
  description={A \emph{Message Queue Server} is a server which provides
    access for message-oriented services. This kind of server is mainly
    used in \emph{message-oriented middlewares}.}
}

\storeglosentry{glo:MOM}{
  name={MOM},
  description={\emph{Message Oriented Middleware}}
}

\storeglosentry{glo:NAT}{
  name={NAT},
  description={\emph{Network Address Translation} or \emph{Network Address
    Translator} (RFC 1631) is used to multiplex one official IP address
  to a whole network of private network addresses (contact the site of the
  IANA for a list of IP addresses that are reserved for private use only).}
}

\storeglosentry{glo:URI}{
  name={URI},
  description={A \emph{Uniform Resource Identifier} is a compact string of
  characters for identifying an abstract or physical resource (taken from
  RFC 2396).}
}

\storeglosentry{glo:SSH}{
  name={SSH},
  description={The \emph{Secure SHell} can be used to gain access to a
    remote computer in a secure manner (i.e~encrypted).}
}

\storeglosentry{glo:XenBEE}{
  name={XenBEE},
  description={\emph{Xen-based Execution Environment}}
}

\storeglosentry{glo:X509}{
  name={X.509},
  description={X.509 is an ITU-T standard for public-key infrastructures
    (PKI). Among other things, X.509 defines the format of public-key
    certificates and a certificate validation path.}
}

\storeglosentry{glo:PKI}{
  name={PKI},
  description={\emph{Public Key Infrastructure}. Certificates and a
    Certificate Authority can be used to build up a PKI. Certificates,
    which have  been  signed by a trusted authority are trusted, too.}
}

\storeglosentry{glo:CA}{
  name={CA},
  description={\emph{Certificate Authority}. A CA issues certificates to
  users it trusts (\eg after verifying their particulars). Users can then
  rely on the validity of a certificate if it has been issued (\ie signed)
  by a trusted CA.}
}

\storeglosentry{glo:XML}{
  name={XML},
  description={\emph{eXtensible Markup Language}}
}

\storeglosentry{glo:UUID}{
  name={UUID},
  description={\emph{Universal Unique IDentifier}. Have a look at RFC~4122
  for more information.}
}

\storeglosentry{glo:STOMP}{
  name={STOMP},
  description={\emph{Streaming Text Oriented Messaging Protocol}}
}

\storeglosentry{glo:FSM}{
  name={FSM},
  description={\emph{Finite State Machine}.}
}

%%% Local Variables: 
%%% mode: latex
%%% TeX-master: "main"
%%% End: 


% Tweak the margins to match requirements. Seems to work OK.
\topmargin-0.5cm
\footskip1cm
\oddsidemargin0.5cm
\evensidemargin0cm
\textwidth16cm
\textheight22cm
\vfuzz1pc
\hfuzz1pc

\parindent 0pt

% Tweak the url-package
%% Define a new 'ap' style for the package that will use a smaller font.
\makeatletter
\def\url@apstyle{%
  \@ifundefined{selectfont}{\def\UrlFont{\sf}}{\def\UrlFont{\small\ttfamily}}}
\makeatother
%%% Now actually use the newly defined style.
\urlstyle{ap}

% Tweak the listings-package
\lstset{basicstyle=\small,
  stringstyle=\ttfamily,
  showstringspaces=false
}

% Tweak tables
\renewcommand{\arraystretch}{1.2}  % more space between rows

% Tweak hyphenation
\hyphenation{mid-dle-ware}

% COLOR definitions
\definecolor{listingcolor}{named}{White}

% prevent latex from putting figures all too often on a single page
\renewcommand{\topfraction}{0.85}
\renewcommand{\textfraction}{0.1}
\renewcommand{\floatpagefraction}{0.75}

% limit the table of contents to chapters and sections
\setcounter{tocdepth}{1}

\pagestyle{fancy}
\fancyhf{}                                %Clears all header and footer fields, in preparation.
\fancyhead[LE,RO]{\textbf{\thepage}}      %Displays the page number in bold in the header,
                                          % to the left on even pages and to the right on odd pages.
\fancyhead[RE]{\nouppercase{\leftmark}}   %Displays the upper-level (chapter) information---
                                          % as determined above---in non-upper case in the header, to the right on even pages.
\fancyhead[LO]{\nouppercase{\rightmark}}  %Displays the lower-level (section) information---as
                                          % determined above---in the header, to the left on odd pages.
\renewcommand{\headrulewidth}{0.5pt}      %Underlines the header. (Set to 0pt if not required).
\renewcommand{\footrulewidth}{0pt}      %Underlines the footer. (Set to 0pt if not required).
\renewcommand*{\partpagestyle}{empty}
\renewcommand*{\chapterpagestyle}{empty}

%\makeindex
\makeglossary

\begin{document}
%  T I T L E   P A G E
\titlehead{{\Large University of Kaiserslautern
    \hfill WS 2006/2007\\}
  Department of Computer Science \\
  Building 48, Room 375\\
  Erwin-Schr\"{o}dinger-Str.\\
  67663 Kaiserslautern}
\subject{Diploma Thesis}
\title{Design and Implementation of a Xen-based Execution
  Environment\footnote{XenBEE: see {\small\texttt{http://xenbee.berlios.de}}}}
\author{Alexander Petry}
\date{April 2007}
\publishers{%
  \begin{tabular}{rl}
    \small Advisor: & \small Juniorprof.~Peter Merz\\
    \small Co-Advisor: & \small Dipl.~Ing.~(Inf.) Mathias Dalheimer
  \end{tabular}%
  \vspace{2cm}
  \vfill%
  \includegraphics[height=1.5cm]{ITWM_rgb_200x80px}%
  \hfill
  \includegraphics[height=1.5cm]{TU-KL-CMYK}%
  \hfill
}
\pdfbookmark[0]{Titlepage}{title} % Sets a PDF bookmark for the title page
\maketitle


% \begin{titlepage}

%   \vspace*{0.1in}
%   \begin{singlespace}
%   \begin{center}
%     \begin{doublespace}
%       {\Huge{\sc{Design and Implementation of a Xen-based Execution Environment}}}
%     \end{doublespace}
%     \vspace{0.5in}
%     by\\
%     \vspace{0.5in}
%    {\Large{\sc{Alexander Petry}}}\\
%     \vspace{1.5in}
%     A thesis submitted to\\
%     The University of Kaiserslautern\\
%     for the degree of
%   \end{center}
%   \vfill
%   \includegraphics[height=2cm]{ITWM_rgb_200x80px}
%   \hfill
%   \parbox{2.8in}{Department of computer science\\
%                  University of Kaiserslautern\\
%                  April 2007}
%   \end{singlespace}

% \end{titlepage}

%%% Local Variables: 
%%% mode: latex
%%% TeX-master: "main"
%%% End: 


\frontmatter
%  P R E F A C E
\setchapterpreamble[o]{%
  \dictum{\textit{Virtualization  is a  concept that  one cannot  think  away from
    computer science anymore.}}}

\chapter{Preface}
\thispagestyle{empty}

This work  is about  the execution of  arbitrary applications on  a remote
system exploiting virtual machines. It addresses the problem that multiple
potentially  broken applications  are typically  executed on  remote hosts
side by  side ---  if one application  behaves ``wrong''  (e.~g.~CPU cycle
consumption,  memory  leakage, etc.)  it  may  involve other  applications
running on the same host.

\textbf{Motto:} \emph{Secure execution by separation with virtual machines.}
\vfill

% chapter overview
\begin{description}
\item[Chapter   \ref{cha:intro}]   The   first   chapter   contains   some
  introductory material that you should  read if you are not familiar with
  virtualization  technologies   and  especially  the  \emph{\gls{glo:XEN}
    hypervisor}.
  
\item[Chapter \ref{cha:requirements}] In  this chapter the requirements of
  the execution environment are discussed.
  
\item[Chapter \ref{cha:design}] This chapter  deals with the design of the
  \emph{\gls{glo:XEN}-based Execution Environment} and its components.
  
\item[Chapter  \ref{cha:comm-prot}]  The   fourth  chapter  discusses  the
  protocols  used for  communication between  the different  parts  of the
  system.
  
\item[Chapter  \ref{cha:secur-cons}]  In  this  chapter  security  related
  topics are discussed, such as  securing the messages sent between client
  and server  using \emph{Message Layer Security}  (\gls{glo:MLS}) and how
  the executing  of user-supplied  scripts are secured  using \emph{chroot
    environment}.
  
\item[Chapter \ref{cha:results}] This chapter  shows some results of using
  the  \emph{\gls{glo:XEN}-based Execution  Environment}  and compares  it
  against running the same programs on a stand-alone machine.
  
\item[Chapter   \ref{cha:conclusions}]  The   final  chapter   draws  some
  conclusions about  the work  that has been  done and provides  ideas for
  future development.
  
\end{description}

\clearpage

%%% Local Variables: 
%%% mode: latex
%%% TeX-master: "main"
%%% End: 


%  C O N T E N T S ,   T A B L E S  ,   F I G U R E S

\newpage
\pdfbookmark[0]{Contents}{contents} % Sets a PDF bookmark for the contents page
\tableofcontents
%\thispagestyle{empty}
\clearpage

\listoftables
%\thispagestyle{empty}
\clearpage

\listoffigures
%\thispagestyle{empty}
\clearpage

\mainmatter
%\setcounter{page}{1}

%  C H A P T E R S
% Each chapter is in its own file that starts
% with \chapter{Things And Widgits}

\setchapterpreamble[o]{%
  \dictum[Intel Corporation]{``Virtualization is a paradigm shift; it
    changes how you think about your resources.''}}

\chapter{Introduction}
\label{cha:intro}

Virtual machine (VM) technology is a major development in computer systems
design   \cite{buzen73}.     They   have   extended    the   multi-access,
multi-programming  and multi-processing  systems  to be  multi-environment
systems  as well by  providing efficient  ``copies'' of  complete computer
systems \cite{goldberg73}.

Todays  personal   computer  systems   are  powerful  enough   to  provide
virtualization  technology, which  has been  long time  reserved  for high
performance  mainframe  systems. The  paper  \emph{Analysis  of the  Intel
  Pentium's    Ability    to   Support    a    Secure   Virtual    Machine
  Monitor}\cite{robin00analysis}  discusses the  problems  of how  virtual
machines can be efficiently supported by the IA32 architecture.

In  the last few  years, the  remote execution  of applications  gained on
importance, because powerful server systems need not be maintained locally
but  could be  deployed elsewhere.  Grid  middle-wares such  as Globus  or
Unicore  are examples  for environments,  in  which users  can send  their
computation to  some computer  connected to the  grid and get  the results
back.

This work  is about  the execution of  arbitrary applications on  a remote
system exploiting virtual machines. It addresses the problem that multiple
potentially  broken applications  are typically  executed on  remote hosts
side by  side ---  if one application  behaves ``wrong''  (e.~g.~CPU cycle
consumption,  memory  leakage, etc.)  it  may  involve other  applications
running on the same host.

Sharing of computing resources must always consider security, availability
of  the resources and  fairness among  the resource  users.  To  make that
clear,  take  two jobs  from  different users  that  are  scheduled to  be
executed on the  same grid-node with overlapping execution  times. Let one
of the jobs have  a memory leak --- which is not  as uncommon as you might
think --- that task will eventually  eat up all the available memory. This
misbehavior not only  harms the other job, but also  may harm the software
used to connect the host to the grid.

The here  proposed solution is separation using  virtual machines provided
by  the Xen  hypervisor. Each  job  will be  executed in  its own  virtual
machine and  is not  able to harm  either other  jobs running on  the same
physical host, or the physical node itself.

\section{The history of virtualization technologies}
\label{sec:virtualization-history}

The history of virtualization starts  with a paper entitled ``Time sharing
in  large,  fast  computers''  \cite{Strachey59}  written  by  Christopher
Strachey in  1959. His idea bases  on a single-CPU  system which processes
jobs one  after each  other. If  a program blocks  due to  some peripheral
access the  next program in the queue  gets started and will  be run until
the next peripheral  access occurs and so forth.   The system presents the
user with  a \emph{logical CPU} and  a scheduler assigns  this logical CPU
transparently  for   the  user  to   a  physical  CPU.   His   concept  of
``time-sharing''  is now  known  as \emph{\gls{glo:multi-programming}}  as
Christopher Strachey states in a letter to Donald E. Knuth in 1974.

This very simple scheduling strategy maximizes the utilization of the most
worthy resource  to that time  --- the CPU  --- and provides the  base for
current scheduling strategies such as \gls{glo:multi-tasking}.

\subsubsection{The Atlas Project}

Later on  in the  early 1960s,  the ``Atlas project''  --- a  joint effort
between the  University of Manchester and  Ferranti~Ltd.~has been founded.
The Atlas  computer has been the  most powerful mainframe  computer in the
world in  those days.   It provided spooling  mechanisms and  pioneered in
\emph{demand  paging}  and  \emph{supervisor  calls}, they  also  invented
``\gls{glo:virtmem}'' --- called ``one-level store'' in
the Atlas system.

The  supervisor calls  where activated  through interrupt  routines  or by
so-called   \emph{\gls{glo:extracode}}  instructions   within   an  object
program.  Atlas made use of two ``virtual machines'' --- one executing the
\gls{glo:supervisor}  and  the  other   was  used  to  run  user  programs
\cite{virtualization-overview}.

\subsubsection{The M44/44X Project}

The IBM Watson  Research Center has been the home  for the M44/44X Project
in the  mid 1960s. The goal of  this project was to  evaluate the upcoming
concepts of  \gls{glo:time-sharing}.

The research team, led by R.~A.~Nelson, developed a way of partitioning an
IBM 7044 machine into sub-machines that  were each images of the 7044 with
less memory  --- the  main machine was  called M44, the  sub-machines 44X,
thus the project's name \cite{virtualization-overview}.

Especially David  Sayre and Belady made extensive  experimental studies to
evaluate the  performance of  \gls{glo:virtmem}, load control  and various
scheduling policies \cite{Belady66,denning81}.

\subsubsection{IBM virtual machines and the ``virtual machine monitor''}

IBM    has   been   perhaps    the   most    important   force    in   the
\gls{glo:virtualization} area  and a  number of IBM-based  virtual machine
systems were developed: the CP-40  (for a modified version of IBM 360/40),
the CP-67 (for  the IBM 360/67) and of course the  famous VM/370, and many
more.

The  VM/370 is  the name  for three  operating systems,  the \emph{Control
  Program}  (CP),  \emph{Conversational  Monitor  System}  (CMS)  and  the
\emph{Remote Spooling and Communications Subsystem} (RSCS) \cite{creasy81}.
Together they provide a way to form virtual machines, which can be used by
many users. The CP therefore  simulates multiple copies of the hardware on
which it is running.  The CMS is the operating system which runs in such a
``virtual machine'' and provides access for the users.

These  virtual  machines  were   typically  identical  ``copies''  of  the
underlying  hardware. A  special  component called  the ``virtual  machine
monitor''  (\gls{glo:VMM}) ran  directly  on the  real hardware.   Several
virtual machines could then be created using the VMM by assigning parts of
the hardware to the virtual machine. The virtual machine could then run an
operating system on its own and  only had access to those parts configured
through  the VMM.  By  this, new  operating  systems could  be tested  and
developed in a stable and secure manner.

\begin{figure}[htbp]
  \begin{center}
  \begin{minipage}{0.75\textwidth}
    \begin{center}
      \includegraphics[scale=.75]{architecture-with-virtualization}
    \end{center}
    \caption[Virtualization   architecture]{The   so-called   bare-metal
      virtualization:   A  virtual   machine  monitor   runs   as  small
      ``operating system'' on the  real hardware and provides access for
      virtual machines.}
    \label{fig:arch-virt}
  \end{minipage}
  \end{center}
\end{figure}

This  kind   of  virtualization  is  called   ``full''  or  ``bare-metal''
virtualization  ---  an operating  system  running  in  a virtual  machine
provided by the  VMM thinks it runs on real  hardware. An abstract picture
of such an architecture is shown in Fig.~\ref{fig:arch-virt}.

\section{Virtualization techniques}
\label{sec:techniques}

According   to    the   Church-Turing~Thesis   \cite{church_turing_thesis}
\emph{``Any real-world  computation can  be translated into  an equivalent
  computation involving  a Turing machine.''}, every  computer can simulate
any other computer.

\bigskip

\subsection{Partitioning}
\label{sec:vt-partitioning}

The first virtualization technique  bases on partitioning of the available
hardware \cite{borden89}.   It is available  since the 1960s and  has been
provided  by  two  approaches  ---  a software  approach  and  a  hardware
approach.

\subsubsection{Software partitioning}
\label{sec:softw-part}

The IBM CP/67  has been the first operating  system which provided virtual
machine support, it  was running on the System/360 Model  67 and was first
available in 1967 \cite{borden89}.

As described earlier, the CP gave each user a virtual machine on which the
CMS (a  single user  operating system) was  running and provided  the user
with command processing and information management functions. Each virtual
machine was ``copy'' of the base hardware architecture, it was possible to
run OS/360 in  a virtual machine and ``in fact, even  CP/67 itself was run
"second  level" in a  virtual machine  for the  purposes of  debugging and
testing \cite{borden89}''.

\subsubsection{Hardware partitioning}
\label{sec:hardw-part}

Hardware partitioning is an enhancement over software partitioning and was
introduced  by  the  IBM   \emph{System/370  158  MP}  and  \emph{168  MP}
systems. In 1967,  IBM introduced multiprocessor versions of  Model 65 and
67, which  provided duplexed hardware to achieve  tolerance against single
hardware failures.  By  splitting up the whole system  into two sides, two
separate systems could be created  which ran totally independent from each
other.

\subsection{Emulation}
\label{sec:emulation}

This kind of virtualization  simulates a complete hardware architecture in
all  details. By  that every  operating system  and  therefore application
which has been  developed for that particular application  may be executed
within the emulator. Some examples are:
\begin{itemize}
\item  \emph{Wine \cite{wine}}  --- Wine  is  not quite  an emulator,  but
  nonetheless I put it in this list, too. It emulates the Windows API, but
  executes many functions directly  on the underlying x86 hardware without
  emulating each instruction.
\item \emph{Bochs  \cite{bochs}} --- this  is a very portable  open source
  IA-32 PC emulator.  Each machine  instruction will be interpreted and is
  handled in software.  This emulator may for instance be  used to run x86
  code on a PowerPC platform.
\item  \emph{QEMU  \cite{qemu}} ---  this  is  both  an emulator  and  an
  virtualizer, since  it support  two modes of  operation. As  an emulator
  each instruction  gets interpreted  as it is  the case  of \emph{bochs},
  e.g.~it is possible  to emulate an ARM processor on  your PC. To achieve
  better  performances a  technique called  \emph{dynamic  translation} is
  used.   Running as  a virtualizer,  QEMU is  able achieve  nearly native
  performance, since most of the instructions are directly executed on the
  host CPU --- to make this possible a kernel module (QEMU accelerator) is
  required.
\end{itemize}

There are several software  and hardware solutions available which provide
a  virtualized  platform.

\begin{description}
\item[Emulation] An  emulator simulates the complete hardware  and as such
  allows the execution of unmodified guest operating systems. Examples are
  Bochs and Qemu (without its acceleration techniques) for instance.
\item[Native vir\-tu\-ali\-za\-tion] The  virtual machine in this approach
  simulates just enough hardware to let  an unmodified guest to be run. As
  mentioned earlier,  this field  was pioneered in  the beginnings  of the
  1960s.
\item[Paravirtualization]  In  this case,  the  virtual  machine does  not
  necessarily need to simulate hardware, instead it provides a special API
  to which a guest system has to  be ported. That means one can not run an
  unmodified guest on  top of a paravirtualized machine.  Examples in this
  field include Xen and VMWare ESX Server.
\item[Operating system-level  virtualization] This kind  of virtualization
  uses the same  kernel for the host and all guest  systems --- the guests
  run  ``within''  the  host.   Examples include  Solaris  Containers  and
  Free-BSD Jails.
\item[Application  virtualization] The  Java virtual  machine is  a widely
  known  and used implementation  of this  technique. Programs  written in
  Java are compiled into ``Java byte  code'' which is then executed by the
  Java   runtime   environment   (virtual   machine  plus   an   operating
  environment).  This  approach made it  possible to create  software that
  can  be run  on every  platform for  which a  port of  the  Java virtual
  machine exists.
\end{description}

In this  work I am  dealing with the ``paravirtualization''  technique and
especially with the Xen hypervisor.

\subsection{Problems on x86 hardware}
\label{sec:x86-problems}
\nocite{robin00analysis}

From \cite{popek75}: New architectures are discussed in \cite{goldberg73},
while hardware and software methods  which have been employed for existing
third generation  architectures are described in  \cite{buzen73}. In these
latter cases, the  virtual machine monitor typically is  run in privileged
mode,  and all  other  software in  user  mode. For  this  approach to  be
successful,  all ``sensitive''  instructions must  trap when  execution is
attempted in user mode, so that they can be simulated.




\section{The Xen hypervisor}
\label{sec:xen-hypervisor}

\cite{hendricks79}: ``The term “hypervisor” is applied to computer systems
that present a very basic  user program interface-one which is so nearly
identical to a  particular computer machine interface that  an operating
system intended  to support such machines  may serve as  a hypervisor user
program without  software modification. The user interface  presented by a
hypervisor is commonly called a  virtual machine; the term “subsystem” may
be applied to the complex of software used within a virtual machine.''

Xen  \cite{xen}  is  a  virtualization  technology to  partition  a  given
physical machine into smaller virtual machines (i.e. \gls{glo:VM}s).  Each
of these  \gls{glo:VM}s has their own  main memory, file  space, access to
one or  more virtual CPUs and everything  else that is required  to run an
operating system.   Xen belongs to the hosted  virtualization group, which
means that the \gls{glo:VMM} still requires an operating system to run and
does not represent a stand-alone operating system itself.

\begin{figure}[htbp]
  \begin{center}
    \includegraphics[height=8cm]{xen-architecture}
  \end{center}
  \caption[Xen architecture]{The structure of a system running the Xen
    hypervisor and several user domains (taken from \cite{xen-art})}
  \label{fig:xen-architecture}
\end{figure}

Xen  virtual  machines  (see Fig.~\ref{fig:xen-architecture})  are  called
``domains''  and   the  top-level  or   most  privileged  one   is  called
\texttt{Domain-0}  --- or  \texttt{dom0} for  short ---  this is  the one,
which runs the control and management programs that are required to create
new virtual machines.  The  virtual machine instances beside \texttt{dom0}
are called ``user  domains'' --- or \texttt{domU}s for  short --- they are
less privileged and their access to the hardware is controlled and managed
by the hypervisor running in \texttt{dom0}.

The  operating  system  running  within  a user  domain  accesses  virtual
hardware  provided by  the Xen-architecture  (SCSI or  IDE  controllers to
access virtual hard drives, network interface cards, virtual CPUs, graphic
card device and so on).


\section{Benefits of virtualization}
\label{sec:benefits}

To  satisfy customer  demands,  not a  single,  general purpose  operating
system  can be  used, over  the time  dozens of  specialized  systems have
evolved --- for instance on  consumer desktop systems one can find Apple's
MacOS,  Microsoft   Windows,  some  Linux  distribution   with  a  desktop
environment,  on cluster  systems  or  on systems  which  have to  provide
outstanding  security and availability  typically other  operating systems
are  used   (Linux  for  instance,  may   be  an  exception   due  to  its
adaptiveness).  Each  system has been designed  to address the  needs of a
large segment of the marketplace \cite{borden89}.

\bigskip

The following list are reasons, why multiple operating systems may need to
coexist and  even be used in the  same establishment at the  same time and
how virtual machines fit into that \cite{borden89,virtualization-overview}:

\begin{itemize}
\item  \emph{Diverse Workload}  --- in  large establishments  it  is often
  required, that  different computing  requirements must be  fulfilled. An
  airport, for instance, requires  a highly responsive reservation system,
  a  database system  for aircraft  maintenance  and parts  and a  general
  purpose system for payroll and planning.
\item  \emph{Test  and  development}   ---  many  companies  require  high
  availability and stability of  their computing components, but they also
  may want to test, develop  and deploy new software or software versions.
  
  Here at  the Fraunhofer ITWM,  for instance, SuSE  Linux is used  as the
  operating system for many desktop systems. To provide stability, updates
  have to be tested prior deploying them everywhere.

  The same holds for application  development --- a company will have some
  productive system running the developed application, and this system has
  to be available 24 hours  day. Clearly, development cannot take place on
  the production machines,  since the probability of a  breakage is simply
  too high --- machines solely dedicated for development are required.
\item \emph{Backup  and recovery} --- server  applications which represent
  important  components   for  the  daily   work  (e.g.~database  systems,
  web-servers etc.) must  be available all the time. A  nice thing to have
  in such  a situation  is automatic recovery  from failure ---  to reduce
  down-time two systems can be used: one of them is the productive system,
  the  other one  is a  backup system  that is  an exact  ``copy''  of the
  productive system  and just waits  for a failure  of the other.   Upon a
  system failure  in the  productive system, the  backup system  will take
  over.
\item  \emph{Platform  independent development}  ---  over  the time  many
  different  operating  systems have  evolved  ---  various Unix  flavours
  (FreeBSD, Linux for  instance), DOS, Microsoft Windows, Mac  OS, just to
  name some of them --- not all of them are compatible to each other.

  Application developers who  want to develop an application  that runs on
  several of these operating systems are required to port and verify their
  application on each  target system and maybe on  several versions of the
  same target system.

  One way  is to have  an extra machine  with the target  operating system
  installed on  it just for testing  and porting issues ---  that not only
  imposes energy  costs but also maintenance  costs.

  The other way is to use  virtual machines instead of actual machines ---
  each virtual machine runs a version of a target operating system.
\item  \emph{Server consolidation} ---  a common  approach for  many small
  companies is to have one server on which runs for instance a web server,
  some content management system with a database as its back-end under the
  same  operating system.

  That approach  may involve security problems,  since if just  one of the
  services   contains  a   security  hole,   the  whole   system   can  be
  compromised. Using virtual  machines for each of the  services, the only
  system  that is  now compromised  is the  one providing  this particular
  service.
\end{itemize}


%%% Local Variables: 
%%% TeX-master: "main.tex"
%%% End:

\setchapterpreamble[o]{%
  \dictum[James  Magary]{\textit{``Computers can figure  out all  kinds of
      problems, except the things in  the world that just don't add up.''}
  }}

\chapter{Requirements Analysis}
\label{cha:requirements}

This chapter  details on the  goals that were  outlined at the end  of the
previous chapter. To analyze  the requirements to the \gls{glo:XenBEE} the
analysis process is divided into two parts: \emph{Functional Requirements}
and \emph{Non-Functional Requirements}.

The section  on \emph{Functional Requirements}  aims to analyze  the first
two  goals of  this work,  \ie  \emph{Batch job  execution semantics}  and
\emph{On-demand server deployment}.  Both are pure functional requirements
that have  to be  designed and implemented  in the  \gls{glo:XenBEE}. This
section will also contain some additional use cases the are related to the
job  execution.   The provided  use  cases  are  analyzed with  regard  to
\emph{Integrability} of the \gls{glo:XenBEE} into grid-like environments.

The \emph{Non-Functional  Requirements} address the  goals \emph{Security}
and \emph{Efficiency}. Therefore some ideas will be presented that provide
a secure and efficient execution of jobs.

\section{Functional Requirements}
\label{sec:req:functional-requirements}

This section discusses the execution semantics that are to be supported by
the \gls{glo:XenBEE}.   In particular  it analyzes how  batch jobs  can be
executed on  a remote virtual machine  and how server  applications can be
deployed on-demand to virtual machines.

The following sections  describe the execution of these  kind of jobs, but
first of all the basic execution semantics are analyzed.

\subsection{Basic execution semantic}

Suppose you  wanted to execute a  job on a remote  resource. The execution
environment should at least provide the following functions: submission of
a  job,  status  retrieval  and   termination  of  a  submitted  job  (see
Figure~\ref{fig:uc-basic-execution}).

In the \gls{glo:XenBEE} a job  is defined by an \emph{execution container}
along with a description of the  job. The execution container is a virtual
machine image that  contains the application to which  the job refers.

\begin{figure}[ht]
  \centering
  \includegraphics[scale=0.65]{uc-basic-execution}
  \caption{Basic execution semantics.}
  \label{fig:uc-basic-execution}
\end{figure}

The \emph{query status}  use case requires the modeling  of a finite state
automaton that describes the current state of a job (job-state model). The
\gls{glo:OGSA}-\emph{Basic  Execution Service} \cite{ogsa-bes}  provides a
stable, generic and extensible specification for such a job-state model (a
detailed description can be found in Section~\ref{sec:fundamentals:bes} on
page~\pageref{sec:fundamentals:bes}).   To support the  integrability with
grid-like  environments   this  specification   should  be  used   in  the
\gls{glo:XenBEE}.

The  \emph{terminate  job}  use  case  must  always  be  available  to  a
user. That  means a  user must be  able to  terminate the execution  of a
previously submitted job  at any time.  The \emph{terminate  job} use case
is also included in the \gls{glo:BES} state model.

\subsection{Batch job execution}
\label{sec:req:batch-job-execution}

A \emph{batch job} is a program that is executed by a computation resource
without further  user input, \ie  the opposite to  \emph{interactive job}.
Batch jobs  typically transform input  data into output data,  whereas the
input data  may also be  absent. If no  programming errors have  been made
these  kind  of jobs  finish  after an  undefined  but  finite time.   The
execution   of  the   POV-Ray   raytracer  \cite{POV-Ray}   to  render   a
user-supplied scene is an example for this kind of jobs.

Figure~\ref{fig:uc-batch-job-execution}  shows  the  individual use  cases
that are involved when submitting a batch job to the \gls{glo:XenBEE}. The
user has  to provide  the virtual  machine image and  the input  data. The
\emph{\gls{glo:xbed}}  must then access  these files  to create  a virtual
machine that executes  the batch job. The generated output  data has to be
made available by the \emph{xbed} so that the user can access it.

\begin{figure}[ht]
  \centering
  \includegraphics[scale=0.65]{uc-submit-task}
  \caption{Batch job execution use cases.}
  \label{fig:uc-batch-job-execution}
\end{figure}

The  following sections specify  the requirements  for the  job submission
description  and  provide  some  ideas  on  how  an  implementation  could
implement the data access.

\subsubsection{Job description}

The  crucial  part  of  this  use  case is  the  description  of  the  job
submission.     In   the    past   each    grid   middlewares    such   as
Condor~\cite{condor},     Unicore~\cite{unicore}     or     the     Globus
Toolkit~\cite{globus}  used  their  own proprietary  submission  language.
This  made  interoperability  between  different  grids  middlewares  very
difficult.

The   \emph{Job    Submission   Description   Language}   (\gls{glo:JSDL},
\cite{jsdl-spec}) is  generic description  language for the  submission of
computational jobs  to a remote  resource. To the  time of the  writing of
this  work the  mentioned  grid  middlewares have  already  moved to  this
language or are in progress to do so.

Since the \gls{glo:XenBEE} should  be integrable into grid environments, a
fixed requirement  for the \gls{glo:XenBEE} is to  use the \gls{glo:JSDL}.
For     more     information     on     the     \gls{glo:JSDL}     consult
Section~\ref{sec:fundamentals:jsdl}                                      on
page~\pageref{sec:fundamentals:jsdl}.

\subsubsection{Selecting a VM image}

The submission of a task includes  the selection of an image that contains
the application  the user  wants to execute.   A sophisticated  process of
image-selection can  be rather complicated, since it  involves matching of
available  images  against  a  description  provided by  the  user.   Such
selection mechanisms are out of the scope of this thesis.

\subsubsection{File provision and access}

In a preliminary step the client must make the \gls{glo:VM} image, as well
as  the  input data  available  to  the  \emph{xbed}.  The  \gls{glo:JSDL}
supports     \emph{Uniform    Resource     Identifiers}    (\gls{glo:URI},
\cite{rfc2396}) that  can be  used to accomplish  this task, \ie  the user
specifies  the location  of  an  input file  with  a \gls{glo:URI}.   The
\emph{xbed} is then able to access  (retrieve) the files.  The same can be
applied for the provision of generated output data, \ie the user specifies
the target location to which an output file should be uploaded.

\subsubsection{Virtual machine creation}

For each submitted job a new  virtual machine has to be instantiated. This
virtual  machine uses  the \gls{glo:VM}  image provided  by the  user. The
application that is to be executed is specified by the client with the use
of the \gls{glo:JSDL}.

To  actually  execute  the  application  in the  virtual  machine  another
component  is required: the  \emph{\gls{glo:xbeinstd}}. This  component is
then used to control and monitor the execution on the virtual machine.

\subsection{On-demand Server deployment}
\label{sec:req:server-deployment}

A  \emph{server application} or  a \emph{service}  is a  remotely executed
program loops over  the following steps indefinitely often:  wait for user
input, execute a  computation on the input data,  generate output data.  A
web server is  an example for such an application: it  waits until a user
(or  some  service)  makes a  request  to  it,  handles the  request  (\ie
retrieval  of a  document)  and  eventually returns  the  result (\ie  the
document).

\begin{figure}[ht]
  \centering
  \includegraphics[scale=0.65]{uc-deploy-server}
  \caption{Server deployment use cases.}
  \label{fig:uc-server-deployment}
\end{figure}

The use cases that are  involved in the \emph{on-demand server deployment}
process  are   depicted  in  Figure~\ref{fig:uc-server-deployment}.   This
process shares  most of  the involved use  cases with the  batch execution
process.

In contrast  to a batch job  submission, the virtual  machine instance may
run ``forever'', \ie the ``job''  runs until the \gls{glo:VM} is shut down
by the  user or  the job  itself is terminated.   This must  be explicitly
stated in the job description.

Another difference  is that the  \gls{glo:VM} must be reachable  through a
standard network  connection (\eg TCP/IP  connectivity).  This can  be the
case for  virtual machines that execute  batch jobs, too, but  it need not
to.

Based on the network connectivity, a login to the \gls{glo:VM} can also be
provided.  For  example by  using the \emph{Secure  Shell} (\gls{glo:SSH},
\cite{openssh}).

\subsection{Caching of data}
\label{sec:uc-data-caching}

Imagine a  user who wants to  execute the same  application several times.
That would mean he has to submit  the same image over and over again. This
imposes a  heavy load on the network  that is connecting the  user and the
server. It would  be wise to provide a caching  mechanism, that allows the
user to  store his image  on server-side. According to  the \emph{Locality
  Principle}  \cite{locality-principle},   the  caching  should  decreases
overall execution time,  too.  The involved steps to  cache data are shown
in  Figure~\ref{fig:uc-data-caching}.

\begin{figure}[h]
  \centering
  \includegraphics[scale=.75]{uc-data-caching}
  \caption[UC  Data  Caching]{A user  who  is  requesting  the caching  of
    (arbitrary) data.}
  \label{fig:uc-data-caching}
\end{figure}

The  \emph{cache data} use  case can  also make  use of  \gls{glo:URI}s to
refer to  the data.   The \emph{access cache}  use case requires  that the
entries can be \emph{listed}  and \emph{referenced}. The entries should be
specified  as \gls{glo:URI}s,  too.   This makes  them  available for  job
submissions.

To provide shorter execution times, the \gls{glo:XenBEE} should provide a
cache with the following requirements: Arbitrary data must be addable, a
cache listing must be available, cache entries must be indentifiable by
\gls{glo:URI}.

\subsection{Support for Calana}
\label{sec:calana-support}

Calana is a new agent-based  Grid scheduler that uses auctions to schedule
job  submissions.   A  short  description   of  Calana  can  be  found  in
Appendix~\ref{app:sec:calana}    and    in    \cite{dalheimer05agentbased,
  petry06}.

Calana  assumes that  a computation  resource supports  reservations. That
means the resource must  provide semantics to \emph{make}, \emph{confirm},
\emph{use} and \emph{cancel} reservations.

Figure~\ref{fig:calana-xenbee} describes how a  user would interact with a
system, that uses Calana for  job scheduling and the \gls{glo:XenBEE} as a
computation resource. This scenario  requires an agent that implements the
Calana  protocol   on  the  one  hand   and  the  protocol   used  in  the
\gls{glo:XenBEE} on the other hand.

\begin{figure}[htbp]
  \centering
  \includegraphics[scale=0.55]{uc-calana-xenbee}
  \caption[Calana and  XenBEE]{The actors and use cases  that are involved
    when a Calana agent uses the \gls{glo:XenBEE} as its resource.}
  \label{fig:calana-xenbee}
\end{figure}

The user requests a resource from the broker which will in turn open up an
auction among  its agents.  One of  those agents are shown  in the figure.
In order  to bid  on the auction  the agent  creates a reservation  on the
\emph{xbed}.   If the auction  is lost,  the reservation  is automatically
canceled.   If the auction  was won,  the user  is eventually  presented a
unique identifier for his reservation.

This reservation can then be used to submit a job to the \emph{xbed}.  The
\emph{xbed} must then check the validity of this identification number.

To support  Calana the \emph{xbed}  has to provide  reservation semantics,
\ie  it must be  possible to  \emph{make}, \emph{confirm},  \emph{use} and
\emph{cancel} reservations.

\section{Non-functional Requirements}

The following sections  describe shortly which non-functional requirements
the \gls{glo:XenBEE} should support.

\subsection{Security}

This  section aims  on  requirements  that are  related  to security.   In
particular  three  requirements  are presented:  \emph{authentication  and
  authorization}, \emph{secure communication} and \emph{secure execution}.

\subsubsection{Authentication and authorization}

Since the \gls{glo:XenBEE} provides a \emph{service}, the provider of this
service   may  want   to  restrict   access   to  a   selected  group   of
\emph{authorized} people.

Before  granting a  user  the  access to  the  execution environment,  the
identity  of  that  user  has  to  be  verified,  \ie  the  user  must  be
\emph{authenticated}. Without authentication  an unauthorized person could
simply pretend to be an authorized person.

\subsubsection{Secure communication}

Secure  communication  between  the  \emph{xbe}  and  the  \emph{xbed}  is
required    to   prevent   \emph{eavesdropping},    \emph{tampering}   and
\emph{message forgery}. 

That means an attacker must  not have the possibility to overhear probably
confidential data  that maybe  included in a  user's job  description, nor
should it be possible that he  can modify or even create new messages that
seem  to come from  this user.

A typical  approach in  Grid middlewares such  as Globus  \cite{globus} or
Unicore   \cite{unicore}   is   to   use  public-key   certificates   (\eg
\gls{glo:X509} certificates) to  provide authentication, authorization and
secure  communication.   The   \gls{glo:XenBEE}  should  follow  the  same
principles.            Section~\ref{sec:secure-communication}           on
page~\ref{sec:secure-communication}   describes   in   detail  how   these
requirements can be provided in the \gls{glo:XenBEE}.

\subsubsection{Secure execution}

This requirement targets at the  actual job execution.  The use of virtual
machines provide  already that  the jobs cannot  harm each  other, because
they are completely separated from each other.

But security has  to be provided on the Xen-host as  well. That means that
data that  belongs to one job (VM  image, input data, output  data, and so
on) must  not be modifiable  or accessible by  any other jobs ---  even if
both jobs belong to the same user.

Since  all files  are accessed  by publicly  reachable  \gls{glo:URI}s, to
ensure the  security of these files  it could be possible  to provide them
encrypted. Before they can be used  by the \emph{xbed} to create a virtual
machine, they have to be decrypted somehow.

\subsection{Efficiency}

Efficiency in the context of  the \gls{glo:XenBEE} means that the overhead
which  is imposed by  the communication  and the  use of  virtual machines
should be kept minimal.

The caching  of virtual machine  images can be  used to decrease  the time
that is  needed to  deploy a  new virtual machine.  This affects  both the
batch  job  execution and  the  on-demand  deployment  of servers  to  key
locations.

% \section{Summary}

% put this figure in the conclusions of this chapter

% \begin{figure}[htbp]
%   \centering
%   \includegraphics[scale=.7]{concrete-concept}
%   \caption[A more concrete concept]{The  evolved concept, using a managing
%     component (\emph{xbed}) to control the virtual machines.  Each virtual
%     machine  is  dedicated  to  a  single  task  submitted  by  some  user
%     (controlled  through the  \emph{xbeinstd} component).   Users  do only
%     interact with the managing component.}
%   \label{fig:concrete-concept}
% \end{figure}

% \section{A more concrete concept}

% The previous  sections dealt with a  rather abstract view  on the proposed
% execution  environment,  this section  however  outlines  a slightly  more
% concrete  concept  for  the  execution  environment  and  its  components.

% Handling of requests, that the users make to the execution environment and
% controlling the  virtual machines, requires an  additional component. This
% component will be called  \emph{Xen-Based Execution Daemon} or \emph{xbed}
% for short. If a user wants to interact with the execution environment, he
% does  so  by  communicating  directly  with the  managing  component  (see
% Figure~\ref{fig:concrete-concept}).

% Another component which  will be required to control  the actual execution
% of a task  within a virtual machine will  be the \emph{Xen-Based Execution
%   Instance Daemon} or \emph{xbeinstd} for short. There are several reasons
% why this  dedicated component  is required to  be running in  each virtual
% machine. The  most important one is,  that the same  virtual machine image
% could  be used  for more  than just  one application.   Another  reason is
% simply flexibility,  the \gls{glo:JSDL} allows a user  to exactly specify
% how and which application shall be run and this component adheres to that.

% Basically  this concept  defines a  client-server architecture,  where the
% client is represented  by the user (or some mediator  such as a web-portal
% or a  command-line client) and the  server is represented  by the managing
% component.

% The interaction between the users and the execution environment requires a
% communication   layer   over   which    the   requests   are   made.    In
% Section~\ref{sec:fundamentals:mom} I have already introduced \emph{Message
%   Oriented  Middlewares} and  how  message-queue servers  can  be used  to
% provide logical  connections between the  involved distributed components.
% Using the  same concept in this  execution environment makes  sense due to
% the following reasons:

% \begin{itemize}
% \item  There could be  more than  one server  providing such  an execution
%   environment.   Message-queue \emph{topics}\footnote{A \emph{topic}  is a
%     special queue to  which arbitrary many consumers can  subscribe.  If a
%     message gets sent to this  queue, all consumers (rather than just one)
%     receive  this message.}  can  be used  to handle  any number  of those
%   servers uniformly. The clients send  their requests to a single queue on
%   the  \gls{glo:MQS} and  the \gls{glo:MQS}  multiplexes it  to  all known
%   servers.
% \item The server could be in  need of sending messages to clients that are
%   not connected anymore. Think of a notification that could be sent when a
%   task finished its execution.
% \item The use of a MQS renders  it possible that the server can be located
%   behind a very restrictive firewall or even a \gls{glo:NAT}-gateway.
% \item  There is  no visible  difference for  a client,  whether it  uses a
%   direct connection or a connection through a \gls{glo:MQS}.
% \end{itemize}

% The  network  layers  that  are  involved when  using  a  message-oriented
% communication    with    a    message-queue    server   are    shown    in
% Figure~\ref{fig:mqs-layers}.



%%% Local Variables: 
%%% mode: latex
%%% TeX-master: "main.tex"
%%% End: 

\setchapterpreamble[o]{%
  \dictum[Lao-tzu]{\textit{``A journey of a thousand miles begins with
      a single step.''}}}

\chapter{Fundamentals}
\label{cha:fundamentals}

This chapter  provides you  with the description  of basic  principles and
concepts  that have  been  used in  this  work. The  chapter is  basically
separated into two  parts: Concepts that have already  been pointed out in
the  previous  chapter  and  descriptions  of technologies  that  lead  to
important design decisions.

The  first part  starts  with an  introduction  into the  \emph{Extensible
  Markup Language}. \gls{glo:XML} is the description language that is used
for  the  \emph{Job  Submission  Description  Language},  the  \emph{Basic
  Execution Service}  and the messages  which are used in  the implemented
communication  protocol  (see Section~\ref{sec:communication-protocol}  on
page \pageref{sec:communication-protocol}).

The second and last part of this chapter provides the reasoning which lead
to  the decision  to base  the \gls{glo:XenBEE}  on  an \emph{asynchronous
  message-passing  communication  model}.    It  also  discusses  how  the
communication between the distributed components can be secured.

\section[The Extensible Markup Language]
{The Extensible Markup Language (\gls{glo:XML})}
\label{sec:fundamentals:xml}

The Extensible Markup Language \cite{xml}  is a simple, yet very flexible,
plain text based description format. The format represents a subset of the
\emph{Standard  Generalized Markup Language}  (SGML).  It  can be  used in
variety of ways  and even more usages are discovered  still. Usages of XML
can be found in  \emph{XHTML}, \emph{RSS}, \emph{Atom}, \emph{Math-ML} and
many more.   Due to the  structured semantics of  XML, more and  more file
formats are  nowadays based  on XML,  thus replacing the  old INI  or Unix
\texttt{rc}  files  ---  a very  popular  example  in  this field  is  the
\emph{OASIS  Open Document  Format for  Office Applications}\footnote{More
  information   about  the   Open  Document   Format  can   be   found  on
  \url{http://www.oasis-open.org/committees/tc_home.php?wg_abbrev=office}}.

An XML-file is  an ordinary plain text file, that  could have been created
by any text  editor.  The most important building  blocks of XML-files are
\emph{elements}, \emph{attributes} and \emph{text}.

Elements   are  \emph{logical  structures},   that  can   have  additional
attributes and  sub-elements or \emph{children}, whereas  the children can
either be other elements or text.  The following example shows you a small
XML document. Every XML  document contains \textbf{exactly one} designated
\emph{root} element, which is simply the first element in the document.

For parsing purposes,  an XML document can be represented  as a tree, this
is  shown in Figure~\ref{fig:xml-example}.   A widely  used model  for the
in-memory  representation of  XML documents  is the  \emph{Document Object
  Model} (DOM).  Most  of the available XML parsers,  provide an interface
for parsing a document into  an in-memory representation that follows this
model. The programmer  can then simply add, delete  or modify elements and
attributes by  using an object-oriented  interface.  Using the  example in
Figure~\ref{fig:xml-example}, a programmer could for instance iterate over
all children  of the \texttt{root}-element, that have  a tagname (\ie the
name  of the  element) equal  to ``child''  --- in  this case,  that would
result in just two elements.

\begin{figure}[h]
  \centering
  \begin{tabular}{rc}
    \begin{minipage}[c]{.35\textwidth}
      \begin{lstlisting}[language=XML]
<root foo="bar">
  <child/>
  Some text
  <child>More text</child>
</root>
      \end{lstlisting}%
    \end{minipage} &
    \begin{minipage}[c]{.35\textwidth}
      \includegraphics[width=5cm]{xml-example}
    \end{minipage}
  \end{tabular}
  \caption{A simple XML example}
  \label{fig:xml-example}
\end{figure}

\subsection{Namespaces}

XML  documents   may  contain  elements  and   attributes  from  different
vocabularies (\ie different document  types). To resolve ambiguity between
the  involved   vocabularies,  the  W3C  recommends  the   use  of  unique
\emph{Namespaces}  that are  assigned  to each  element.   A document  may
contain a  \emph{default}-namespace to which  all elements belong  that do
not  have  a  special  namespace  assigned.   Within  a  single  document,
namespaces are given  a logical name. The logical  name itself is assigned
the unique Namespace identifier (\eg a \gls{glo:URI}). Some namespaces and
common   ``names''   for  them   are   given   in   the  following   table
(Table~\ref{tab:namespaces}):

\medskip

\begin{table}[ht]
  \centering
  \begin{tabular}{@{}ll@{}}\toprule
    logical name        & \multicolumn{1}{l}{namespace URI} \\ \midrule % header
    \texttt{xsd}        & \url{http://www.w3.org/2001/XMLSchema} \\
    \texttt{jsdl}       & \url{http://schemas.ggf.org/jsdl/2005/11/jsdl} \\
    \texttt{jsdl-posix} &  \url{http://schemas.ggf.org/jsdl/2005/11/jsdl-posix} \\
    \texttt{dsig}       & \url{http://www.w3.org/2000/09/xmldsig#} \\
    \texttt{bes}        & \url{http://schemas.ggf.org/bes/2006/08/bes-activity} \\
    \texttt{xbe}        & \url{http://xenbee.berlios.de/schemas/xbe/2007/01/xbe} \\
    \texttt{xbe-sec}    & \url{http://xenbee.berlios.de/schemas/xbe-sec/2007/01/xbe-sec} \\
    \texttt{xsdl}       & \url{http://xenbee.berlios.de/schemas/xsdl/2007/01/xsdl} \\
    \bottomrule
  \end{tabular}
  \caption[XML Namespaces used in this work]{Important namespaces}
  \label{tab:namespaces}
\end{table}

Namespaces  are   specified  in  the  XML  using   the  special  attribute
\texttt{xmlns}.     An   attribute    of   an    element,    which   reads
\texttt{xmlns="www.example.com"}, sets the  default namespace to the given
URI, while  \texttt{xmlns:foo="www.example.com"} makes the  same namespace
known as the logical name  ``foo''. In another document the same namespace
could have been assigned the logical name ``bar'' as well.

To specify  that a  given element  belongs to a  namespace other  than the
default namespace, the  element's name is prefixed by  the logical name of
the namespace,  \eg \texttt{foo:child} means, that  the ``child'' element
belongs to the namespace defined by ``foo''.

\subsection{Validation of XML documents}
\label{sec:xml-validation}

A  really nice  and very  useful  addition to  XML is  the possibility  to
\emph{validate}  an XML  document.  There  are two  mechanisms  to provide
validation,    the    \emph{Document    Type   Definition}    (DTD)    and
\emph{XML-Schema}.

\subsubsection{Document Type Definition}

A DTD defines for a particular  document what elements are allowed and how
their attributes look like. The  composition of elements to form container
(\ie parent)   elements  can   be   described  in   a  rudimentary   way.
Unfortunately, the  DTD uses  its own syntax,  that has nothing  in common
with  the syntax of  an XML  document. For  an author  of an  XML document
type\footnote{for example the configuration file format of an application}
that means in particular, that he has to learn two different syntaxes.

\subsubsection{XML-Schema Definition}

XML-Schema is  itself defined  using XML as  its description  language and
obsoletes the  DTD. It is much  more powerful, for instance,  an author is
able  to   restrict  the   actual  data  an   element  or   attribute  may
contain.  Let's for  instance  say, a  given  attribute can  only take  on
non-negative  integers.    To  reflect  this  constraint   in  the  schema
definition,  the   author  would  set   the  type  of  the   attribute  to
\texttt{xsd:nonNegativeInteger}.

There are many predefined data types, an author may use to create new data
type constraints.  An XML-Schema validator complains, if  a document, that
is  supposed to  conform  to  that schema,  contains  the just  introduced
attribute with a negative value, \eg $-1$.

The advantages of XML-Schema over a DTD are obvious.  When using DTDs, the
application  itself was  responsible to  check  the validity  of each  XML
document  that it  used.  That  means, the  same functionality  had  to be
implemented over and over again, \ie  each time a new application wants to
make use of a given XML document type. While using XML-Schema definitions,
the author of  an XML document type defines  the validity constraints just
once and any application  may rely on that.

To make that clear, here is a short example: Suppose there is a definition
for  an  element  called  ``\texttt{entry-id}''  which may  only  take  on
positive integer values.  Since a DTD does not support constraints on data
types, each  application must check for  itself if the  value matches that
type.  Now suppose the constraint for  that element is changed so that the
number $0$  is included  as well ---  in each application,  the validation
code must be modified to match the new constraint.

\section[The Job Submission Description Language]
{The Job Submission Description Language (\gls{glo:JSDL})}
\label{sec:fundamentals:jsdl}

\gls{glo:JSDL} is a very extensible XML-based description language for the
submission  of computational  jobs \cite{jsdl-spec}.   With \gls{glo:JSDL}
you are  able to  describe all requirements  that a computational  job may
need for the submission to a remote resource --- mainly the \gls{glo:JSDL}
addresses grid resources but it is not limited to that.

Nearly  every  element of  the  \gls{glo:JSDL}  specification may  contain
arbitrary  many  user-defined  elements  from  other  XML  specifications.
Therefore  is  the JSDL  adoptable  to  upcoming  user requirements.   The
\gls{glo:JSDL} has been  extended in this work to  support the description
of   virtual  machines   (see   Section~\ref{sec:xen-based-submission}  on
page~\pageref{sec:xen-based-submission}).

To explain  this, consider the  following example. To  execute a job  on a
resource  a previously  acquired  reservation is  required.   To add  this
information  to the job  submission, the  \gls{glo:JSDL} would  include an
additional   element   which   holds   all   information   regarding   the
reservation. Such extension elements  are purely optional and systems that
are unaware of a particular extension element may just neglect it.

The following section roughly describes the most important components that
are needed to form a useful submission description.

\subsection{``Hello World'' with the JSDL}

A    \gls{glo:JSDL}    document     does    always    start    with    the
\texttt{JobDefinition} element,  which is the top-level  element and holds
all required information about the job.

Let's assume a user wants to execute a small program on a remote resource.
The  program  will  indeed be  very  simple,  it  just prints  the  string
``\texttt{Hello World}'' to its  standard output stream.  The execution of
this application on the user's local computer could be similar to:

\begin{minipage}{0.75\textwidth}
  \begin{lstlisting}[language=ksh]
    $ echo "Hello World"
    Hello World
    $
  \end{lstlisting}
\end{minipage}

This  excerpt  represents  the   execution  in  a  standard  UNIX  command
shell. Note that the \texttt{echo}  program does nothing more than writing
the parameters passed to it to its \texttt{stdout} stream. Now suppose the
user  desires  to execute  the  same program  on  a  remote resource.  The
\gls{glo:JSDL}  document will  look somewhat  like the  document  shown in
Listing~\ref{lst:jsdl-example}.

\medskip
\begin{center}
%  \begin{minipage}{.75\textwidth}
    \begin{lstlisting}[captionpos=b,float=t,backgroundcolor=\color{listingcolor},frame=lines,numbers=none,stepnumber=5,numberfirstline=false,numberstyle=\tiny,caption={A
      small ``Hello World''-example written in
      \gls{glo:JSDL}},label={lst:jsdl-example},language=XML]
<jsdl:JobDefinition>
 <jsdl:JobDescription>
    <jsdl:Application>
      <jsdl-posix:POSIXApplication>
        <jsdl-posix:Executable>
          /bin/echo
        </jsdl-posix:Executable>
        <jsdl-posix:Argument>Hello World</jsdl-posix:Argument>
      </jsdl-posix:POSIXApplication>
    </jsdl:Application>
  </jsdl:JobDescription>
</jsdl:JobDefinition>
     \end{lstlisting}
%   \end{minipage}
\end{center}

Note, that  the shown \gls{glo:JSDL}  document describes exactly  the same
execution  the  user  had  previously  performed  locally.   The  executed
\texttt{echo} program  again writes  its arguments to  its \texttt{stdout}
stream.   Different from the  local execution  is in  this case,  that the
output will  be lost, since the user  did not specify what  should be done
with the  generated output.  If the  user was interested  in the program's
output, he  had to specify  the redirection of  the output stream  to some
file  and a  staging operation  that transfers  the created  file  to some
location he has access to.

\subsection{Important elements}

A typical \gls{glo:JSDL} document consists  of the following parts --- job
identification,  application description,  resource descriptions  and data
staging elements. To keep the example simple, only the second one has been
used in the example  in Listing~\ref{lst:jsdl-example}.

\subsubsection{Job identification}

The \texttt{JobIdentification}  element is used to  hold information about
the job, such as a descriptive  name, that is mostly interesting for human
beings.  Nonetheless it  may hold additional information that  could be of
interest to  applications processing the document ---  such as annotations
(\eg a  unique task  identification number  could  be stored  in such  an
annotation).

\subsubsection{Application}

With the \texttt{Application} element, a user describes the program itself
---  \ie the  real executable,  that  is going  to  be  used.  A  special
extension --- \texttt{POSIXApplication}, also defined in the specification
\cite{jsdl-spec}   ---   can  be   used   to   describe  executables   for
\gls{glo:POSIX}-compliant   operating  systems  \cite{posix}.    You  have
already      seen     the     usage      of     this      extension     in
Listing~\ref{lst:jsdl-example},  where it  had  been used  to specify  the
execution of the \texttt{echo} command line program.

\subsubsection{Resources}

This element can be used  to describe various resource requirements of the
application. Some of the many resource types one can use are listed below.

\begin{itemize}
\item the number of CPUs the job requires
\item the operating system required by the job
\item amount of virtual memory that must be available for the job
\item file-systems  and their expected  mount-points
\end{itemize}

All  specified file-systems  must  be  made available  for  the job  prior
execution. Every file-system specification  defines a unique name that can
be used to refer to that particular file-system in other elements, such as
staging operations.  Thereby the  user can define \emph{logical names} for
special directories within the execution environment of the task.
  
\subsubsection{Staging operations}

The   \texttt{DataStaging}    element   is   used    to   define   staging
operations.  These operations  can either  be  \emph{Stage-In} operations,
which  refer  to files  that  have to  be  transfered  into the  execution
environment  prior   the  execution  of  the   task,  or  \emph{Stage-Out}
operations, which refer  to transfers that have to be  made after the task
has been executed.

A user may  specify the \texttt{DataStaging} element as  often as he likes
to.   The most  relevant elements  within  a staging  instruction are  the
\texttt{Source}  and \texttt{Target}  elements, both  of them  can  hold a
\texttt{URI}  element  to  specify  a  generic  location.   The  mandatory
\texttt{FileName} element points to  an actual file within the file-system
hierarchy of the execution environment of the task. The actual location of
a file can be given relative  to a previously defined file-system, in this
case the \texttt{FilesystemName} element must be specified and is required
to contain the logical name of \texttt{FileSystem} resource.

\subsubsection{Conclusions}

The \gls{glo:JSDL}  is a  powerful description language  for computational
jobs. It aims  to cover the description of the  job submission for typical
computational  jobs. The \gls{glo:XenBEE}  is such  an example,  since the
\gls{glo:JSDL}  does not  know anything  about executing  jobs  on virtual
machines.  The extensions  to the \gls{glo:JSDL} that I  have designed and
implemented  are   covered  in  Section~\ref{sec:xen-based-submission}  on
page~\pageref{sec:xen-based-submission}.

The  next  section  deals  with  the  \emph{Basic  Execution  Service},  a
specification that  provides a common execution  semantic of computational
jobs. This semantic is backed  up by an extensible state-model to describe
the job execution.

\section[The Basic Execution Service]{The Basic Execution Service (BES)}
\label{sec:fundamentals:bes}

The \emph{Basic Execution Service} \cite{ogsa-bes} is a specification that
defines  a service  (\eg a  web service)  which provides  functionality to
control \emph{Activities}.  An \emph{activity}  can be seen as an abstract
view on a computational job. A user  is able to submit an activity to the
execution service and  can later on control and  monitor that activity ---
using web service calls, for instance.

Control  of  the  activity  is   basically  limited  to  a  request  which
\emph{terminates} the  activity. The monitoring of an  activity results in
returning the activity's current state.

The state  of an activity is  modeled using a finite  state automaton. The
specification  of  the  \gls{glo:BES}  incorporates  a  simple,  but  very
extensible, state  machine for  activities. It comprises  a total  of just
five  states  an  activity  can  be  in  at  any  time:  \texttt{Pending},
\texttt{Running},       \texttt{Finished},       \texttt{Failed}       and
\texttt{Terminated}. 

To be extensible and integrable  into existing environments, the states of
the \gls{glo:BES}  are represented using a XML  specification. The element
which   represents   the  current   state   of   an   activity  is   named
\texttt{ActivityStatus} and belongs to  the \emph{bes} namespace as it has
been  defined  in Table~\ref{tab:namespaces},  the  state  itself is  then
specified using the \texttt{state} attribute:

\begin{lstlisting}[language=XML]
  <bes:ActivityStatus state="Running"/>
\end{lstlisting}

The    basic    state   machine,    or    state-model,    is   shown    in
Figure~\ref{fig:bes-basic}. To  reflect the request for  termination of an
activity, each  non-terminal state provides an outgoing  transition to the
\texttt{Terminated} state.

\begin{figure}[h]
  \centering
  \includegraphics[scale=.6]{bes-basic-job-model}
  \caption[Basic BES Job-State-Model]{This is the job-state-model as it is
    defined in the BES specification \cite{ogsa-bes}}
  \label{fig:bes-basic}
\end{figure}

This  basic state-model  represents everything  a possible  client  of the
execution service needs  to know. An actual execution  service may require
to  use  additional  states.   It  can  define both  new  states  and  new
transitions,  as  long  as  it  conforms  to a  rather  simple  rule:  the
\textbf{visual behavior}, as it is experienced by some client, must not be
altered. A breakage of this rule would be the introduction of a transition
from  the \texttt{Running}  state  back into  the \texttt{Pending}  state.
Clearly, the visual behavior a  client experiences, has changed, since the
client  simply  does   not  expect  that  the  activity   is  suddenly  in
\texttt{Pending} again.

\subsection{Extending the state-model}

New  states can  be added  by  splitting up  either one  of the  ``basic''
states, or  a state that is  an extension itself.   Among these sub-states
any number  of new transitions  may be introduced.  The  following example
for  an extension  has  been taken  from  the \gls{glo:BES}  specification
\cite{ogsa-bes}.

Suppose the execution service provides  a way to suspend a given activity.
This  requires  not only  additional  user  requests  --- one  to  request
\emph{suspension} and  one to request  \emph{resumption} --- but  also two
new states. These states are modeled as sub-states of the \texttt{Running}
state, since an  activity may only be suspended  while it already running.
The  \texttt{Running} state is  now internally  split into  the sub-states
\texttt{Executing} and \texttt{Suspended}.   The extended state machine is
shown in Figure~\ref{fig:bes-suspend-model}

\begin{figure}[h]
  \centering
  \includegraphics[scale=.6]{bes-suspend-job-model}
  \caption[Extended  BES Job-State-Model]{The  basic state-model  has been
    extended  to  support suspension  (taken  from  the BES  specification
    \cite{ogsa-bes}).}
  \label{fig:bes-suspend-model}
\end{figure}

The nice  thing about  the extensibility of  this state-model is  that any
client that ``understands'' the  basic model, will understand any extended
model as well. That is  because the \emph{visual behavior} does not change
and  therefore  will never  ``anything  unexpected''  happen. This  visual
behavior is  directly reflected  in the XML  specification of  the current
state.  Any  additional states are represented  by user-definable elements
added  to  the  \texttt{ActivityStatus}  element as  sub-elements.   Let's
assume  the just modeled  extension defines  its own  state representation
using its own namespace.  The  current state of a suspended activity could
then be written as:

\begin{lstlisting}[float={h!},language=XML]
  <bes:ActivityStatus state="Running">
     <ext:Suspended/>
  </bes:ActivityStatus>
\end{lstlisting}

The state is still \texttt{Running}, but  any client that is aware of this
extension   will  know   how  to   interpret   the  \texttt{ext:Suspended}
sub-element, \ie it could provide  a user with the possibility to resume
the action.

The state-model  that is  used by the  \gls{glo:XenBEE} extends  the basic
model  with  support  for  staging operations,  reservations  and  virtual
machines,  it   can  be  found   on  page~\pageref{sec:xbed:job-model}  in
Section~\ref{sec:xbed:job-model}.

This section closes the description of the technologies that were required
for  the \gls{glo:XenBEE}  to provide  batch job  execution  semantics and
integrability into grid-environments.  The following sections motivate the
applied communication model and how  this model can be enhanced to provide
secure communication.

\section{Communication Model}
\label{sec:fundamental:communication-model}

Communication in distributed systems  can be either \emph{synchronous}, or
\emph{asynchronous}  \cite{MeCa:2005:Taxonomy}.   This section  summarizes
these two models and results  in the definition of the communication model
used by the \gls{glo:XenBEE}.

\subsection{Programming Models}

The programming model of  synchronous communication is called \emph{Remote
  Procedure Calls}  (RPC, \cite{rpc}). It provides the  same function call
semantics as a local function call, \ie the program waits until the result
is   computed  and  returned   by  the   remote  system.    Commonly  used
implementations of  this model are the \emph{Common  Object Request Broker
  Architecture} (CORBA, \cite{corba}), the \emph{Remote Method Invocation}
(RMI, \cite{rmi}) or the  \emph{Distributed Component Object Model} (DCOM,
\cite{dcom}).

Asynchronous   communication  follows   the  emerging   paradigm   of  the
\emph{event-based   communication  model}   \cite{MeCa:2005:Taxonomy}.   A
request  that is  sent  to a  remote  system does  not  have an  immediate
result. The  result, if any, is  received by the  caller asynchronously to
his current computation. The  distributed components that use asynchronous
communication  are interconnected  by  using message-passing  technologies
such as the \emph{Message Passing Interface} (MPI, \cite{mpi}).

The  \gls{glo:XenBEE}  can be  implemented  using  either  of two  models.
Consider for  example the request for  the termination of  a submitted job
using a single-threaded client application. The client could make a remote
procedure call  and wait until the  termination has been  performed on the
remote site.   Or the client just sends  the request to the  server and is
able to accept further input from the user.

The   \gls{glo:XenBEE}  will   be   designed  to   use  the   asynchronous
communication  model.    This  implies   the  use  of   a  message-passing
technology.

\citet{dad-mom}  state, that:
\begin{quotation}
  \emph{``Every  DAD  (Distributed  Application  Developer)  needs  a  MOM
    (Message Oriented Middleware)''}.
\end{quotation}


\subsection[Message Oriented Middleware]{Message Oriented Middleware (MOM)}
\label{sec:fundamentals:mom}

Message-passing in  distributed systems need  not be based on  direct (\eg
\gls{glo:TCP}) connections between each component. The messages can easily
be transmitted to an intermediate system which forwards the message to the
target  system.   This section  describes  a  middleware component  called
\emph{Message-Queue Server} (\gls{glo:MQS}) which can be used in this kind
of distributed systems to improve communication quality.

The abstraction  from direct connections  between each component  leads to
the definition  of \emph{logical connections}.  A logical  connection is a
connection between distributed components that pass messages to each other
while not being directly connected to each other.

To send a message to some component using logical connections, the message
is addressed to  a logical destination and sent  to an intermediate server
that hopefully  knows the  actual target. To  receive messages  from other
components, a component registers itself with the intermediate server.

Such   an   intermediate   server   is   a   \emph{Message-Queue   Server}
(\gls{glo:MQS}). The  protocol layers  that are involved  when application
data is  to be transmitted from  one component to another  are depicted in
Figure~\ref{fig:mqs-layers}.

\begin{figure}[ht]
  \centering
  \includegraphics[scale=0.7]{mqs-layers}
  \caption{Protocol layers in an \gls{glo:MQS}-based communication.}
  \label{fig:mqs-layers}
\end{figure}

The  usage of  such an  intermediate \gls{glo:MQS}  has  several important
advantages over direct connections between the distributed components:

\begin{itemize}
\item All  messages are sent to \textbf{logical  queues} (\ie end-points).
  This means that  the details of the connection of  a remote component is
  hidden to other components.  For  example, the IP address of a component
  may  change between  two subsequent  messages sent  to it  without being
  noticed by the sender.
\item All  connections are \textbf{outbound} which  effectively means that
  all  components  may  reside   behind  a  (restrictive)  firewall  or  a
  \gls{glo:NAT}-gateway.   This not  only  increases the  security of  the
  \emph{xbed},   but    also   targets   the    problems   which   typical
  network-policies     and    hence    resulting     network-layouts    of
  grid-environments or companies impose.
\item The  \gls{glo:MQS} need not  to be on  a single machine, but  can be
  distributed  over  many computers  to  implement \textbf{fail-over}  and
  \textbf{load-balancing}.
\item Messages  can be  kept in a  \textbf{consistent storage}  within the
  \gls{glo:MQS} if they  cannot be delivered right now.   That may happen,
  if the communication partner is temporarily disconnected --- all pending
  messages will be delivered as soon as the end-point connects again.
\item    Multiple   \gls{glo:MQS}   can    be   configured    to   provide
  \textbf{forwarding and  routing} of  messages destined for  a particular
  queue --- that means independence from the actual network-topology.
\item A \gls{glo:MQS} can be configured to provide \textbf{authentication}
  and \textbf{authorization} to limit access to particular queues.
\item   Messages   sent   from   one   component   to   another   can   be
  \textbf{transformed}  while passing  the \gls{glo:MQS}.   That  means in
  particular,  that each  component may  send messages  in its  own native
  format and the \gls{glo:MQS}  intelligently transforms the messages into
  the native format of the receiver.
\end{itemize}

The drawbacks  of the  usage of \gls{glo:MQS}  are: An increased  delay in
message  transmission,  because all  messages  must  be  processed by  the
\gls{glo:MQS}  before they  can be  delivered.  And  that the  secure (\ie
encrypted)  transmission  of  messages   has  to  be  implemented  by  the
applications themselves.  The latter issue is discussed in more detail and
with        regard         to        the        \gls{glo:XenBEE}        in
Section~\ref{sec:secure-communication}.

\begin{figure}[htbp]
  \centering
  \includegraphics[scale=.75]{mqs-topology}
  \caption[Example  MQS topology]{A simple  message-based system  which is
    using a \gls{glo:MQS}.}
  \label{fig:mqs-topology}
\end{figure}

An example for a distributed system  that uses a \gls{glo:MQS} is shown in
Figure~\ref{fig:mqs-topology}.  Site~$B$ has  some services connected to a
\gls{glo:MQS}.   These services can  be reached  by clients  from site~$A$
through an  internet connection.  The  steps involved in building  up this
communication scheme are:
\begin{enumerate}
\item  Each service  connects to  the \gls{glo:MQS}  and \emph{subscribes}
  itself to a unique queue (\eg service.\emph{X}).
\item    Clients   subscribe    themselves   to    unique    queues,   too
  (\eg client.\emph{Y}).
\end{enumerate}

Now  that each  party is  subscribed to  its own  unique queue,  a two-way
communication is possible:
\begin{enumerate}
\item A client  that wants to communicate with one  of the services, sends
  its messages  to the unique queue  of that particular  service. The sent
  message  contains a  special \emph{reply-to}  field that  is set  to the
  unique queue of the client
\item Answers from  a service to a connected client are  sent to the queue
  specified in the reply-to field of received messages.
\end{enumerate}

The same  communication scheme  will be used  in the  \gls{glo:XenBEE}, as
well.  Each  component subscribes  itself to a  unique queue,  whereas the
queue of the \emph{xbed} will be configurable by an administrator.

The  next section  describes how  the message-based  communication  can be
secured, \ie  eavesdropping, modification and forgery of  messages must be
prevented.

\section{Secure Communication}
\label{sec:secure-communication}

Since the proposed system uses message queues to transfer messages between
clients  and  the   server,  all  transmitted  messages  are   sent  to  a
message-queue server  first. If  this intermediate server  is compromised,
all messages that pass through it can also be read by the intruder.

There are  two different  approaches to ensure  a secure transport  of the
messages from a  client to a server and  vice versa: \emph{Transport Layer
  Security} (TLS, \cite{rfc2246})  and \emph{Message Layer Security} (MLS,
\cite{mls, oasis-wss}). 

Both of them use public-key certificates, \eg \gls{glo:X509} certificates.
A public-key certificate is a data  structure that binds a public-key to a
subject (person  or system)  \cite{rfc2459}. If a  communication is  to be
secured by the use of public-key certificates, the ``users of a public-key
must be confident that the associated private-key is actually owned by the
correct  remote subject''  \cite{rfc2459}.   This can  be accomplished  by
having a trusted authority  digitally sign the involved certificates. Such
an infrastructure  is a called  a \emph{Public Key  Infrastructure} (PKI).
More   information  about   public-key  cryptography   can  be   found  in
Appendix~\ref{app:sec:public-key-cryptography}                           on
page~\pageref{app:sec:public-key-cryptography}.

The following  section describes  the \emph{Public Key  Infrastructure} in
detail.  After that the  \emph{Transport Layer Security} and \emph{Message
  Layer  Security} protocols  are discussed  and analyzed.   Subsequent to
that the implications for the \gls{glo:XenBEE} are discussed.

\subsection[Public Key Infrastructure]{Public Key Infrastructure (PKI)}

A  \emph{Public Key  Infrastructure} provides  the authentication  of user
identities using  public-key certificates.  The main aspect  is that there
are special \textbf{trusted  third parties} (\emph{Certificate Authority},
\gls{glo:CA})  that   are  permitted  to   \textbf{digitally  sign}  other
certificates. If a user\footnote{or some server, etc.}  wants to prove his
identity to  another entity, his  certificate is validated by  that entity
against the CA's certificate.  If the validity could be verified, the user
successfully proved that he is  in possess of the private-key that belongs
to this public-key \cite{rfc2459}.

The \gls{glo:CA} is responsible for checking that the public-key contained
in  the certificate  actually belongs  to the  requesting user,  server or
other  entity denoted  in the  certificate.  This  process is  for example
performed by verifying the credentials of a user (\eg with help of a photo
identification).

Any third-party that trusts a given CA will transparently trust any entity
that offers a certificate signed by that particular CA.

Validation is performed by verifying  that the certificate itself has been
signed  by a  trusted authority  --- the  actual validation  process  is a
little   bit  more  complex,   since  it   involves  checking   against  a
\emph{Revocation List} and  a ``best before'' date (\ie life  time of the
certificate), too.

The  signing  process  uses  the  authority's private  key  to  compute  a
cryptographic signature. This private-key must of course be kept in a very
secure  location  (\eg on a  physically  from  the Internet  disconnected
computer) ---  if it would fall into  the wrong hands, the  whole chain of
trust is compromised.

\begin{figure}[h]
  \centering
  \includegraphics[scale=.55]{pki}
  \caption[Public  Key Infrastructure]{Alice  proofs her  identity  to Bob
    using a certificate  that is signed by a CA that  both, Alice and Bob,
    trust.}
  \label{fig:pki}
\end{figure}

An  example verification  process  is shown  in Figure~\ref{fig:pki},  the
steps can be described as follows:
\begin{enumerate}
\item \emph{Alice} request the signing of her certificate by a CA and thus
  sends a certificate request to the CA containing her public-key.
\item The CA in turn verifies Alice's credentials and eventually signs the
  certificate with its private key.
\item The signed certificate is sent back to \emph{Alice} for her later use.
\item Now,  \emph{Alice} wants to prove  her identity to a  friend of her,
  \emph{Bob},   therefore   \emph{Alice}    sends   her   certificate   to
  \emph{Bob}.  The   proof  may  be   necessary  to  establish   a  secure
  communication over an insecure channel, \eg the Internet.
\item \emph{Bob}  verifies the received certificate against  the very same
  CA by  which \emph{Alice} had her certificate  signed.  Since \emph{Bob}
  trusts the  CA and the received  certificate states, that  it belongs to
  his  friend \emph{Alice},  he  can be  assured,  that he  is talking  to
  \emph{Alice}.
\end{enumerate}

Both  of  the following  protocols  can make  use  of  a \gls{glo:PKI}  to
authenticate  communication partners.  Once  the communication  partner is
authenticated a public-key based secure communication can be set up by the
protocols.

\subsection[Transport Layer Security]{Transport Layer Security (TLS)}
\label{sec:fundamentals:tls}

The \gls{glo:TLS} protocol (RFC~4346, \cite{rfc4346}) can use certificates
on  both sides,  \ie  client  and server  side.  For websites  server-side
certificates are used so that  clients can validate that they are actually
communicating  with  the correct  server.  The  server's certificate  must
therefore  be  signed   by  an  authority  that  the   user  trusts.   For
authentication to  the server client-side certificates are  used.  In this
case the  client's certificate  must be signed  by an authority  which the
server trusts.

The  protocol is  split  into three  phases  \cite{rfc4346}.  Firstly  the
communication partners negotiate  the cryptographic algorithms that should
be used.  Secondly certificate-based authentication and a public-key based
key exchange  are performed.  The  last phase is the  actual communication
phase. In  this phase the transmitted  data is encrypted  with a symmetric
encryption algorithm.  The shared key that  is used had  been exchanged in
the second phase.

Since the \gls{glo:TLS} aims to secure the transport layer (\eg TCP), only
direct  connections  between two  systems  are  secured.   When sending  a
message  over such  a connection,  the  whole message  is encrypted  prior
transmitting it.  On receiving a message, it is automatically decrypted.

Figure~\ref{fig:tls-communication}  shows   the  communication  between  a
client and a  server with an intermediate host. The  client and the server
do  not have  a  direct connection  to  each other  which  means that  all
messages  have to  be transmitted  to  the intermediate  host first.   The
individual  connections   to  the  intermediate  host   are  secured  with
\gls{glo:TLS}.

\begin{figure}[ht]
  \centering
  \includegraphics[scale=0.75]{tls-communication}
  \caption[Secure   communication  with  TLS]{Secure   communication  with
    Transport Layer Security (derived from \cite{mls}).}
  \label{fig:tls-communication}
\end{figure}

The  actual  communication between  the  client  and  the server  is  only
partially  secured. All  transmitted  messages can  be  accessed in  their
unencrypted version on the intermediate host.

In message-queue based  systems there is always at  least one intermediate
server --- the message-queue server.  This means that \gls{glo:TLS} cannot
be  used   to  provide  end-to-end  communication   security  between  the
distributed components.  But it can still be used to secure the individual
connections from each component to the message-queue server.

\subsection[Message Layer Security]{Message Layer Security (MLS)}
\label{sec:fundamentals:mls}

In  contrast   to  the   \emph{Transport  Layer  Security}   protocol  the
\emph{Message Layer Security} protocol  aims directly at the messages that
are sent between two systems (\cite{oasis-wss, mls}, MLS).

\gls{glo:MLS}  is  an  approach  that encapsulates  all  security  related
information within the transmitted message itself \cite{mls}. Securing the
message  instead  of the  transport  layer  has  several advantages.   The
following  list  is  based  on  the  information  that  can  be  found  in
\cite{mls}:

\begin{itemize}
\item \textbf{Increased  flexibility}. It  is possible to  secure selected
  parts of  a message  only \cite{mls}.  An  \gls{glo:MQS} has  to inspect
  received   messages   in   order   to   forward  it   to   the   correct
  destination. This part of the  message can be left unencrypted while the
  remaining part of the message is encrypted.
\item  \textbf{Extensibility}.  Intermediate systems  or services  can add
  their  own (signed) headers  to the  message without  breaking unrelated
  (\eg  encrypted) parts  of the  message. An  example that  is  listed in
  \cite{mls} is \emph{audit logging}.
\item \textbf{Support  for multiple protocols}. \gls{glo:MLS}  can be used
  to send messages securely over a  variety of protocols such as SMTP, FTP
  or  TCP  without relying  on  the  security  of the  transport  protocol
  \cite{mls}.
\end{itemize}

The  major strength of  \gls{glo:MLS} is  also its  greatest disadvantage.
Since the security information is integrated into the messages, the layout
of the  messages must  be known  to the security  layer. That  means, each
different message layout requires  an own implementation of \gls{glo:MLS}.
Another disadvantage is the complexity of this protocol which imposes some
overhead to the message processing step.

In  \cite{oasis-wss}  a  specification  for securing  \emph{Simple  Object
  Access  Protocol}  (SOAP)  messages  with \gls{glo:MLS}  can  be  found.
Figure~\ref{fig:mls-communication} depicts  the same communication problem
as Figure~\ref{fig:tls-communication},  \ie the communication  between two
systems with the usage of an intermediate host.

\begin{figure}[ht]
  \centering
  \includegraphics[scale=0.75]{mls-communication}
  \caption[Secure   communication  with  MLS]{Secure   communication  with
    Message Layer Security (based on the picture found in \cite{mls}).}
  \label{fig:mls-communication}
\end{figure}

The message is encrypted by the application running on the client host. In
contrast  to the  \gls{glo:TLS}, this  step is  performed only  once.  The
encrypted message is  sent to the intermediate host  and then forwarded to
its  final  destination.   Since  strong  cryptography  is  involved,  the
intermediate  host cannot  access (\ie  read) the  encrypted parts  of the
message.

Consequently  can \gls{glo:MLS}  be used  to provide  a  secure end-to-end
communication for distributed applications that use message-passing.


\subsection{Implications for the \gls{glo:XenBEE}}

The  previous sections  have shown  that only  the  \gls{glo:MLS} provides
secure  end-to-end  communication for  systems  that  use an  intermediate
message-queue server.   Authentication, as well as  strong cryptography is
in both protocols provided by a \emph{Public-Key Infrastructure}.

To  ensure   a  secure  communication   between  \emph{\gls{glo:xbe}}  and
\emph{\gls{glo:xbed}}  \emph{Message Layer  Security} has  to be  used. To
actually  make sure that  a client  ``talks'' to  correct server  and vice
versa, authentication  must be provided  in both directions.  This implies
that   the   \gls{glo:XenBEE}   uses   public-key   certificates   and   a
\gls{glo:PKI}.

% \section{Summary}

% This  section  summarize  the  technologies  that  will  be  used  in  the
% \gls{glo:XenBEE}.

% The \gls{glo:XenBEE} will use a message

%%% Local Variables: 
%%% mode: latex
%%% TeX-master: "main.tex"
%%% End: 


\chapter{Design}
\label{cha:design}

\begin{figure}
  \begin{center}
    \includegraphics[scale=.75]{msc-establish-mls}
  \end{center}
  \caption[MSC Message Layer Security]{TODO: fill me in}
  \label{fig:msc-establish-mls}
\end{figure}

\begin{figure}
  \begin{center}
    \includegraphics[scale=.75]{msc-reserve}
  \end{center}
  \caption[MSC Make Reservation]{TODO: fill me in}
  \label{fig:msc-reserve}
\end{figure}

\begin{figure}
  \begin{center}
    \includegraphics[scale=.75]{msc-confirm}
  \end{center}
  \caption[MSC Confirm Reservation]{TODO: fill me in}
  \label{fig:msc-confirm}
\end{figure}

\begin{figure}
  \begin{center}
    \includegraphics[scale=.75]{msc-list-cache}
  \end{center}
  \caption[MSC List Cache Entries]{TODO: fill me in}
  \label{fig:msc-list-cache}
\end{figure}

\begin{figure}
  \begin{center}
    \includegraphics[scale=.75]{msc-status-request}
  \end{center}
  \caption[MSC Request Task Status]{TODO: fill me in}
  \label{fig:msc-status-request}
\end{figure}

\begin{figure}
  \begin{center}
    \includegraphics[scale=.75]{msc-terminate}
  \end{center}
  \caption[MSC Terminate Task Request]{TODO: fill me in}
  \label{fig:msc-terminate}
\end{figure}

\begin{figure}
  \begin{center}
    \includegraphics[scale=.75]{act-start-task}
  \end{center}
  \caption[Start Task Activity]{TODO: fill me in}
  \label{fig:act-start-task}
\end{figure}

\begin{figure}
  \begin{center}
    \includegraphics[scale=.75]{act-start-instance}
  \end{center}
  \caption[Start Instance Activity]{TODO: fill me in}
  \label{fig:act-start-instance}
\end{figure}

\begin{figure}
  \begin{center}
    \includegraphics[scale=.75]{act-stage-in}
  \end{center}
  \caption[Stage-In Activity]{TODO: fill me in}
  \label{fig:act-stage-in}
\end{figure}

\begin{figure}
  \begin{center}
    \includegraphics[scale=.75]{act-retrieve-file}
  \end{center}
  \caption[File Retrieval Activity]{TODO: fill me in}
  \label{fig:act-retrieve-file}
\end{figure}


%%% Local Variables: 
%%% mode: latex
%%% TeX-master: "main"
%%% End: 

\setchapterpreamble[o]{%
  \dictum[Benjamin Disraeli]{\textit{``There are three kinds of lies: lies, damn
    lies, and statistics''}}}

\chapter{Results}
\label{cha:results}

\begin{figure}[h]
  \begin{center}
    \includegraphics[height=5cm]{povray/optics}
  \end{center}
  \caption[An advanced POV-Ray picture]{TODO: fill me in}
  \label{fig:pov-optics}
\end{figure}

%%% Local Variables: 
%%% mode: latex
%%% TeX-master: "main"
%%% End: 


\chapter{Conclusions and Future Work}
\label{cha:conclusions}

This  chapter analyzes  the work  that  has been  done in  respect of  the
initial goals  that have been claimed.  It also provides  hints for future
developments.

\section*{Conclusions}

The target of  this work was to design and  implement a complete execution
environment based  on virtual machines. The  first step in this was  to define the
execution semantics that have to be supported 

\begin{itemize}
\item job semantics: batch, server deployment
\item access to required data: image, kernel, initrd, input files
\item caching and compression of files
\item uniform way to access files (URI)
\item standardized job description
\item standardized job model
\item reservations
\item hooks to modify input and output files (encryption)
\item secure end-to-end communication xml
\item task security with chroot environment
\item user has full control over the vm
\item results have shown that the implementation is able to execute batch
  jobs and server deployment
\item  also: starting/stopping  of  virtual machines  is  fast enough  for
  on-demand server deployment
\item 
\end{itemize}

\section*{Future work}

The current  implementation of the \gls{glo:XenBEE} is  already usable for
real world  problems as  it has  been shown in  the previous  chapter. But
there are still many aspects that could be implemented and analyzed in the
future. The following sections provide a few ideas for future works.

\subsubsection{Integration into Calana}

The most crucial future development step is the actual integration into an
existing  grid  environment.   The  execution  environment  understands  a
commonly used language for the  job submission and a formalized job-model.
The  basic  requirements  for   the  integration  into  Calana  have  been
implemented,  as well.   But  the glue  between  the \gls{glo:XenBEE}  and
Calana (or some  other grid middleware) --- the  Calana-agent --- is still
missing.

\subsubsection{Unattended Updates and Support for Work flows}

Completely out  of the scope  of this work  was the administration  of the
used virtual  machines images.   It could for  instance be possible  to do
automated  updates of stored  images.  These  updates should  be performed
regularly and without interaction.  The  update process would also need to
verify the updated applications, \ie a test suite must be executed on that
image.   The execution environment  could therefore  be enhanced  with the
support of a complete work flow description language.

\subsubsection{Cache hierarchies}

A cluster of machines that are used for the \gls{glo:XenBEE} could use one
or more shared  caches. If an user wants to execute  a particular job many
times or  on several machines  at the same  time, he could load  the image
into  the shared  cache first.   Each involved  execution host  could then
retrieve the image into its local cache.

\subsubsection{Advanced file system support}

The current implementation  makes the assumption that only  a single image
file is  involved.  But  it could  also be possible  to compose  a virtual
machine out of  several images. A basic image  that contains the operating
system and application installation and a second image that can be used to
store input and output data.  Actually, the additional image could also be
a network file system.

\subsubsection{tools}
\label{sec:tools}



%%% Local Variables: 
%%% mode: latex
%%% TeX-master: "main"
%%% End: 


%  A P P E N D I C E S
\appendix
%\def\chaptername{Appendix}
\chapter{Additional Background Information}
\label{app:cha:background}

This chapter  provides some additional background information  that may be
useful to fully understand particular design and implementation aspects.

\section{Public-key cryptography}
\label{app:sec:public-key-cryptography}

Public-key  cryptography describes  a form  of cryptography  where  a user
holds  two  different  keys,  a  \emph{private  key}  and  a  \emph{public
  key}.  These two  keys are  mathematically  related to  each other,  but
nobody can practically derive the private key from the public key.

The public key can be made  publicly available without any risk, while the
private key must  be kept very secret. A widely used  algorithm is the RSA
algorithm named  after its  creators \citet*{rivest77method}. It  has been
the first algorithm, that  was suitable for both encryption/decryption and
signing.  For more background information on public-key cryptosystems, you
are encouraged to read \cite{rivest77method,diffie76new}.

The RSA algorithm relies on the fact, that the factorization of reasonably
large numbers is  computationally very hard and no  efficient algorithm is
publicly known.  Especially  hard to factor are numbers  whose factors are
two randomly-chosen prime numbers of sufficient length.

In the following  I am going to describe  how the keys are set  up and how
they  are  used  to  encrypt/decrypt  or to  sign/validate  a  clear  text
message.  The provided  material  is  based on  the  information found  in
\cite{rivest77method} and \cite{diffie76new}.

According  to  \cite{rivest77method}   places  each  user  his  encryption
procedure $E$  in a publicly  accessible file (\eg database).   Using this
public file, any  other user is able to  retrieve the encryption procedure
of some other user (\ie the one he wants to send encrypted messages). Each
user keeps his decryption procedure $D$ secret.

The mentioned procedures $D$ and $E$ have the following properties:
\begin{enumerate}
\renewcommand{\theenumi}{\alph{enumi}}
\renewcommand{\labelenumi}{(\theenumi)}

\item Deciphering  the enciphered form of  a message $M$  yields $M$. That
  is,
  \begin{equation}
    \label{eq:1}
    D(E(M)) = M.
  \end{equation}
  \label{pubkey-cryptosystem-prop-1}
\item Both $D$ and $E$ are easy to compute.
  \label{pubkey-cryptosystem-prop-2}
\item The user does not reveal an  easy way to compute $D$ if he makes $E$
  publicly available.
  \label{pubkey-cryptosystem-prop-3}
\item  The enciphering  of a  previously ciphered  message $M$  results in
  $M$. That is,
  \begin{equation}
    \label{eq:2}
    E(D(M)) = M.
  \end{equation}
  \label{pubkey-cryptosystem-prop-4}
\end{enumerate}

A                  function                 $E$                 satisfying
(\ref{pubkey-cryptosystem-prop-1})--(\ref{pubkey-cryptosystem-prop-3})   is
said  to  be  a  \emph{``trap-door  one-way function''}  and  if  it  also
satisfies  (\ref{pubkey-cryptosystem-prop-4})  it  is a  \emph{``trap-door
  one-way permutation''} \cite{rivest77method,diffie76new}.

\subsection{Key setup}

The \emph{encryption  key} consists  of a pair  of positive  integers $(e,
n)$,  where $e$  is the  \emph{encryption exponent}  and $n$  is  used for
modulo  operations.  The \emph{decryption  key}  is  also  a pair  of  two
integers, where only the exponent differs, thus $(d, n)$ is the decryption
key and $d$  represents the \emph{decryption exponent}. $(e,  n)$ are made
publicly available.

\citet*{rivest77method} suggest the  following approach for the generation
of $(e, n)$ and  $(d, n)$. The fist step is to  compute $n$ as the product
of two very large, ``random'' primes $p$ and $q$:
\begin{equation*}
  \label{eq:compute-n}
n = p \cdot q.
\end{equation*}

Although you  publish $n$, nobody is  able to compute the  factors $p$ and
$q$ in  reasonably time due to  the enormous difficulty  of factoring $n$.
In \cite{rivest77method}  it is assumed,  that the computation of  $p$ and
$q$ from a given $n$ takes $1.5 \times 10^{29}$ operations, given that $n$
has a length  of $300$ digits. If one operation  took one microsecond, the
whole computation takes $4.9 \times 10^{15}$ years.

The  next step  is to  choose  $d$, therefore  one picks  a large,  random
integer that is \emph{co-prime}\footnote{two integers $a$ and $b$ are said
  to be co-prime  if they do not  have a common factor other  than $1$ and
  $-1$, \ie their greatest common divisor  (gcd) is $1$.} to $(p-1) \cdot
(q-1)$.\footnote{the term $(p-1)\cdot(q-1)$ is the result of \emph{Euler's
    Phi} or  \emph{Euler's totient function}  ($\phi$) applied to  $n$} In
other words, $d$ has to satisfy:
\begin{equation*}
  \label{eq:compute-d}
  gcd(d, (p-1) \cdot (q-1)) = 1.
\end{equation*}

Finally,  the integer $e$  is computed  from $p$,  $q$ and  $d$ to  be the
\emph{multiplicative inverse} of $d$, modulo $\phi(n)$:
\begin{equation*}
  \label{eq:compute-e}
  e \cdot d \equiv 1\ \ (mod\ (p-1) \cdot (q-1))
\end{equation*}

\subsection{Encryption and Decryption}

If  two  persons,  Alice  and   Bob,  want  to  send  each  other  private
(\ie encrypted)  messages,  they   both  retrieve  the  other's  publicly
available encryption  key first ---  Bob retrieves $(e_a, n_a)$  and Alice
retrieves $(e_b, n_b)$.

Let's say  Alice wants to send a  private message to Bob.   To encrypt the
message, she has to  represent it as an integer between $0$  and $n_b - 1$
(long  messages can  be broken  into smaller  blocks, so  that  each block
fulfills the requirement). Alice then  encrypts the message $M$ by raising
it to the $e_b$th power modulo $n_b$, the result is the cyphertext $C$:
\begin{equation*}
  \label{eq:encrypt-message}
  C \equiv E(M) \equiv M^{e_b}\ \ (mod\ \ n_b).
\end{equation*}

On reception  of the cyphertext  $C$, Bob raises  it to the  $d_b$th power
modulo $n_b$. He  is the only person, who knows $d_b$  and therefore he is
solely able to decrypt $C$:
\begin{equation*}
  \label{eq:decrypt-message}
  M \equiv D(C) \equiv C^{d_b}\ \ (mod\ \ n_b).
\end{equation*}

\subsection{Signing and Validating}

Electronic  signatures, \eg in electronic  mail systems,  especially when
used in  business transactions, must  provide provability to  the receiver
that  the message  originated from  the sender.   This is  more  than just
provide  mere \emph{authentication},  where the  recipient of  a digitally
signed message can  verify that the message came  from the sender. Digital
signatures must be  able to be used to convince  a ``judge'', that neither
the recipient did  forge the message, nor the sender  can deny sending the
message.

That means,  an electronic signature must  be \emph{message}-dependent, as
well as  \emph{signer}-dependent. If the  signature did not depend  on the
message itself, a dishonorable recipient  could just change the message or
attach the signature to a  completely different message before showing the
message/signature pair  to a judge. If  the signature would  not depend on
the \emph{signer}, obviously anybody could have signed the message.

\medskip

If   Bob  want  to   send  Alice   a  signed   message,  he   applies  his
\emph{decryption}  function $D_b$  to the  clear text  message  $M$, which
results in the signature $S$:

\begin{equation*}
  \label{eq:compute-signature}
  S = D_b(M).
\end{equation*}

To   perform  this,   the  cryptosystem   has  to   be   implemented  with
\emph{trap-door         one-way        permutations},        \ie property
(\ref{pubkey-cryptosystem-prop-4}) must hold.

This signature  can now  be encrypted using  Alice's public key  to ensure
privacy, there  is no need to  send the message along  with the signature,
since it can  be computed from it. On reception,  Alice first decrypts the
message which  results in the  plain signature $S$ again.   Applying Bob's
\emph{encryption} function to the  received signature (Alice knows who the
presumed sender  of the  message is) makes  perfect sense due  to property
(\ref{pubkey-cryptosystem-prop-4}):

\begin{equation*}
  \label{eq:validate-signature}
  M = E_b(S)
\end{equation*}

Bob cannot  later on deny that he  sent the message, since  nobody but him
could  have generated  the signature  $S$.  Alice  is able  to  convince a
``judge'', that  Bob did send the  message, since $E_b(S) =  M$. But Alice
cannot modify  $M$ or  provide a different  message $M'$ because  then she
would also need to compute $S' = D_b(M')$ as well.

\section{Calana}
\label{app:sec:calana}

\emph{Calana} is  a new Grid  scheduler approach proposed  by M.~Dalheimer
\cite{dalheimer05agentbased}.   The scheduler  uses  several \emph{agents}
and at  least one  \emph{broker}.

The  agents are  responsible for  single resource.   That means  they know
whether the resource is free or  not. They are also capable to acquire new
or cancel previously made reservations on this resource.

If  a user  wants to  submit a  job to  the grid  environment,  the broker
initiates an  \emph{auction} among the  connected agents. The job  is then
assigned to the resource that belongs to the agent that won the auction.

\subsection{Architecture}

An  abstract   view  over   the  architecture  of   Calana  is   shown  in
Figure~\ref{fig:calana-architecture}.   For a  detailed discussion  of the
protocol  that  is   used  to  perform  the  auctions,   have  a  look  at
\cite{dalheimer06calanaprotocol} and \cite{petry06}.

\begin{figure}[ht]
  \centering
  \includegraphics[scale=0.5]{calana-architecture}
  \caption{Architecture of Calana}
  \label{fig:calana-architecture}
\end{figure}

The main steps of such an auction are can be described as follows:

\begin{enumerate}
\item When a user submits a job to the Calana-broker, the broker will open
  up an auction and try to \emph{book} a resource for the task.
\item    For   each   task    an   auction    is   created    by   sending
  \texttt{BookingReq}-messages to the connected agents.
\item The agents will make  one or more \emph{reservations} on their local
  scheduler  and  answer with  a  \texttt{AuctionBid}.   Bids contain  for
  example  the  cost  of   using  the  resource  and  various  reservation
  parameters  such  as  the   earliest  start-time  and  duration  of  the
  reservation.
\item To make a decision, the  broker judges all received bids and chooses
  the     best     one      according     to     some     preference-model
  \cite{dalheimer05agentbased, petry06}.
\item If  the user  accepts the decision,  the broker  \emph{confirms} the
  reservation.
\end{enumerate}

\subsection{Job-state model}
\label{app:sec:calana-job-model}

To reflect the possibility  to \emph{make}, \emph{confirm}, \emph{use} and
\emph{cancel} reservations on some resource, the job-state model had to be
extended. We discussed about a common state-model for the jobs and came to
the  consensus of  adopting  the  \gls{glo:BES} model  to  our needs  (see
Figure~\ref{fig:bes-calana-job-model}).

\begin{figure}[h]
  \centering
  \includegraphics[scale=.55]{bes-calana-job-model}
  \caption[Calana Job Model]{The common job model proposed by the Calana
    Grid scheduler.}
  \label{fig:bes-calana-job-model}
\end{figure}

%%% Local Variables: 
%%% mode: latex
%%% TeX-master: "main.tex"
%%% End: 

%\include{appendix2}

% G L O S S A R Y

% \printindex
%\renewcommand{\glossarypreamble}{You need not to know everything, you just
%  need to know where to find it.\par}
\printglossary


%  R E F E R E N C E S

\renewcommand{\bibname}{References}
%\addcontentsline{toc}{chapter}{\quad\,\,{List of References}}

\bibliographystyle{plainnat}
\bibliography{bibliography}

% Supplementary references?
% Theses do not have indices.

% include the affidavit
%% Eidesstattliche Erklärung
\newpage

\thispagestyle{empty}
\begin{center}  
{\Large Erklärung an Eides Statt}
\end{center}

\vspace{1cm}

\begin{center}
\fbox{
  \begin{minipage}{0.9\textwidth}
    Ich versichere hiermit, dass ich die vorliegende Diplomarbeit mit dem
    Thema ,,Design and Implementation of a Xen-Based Execution Environment''
    selbständig verfasst und keine anderen als die angegebenen Hilfsmittel
    benutzt habe.  Die Stellen, die anderen Werken dem Wortlaut oder dem
    Sinn nach entnommen wurden, habe ich durch die Angabe der Quelle, auch
    der benutzten Sekundärliteratur, als Entlehnung kenntlich gemacht.

    \vspace{1.5cm}

    \begin{tabular*}{.9\textwidth}%
      {ll}%
      \rule{0.45\textwidth}{.5pt} & \rule{0.45\textwidth}{.5pt} \\
      (Ort, Datum) & (Name, Unterschrift)
    \end{tabular*}

\end{minipage}
}
\end{center}

\newpage


\end{document}
