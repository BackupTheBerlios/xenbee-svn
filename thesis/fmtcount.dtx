%\iffalse
% fmtcount.dtx generated using makedtx version 0.91b (c) Nicola Talbot
% Command line args:
%   -src "(.+)\.(def)=>\1.\2"
%   -src "(.+)\.(sty)=>\1.\2"
%   -doc "manual.tex"
%   -author "Nicola Talbot"
%   -dir "source"
%   fmtcount
% Created on 2007/6/22 16:48
%\fi
%\iffalse
%<*package>
%% \CharacterTable
%%  {Upper-case    \A\B\C\D\E\F\G\H\I\J\K\L\M\N\O\P\Q\R\S\T\U\V\W\X\Y\Z
%%   Lower-case    \a\b\c\d\e\f\g\h\i\j\k\l\m\n\o\p\q\r\s\t\u\v\w\x\y\z
%%   Digits        \0\1\2\3\4\5\6\7\8\9
%%   Exclamation   \!     Double quote  \"     Hash (number) \#
%%   Dollar        \$     Percent       \%     Ampersand     \&
%%   Acute accent  \'     Left paren    \(     Right paren   \)
%%   Asterisk      \*     Plus          \+     Comma         \,
%%   Minus         \-     Point         \.     Solidus       \/
%%   Colon         \:     Semicolon     \;     Less than     \<
%%   Equals        \=     Greater than  \>     Question mark \?
%%   Commercial at \@     Left bracket  \[     Backslash     \\
%%   Right bracket \]     Circumflex    \^     Underscore    \_
%%   Grave accent  \`     Left brace    \{     Vertical bar  \|
%%   Right brace   \}     Tilde         \~}
%</package>
%\fi
% \iffalse
% Doc-Source file to use with LaTeX2e
% Copyright (C) 2007 Nicola Talbot, all rights reserved.
% \fi
% \iffalse
%<*driver>
\documentclass{ltxdoc}

\usepackage[colorlinks,
            bookmarks,
            bookmarksopen,
            pdfauthor={N.L.C. Talbot},
            pdftitle={fmtcount.sty: Displaying the Values of LaTeX Counters},
            pdfkeywords={LaTeX,counter}]{hyperref}



\newcommand{\styname}[1]{\textsf{#1}}\newcommand{\clsname}[1]{\textsf{#1}}\newcommand{\cmdname}[1]{\texttt{\symbol{92}#1}}

\begin{document}
\DocInput{fmtcount.dtx}
\end{document}
%</driver>
%\fi
%
%\RecordChanges
%\CheckSum{7751}
%\OnlyDescription
%\def\filedate{22 June 2007}
%\def\fileversion{1.2}
%\def\filename{fmtcount.dtx}
%\def\docdate{22nd June 2007}
%
% \title{fmtcount.sty v\fileversion: Displaying the Values of 
%\LaTeX\ Counters}
% \author{Nicola L.C. Talbot}
% \date{\docdate}
% \maketitle
% \tableofcontents
% \section{Introduction}
%The \styname{fmtcount} package provides commands to display
%the values of \LaTeX\ counters in a variety of formats. It also
%provides equivalent commands for actual numbers rather than 
%counter names. Limited multilingual support is available.
%
%\section{Installation}
%
%This package is distributed with the files \texttt{fmtcount.dtx}
%and \texttt{fmtcount.ins}.  To extract the code do:
%\begin{verbatim}
%latex fmtcount.ins
%\end{verbatim}
%This will create the files \texttt{fmtcount.sty} and 
%\texttt{fmtcount.perl}, along with several \texttt{.def} files.
%Place \texttt{fmtcount.sty} and the \texttt{.def} files somewhere
%where \LaTeX\ will find them (e.g.\ \texttt{texmf/tex/latex/fmtcount/}) and place \texttt{fmtcount.perl} somewhere where \LaTeX2HTML
%will find it (e.g.\ latex2html/styles). Remember to refresh
%the \TeX\ database (using \texttt{texhash} under Linux, for
%other operating systems check the manual.)
%
%\section{Available Commands}
%
%The commands can be divided into two categories: those that
%take the name of a counter as the argument, and those
%that take a number as the argument.
%
% \DescribeMacro{\ordinal}
% The macro \verb"\ordinal{"\meta{counter}\verb!}! will
% print the value of a \LaTeX\ counter \meta{counter} as an ordinal,
% \DescribeMacro{\fmtord}
% where the macro \verb"\fmtord{"\meta{text}\verb!}! is used to format the st,nd,rd,th bit.
% By default the ordinal is formatted as a superscript, if the package option \texttt{level}
% is used, it is level with the text.
% For example, if the current section is 3, then \verb"\ordinal{section}" will produce
% the output: 3\textsuperscript{rd}.
%
%\textbf{Note:} the \clsname{memoir} class also defines a command called
%\verb"\ordinal" which takes a number as an argument instead of a
%counter. In order to overcome this incompatiblity, if you want
%to use the \styname{fmtcount} package with the \clsname{memoir} class 
%you should use \verb"\FCordinal" to access \styname{fmtcount}'s 
%version of \verb"\ordinal", and use \verb"\ordinal" to use
%\clsname{memoir}'s version of that command.
%
%\DescribeMacro{\ordinalnum}
%The macro \verb"\ordinalnum" is like \verb!\ordinal!
%but takes an actual number rather than a counter as the
%argument. For example: \verb"\ordinalnum{3}" will
%produce: 3\textsuperscript{rd}.
%
% \DescribeMacro{\numberstring}
% The macro \verb"\numberstring{"\meta{counter}\verb!}! will print the value
% of \meta{counter} as text.  E.g.\ \verb"\numberstring{section}" will produce:
% three.
% \DescribeMacro{\Numberstring}
% The macro \verb"\Numberstring{"\meta{counter}\verb!}! does the same as
% \verb"\numberstring", but with initial letters in uppercase.  For
% example, \verb"\Numberstring{section}" will produce: Three.
%\DescribeMacro{\NUMBERstring}
%The macro \verb"\NUMBERstring{"\meta{counter}\verb'}' does the same
%as \verb"\numberstring", but converted to upper case. Note that
%\verb"\MakeUppercase{\NUMBERstring{"\meta{counter}\verb'}}' doesn't
%work, due to the way that \verb"\MakeUppercase" expands its 
%argument\footnote{See all the various postings to 
%\texttt{comp.text.tex} about \cmdname{MakeUppercase}}.
%
%\DescribeMacro{\numberstringnum}
%\DescribeMacro{\Numberstringnum}
%\DescribeMacro{\NUMBERstringnum}
%The macros \verb!\numberstringnum!, 
%\verb!\Numberstringnum! and
%\verb"\NUMBERstringnum" work like 
%\verb!\numberstring!, 
%\verb!\Numberstring! and
%\verb"\NUMBERstring", respectively, but take an actual number
%rather than a counter as the argument. For example:
%\verb'\Numberstringnum{105}' will produce: One Hundred and Five.
%
%
% \DescribeMacro{\ordinalstring}
% The macro \verb"\ordinalstring{"\meta{counter}\verb!}! will print the
% value of \meta{counter} as a textual ordinal.  E.g.\
% \verb"\ordinalstring{section}" will produce: third.
% \DescribeMacro{\Ordinalstring}
% The macro \verb"\Ordinalstring{"\meta{counter}\verb!}! does the same as 
% \verb"\ordinalstring", but with initial letters in uppercase.  For example,
% \verb"\Ordinalstring{section}" will produce: Third.
% \DescribeMacro{\ORDINALstring}
% The macro \verb"\ORDINALstring{"\meta{counter}\verb!}! does the same as 
%\verb"\ordinalstring", but with all words in upper case (see previous
%note about \cmdname{MakeUppercase}).
%
%\DescribeMacro{\ordinalstringnum}
%\DescribeMacro{\Ordinalstringnum}
%\DescribeMacro{\ORDINALstringnum}
%The macros \verb!\ordinalstringnum!, 
%\verb!\Ordinalstringnum! and \cmdname{ORDINALstringnum} work like 
%\verb!\ordinalstring!, 
%\verb!\Ordinalstring! and \cmdname{ORDINALstring}, respectively, but take an actual number
%rather than a counter as the argument. For example,
%\verb"\ordinalstringnum{3}" will produce: third.
%
%\changes{v.1.09}{21 Apr 2007}{store facility added}
%As from version 1.09, textual representations can be stored for
%later use. This overcomes the problems encountered when you
%attempt to use one of the above commands in \cmdname{edef}.
%
%Each of the following commands takes a label as the first argument,
%the other arguments are as the analogous commands above. These
%commands do not display anything, but store the textual 
%representation. This can later be retrieved using
%\DescribeMacro{\FMCuse}\cmdname{FMCuse}\{\meta{label}\}.
%Note: with \cmdname{storeordinal} and \cmdname{storeordinalnum}, the 
%only bit that doesn't get expanded is \cmdname{fmtord}. So, for 
%example, \verb"\storeordinalnum{mylabel}{3}" will be stored as
%\verb"3\relax \fmtord{rd}".
%
%\DescribeMacro{\storeordinal}
%\cmdname{storeordinal}\{\meta{label}\}\{\meta{counter}\}[\meta{gender}]
%\par
%\DescribeMacro{\storeordinalstring}
%\cmdname{storeordinalstring}\{\meta{label}\}\{\meta{counter}\}[\meta{gender}]
%\par
%\DescribeMacro{\storeOrdinalstring}
%\cmdname{storeOrdinalstring}\{\meta{label}\}\{\meta{counter}\}[\meta{gender}]
%\par
%\DescribeMacro{\storeORDINALstring}
%\cmdname{storeORDINALstring}\{\meta{label}\}\{\meta{counter}\}[\meta{gender}]
%\par
%\DescribeMacro{\storenumberstring}
%\cmdname{storenumberstring}\{\meta{label}\}\{\meta{counter}\}[\meta{gender}]
%\par
%\DescribeMacro{\storeNumberstring}
%\cmdname{storeNumberstring}\{\meta{label}\}\{\meta{counter}\}[\meta{gender}]
%\par
%\DescribeMacro{\storeNUMBERstring}
%\cmdname{storeNUMBERstring}\{\meta{label}\}\{\meta{counter}\}[\meta{gender}]
%\par
%\DescribeMacro{\storeordinalnum}
%\cmdname{storeordinalnum}\{\meta{label}\}\{\meta{number}\}[\meta{gender}]
%\par
%\DescribeMacro{\storeordinalstringnum}
%\cmdname{storeordinalstring}\{\meta{label}\}\{\meta{number}\}[\meta{gender}]
%\par
%\DescribeMacro{\storeOrdinalstringnum}
%\cmdname{storeOrdinalstringnum}\{\meta{label}\}\{\meta{number}\}[\meta{gender}]
%\par
%\DescribeMacro{\storeORDINALstringnum}
%\cmdname{storeORDINALstringnum}\{\meta{label}\}\{\meta{number}\}[\meta{gender}]
%\par
%\DescribeMacro{\storenumberstringnum}
%\cmdname{storenumberstring}\{\meta{label}\}\{\meta{number}\}[\meta{gender}]
%\par
%\DescribeMacro{\storeNumberstringnum}
%\cmdname{storeNumberstring}\{\meta{label}\}\{\meta{number}\}[\meta{gender}]
%\par
%\DescribeMacro{\storeNUMBERstringnum}
%\cmdname{storeNUMBERstring}\{\meta{label}\}\{\meta{number}\}[\meta{gender}]
%
% \DescribeMacro{\binary}
%\changes{v2.4}{25 Sept 2004}{'binary added}
% The macro \verb"\binary{"\meta{counter}\verb!}! will print the 
%value of \meta{counter} as a binary number.  
%E.g.\ \verb"\binary{section}" will produce: 11.  The declaration 
%\DescribeMacro{\padzeroes}\verb"\padzeroes["\meta{n}\verb!]! 
%will ensure numbers are written to \meta{n} digits, padding with 
%zeroes if necessary.  E.g.\ \verb"\padzeroes[8]\binary{section}" 
%will produce: 00000011.
% The default value for \meta{n} is 17.
%
%\DescribeMacro{\binarynum}
%The macro \verb"\binarynum" is like \verb!\binary!
%but takes an actual number rather than a counter as the
%argument. For example: \verb"\binarynum{5}" will
%produce: 101.
%
% \DescribeMacro{\octal}
%\changes{v2.4}{25 Sept 2004}{'octal added}
% The macro \verb"\octal{"\meta{counter}\verb!}! will print the 
%value of \meta{counter} as an octal number.  For example, if you 
%have a counter called, say \texttt{mycounter}, and you set the 
%value to 125, then \verb"\octal{mycounter}" will produce: 177.  
% Again, the number will be padded with zeroes if necessary, 
%depending on whether \verb"\padzeroes" has been used.
%
%\DescribeMacro{\octalnum}
%The macro \verb"\octalnum" is like \verb!\octal!
%but takes an actual number rather than a counter as the
%argument. For example: \verb"\octalnum{125}" will
%produce: 177.
%
% \DescribeMacro{\hexadecimal}
%\changes{v2.4}{25 Sept 2004}{'hexadecimal added}
% The macro \verb"\hexadecimal{"\meta{counter}\verb!}! will print 
%the value of \meta{counter} as a hexadecimal number.  Going back 
%to the previous example, \verb"\hexadecimal{mycounter}" will
% produce: 7d. Again, the number will be padded with zeroes if 
%necessary, depending on whether \verb"\padzeroes" has been used.
% \DescribeMacro{\Hexadecimal}
% \verb"\Hexadecimal{"\meta{counter}\verb!}! does the same thing, 
%but uses uppercase characters, e.g.\ 
% \verb"\Hexadecimal{mycounter}" will produce: 7D.
%
%\DescribeMacro{\hexadecimalnum}
%\DescribeMacro{\Hexadecimalnum}
%The macros \verb"\hexadecimalnum" and
%\verb"\Hexadecimalnum" are like 
%\verb!\hexadecimal! and \verb"\Hexadecimal"
%but take an actual number rather than a counter as the
%argument. For example: \verb"\hexadecimalnum{125}" will
%produce: 7d, and \verb"\Hexadecimalnum{125}" will 
%produce: 7D.
%
%\DescribeMacro{\decimal}
%\changes{v2.41}{22 Oct 2004}{'decimal added}
%The macro \verb"\decimal{"\meta{counter}\verb!}! is similar to 
%\verb"\arabic" but the number can be padded with zeroes
% depending on whether \verb"\padzeroes" has been used.  
%For example: \verb"\padzeroes[8]\decimal{section}" will
% produce: 00000005.
%
%\DescribeMacro{\decimalnum}
%The macro \verb"\decimalnum" is like \verb!\decimal!
%but takes an actual number rather than a counter as the
%argument. For example: \verb"\padzeroes[8]\decimalnum{5}" will
%produce: 00000005.
%
% \DescribeMacro{\aaalph}
%\changes{v2.4}{25 Sept 2004}{'aaalph added}
% The macro \verb"\aaalph{"\meta{counter}\verb!}! will print the 
%value of \meta{counter} as: a b \ldots\ z aa bb \ldots\ zz etc.
% For example, \verb"\aaalpha{mycounter}" will produce: uuuuu if 
%\texttt{mycounter} is set to 125.
% \DescribeMacro{\AAAlph}
% \verb"\AAAlph{"\meta{counter}\verb!}! does the same thing, but 
%uses uppercase characters, e.g.\ 
% \verb"\AAAlph{mycounter}" will produce: UUUUU.
%
%\DescribeMacro{\aaalphnum}
%\DescribeMacro{\AAAlphnum}
%The macros \verb"\aaalphnum" and
%\verb"\AAAlphnum" are like 
%\verb!\aaalph! and \verb"\AAAlph"
%but take an actual number rather than a counter as the
%argument. For example: \verb"\aaalphnum{125}" will
%produce: uuuuu, and \verb"\AAAlphnum{125}" will 
%produce: UUUUU.
%
% \DescribeMacro{\abalph}
%\changes{v2.4}{25 Sept 2004}{'abalph added}
% The macro \verb"\abalph{"\meta{counter}\verb!}! will print the 
%value of \meta{counter} as: a b \ldots\ z aa ab \ldots\ az etc.
% For example, \verb"\abalpha{mycounter}" will produce: du if 
%\texttt{mycounter} is set to 125.
% \DescribeMacro{\ABAlph}
% \verb"\ABAlph{"\meta{counter}\verb!}! does the same thing, but 
%uses uppercase characters, e.g.\ 
% \verb"\ABAlph{mycounter}" will produce: DU.
%
%\DescribeMacro{\abalphnum}
%\DescribeMacro{\ABAlphnum}
%The macros \verb"\abalphnum" and
%\verb"\ABAlphnum" are like 
%\verb!\abalph! and \verb"\ABAlph"
%but take an actual number rather than a counter as the
%argument. For example: \verb"\abalphnum{125}" will
%produce: du, and \verb"\ABAlphnum{125}" will 
%produce: DU.
%
%\section{Package Options}
%
%The following options can be passed to this package:
%
%\begin{tabular}{ll}
% raise    & make ordinal st,nd,rd,th appear as superscript\\
% level    & make ordinal st,nd,rd,th appear level with rest of 
%text
%\end{tabular}
%
%\noindent These can also be set using the command:
%
%\DescribeMacro{\fmtcountsetoptions}
%\verb"\fmtcountsetoptions{fmtord="\meta{type}\verb'}'
%
%\noindent where \meta{type} is either \texttt{level} or 
%\texttt{raise}.
%
%\section{Multilingual Support}
%
%Version 1.02 of the \styname{fmtcount} package now has
%limited multilingual support.  The following languages are
%implemented: English, Spanish, Portuguese, French, French (Swiss)
%and French (Belgian). German support was added in version 
%1.1\footnote{Thanks to K. H. Fricke for supplying the information}.
%
% The package checks to see if the
%command \verb"\date"\meta{language} is defined\footnote{this
%will be true if you have loaded \styname{babel}}, and will
%load the code for those languages.  The commands \verb"\ordinal",
%\verb"\ordinalstring" and \verb'\numberstring' (and their 
%variants) will then be formatted in the currently
%selected language.
%
%If the French language is selected, the French (France) version
%will be used by default (e.g.\ soxiante-dix for 70).  To
%select the Swiss or Belgian variants (e.g.\ septente for 70) use:
%\verb"\fmtcountsetoptions{french="\meta{dialect}\verb'}'
%where \meta{dialect} is either \texttt{swiss} or \texttt{belgian}.
%You can also use this command to change the action of 
%\verb"\ordinal".
%\verb"\fmtcountsetoptions{abbrv=true}" to produce ordinals
%of the form 2\textsuperscript{e} or
%\verb"\fmtcountsetoptions{abbrv=false}" to produce ordinals
%of the form 2\textsuperscript{eme} (default).
%
%The \texttt{french} and \texttt{abbrv} settings only have an
%effect if the French language has been defined.
%
%The male gender for all languages is used by default, however the
%feminine or neuter forms can be obtained by passing \texttt{f} or
%\texttt{n} as an optional argument to \verb"\ordinal",
%\verb!\ordinalnum! etc.  For example:
%\verb"\numberstring{section}[f]". Note that the optional argument
%comes \emph{after} the compulsory argument.  If a gender is
%not defined in a given language, the masculine version will
%be used instead.
%
%Let me know if you find any spelling mistakes (has been known
%to happen in English, let alone other languages I'm not so
%familiar with.) If you want to add support for another language,
%you will need to let me know how to form the numbers and ordinals 
%from 0 to 99999 in that language for each gender.
%
%\section{Configuration File \texttt{fmtcount.cfg}}
%
%You can save your preferred default settings to a file called
%\texttt{fmtcount.cfg}, and place it on the \TeX\ path.  These
%settings will then be loaded by the \styname{fmtcount}
%package.
%
%Note that if you are using the \styname{datetime} package,
%the \texttt{datetime.cfg} configuration file will override
%the \texttt{fmtcount.cfg} configuration file.
%For example, if \texttt{datetime.cfg} has the line:
%\begin{verbatim}
%\renewcommand{\fmtord}[1]{\textsuperscript{\underline{#1}}}
%\end{verbatim}
%and if \texttt{fmtcount.cfg} has the line:
%\begin{verbatim}
%\fmtcountsetoptions{fmtord=level}
%\end{verbatim}
%then the former definition of \verb"\fmtord" will take
%precedence.
%
%\section{LaTeX2HTML style}
%
%The \LaTeX2HTML\ style file \texttt{fmtcount.perl} is provided.
%The following limitations apply:
%
%\begin{itemize}
%\item \verb"\padzeroes" only has an effect in the preamble.
%
%\item The configuration file 
%\texttt{fmtcount.cfg} is currently ignored. (This is because
%I can't work out the correct code to do this.  If you
%know how to do this, please let me know.)  You can however
%do:
%\begin{verbatim}
%\usepackage{fmtcount}
%\html{%\iffalse
% fmtcount.dtx generated using makedtx version 0.91b (c) Nicola Talbot
% Command line args:
%   -src "(.+)\.(def)=>\1.\2"
%   -src "(.+)\.(sty)=>\1.\2"
%   -doc "manual.tex"
%   -author "Nicola Talbot"
%   -dir "source"
%   fmtcount
% Created on 2007/6/22 16:48
%\fi
%\iffalse
%<*package>
%% \CharacterTable
%%  {Upper-case    \A\B\C\D\E\F\G\H\I\J\K\L\M\N\O\P\Q\R\S\T\U\V\W\X\Y\Z
%%   Lower-case    \a\b\c\d\e\f\g\h\i\j\k\l\m\n\o\p\q\r\s\t\u\v\w\x\y\z
%%   Digits        \0\1\2\3\4\5\6\7\8\9
%%   Exclamation   \!     Double quote  \"     Hash (number) \#
%%   Dollar        \$     Percent       \%     Ampersand     \&
%%   Acute accent  \'     Left paren    \(     Right paren   \)
%%   Asterisk      \*     Plus          \+     Comma         \,
%%   Minus         \-     Point         \.     Solidus       \/
%%   Colon         \:     Semicolon     \;     Less than     \<
%%   Equals        \=     Greater than  \>     Question mark \?
%%   Commercial at \@     Left bracket  \[     Backslash     \\
%%   Right bracket \]     Circumflex    \^     Underscore    \_
%%   Grave accent  \`     Left brace    \{     Vertical bar  \|
%%   Right brace   \}     Tilde         \~}
%</package>
%\fi
% \iffalse
% Doc-Source file to use with LaTeX2e
% Copyright (C) 2007 Nicola Talbot, all rights reserved.
% \fi
% \iffalse
%<*driver>
\documentclass{ltxdoc}

\usepackage[colorlinks,
            bookmarks,
            bookmarksopen,
            pdfauthor={N.L.C. Talbot},
            pdftitle={fmtcount.sty: Displaying the Values of LaTeX Counters},
            pdfkeywords={LaTeX,counter}]{hyperref}



\newcommand{\styname}[1]{\textsf{#1}}\newcommand{\clsname}[1]{\textsf{#1}}\newcommand{\cmdname}[1]{\texttt{\symbol{92}#1}}

\begin{document}
\DocInput{fmtcount.dtx}
\end{document}
%</driver>
%\fi
%
%\RecordChanges
%\CheckSum{7751}
%\OnlyDescription
%\def\filedate{22 June 2007}
%\def\fileversion{1.2}
%\def\filename{fmtcount.dtx}
%\def\docdate{22nd June 2007}
%
% \title{fmtcount.sty v\fileversion: Displaying the Values of 
%\LaTeX\ Counters}
% \author{Nicola L.C. Talbot}
% \date{\docdate}
% \maketitle
% \tableofcontents
% \section{Introduction}
%The \styname{fmtcount} package provides commands to display
%the values of \LaTeX\ counters in a variety of formats. It also
%provides equivalent commands for actual numbers rather than 
%counter names. Limited multilingual support is available.
%
%\section{Installation}
%
%This package is distributed with the files \texttt{fmtcount.dtx}
%and \texttt{fmtcount.ins}.  To extract the code do:
%\begin{verbatim}
%latex fmtcount.ins
%\end{verbatim}
%This will create the files \texttt{fmtcount.sty} and 
%\texttt{fmtcount.perl}, along with several \texttt{.def} files.
%Place \texttt{fmtcount.sty} and the \texttt{.def} files somewhere
%where \LaTeX\ will find them (e.g.\ \texttt{texmf/tex/latex/fmtcount/}) and place \texttt{fmtcount.perl} somewhere where \LaTeX2HTML
%will find it (e.g.\ latex2html/styles). Remember to refresh
%the \TeX\ database (using \texttt{texhash} under Linux, for
%other operating systems check the manual.)
%
%\section{Available Commands}
%
%The commands can be divided into two categories: those that
%take the name of a counter as the argument, and those
%that take a number as the argument.
%
% \DescribeMacro{\ordinal}
% The macro \verb"\ordinal{"\meta{counter}\verb!}! will
% print the value of a \LaTeX\ counter \meta{counter} as an ordinal,
% \DescribeMacro{\fmtord}
% where the macro \verb"\fmtord{"\meta{text}\verb!}! is used to format the st,nd,rd,th bit.
% By default the ordinal is formatted as a superscript, if the package option \texttt{level}
% is used, it is level with the text.
% For example, if the current section is 3, then \verb"\ordinal{section}" will produce
% the output: 3\textsuperscript{rd}.
%
%\textbf{Note:} the \clsname{memoir} class also defines a command called
%\verb"\ordinal" which takes a number as an argument instead of a
%counter. In order to overcome this incompatiblity, if you want
%to use the \styname{fmtcount} package with the \clsname{memoir} class 
%you should use \verb"\FCordinal" to access \styname{fmtcount}'s 
%version of \verb"\ordinal", and use \verb"\ordinal" to use
%\clsname{memoir}'s version of that command.
%
%\DescribeMacro{\ordinalnum}
%The macro \verb"\ordinalnum" is like \verb!\ordinal!
%but takes an actual number rather than a counter as the
%argument. For example: \verb"\ordinalnum{3}" will
%produce: 3\textsuperscript{rd}.
%
% \DescribeMacro{\numberstring}
% The macro \verb"\numberstring{"\meta{counter}\verb!}! will print the value
% of \meta{counter} as text.  E.g.\ \verb"\numberstring{section}" will produce:
% three.
% \DescribeMacro{\Numberstring}
% The macro \verb"\Numberstring{"\meta{counter}\verb!}! does the same as
% \verb"\numberstring", but with initial letters in uppercase.  For
% example, \verb"\Numberstring{section}" will produce: Three.
%\DescribeMacro{\NUMBERstring}
%The macro \verb"\NUMBERstring{"\meta{counter}\verb'}' does the same
%as \verb"\numberstring", but converted to upper case. Note that
%\verb"\MakeUppercase{\NUMBERstring{"\meta{counter}\verb'}}' doesn't
%work, due to the way that \verb"\MakeUppercase" expands its 
%argument\footnote{See all the various postings to 
%\texttt{comp.text.tex} about \cmdname{MakeUppercase}}.
%
%\DescribeMacro{\numberstringnum}
%\DescribeMacro{\Numberstringnum}
%\DescribeMacro{\NUMBERstringnum}
%The macros \verb!\numberstringnum!, 
%\verb!\Numberstringnum! and
%\verb"\NUMBERstringnum" work like 
%\verb!\numberstring!, 
%\verb!\Numberstring! and
%\verb"\NUMBERstring", respectively, but take an actual number
%rather than a counter as the argument. For example:
%\verb'\Numberstringnum{105}' will produce: One Hundred and Five.
%
%
% \DescribeMacro{\ordinalstring}
% The macro \verb"\ordinalstring{"\meta{counter}\verb!}! will print the
% value of \meta{counter} as a textual ordinal.  E.g.\
% \verb"\ordinalstring{section}" will produce: third.
% \DescribeMacro{\Ordinalstring}
% The macro \verb"\Ordinalstring{"\meta{counter}\verb!}! does the same as 
% \verb"\ordinalstring", but with initial letters in uppercase.  For example,
% \verb"\Ordinalstring{section}" will produce: Third.
% \DescribeMacro{\ORDINALstring}
% The macro \verb"\ORDINALstring{"\meta{counter}\verb!}! does the same as 
%\verb"\ordinalstring", but with all words in upper case (see previous
%note about \cmdname{MakeUppercase}).
%
%\DescribeMacro{\ordinalstringnum}
%\DescribeMacro{\Ordinalstringnum}
%\DescribeMacro{\ORDINALstringnum}
%The macros \verb!\ordinalstringnum!, 
%\verb!\Ordinalstringnum! and \cmdname{ORDINALstringnum} work like 
%\verb!\ordinalstring!, 
%\verb!\Ordinalstring! and \cmdname{ORDINALstring}, respectively, but take an actual number
%rather than a counter as the argument. For example,
%\verb"\ordinalstringnum{3}" will produce: third.
%
%\changes{v.1.09}{21 Apr 2007}{store facility added}
%As from version 1.09, textual representations can be stored for
%later use. This overcomes the problems encountered when you
%attempt to use one of the above commands in \cmdname{edef}.
%
%Each of the following commands takes a label as the first argument,
%the other arguments are as the analogous commands above. These
%commands do not display anything, but store the textual 
%representation. This can later be retrieved using
%\DescribeMacro{\FMCuse}\cmdname{FMCuse}\{\meta{label}\}.
%Note: with \cmdname{storeordinal} and \cmdname{storeordinalnum}, the 
%only bit that doesn't get expanded is \cmdname{fmtord}. So, for 
%example, \verb"\storeordinalnum{mylabel}{3}" will be stored as
%\verb"3\relax \fmtord{rd}".
%
%\DescribeMacro{\storeordinal}
%\cmdname{storeordinal}\{\meta{label}\}\{\meta{counter}\}[\meta{gender}]
%\par
%\DescribeMacro{\storeordinalstring}
%\cmdname{storeordinalstring}\{\meta{label}\}\{\meta{counter}\}[\meta{gender}]
%\par
%\DescribeMacro{\storeOrdinalstring}
%\cmdname{storeOrdinalstring}\{\meta{label}\}\{\meta{counter}\}[\meta{gender}]
%\par
%\DescribeMacro{\storeORDINALstring}
%\cmdname{storeORDINALstring}\{\meta{label}\}\{\meta{counter}\}[\meta{gender}]
%\par
%\DescribeMacro{\storenumberstring}
%\cmdname{storenumberstring}\{\meta{label}\}\{\meta{counter}\}[\meta{gender}]
%\par
%\DescribeMacro{\storeNumberstring}
%\cmdname{storeNumberstring}\{\meta{label}\}\{\meta{counter}\}[\meta{gender}]
%\par
%\DescribeMacro{\storeNUMBERstring}
%\cmdname{storeNUMBERstring}\{\meta{label}\}\{\meta{counter}\}[\meta{gender}]
%\par
%\DescribeMacro{\storeordinalnum}
%\cmdname{storeordinalnum}\{\meta{label}\}\{\meta{number}\}[\meta{gender}]
%\par
%\DescribeMacro{\storeordinalstringnum}
%\cmdname{storeordinalstring}\{\meta{label}\}\{\meta{number}\}[\meta{gender}]
%\par
%\DescribeMacro{\storeOrdinalstringnum}
%\cmdname{storeOrdinalstringnum}\{\meta{label}\}\{\meta{number}\}[\meta{gender}]
%\par
%\DescribeMacro{\storeORDINALstringnum}
%\cmdname{storeORDINALstringnum}\{\meta{label}\}\{\meta{number}\}[\meta{gender}]
%\par
%\DescribeMacro{\storenumberstringnum}
%\cmdname{storenumberstring}\{\meta{label}\}\{\meta{number}\}[\meta{gender}]
%\par
%\DescribeMacro{\storeNumberstringnum}
%\cmdname{storeNumberstring}\{\meta{label}\}\{\meta{number}\}[\meta{gender}]
%\par
%\DescribeMacro{\storeNUMBERstringnum}
%\cmdname{storeNUMBERstring}\{\meta{label}\}\{\meta{number}\}[\meta{gender}]
%
% \DescribeMacro{\binary}
%\changes{v2.4}{25 Sept 2004}{'binary added}
% The macro \verb"\binary{"\meta{counter}\verb!}! will print the 
%value of \meta{counter} as a binary number.  
%E.g.\ \verb"\binary{section}" will produce: 11.  The declaration 
%\DescribeMacro{\padzeroes}\verb"\padzeroes["\meta{n}\verb!]! 
%will ensure numbers are written to \meta{n} digits, padding with 
%zeroes if necessary.  E.g.\ \verb"\padzeroes[8]\binary{section}" 
%will produce: 00000011.
% The default value for \meta{n} is 17.
%
%\DescribeMacro{\binarynum}
%The macro \verb"\binarynum" is like \verb!\binary!
%but takes an actual number rather than a counter as the
%argument. For example: \verb"\binarynum{5}" will
%produce: 101.
%
% \DescribeMacro{\octal}
%\changes{v2.4}{25 Sept 2004}{'octal added}
% The macro \verb"\octal{"\meta{counter}\verb!}! will print the 
%value of \meta{counter} as an octal number.  For example, if you 
%have a counter called, say \texttt{mycounter}, and you set the 
%value to 125, then \verb"\octal{mycounter}" will produce: 177.  
% Again, the number will be padded with zeroes if necessary, 
%depending on whether \verb"\padzeroes" has been used.
%
%\DescribeMacro{\octalnum}
%The macro \verb"\octalnum" is like \verb!\octal!
%but takes an actual number rather than a counter as the
%argument. For example: \verb"\octalnum{125}" will
%produce: 177.
%
% \DescribeMacro{\hexadecimal}
%\changes{v2.4}{25 Sept 2004}{'hexadecimal added}
% The macro \verb"\hexadecimal{"\meta{counter}\verb!}! will print 
%the value of \meta{counter} as a hexadecimal number.  Going back 
%to the previous example, \verb"\hexadecimal{mycounter}" will
% produce: 7d. Again, the number will be padded with zeroes if 
%necessary, depending on whether \verb"\padzeroes" has been used.
% \DescribeMacro{\Hexadecimal}
% \verb"\Hexadecimal{"\meta{counter}\verb!}! does the same thing, 
%but uses uppercase characters, e.g.\ 
% \verb"\Hexadecimal{mycounter}" will produce: 7D.
%
%\DescribeMacro{\hexadecimalnum}
%\DescribeMacro{\Hexadecimalnum}
%The macros \verb"\hexadecimalnum" and
%\verb"\Hexadecimalnum" are like 
%\verb!\hexadecimal! and \verb"\Hexadecimal"
%but take an actual number rather than a counter as the
%argument. For example: \verb"\hexadecimalnum{125}" will
%produce: 7d, and \verb"\Hexadecimalnum{125}" will 
%produce: 7D.
%
%\DescribeMacro{\decimal}
%\changes{v2.41}{22 Oct 2004}{'decimal added}
%The macro \verb"\decimal{"\meta{counter}\verb!}! is similar to 
%\verb"\arabic" but the number can be padded with zeroes
% depending on whether \verb"\padzeroes" has been used.  
%For example: \verb"\padzeroes[8]\decimal{section}" will
% produce: 00000005.
%
%\DescribeMacro{\decimalnum}
%The macro \verb"\decimalnum" is like \verb!\decimal!
%but takes an actual number rather than a counter as the
%argument. For example: \verb"\padzeroes[8]\decimalnum{5}" will
%produce: 00000005.
%
% \DescribeMacro{\aaalph}
%\changes{v2.4}{25 Sept 2004}{'aaalph added}
% The macro \verb"\aaalph{"\meta{counter}\verb!}! will print the 
%value of \meta{counter} as: a b \ldots\ z aa bb \ldots\ zz etc.
% For example, \verb"\aaalpha{mycounter}" will produce: uuuuu if 
%\texttt{mycounter} is set to 125.
% \DescribeMacro{\AAAlph}
% \verb"\AAAlph{"\meta{counter}\verb!}! does the same thing, but 
%uses uppercase characters, e.g.\ 
% \verb"\AAAlph{mycounter}" will produce: UUUUU.
%
%\DescribeMacro{\aaalphnum}
%\DescribeMacro{\AAAlphnum}
%The macros \verb"\aaalphnum" and
%\verb"\AAAlphnum" are like 
%\verb!\aaalph! and \verb"\AAAlph"
%but take an actual number rather than a counter as the
%argument. For example: \verb"\aaalphnum{125}" will
%produce: uuuuu, and \verb"\AAAlphnum{125}" will 
%produce: UUUUU.
%
% \DescribeMacro{\abalph}
%\changes{v2.4}{25 Sept 2004}{'abalph added}
% The macro \verb"\abalph{"\meta{counter}\verb!}! will print the 
%value of \meta{counter} as: a b \ldots\ z aa ab \ldots\ az etc.
% For example, \verb"\abalpha{mycounter}" will produce: du if 
%\texttt{mycounter} is set to 125.
% \DescribeMacro{\ABAlph}
% \verb"\ABAlph{"\meta{counter}\verb!}! does the same thing, but 
%uses uppercase characters, e.g.\ 
% \verb"\ABAlph{mycounter}" will produce: DU.
%
%\DescribeMacro{\abalphnum}
%\DescribeMacro{\ABAlphnum}
%The macros \verb"\abalphnum" and
%\verb"\ABAlphnum" are like 
%\verb!\abalph! and \verb"\ABAlph"
%but take an actual number rather than a counter as the
%argument. For example: \verb"\abalphnum{125}" will
%produce: du, and \verb"\ABAlphnum{125}" will 
%produce: DU.
%
%\section{Package Options}
%
%The following options can be passed to this package:
%
%\begin{tabular}{ll}
% raise    & make ordinal st,nd,rd,th appear as superscript\\
% level    & make ordinal st,nd,rd,th appear level with rest of 
%text
%\end{tabular}
%
%\noindent These can also be set using the command:
%
%\DescribeMacro{\fmtcountsetoptions}
%\verb"\fmtcountsetoptions{fmtord="\meta{type}\verb'}'
%
%\noindent where \meta{type} is either \texttt{level} or 
%\texttt{raise}.
%
%\section{Multilingual Support}
%
%Version 1.02 of the \styname{fmtcount} package now has
%limited multilingual support.  The following languages are
%implemented: English, Spanish, Portuguese, French, French (Swiss)
%and French (Belgian). German support was added in version 
%1.1\footnote{Thanks to K. H. Fricke for supplying the information}.
%
% The package checks to see if the
%command \verb"\date"\meta{language} is defined\footnote{this
%will be true if you have loaded \styname{babel}}, and will
%load the code for those languages.  The commands \verb"\ordinal",
%\verb"\ordinalstring" and \verb'\numberstring' (and their 
%variants) will then be formatted in the currently
%selected language.
%
%If the French language is selected, the French (France) version
%will be used by default (e.g.\ soxiante-dix for 70).  To
%select the Swiss or Belgian variants (e.g.\ septente for 70) use:
%\verb"\fmtcountsetoptions{french="\meta{dialect}\verb'}'
%where \meta{dialect} is either \texttt{swiss} or \texttt{belgian}.
%You can also use this command to change the action of 
%\verb"\ordinal".
%\verb"\fmtcountsetoptions{abbrv=true}" to produce ordinals
%of the form 2\textsuperscript{e} or
%\verb"\fmtcountsetoptions{abbrv=false}" to produce ordinals
%of the form 2\textsuperscript{eme} (default).
%
%The \texttt{french} and \texttt{abbrv} settings only have an
%effect if the French language has been defined.
%
%The male gender for all languages is used by default, however the
%feminine or neuter forms can be obtained by passing \texttt{f} or
%\texttt{n} as an optional argument to \verb"\ordinal",
%\verb!\ordinalnum! etc.  For example:
%\verb"\numberstring{section}[f]". Note that the optional argument
%comes \emph{after} the compulsory argument.  If a gender is
%not defined in a given language, the masculine version will
%be used instead.
%
%Let me know if you find any spelling mistakes (has been known
%to happen in English, let alone other languages I'm not so
%familiar with.) If you want to add support for another language,
%you will need to let me know how to form the numbers and ordinals 
%from 0 to 99999 in that language for each gender.
%
%\section{Configuration File \texttt{fmtcount.cfg}}
%
%You can save your preferred default settings to a file called
%\texttt{fmtcount.cfg}, and place it on the \TeX\ path.  These
%settings will then be loaded by the \styname{fmtcount}
%package.
%
%Note that if you are using the \styname{datetime} package,
%the \texttt{datetime.cfg} configuration file will override
%the \texttt{fmtcount.cfg} configuration file.
%For example, if \texttt{datetime.cfg} has the line:
%\begin{verbatim}
%\renewcommand{\fmtord}[1]{\textsuperscript{\underline{#1}}}
%\end{verbatim}
%and if \texttt{fmtcount.cfg} has the line:
%\begin{verbatim}
%\fmtcountsetoptions{fmtord=level}
%\end{verbatim}
%then the former definition of \verb"\fmtord" will take
%precedence.
%
%\section{LaTeX2HTML style}
%
%The \LaTeX2HTML\ style file \texttt{fmtcount.perl} is provided.
%The following limitations apply:
%
%\begin{itemize}
%\item \verb"\padzeroes" only has an effect in the preamble.
%
%\item The configuration file 
%\texttt{fmtcount.cfg} is currently ignored. (This is because
%I can't work out the correct code to do this.  If you
%know how to do this, please let me know.)  You can however
%do:
%\begin{verbatim}
%\usepackage{fmtcount}
%\html{%\iffalse
% fmtcount.dtx generated using makedtx version 0.91b (c) Nicola Talbot
% Command line args:
%   -src "(.+)\.(def)=>\1.\2"
%   -src "(.+)\.(sty)=>\1.\2"
%   -doc "manual.tex"
%   -author "Nicola Talbot"
%   -dir "source"
%   fmtcount
% Created on 2007/6/22 16:48
%\fi
%\iffalse
%<*package>
%% \CharacterTable
%%  {Upper-case    \A\B\C\D\E\F\G\H\I\J\K\L\M\N\O\P\Q\R\S\T\U\V\W\X\Y\Z
%%   Lower-case    \a\b\c\d\e\f\g\h\i\j\k\l\m\n\o\p\q\r\s\t\u\v\w\x\y\z
%%   Digits        \0\1\2\3\4\5\6\7\8\9
%%   Exclamation   \!     Double quote  \"     Hash (number) \#
%%   Dollar        \$     Percent       \%     Ampersand     \&
%%   Acute accent  \'     Left paren    \(     Right paren   \)
%%   Asterisk      \*     Plus          \+     Comma         \,
%%   Minus         \-     Point         \.     Solidus       \/
%%   Colon         \:     Semicolon     \;     Less than     \<
%%   Equals        \=     Greater than  \>     Question mark \?
%%   Commercial at \@     Left bracket  \[     Backslash     \\
%%   Right bracket \]     Circumflex    \^     Underscore    \_
%%   Grave accent  \`     Left brace    \{     Vertical bar  \|
%%   Right brace   \}     Tilde         \~}
%</package>
%\fi
% \iffalse
% Doc-Source file to use with LaTeX2e
% Copyright (C) 2007 Nicola Talbot, all rights reserved.
% \fi
% \iffalse
%<*driver>
\documentclass{ltxdoc}

\usepackage[colorlinks,
            bookmarks,
            bookmarksopen,
            pdfauthor={N.L.C. Talbot},
            pdftitle={fmtcount.sty: Displaying the Values of LaTeX Counters},
            pdfkeywords={LaTeX,counter}]{hyperref}



\newcommand{\styname}[1]{\textsf{#1}}\newcommand{\clsname}[1]{\textsf{#1}}\newcommand{\cmdname}[1]{\texttt{\symbol{92}#1}}

\begin{document}
\DocInput{fmtcount.dtx}
\end{document}
%</driver>
%\fi
%
%\RecordChanges
%\CheckSum{7751}
%\OnlyDescription
%\def\filedate{22 June 2007}
%\def\fileversion{1.2}
%\def\filename{fmtcount.dtx}
%\def\docdate{22nd June 2007}
%
% \title{fmtcount.sty v\fileversion: Displaying the Values of 
%\LaTeX\ Counters}
% \author{Nicola L.C. Talbot}
% \date{\docdate}
% \maketitle
% \tableofcontents
% \section{Introduction}
%The \styname{fmtcount} package provides commands to display
%the values of \LaTeX\ counters in a variety of formats. It also
%provides equivalent commands for actual numbers rather than 
%counter names. Limited multilingual support is available.
%
%\section{Installation}
%
%This package is distributed with the files \texttt{fmtcount.dtx}
%and \texttt{fmtcount.ins}.  To extract the code do:
%\begin{verbatim}
%latex fmtcount.ins
%\end{verbatim}
%This will create the files \texttt{fmtcount.sty} and 
%\texttt{fmtcount.perl}, along with several \texttt{.def} files.
%Place \texttt{fmtcount.sty} and the \texttt{.def} files somewhere
%where \LaTeX\ will find them (e.g.\ \texttt{texmf/tex/latex/fmtcount/}) and place \texttt{fmtcount.perl} somewhere where \LaTeX2HTML
%will find it (e.g.\ latex2html/styles). Remember to refresh
%the \TeX\ database (using \texttt{texhash} under Linux, for
%other operating systems check the manual.)
%
%\section{Available Commands}
%
%The commands can be divided into two categories: those that
%take the name of a counter as the argument, and those
%that take a number as the argument.
%
% \DescribeMacro{\ordinal}
% The macro \verb"\ordinal{"\meta{counter}\verb!}! will
% print the value of a \LaTeX\ counter \meta{counter} as an ordinal,
% \DescribeMacro{\fmtord}
% where the macro \verb"\fmtord{"\meta{text}\verb!}! is used to format the st,nd,rd,th bit.
% By default the ordinal is formatted as a superscript, if the package option \texttt{level}
% is used, it is level with the text.
% For example, if the current section is 3, then \verb"\ordinal{section}" will produce
% the output: 3\textsuperscript{rd}.
%
%\textbf{Note:} the \clsname{memoir} class also defines a command called
%\verb"\ordinal" which takes a number as an argument instead of a
%counter. In order to overcome this incompatiblity, if you want
%to use the \styname{fmtcount} package with the \clsname{memoir} class 
%you should use \verb"\FCordinal" to access \styname{fmtcount}'s 
%version of \verb"\ordinal", and use \verb"\ordinal" to use
%\clsname{memoir}'s version of that command.
%
%\DescribeMacro{\ordinalnum}
%The macro \verb"\ordinalnum" is like \verb!\ordinal!
%but takes an actual number rather than a counter as the
%argument. For example: \verb"\ordinalnum{3}" will
%produce: 3\textsuperscript{rd}.
%
% \DescribeMacro{\numberstring}
% The macro \verb"\numberstring{"\meta{counter}\verb!}! will print the value
% of \meta{counter} as text.  E.g.\ \verb"\numberstring{section}" will produce:
% three.
% \DescribeMacro{\Numberstring}
% The macro \verb"\Numberstring{"\meta{counter}\verb!}! does the same as
% \verb"\numberstring", but with initial letters in uppercase.  For
% example, \verb"\Numberstring{section}" will produce: Three.
%\DescribeMacro{\NUMBERstring}
%The macro \verb"\NUMBERstring{"\meta{counter}\verb'}' does the same
%as \verb"\numberstring", but converted to upper case. Note that
%\verb"\MakeUppercase{\NUMBERstring{"\meta{counter}\verb'}}' doesn't
%work, due to the way that \verb"\MakeUppercase" expands its 
%argument\footnote{See all the various postings to 
%\texttt{comp.text.tex} about \cmdname{MakeUppercase}}.
%
%\DescribeMacro{\numberstringnum}
%\DescribeMacro{\Numberstringnum}
%\DescribeMacro{\NUMBERstringnum}
%The macros \verb!\numberstringnum!, 
%\verb!\Numberstringnum! and
%\verb"\NUMBERstringnum" work like 
%\verb!\numberstring!, 
%\verb!\Numberstring! and
%\verb"\NUMBERstring", respectively, but take an actual number
%rather than a counter as the argument. For example:
%\verb'\Numberstringnum{105}' will produce: One Hundred and Five.
%
%
% \DescribeMacro{\ordinalstring}
% The macro \verb"\ordinalstring{"\meta{counter}\verb!}! will print the
% value of \meta{counter} as a textual ordinal.  E.g.\
% \verb"\ordinalstring{section}" will produce: third.
% \DescribeMacro{\Ordinalstring}
% The macro \verb"\Ordinalstring{"\meta{counter}\verb!}! does the same as 
% \verb"\ordinalstring", but with initial letters in uppercase.  For example,
% \verb"\Ordinalstring{section}" will produce: Third.
% \DescribeMacro{\ORDINALstring}
% The macro \verb"\ORDINALstring{"\meta{counter}\verb!}! does the same as 
%\verb"\ordinalstring", but with all words in upper case (see previous
%note about \cmdname{MakeUppercase}).
%
%\DescribeMacro{\ordinalstringnum}
%\DescribeMacro{\Ordinalstringnum}
%\DescribeMacro{\ORDINALstringnum}
%The macros \verb!\ordinalstringnum!, 
%\verb!\Ordinalstringnum! and \cmdname{ORDINALstringnum} work like 
%\verb!\ordinalstring!, 
%\verb!\Ordinalstring! and \cmdname{ORDINALstring}, respectively, but take an actual number
%rather than a counter as the argument. For example,
%\verb"\ordinalstringnum{3}" will produce: third.
%
%\changes{v.1.09}{21 Apr 2007}{store facility added}
%As from version 1.09, textual representations can be stored for
%later use. This overcomes the problems encountered when you
%attempt to use one of the above commands in \cmdname{edef}.
%
%Each of the following commands takes a label as the first argument,
%the other arguments are as the analogous commands above. These
%commands do not display anything, but store the textual 
%representation. This can later be retrieved using
%\DescribeMacro{\FMCuse}\cmdname{FMCuse}\{\meta{label}\}.
%Note: with \cmdname{storeordinal} and \cmdname{storeordinalnum}, the 
%only bit that doesn't get expanded is \cmdname{fmtord}. So, for 
%example, \verb"\storeordinalnum{mylabel}{3}" will be stored as
%\verb"3\relax \fmtord{rd}".
%
%\DescribeMacro{\storeordinal}
%\cmdname{storeordinal}\{\meta{label}\}\{\meta{counter}\}[\meta{gender}]
%\par
%\DescribeMacro{\storeordinalstring}
%\cmdname{storeordinalstring}\{\meta{label}\}\{\meta{counter}\}[\meta{gender}]
%\par
%\DescribeMacro{\storeOrdinalstring}
%\cmdname{storeOrdinalstring}\{\meta{label}\}\{\meta{counter}\}[\meta{gender}]
%\par
%\DescribeMacro{\storeORDINALstring}
%\cmdname{storeORDINALstring}\{\meta{label}\}\{\meta{counter}\}[\meta{gender}]
%\par
%\DescribeMacro{\storenumberstring}
%\cmdname{storenumberstring}\{\meta{label}\}\{\meta{counter}\}[\meta{gender}]
%\par
%\DescribeMacro{\storeNumberstring}
%\cmdname{storeNumberstring}\{\meta{label}\}\{\meta{counter}\}[\meta{gender}]
%\par
%\DescribeMacro{\storeNUMBERstring}
%\cmdname{storeNUMBERstring}\{\meta{label}\}\{\meta{counter}\}[\meta{gender}]
%\par
%\DescribeMacro{\storeordinalnum}
%\cmdname{storeordinalnum}\{\meta{label}\}\{\meta{number}\}[\meta{gender}]
%\par
%\DescribeMacro{\storeordinalstringnum}
%\cmdname{storeordinalstring}\{\meta{label}\}\{\meta{number}\}[\meta{gender}]
%\par
%\DescribeMacro{\storeOrdinalstringnum}
%\cmdname{storeOrdinalstringnum}\{\meta{label}\}\{\meta{number}\}[\meta{gender}]
%\par
%\DescribeMacro{\storeORDINALstringnum}
%\cmdname{storeORDINALstringnum}\{\meta{label}\}\{\meta{number}\}[\meta{gender}]
%\par
%\DescribeMacro{\storenumberstringnum}
%\cmdname{storenumberstring}\{\meta{label}\}\{\meta{number}\}[\meta{gender}]
%\par
%\DescribeMacro{\storeNumberstringnum}
%\cmdname{storeNumberstring}\{\meta{label}\}\{\meta{number}\}[\meta{gender}]
%\par
%\DescribeMacro{\storeNUMBERstringnum}
%\cmdname{storeNUMBERstring}\{\meta{label}\}\{\meta{number}\}[\meta{gender}]
%
% \DescribeMacro{\binary}
%\changes{v2.4}{25 Sept 2004}{'binary added}
% The macro \verb"\binary{"\meta{counter}\verb!}! will print the 
%value of \meta{counter} as a binary number.  
%E.g.\ \verb"\binary{section}" will produce: 11.  The declaration 
%\DescribeMacro{\padzeroes}\verb"\padzeroes["\meta{n}\verb!]! 
%will ensure numbers are written to \meta{n} digits, padding with 
%zeroes if necessary.  E.g.\ \verb"\padzeroes[8]\binary{section}" 
%will produce: 00000011.
% The default value for \meta{n} is 17.
%
%\DescribeMacro{\binarynum}
%The macro \verb"\binarynum" is like \verb!\binary!
%but takes an actual number rather than a counter as the
%argument. For example: \verb"\binarynum{5}" will
%produce: 101.
%
% \DescribeMacro{\octal}
%\changes{v2.4}{25 Sept 2004}{'octal added}
% The macro \verb"\octal{"\meta{counter}\verb!}! will print the 
%value of \meta{counter} as an octal number.  For example, if you 
%have a counter called, say \texttt{mycounter}, and you set the 
%value to 125, then \verb"\octal{mycounter}" will produce: 177.  
% Again, the number will be padded with zeroes if necessary, 
%depending on whether \verb"\padzeroes" has been used.
%
%\DescribeMacro{\octalnum}
%The macro \verb"\octalnum" is like \verb!\octal!
%but takes an actual number rather than a counter as the
%argument. For example: \verb"\octalnum{125}" will
%produce: 177.
%
% \DescribeMacro{\hexadecimal}
%\changes{v2.4}{25 Sept 2004}{'hexadecimal added}
% The macro \verb"\hexadecimal{"\meta{counter}\verb!}! will print 
%the value of \meta{counter} as a hexadecimal number.  Going back 
%to the previous example, \verb"\hexadecimal{mycounter}" will
% produce: 7d. Again, the number will be padded with zeroes if 
%necessary, depending on whether \verb"\padzeroes" has been used.
% \DescribeMacro{\Hexadecimal}
% \verb"\Hexadecimal{"\meta{counter}\verb!}! does the same thing, 
%but uses uppercase characters, e.g.\ 
% \verb"\Hexadecimal{mycounter}" will produce: 7D.
%
%\DescribeMacro{\hexadecimalnum}
%\DescribeMacro{\Hexadecimalnum}
%The macros \verb"\hexadecimalnum" and
%\verb"\Hexadecimalnum" are like 
%\verb!\hexadecimal! and \verb"\Hexadecimal"
%but take an actual number rather than a counter as the
%argument. For example: \verb"\hexadecimalnum{125}" will
%produce: 7d, and \verb"\Hexadecimalnum{125}" will 
%produce: 7D.
%
%\DescribeMacro{\decimal}
%\changes{v2.41}{22 Oct 2004}{'decimal added}
%The macro \verb"\decimal{"\meta{counter}\verb!}! is similar to 
%\verb"\arabic" but the number can be padded with zeroes
% depending on whether \verb"\padzeroes" has been used.  
%For example: \verb"\padzeroes[8]\decimal{section}" will
% produce: 00000005.
%
%\DescribeMacro{\decimalnum}
%The macro \verb"\decimalnum" is like \verb!\decimal!
%but takes an actual number rather than a counter as the
%argument. For example: \verb"\padzeroes[8]\decimalnum{5}" will
%produce: 00000005.
%
% \DescribeMacro{\aaalph}
%\changes{v2.4}{25 Sept 2004}{'aaalph added}
% The macro \verb"\aaalph{"\meta{counter}\verb!}! will print the 
%value of \meta{counter} as: a b \ldots\ z aa bb \ldots\ zz etc.
% For example, \verb"\aaalpha{mycounter}" will produce: uuuuu if 
%\texttt{mycounter} is set to 125.
% \DescribeMacro{\AAAlph}
% \verb"\AAAlph{"\meta{counter}\verb!}! does the same thing, but 
%uses uppercase characters, e.g.\ 
% \verb"\AAAlph{mycounter}" will produce: UUUUU.
%
%\DescribeMacro{\aaalphnum}
%\DescribeMacro{\AAAlphnum}
%The macros \verb"\aaalphnum" and
%\verb"\AAAlphnum" are like 
%\verb!\aaalph! and \verb"\AAAlph"
%but take an actual number rather than a counter as the
%argument. For example: \verb"\aaalphnum{125}" will
%produce: uuuuu, and \verb"\AAAlphnum{125}" will 
%produce: UUUUU.
%
% \DescribeMacro{\abalph}
%\changes{v2.4}{25 Sept 2004}{'abalph added}
% The macro \verb"\abalph{"\meta{counter}\verb!}! will print the 
%value of \meta{counter} as: a b \ldots\ z aa ab \ldots\ az etc.
% For example, \verb"\abalpha{mycounter}" will produce: du if 
%\texttt{mycounter} is set to 125.
% \DescribeMacro{\ABAlph}
% \verb"\ABAlph{"\meta{counter}\verb!}! does the same thing, but 
%uses uppercase characters, e.g.\ 
% \verb"\ABAlph{mycounter}" will produce: DU.
%
%\DescribeMacro{\abalphnum}
%\DescribeMacro{\ABAlphnum}
%The macros \verb"\abalphnum" and
%\verb"\ABAlphnum" are like 
%\verb!\abalph! and \verb"\ABAlph"
%but take an actual number rather than a counter as the
%argument. For example: \verb"\abalphnum{125}" will
%produce: du, and \verb"\ABAlphnum{125}" will 
%produce: DU.
%
%\section{Package Options}
%
%The following options can be passed to this package:
%
%\begin{tabular}{ll}
% raise    & make ordinal st,nd,rd,th appear as superscript\\
% level    & make ordinal st,nd,rd,th appear level with rest of 
%text
%\end{tabular}
%
%\noindent These can also be set using the command:
%
%\DescribeMacro{\fmtcountsetoptions}
%\verb"\fmtcountsetoptions{fmtord="\meta{type}\verb'}'
%
%\noindent where \meta{type} is either \texttt{level} or 
%\texttt{raise}.
%
%\section{Multilingual Support}
%
%Version 1.02 of the \styname{fmtcount} package now has
%limited multilingual support.  The following languages are
%implemented: English, Spanish, Portuguese, French, French (Swiss)
%and French (Belgian). German support was added in version 
%1.1\footnote{Thanks to K. H. Fricke for supplying the information}.
%
% The package checks to see if the
%command \verb"\date"\meta{language} is defined\footnote{this
%will be true if you have loaded \styname{babel}}, and will
%load the code for those languages.  The commands \verb"\ordinal",
%\verb"\ordinalstring" and \verb'\numberstring' (and their 
%variants) will then be formatted in the currently
%selected language.
%
%If the French language is selected, the French (France) version
%will be used by default (e.g.\ soxiante-dix for 70).  To
%select the Swiss or Belgian variants (e.g.\ septente for 70) use:
%\verb"\fmtcountsetoptions{french="\meta{dialect}\verb'}'
%where \meta{dialect} is either \texttt{swiss} or \texttt{belgian}.
%You can also use this command to change the action of 
%\verb"\ordinal".
%\verb"\fmtcountsetoptions{abbrv=true}" to produce ordinals
%of the form 2\textsuperscript{e} or
%\verb"\fmtcountsetoptions{abbrv=false}" to produce ordinals
%of the form 2\textsuperscript{eme} (default).
%
%The \texttt{french} and \texttt{abbrv} settings only have an
%effect if the French language has been defined.
%
%The male gender for all languages is used by default, however the
%feminine or neuter forms can be obtained by passing \texttt{f} or
%\texttt{n} as an optional argument to \verb"\ordinal",
%\verb!\ordinalnum! etc.  For example:
%\verb"\numberstring{section}[f]". Note that the optional argument
%comes \emph{after} the compulsory argument.  If a gender is
%not defined in a given language, the masculine version will
%be used instead.
%
%Let me know if you find any spelling mistakes (has been known
%to happen in English, let alone other languages I'm not so
%familiar with.) If you want to add support for another language,
%you will need to let me know how to form the numbers and ordinals 
%from 0 to 99999 in that language for each gender.
%
%\section{Configuration File \texttt{fmtcount.cfg}}
%
%You can save your preferred default settings to a file called
%\texttt{fmtcount.cfg}, and place it on the \TeX\ path.  These
%settings will then be loaded by the \styname{fmtcount}
%package.
%
%Note that if you are using the \styname{datetime} package,
%the \texttt{datetime.cfg} configuration file will override
%the \texttt{fmtcount.cfg} configuration file.
%For example, if \texttt{datetime.cfg} has the line:
%\begin{verbatim}
%\renewcommand{\fmtord}[1]{\textsuperscript{\underline{#1}}}
%\end{verbatim}
%and if \texttt{fmtcount.cfg} has the line:
%\begin{verbatim}
%\fmtcountsetoptions{fmtord=level}
%\end{verbatim}
%then the former definition of \verb"\fmtord" will take
%precedence.
%
%\section{LaTeX2HTML style}
%
%The \LaTeX2HTML\ style file \texttt{fmtcount.perl} is provided.
%The following limitations apply:
%
%\begin{itemize}
%\item \verb"\padzeroes" only has an effect in the preamble.
%
%\item The configuration file 
%\texttt{fmtcount.cfg} is currently ignored. (This is because
%I can't work out the correct code to do this.  If you
%know how to do this, please let me know.)  You can however
%do:
%\begin{verbatim}
%\usepackage{fmtcount}
%\html{%\iffalse
% fmtcount.dtx generated using makedtx version 0.91b (c) Nicola Talbot
% Command line args:
%   -src "(.+)\.(def)=>\1.\2"
%   -src "(.+)\.(sty)=>\1.\2"
%   -doc "manual.tex"
%   -author "Nicola Talbot"
%   -dir "source"
%   fmtcount
% Created on 2007/6/22 16:48
%\fi
%\iffalse
%<*package>
%% \CharacterTable
%%  {Upper-case    \A\B\C\D\E\F\G\H\I\J\K\L\M\N\O\P\Q\R\S\T\U\V\W\X\Y\Z
%%   Lower-case    \a\b\c\d\e\f\g\h\i\j\k\l\m\n\o\p\q\r\s\t\u\v\w\x\y\z
%%   Digits        \0\1\2\3\4\5\6\7\8\9
%%   Exclamation   \!     Double quote  \"     Hash (number) \#
%%   Dollar        \$     Percent       \%     Ampersand     \&
%%   Acute accent  \'     Left paren    \(     Right paren   \)
%%   Asterisk      \*     Plus          \+     Comma         \,
%%   Minus         \-     Point         \.     Solidus       \/
%%   Colon         \:     Semicolon     \;     Less than     \<
%%   Equals        \=     Greater than  \>     Question mark \?
%%   Commercial at \@     Left bracket  \[     Backslash     \\
%%   Right bracket \]     Circumflex    \^     Underscore    \_
%%   Grave accent  \`     Left brace    \{     Vertical bar  \|
%%   Right brace   \}     Tilde         \~}
%</package>
%\fi
% \iffalse
% Doc-Source file to use with LaTeX2e
% Copyright (C) 2007 Nicola Talbot, all rights reserved.
% \fi
% \iffalse
%<*driver>
\documentclass{ltxdoc}

\usepackage[colorlinks,
            bookmarks,
            bookmarksopen,
            pdfauthor={N.L.C. Talbot},
            pdftitle={fmtcount.sty: Displaying the Values of LaTeX Counters},
            pdfkeywords={LaTeX,counter}]{hyperref}



\newcommand{\styname}[1]{\textsf{#1}}\newcommand{\clsname}[1]{\textsf{#1}}\newcommand{\cmdname}[1]{\texttt{\symbol{92}#1}}

\begin{document}
\DocInput{fmtcount.dtx}
\end{document}
%</driver>
%\fi
%
%\RecordChanges
%\CheckSum{7751}
%\OnlyDescription
%\def\filedate{22 June 2007}
%\def\fileversion{1.2}
%\def\filename{fmtcount.dtx}
%\def\docdate{22nd June 2007}
%
% \title{fmtcount.sty v\fileversion: Displaying the Values of 
%\LaTeX\ Counters}
% \author{Nicola L.C. Talbot}
% \date{\docdate}
% \maketitle
% \tableofcontents
% \section{Introduction}
%The \styname{fmtcount} package provides commands to display
%the values of \LaTeX\ counters in a variety of formats. It also
%provides equivalent commands for actual numbers rather than 
%counter names. Limited multilingual support is available.
%
%\section{Installation}
%
%This package is distributed with the files \texttt{fmtcount.dtx}
%and \texttt{fmtcount.ins}.  To extract the code do:
%\begin{verbatim}
%latex fmtcount.ins
%\end{verbatim}
%This will create the files \texttt{fmtcount.sty} and 
%\texttt{fmtcount.perl}, along with several \texttt{.def} files.
%Place \texttt{fmtcount.sty} and the \texttt{.def} files somewhere
%where \LaTeX\ will find them (e.g.\ \texttt{texmf/tex/latex/fmtcount/}) and place \texttt{fmtcount.perl} somewhere where \LaTeX2HTML
%will find it (e.g.\ latex2html/styles). Remember to refresh
%the \TeX\ database (using \texttt{texhash} under Linux, for
%other operating systems check the manual.)
%
%\section{Available Commands}
%
%The commands can be divided into two categories: those that
%take the name of a counter as the argument, and those
%that take a number as the argument.
%
% \DescribeMacro{\ordinal}
% The macro \verb"\ordinal{"\meta{counter}\verb!}! will
% print the value of a \LaTeX\ counter \meta{counter} as an ordinal,
% \DescribeMacro{\fmtord}
% where the macro \verb"\fmtord{"\meta{text}\verb!}! is used to format the st,nd,rd,th bit.
% By default the ordinal is formatted as a superscript, if the package option \texttt{level}
% is used, it is level with the text.
% For example, if the current section is 3, then \verb"\ordinal{section}" will produce
% the output: 3\textsuperscript{rd}.
%
%\textbf{Note:} the \clsname{memoir} class also defines a command called
%\verb"\ordinal" which takes a number as an argument instead of a
%counter. In order to overcome this incompatiblity, if you want
%to use the \styname{fmtcount} package with the \clsname{memoir} class 
%you should use \verb"\FCordinal" to access \styname{fmtcount}'s 
%version of \verb"\ordinal", and use \verb"\ordinal" to use
%\clsname{memoir}'s version of that command.
%
%\DescribeMacro{\ordinalnum}
%The macro \verb"\ordinalnum" is like \verb!\ordinal!
%but takes an actual number rather than a counter as the
%argument. For example: \verb"\ordinalnum{3}" will
%produce: 3\textsuperscript{rd}.
%
% \DescribeMacro{\numberstring}
% The macro \verb"\numberstring{"\meta{counter}\verb!}! will print the value
% of \meta{counter} as text.  E.g.\ \verb"\numberstring{section}" will produce:
% three.
% \DescribeMacro{\Numberstring}
% The macro \verb"\Numberstring{"\meta{counter}\verb!}! does the same as
% \verb"\numberstring", but with initial letters in uppercase.  For
% example, \verb"\Numberstring{section}" will produce: Three.
%\DescribeMacro{\NUMBERstring}
%The macro \verb"\NUMBERstring{"\meta{counter}\verb'}' does the same
%as \verb"\numberstring", but converted to upper case. Note that
%\verb"\MakeUppercase{\NUMBERstring{"\meta{counter}\verb'}}' doesn't
%work, due to the way that \verb"\MakeUppercase" expands its 
%argument\footnote{See all the various postings to 
%\texttt{comp.text.tex} about \cmdname{MakeUppercase}}.
%
%\DescribeMacro{\numberstringnum}
%\DescribeMacro{\Numberstringnum}
%\DescribeMacro{\NUMBERstringnum}
%The macros \verb!\numberstringnum!, 
%\verb!\Numberstringnum! and
%\verb"\NUMBERstringnum" work like 
%\verb!\numberstring!, 
%\verb!\Numberstring! and
%\verb"\NUMBERstring", respectively, but take an actual number
%rather than a counter as the argument. For example:
%\verb'\Numberstringnum{105}' will produce: One Hundred and Five.
%
%
% \DescribeMacro{\ordinalstring}
% The macro \verb"\ordinalstring{"\meta{counter}\verb!}! will print the
% value of \meta{counter} as a textual ordinal.  E.g.\
% \verb"\ordinalstring{section}" will produce: third.
% \DescribeMacro{\Ordinalstring}
% The macro \verb"\Ordinalstring{"\meta{counter}\verb!}! does the same as 
% \verb"\ordinalstring", but with initial letters in uppercase.  For example,
% \verb"\Ordinalstring{section}" will produce: Third.
% \DescribeMacro{\ORDINALstring}
% The macro \verb"\ORDINALstring{"\meta{counter}\verb!}! does the same as 
%\verb"\ordinalstring", but with all words in upper case (see previous
%note about \cmdname{MakeUppercase}).
%
%\DescribeMacro{\ordinalstringnum}
%\DescribeMacro{\Ordinalstringnum}
%\DescribeMacro{\ORDINALstringnum}
%The macros \verb!\ordinalstringnum!, 
%\verb!\Ordinalstringnum! and \cmdname{ORDINALstringnum} work like 
%\verb!\ordinalstring!, 
%\verb!\Ordinalstring! and \cmdname{ORDINALstring}, respectively, but take an actual number
%rather than a counter as the argument. For example,
%\verb"\ordinalstringnum{3}" will produce: third.
%
%\changes{v.1.09}{21 Apr 2007}{store facility added}
%As from version 1.09, textual representations can be stored for
%later use. This overcomes the problems encountered when you
%attempt to use one of the above commands in \cmdname{edef}.
%
%Each of the following commands takes a label as the first argument,
%the other arguments are as the analogous commands above. These
%commands do not display anything, but store the textual 
%representation. This can later be retrieved using
%\DescribeMacro{\FMCuse}\cmdname{FMCuse}\{\meta{label}\}.
%Note: with \cmdname{storeordinal} and \cmdname{storeordinalnum}, the 
%only bit that doesn't get expanded is \cmdname{fmtord}. So, for 
%example, \verb"\storeordinalnum{mylabel}{3}" will be stored as
%\verb"3\relax \fmtord{rd}".
%
%\DescribeMacro{\storeordinal}
%\cmdname{storeordinal}\{\meta{label}\}\{\meta{counter}\}[\meta{gender}]
%\par
%\DescribeMacro{\storeordinalstring}
%\cmdname{storeordinalstring}\{\meta{label}\}\{\meta{counter}\}[\meta{gender}]
%\par
%\DescribeMacro{\storeOrdinalstring}
%\cmdname{storeOrdinalstring}\{\meta{label}\}\{\meta{counter}\}[\meta{gender}]
%\par
%\DescribeMacro{\storeORDINALstring}
%\cmdname{storeORDINALstring}\{\meta{label}\}\{\meta{counter}\}[\meta{gender}]
%\par
%\DescribeMacro{\storenumberstring}
%\cmdname{storenumberstring}\{\meta{label}\}\{\meta{counter}\}[\meta{gender}]
%\par
%\DescribeMacro{\storeNumberstring}
%\cmdname{storeNumberstring}\{\meta{label}\}\{\meta{counter}\}[\meta{gender}]
%\par
%\DescribeMacro{\storeNUMBERstring}
%\cmdname{storeNUMBERstring}\{\meta{label}\}\{\meta{counter}\}[\meta{gender}]
%\par
%\DescribeMacro{\storeordinalnum}
%\cmdname{storeordinalnum}\{\meta{label}\}\{\meta{number}\}[\meta{gender}]
%\par
%\DescribeMacro{\storeordinalstringnum}
%\cmdname{storeordinalstring}\{\meta{label}\}\{\meta{number}\}[\meta{gender}]
%\par
%\DescribeMacro{\storeOrdinalstringnum}
%\cmdname{storeOrdinalstringnum}\{\meta{label}\}\{\meta{number}\}[\meta{gender}]
%\par
%\DescribeMacro{\storeORDINALstringnum}
%\cmdname{storeORDINALstringnum}\{\meta{label}\}\{\meta{number}\}[\meta{gender}]
%\par
%\DescribeMacro{\storenumberstringnum}
%\cmdname{storenumberstring}\{\meta{label}\}\{\meta{number}\}[\meta{gender}]
%\par
%\DescribeMacro{\storeNumberstringnum}
%\cmdname{storeNumberstring}\{\meta{label}\}\{\meta{number}\}[\meta{gender}]
%\par
%\DescribeMacro{\storeNUMBERstringnum}
%\cmdname{storeNUMBERstring}\{\meta{label}\}\{\meta{number}\}[\meta{gender}]
%
% \DescribeMacro{\binary}
%\changes{v2.4}{25 Sept 2004}{'binary added}
% The macro \verb"\binary{"\meta{counter}\verb!}! will print the 
%value of \meta{counter} as a binary number.  
%E.g.\ \verb"\binary{section}" will produce: 11.  The declaration 
%\DescribeMacro{\padzeroes}\verb"\padzeroes["\meta{n}\verb!]! 
%will ensure numbers are written to \meta{n} digits, padding with 
%zeroes if necessary.  E.g.\ \verb"\padzeroes[8]\binary{section}" 
%will produce: 00000011.
% The default value for \meta{n} is 17.
%
%\DescribeMacro{\binarynum}
%The macro \verb"\binarynum" is like \verb!\binary!
%but takes an actual number rather than a counter as the
%argument. For example: \verb"\binarynum{5}" will
%produce: 101.
%
% \DescribeMacro{\octal}
%\changes{v2.4}{25 Sept 2004}{'octal added}
% The macro \verb"\octal{"\meta{counter}\verb!}! will print the 
%value of \meta{counter} as an octal number.  For example, if you 
%have a counter called, say \texttt{mycounter}, and you set the 
%value to 125, then \verb"\octal{mycounter}" will produce: 177.  
% Again, the number will be padded with zeroes if necessary, 
%depending on whether \verb"\padzeroes" has been used.
%
%\DescribeMacro{\octalnum}
%The macro \verb"\octalnum" is like \verb!\octal!
%but takes an actual number rather than a counter as the
%argument. For example: \verb"\octalnum{125}" will
%produce: 177.
%
% \DescribeMacro{\hexadecimal}
%\changes{v2.4}{25 Sept 2004}{'hexadecimal added}
% The macro \verb"\hexadecimal{"\meta{counter}\verb!}! will print 
%the value of \meta{counter} as a hexadecimal number.  Going back 
%to the previous example, \verb"\hexadecimal{mycounter}" will
% produce: 7d. Again, the number will be padded with zeroes if 
%necessary, depending on whether \verb"\padzeroes" has been used.
% \DescribeMacro{\Hexadecimal}
% \verb"\Hexadecimal{"\meta{counter}\verb!}! does the same thing, 
%but uses uppercase characters, e.g.\ 
% \verb"\Hexadecimal{mycounter}" will produce: 7D.
%
%\DescribeMacro{\hexadecimalnum}
%\DescribeMacro{\Hexadecimalnum}
%The macros \verb"\hexadecimalnum" and
%\verb"\Hexadecimalnum" are like 
%\verb!\hexadecimal! and \verb"\Hexadecimal"
%but take an actual number rather than a counter as the
%argument. For example: \verb"\hexadecimalnum{125}" will
%produce: 7d, and \verb"\Hexadecimalnum{125}" will 
%produce: 7D.
%
%\DescribeMacro{\decimal}
%\changes{v2.41}{22 Oct 2004}{'decimal added}
%The macro \verb"\decimal{"\meta{counter}\verb!}! is similar to 
%\verb"\arabic" but the number can be padded with zeroes
% depending on whether \verb"\padzeroes" has been used.  
%For example: \verb"\padzeroes[8]\decimal{section}" will
% produce: 00000005.
%
%\DescribeMacro{\decimalnum}
%The macro \verb"\decimalnum" is like \verb!\decimal!
%but takes an actual number rather than a counter as the
%argument. For example: \verb"\padzeroes[8]\decimalnum{5}" will
%produce: 00000005.
%
% \DescribeMacro{\aaalph}
%\changes{v2.4}{25 Sept 2004}{'aaalph added}
% The macro \verb"\aaalph{"\meta{counter}\verb!}! will print the 
%value of \meta{counter} as: a b \ldots\ z aa bb \ldots\ zz etc.
% For example, \verb"\aaalpha{mycounter}" will produce: uuuuu if 
%\texttt{mycounter} is set to 125.
% \DescribeMacro{\AAAlph}
% \verb"\AAAlph{"\meta{counter}\verb!}! does the same thing, but 
%uses uppercase characters, e.g.\ 
% \verb"\AAAlph{mycounter}" will produce: UUUUU.
%
%\DescribeMacro{\aaalphnum}
%\DescribeMacro{\AAAlphnum}
%The macros \verb"\aaalphnum" and
%\verb"\AAAlphnum" are like 
%\verb!\aaalph! and \verb"\AAAlph"
%but take an actual number rather than a counter as the
%argument. For example: \verb"\aaalphnum{125}" will
%produce: uuuuu, and \verb"\AAAlphnum{125}" will 
%produce: UUUUU.
%
% \DescribeMacro{\abalph}
%\changes{v2.4}{25 Sept 2004}{'abalph added}
% The macro \verb"\abalph{"\meta{counter}\verb!}! will print the 
%value of \meta{counter} as: a b \ldots\ z aa ab \ldots\ az etc.
% For example, \verb"\abalpha{mycounter}" will produce: du if 
%\texttt{mycounter} is set to 125.
% \DescribeMacro{\ABAlph}
% \verb"\ABAlph{"\meta{counter}\verb!}! does the same thing, but 
%uses uppercase characters, e.g.\ 
% \verb"\ABAlph{mycounter}" will produce: DU.
%
%\DescribeMacro{\abalphnum}
%\DescribeMacro{\ABAlphnum}
%The macros \verb"\abalphnum" and
%\verb"\ABAlphnum" are like 
%\verb!\abalph! and \verb"\ABAlph"
%but take an actual number rather than a counter as the
%argument. For example: \verb"\abalphnum{125}" will
%produce: du, and \verb"\ABAlphnum{125}" will 
%produce: DU.
%
%\section{Package Options}
%
%The following options can be passed to this package:
%
%\begin{tabular}{ll}
% raise    & make ordinal st,nd,rd,th appear as superscript\\
% level    & make ordinal st,nd,rd,th appear level with rest of 
%text
%\end{tabular}
%
%\noindent These can also be set using the command:
%
%\DescribeMacro{\fmtcountsetoptions}
%\verb"\fmtcountsetoptions{fmtord="\meta{type}\verb'}'
%
%\noindent where \meta{type} is either \texttt{level} or 
%\texttt{raise}.
%
%\section{Multilingual Support}
%
%Version 1.02 of the \styname{fmtcount} package now has
%limited multilingual support.  The following languages are
%implemented: English, Spanish, Portuguese, French, French (Swiss)
%and French (Belgian). German support was added in version 
%1.1\footnote{Thanks to K. H. Fricke for supplying the information}.
%
% The package checks to see if the
%command \verb"\date"\meta{language} is defined\footnote{this
%will be true if you have loaded \styname{babel}}, and will
%load the code for those languages.  The commands \verb"\ordinal",
%\verb"\ordinalstring" and \verb'\numberstring' (and their 
%variants) will then be formatted in the currently
%selected language.
%
%If the French language is selected, the French (France) version
%will be used by default (e.g.\ soxiante-dix for 70).  To
%select the Swiss or Belgian variants (e.g.\ septente for 70) use:
%\verb"\fmtcountsetoptions{french="\meta{dialect}\verb'}'
%where \meta{dialect} is either \texttt{swiss} or \texttt{belgian}.
%You can also use this command to change the action of 
%\verb"\ordinal".
%\verb"\fmtcountsetoptions{abbrv=true}" to produce ordinals
%of the form 2\textsuperscript{e} or
%\verb"\fmtcountsetoptions{abbrv=false}" to produce ordinals
%of the form 2\textsuperscript{eme} (default).
%
%The \texttt{french} and \texttt{abbrv} settings only have an
%effect if the French language has been defined.
%
%The male gender for all languages is used by default, however the
%feminine or neuter forms can be obtained by passing \texttt{f} or
%\texttt{n} as an optional argument to \verb"\ordinal",
%\verb!\ordinalnum! etc.  For example:
%\verb"\numberstring{section}[f]". Note that the optional argument
%comes \emph{after} the compulsory argument.  If a gender is
%not defined in a given language, the masculine version will
%be used instead.
%
%Let me know if you find any spelling mistakes (has been known
%to happen in English, let alone other languages I'm not so
%familiar with.) If you want to add support for another language,
%you will need to let me know how to form the numbers and ordinals 
%from 0 to 99999 in that language for each gender.
%
%\section{Configuration File \texttt{fmtcount.cfg}}
%
%You can save your preferred default settings to a file called
%\texttt{fmtcount.cfg}, and place it on the \TeX\ path.  These
%settings will then be loaded by the \styname{fmtcount}
%package.
%
%Note that if you are using the \styname{datetime} package,
%the \texttt{datetime.cfg} configuration file will override
%the \texttt{fmtcount.cfg} configuration file.
%For example, if \texttt{datetime.cfg} has the line:
%\begin{verbatim}
%\renewcommand{\fmtord}[1]{\textsuperscript{\underline{#1}}}
%\end{verbatim}
%and if \texttt{fmtcount.cfg} has the line:
%\begin{verbatim}
%\fmtcountsetoptions{fmtord=level}
%\end{verbatim}
%then the former definition of \verb"\fmtord" will take
%precedence.
%
%\section{LaTeX2HTML style}
%
%The \LaTeX2HTML\ style file \texttt{fmtcount.perl} is provided.
%The following limitations apply:
%
%\begin{itemize}
%\item \verb"\padzeroes" only has an effect in the preamble.
%
%\item The configuration file 
%\texttt{fmtcount.cfg} is currently ignored. (This is because
%I can't work out the correct code to do this.  If you
%know how to do this, please let me know.)  You can however
%do:
%\begin{verbatim}
%\usepackage{fmtcount}
%\html{\input{fmtcount.cfg}}
%\end{verbatim}
%This, I agree, is an unpleasant cludge.
%
%\end{itemize}
%
%\section{Acknowledgements}
%
%I would like to thank my mother for the French and Portuguese
%support and my Spanish dictionary for the Spanish support.
%Thank you to K. H. Fricke for providing me with the German
%translations.
%
%\section{Troubleshooting}
%
%There is a FAQ available at: \url{http://theoval.cmp.uea.ac.uk/~nlct/latex/packages/faq/}.
%
% \section{Contact Details}
% Dr Nicola Talbot\\
% School of Computing Sciences\\
% University of East Anglia\\
% Norwich.  NR4 7TJ.\\
% United Kingdom.\\
% \url{http://theoval.cmp.uea.ac.uk/~nlct/}
%
%
%\StopEventually{}
%\section{The Code}
%\iffalse
%    \begin{macrocode}
%<*fc-british.def>
%    \end{macrocode}
%\fi
% \subsection{fc-british.def}
% British definitions
%    \begin{macrocode}
\ProvidesFile{fc-british}[2007/06/14]
%    \end{macrocode}
% Check that fc-english.def has been loaded
%    \begin{macrocode}
\@ifundefined{@ordinalMenglish}{\input{fc-english.def}}{}
%    \end{macrocode}
% These are all just synonyms for the commands provided by
% fc-english.def.
%    \begin{macrocode}
\let\@ordinalMbritish\@ordinalMenglish
\let\@ordinalFbritish\@ordinalMenglish
\let\@ordinalNbritish\@ordinalMenglish
\let\@numberstringMbritish\@numberstringMenglish
\let\@numberstringFbritish\@numberstringMenglish
\let\@numberstringNbritish\@numberstringMenglish
\let\@NumberstringMbritish\@NumberstringMenglish
\let\@NumberstringFbritish\@NumberstringMenglish
\let\@NumberstringNbritish\@NumberstringMenglish
\let\@ordinalstringMbritish\@ordinalstringMenglish
\let\@ordinalstringFbritish\@ordinalstringMenglish
\let\@ordinalstringNbritish\@ordinalstringMenglish
\let\@OrdinalstringMbritish\@OrdinalstringMenglish
\let\@OrdinalstringFbritish\@OrdinalstringMenglish
\let\@OrdinalstringNbritish\@OrdinalstringMenglish
%    \end{macrocode}
%\iffalse
%    \begin{macrocode}
%</fc-british.def>
%    \end{macrocode}
%\fi
%\iffalse
%    \begin{macrocode}
%<*fc-english.def>
%    \end{macrocode}
%\fi
% \subsection{fc-english.def}
% English definitions
%    \begin{macrocode}
\ProvidesFile{fc-english}[2007/05/26]
%    \end{macrocode}
% Define macro that converts a number or count register (first 
% argument) to an ordinal, and stores the result in the 
% second argument, which should be a control sequence.
%    \begin{macrocode}
\newcommand*{\@ordinalMenglish}[2]{%
\def\@fc@ord{}%
\@orgargctr=#1\relax
\@ordinalctr=#1%
\@modulo{\@ordinalctr}{100}%
\ifnum\@ordinalctr=11\relax
  \def\@fc@ord{th}%
\else
  \ifnum\@ordinalctr=12\relax
    \def\@fc@ord{th}%
  \else
    \ifnum\@ordinalctr=13\relax
      \def\@fc@ord{th}%
    \else
      \@modulo{\@ordinalctr}{10}%
      \ifcase\@ordinalctr
        \def\@fc@ord{th}%      case 0
        \or \def\@fc@ord{st}%  case 1
        \or \def\@fc@ord{nd}%  case 2
        \or \def\@fc@ord{rd}%  case 3
      \else
        \def\@fc@ord{th}%      default case
      \fi
    \fi
  \fi
\fi
\edef#2{\number#1\relax\noexpand\fmtord{\@fc@ord}}%
}
%    \end{macrocode}
% There is no gender difference in English, so make feminine and
% neuter the same as the masculine.
%    \begin{macrocode}
\let\@ordinalFenglish=\@ordinalMenglish
\let\@ordinalNenglish=\@ordinalMenglish
%    \end{macrocode}
% Define the macro that prints the value of a \TeX\ count register
% as text. To make it easier, break it up into units, teens and
% tens. First, the units: the argument should be between 0 and 9
% inclusive.
%    \begin{macrocode}
\newcommand*{\@@unitstringenglish}[1]{%
\ifcase#1\relax
zero%
\or one%
\or two%
\or three%
\or four%
\or five%
\or six%
\or seven%
\or eight%
\or nine%
\fi
}
%    \end{macrocode}
% Next the tens, again the argument should be between 0 and 9
% inclusive.
%    \begin{macrocode}
\newcommand*{\@@tenstringenglish}[1]{%
\ifcase#1\relax
\or ten%
\or twenty%
\or thirty%
\or forty%
\or fifty%
\or sixty%
\or seventy%
\or eighty%
\or ninety%
\fi
}
%    \end{macrocode}
% Finally the teens, again the argument should be between 0 and 9
% inclusive.
%    \begin{macrocode}
\newcommand*{\@@teenstringenglish}[1]{%
\ifcase#1\relax
ten%
\or eleven%
\or twelve%
\or thirteen%
\or fourteen%
\or fifteen%
\or sixteen%
\or seventeen%
\or eighteen%
\or nineteen%
\fi
}
%    \end{macrocode}
% As above, but with the initial letter in uppercase. The units:
%    \begin{macrocode}
\newcommand*{\@@Unitstringenglish}[1]{%
\ifcase#1\relax
Zero%
\or One%
\or Two%
\or Three%
\or Four%
\or Five%
\or Six%
\or Seven%
\or Eight%
\or Nine%
\fi
}
%    \end{macrocode}
% The tens:
%    \begin{macrocode}
\newcommand*{\@@Tenstringenglish}[1]{%
\ifcase#1\relax
\or Ten%
\or Twenty%
\or Thirty%
\or Forty%
\or Fifty%
\or Sixty%
\or Seventy%
\or Eighty%
\or Ninety%
\fi
}
%    \end{macrocode}
% The teens:
%    \begin{macrocode}
\newcommand*{\@@Teenstringenglish}[1]{%
\ifcase#1\relax
Ten%
\or Eleven%
\or Twelve%
\or Thirteen%
\or Fourteen%
\or Fifteen%
\or Sixteen%
\or Seventeen%
\or Eighteen%
\or Nineteen%
\fi
}
%    \end{macrocode}
% This has changed in version 1.09, so that it now stores
% the result in the second argument, but doesn't display anything.
% Since it only affects internal macros, it shouldn't affect
% documents created with older versions. (These internal macros are
% not meant for use in documents.)
%    \begin{macrocode}
\newcommand*{\@@numberstringenglish}[2]{%
\ifnum#1>99999
\PackageError{fmtcount}{Out of range}%
{This macro only works for values less than 100000}%
\else
\ifnum#1<0
\PackageError{fmtcount}{Negative numbers not permitted}%
{This macro does not work for negative numbers, however
you can try typing "minus" first, and then pass the modulus of
this number}%
\fi
\fi
\def#2{}%
\@strctr=#1\relax \divide\@strctr by 1000\relax
\ifnum\@strctr>9
% #1 is greater or equal to 10000
  \divide\@strctr by 10
  \ifnum\@strctr>1\relax
    \let\@@fc@numstr#2\relax
    \edef#2{\@@fc@numstr\@tenstring{\@strctr}}%
    \@strctr=#1 \divide\@strctr by 1000\relax
    \@modulo{\@strctr}{10}%
    \ifnum\@strctr>0\relax
      \let\@@fc@numstr#2\relax
      \edef#2{\@@fc@numstr-\@unitstring{\@strctr}}%
    \fi
  \else
    \@strctr=#1\relax
    \divide\@strctr by 1000\relax
    \@modulo{\@strctr}{10}%
    \let\@@fc@numstr#2\relax
    \edef#2{\@@fc@numstr\@teenstring{\@strctr}}%
  \fi
  \let\@@fc@numstr#2\relax
  \edef#2{\@@fc@numstr\ \@thousand}%
\else
  \ifnum\@strctr>0\relax
    \let\@@fc@numstr#2\relax
    \edef#2{\@@fc@numstr\@unitstring{\@strctr}\ \@thousand}%
  \fi
\fi
\@strctr=#1\relax \@modulo{\@strctr}{1000}%
\divide\@strctr by 100
\ifnum\@strctr>0\relax
   \ifnum#1>1000\relax
      \let\@@fc@numstr#2\relax
      \edef#2{\@@fc@numstr\ }%
   \fi
   \let\@@fc@numstr#2\relax
   \edef#2{\@@fc@numstr\@unitstring{\@strctr}\ \@hundred}%
\fi
\@strctr=#1\relax \@modulo{\@strctr}{100}%
\ifnum#1>100\relax
  \ifnum\@strctr>0\relax
    \let\@@fc@numstr#2\relax
    \edef#2{\@@fc@numstr\ \@andname\ }%
  \fi
\fi
\ifnum\@strctr>19\relax
  \divide\@strctr by 10\relax
  \let\@@fc@numstr#2\relax
  \edef#2{\@@fc@numstr\@tenstring{\@strctr}}%
  \@strctr=#1\relax \@modulo{\@strctr}{10}%
  \ifnum\@strctr>0\relax
    \let\@@fc@numstr#2\relax
    \edef#2{\@@fc@numstr-\@unitstring{\@strctr}}%
  \fi
\else
  \ifnum\@strctr<10\relax
    \ifnum\@strctr=0\relax
       \ifnum#1<100\relax
          \let\@@fc@numstr#2\relax
          \edef#2{\@@fc@numstr\@unitstring{\@strctr}}%
       \fi
    \else
      \let\@@fc@numstr#2\relax
      \edef#2{\@@fc@numstr\@unitstring{\@strctr}}%
    \fi
  \else
    \@modulo{\@strctr}{10}%
    \let\@@fc@numstr#2\relax
    \edef#2{\@@fc@numstr\@teenstring{\@strctr}}%
  \fi
\fi
}
%    \end{macrocode}
% All lower case version, the second argument must be a 
% control sequence.
%    \begin{macrocode}
\DeclareRobustCommand{\@numberstringMenglish}[2]{%
\let\@unitstring=\@@unitstringenglish 
\let\@teenstring=\@@teenstringenglish 
\let\@tenstring=\@@tenstringenglish
\def\@hundred{hundred}\def\@thousand{thousand}%
\def\@andname{and}%
\@@numberstringenglish{#1}{#2}%
}
%    \end{macrocode}
% There is no gender in English, so make feminine and neuter the same
% as the masculine.
%    \begin{macrocode}
\let\@numberstringFenglish=\@numberstringMenglish
\let\@numberstringNenglish=\@numberstringMenglish
%    \end{macrocode}
% This version makes the first letter of each word an uppercase
% character (except ``and''). The second argument must be a control 
% sequence.
%    \begin{macrocode}
\newcommand*{\@NumberstringMenglish}[2]{%
\let\@unitstring=\@@Unitstringenglish 
\let\@teenstring=\@@Teenstringenglish 
\let\@tenstring=\@@Tenstringenglish
\def\@hundred{Hundred}\def\@thousand{Thousand}%
\def\@andname{and}%
\@@numberstringenglish{#1}{#2}}
%    \end{macrocode}
% There is no gender in English, so make feminine and neuter the same
% as the masculine.
%    \begin{macrocode}
\let\@NumberstringFenglish=\@NumberstringMenglish
\let\@NumberstringNenglish=\@NumberstringMenglish
%    \end{macrocode}
% Define a macro that produces an ordinal as a string. Again, break
% it up into units, teens and tens. First the units:
%    \begin{macrocode}
\newcommand*{\@@unitthstringenglish}[1]{%
\ifcase#1\relax
zeroth%
\or first%
\or second%
\or third%
\or fourth%
\or fifth%
\or sixth%
\or seventh%
\or eighth%
\or ninth%
\fi
}
%    \end{macrocode}
% Next the tens:
%    \begin{macrocode}
\newcommand*{\@@tenthstringenglish}[1]{%
\ifcase#1\relax
\or tenth%
\or twentieth%
\or thirtieth%
\or fortieth%
\or fiftieth%
\or sixtieth%
\or seventieth%
\or eightieth%
\or ninetieth%
\fi
}
%   \end{macrocode}
% The teens:
%   \begin{macrocode}
\newcommand*{\@@teenthstringenglish}[1]{%
\ifcase#1\relax
tenth%
\or eleventh%
\or twelfth%
\or thirteenth%
\or fourteenth%
\or fifteenth%
\or sixteenth%
\or seventeenth%
\or eighteenth%
\or nineteenth%
\fi
}
%   \end{macrocode}
% As before, but with the first letter in upper case. The units:
%   \begin{macrocode}
\newcommand*{\@@Unitthstringenglish}[1]{%
\ifcase#1\relax
Zeroth%
\or First%
\or Second%
\or Third%
\or Fourth%
\or Fifth%
\or Sixth%
\or Seventh%
\or Eighth%
\or Ninth%
\fi
}
%    \end{macrocode}
% The tens:
%    \begin{macrocode}
\newcommand*{\@@Tenthstringenglish}[1]{%
\ifcase#1\relax
\or Tenth%
\or Twentieth%
\or Thirtieth%
\or Fortieth%
\or Fiftieth%
\or Sixtieth%
\or Seventieth%
\or Eightieth%
\or Ninetieth%
\fi
}
%    \end{macrocode}
% The teens:
%    \begin{macrocode}
\newcommand*{\@@Teenthstringenglish}[1]{%
\ifcase#1\relax
Tenth%
\or Eleventh%
\or Twelfth%
\or Thirteenth%
\or Fourteenth%
\or Fifteenth%
\or Sixteenth%
\or Seventeenth%
\or Eighteenth%
\or Nineteenth%
\fi
}
%    \end{macrocode}
% Again, as from version 1.09, this has been changed to take two
% arguments, where the second argument is a control sequence.
% The resulting text is stored in the control sequence, and nothing
% is displayed.
%    \begin{macrocode}
\newcommand*{\@@ordinalstringenglish}[2]{%
\@strctr=#1\relax
\ifnum#1>99999
\PackageError{fmtcount}{Out of range}%
{This macro only works for values less than 100000 (value given: \number\@strctr)}%
\else
\ifnum#1<0
\PackageError{fmtcount}{Negative numbers not permitted}%
{This macro does not work for negative numbers, however
you can try typing "minus" first, and then pass the modulus of
this number}%
\fi
\def#2{}%
\fi
\@strctr=#1\relax \divide\@strctr by 1000\relax
\ifnum\@strctr>9\relax
% #1 is greater or equal to 10000
  \divide\@strctr by 10
  \ifnum\@strctr>1\relax
    \let\@@fc@ordstr#2\relax
    \edef#2{\@@fc@ordstr\@tenstring{\@strctr}}%
    \@strctr=#1\relax
    \divide\@strctr by 1000\relax
    \@modulo{\@strctr}{10}%
    \ifnum\@strctr>0\relax
      \let\@@fc@ordstr#2\relax
      \edef#2{\@@fc@ordstr-\@unitstring{\@strctr}}%
    \fi
  \else
    \@strctr=#1\relax \divide\@strctr by 1000\relax
    \@modulo{\@strctr}{10}%
    \let\@@fc@ordstr#2\relax
    \edef#2{\@@fc@ordstr\@teenstring{\@strctr}}%
  \fi
  \@strctr=#1\relax \@modulo{\@strctr}{1000}%
  \ifnum\@strctr=0\relax
    \let\@@fc@ordstr#2\relax
    \edef#2{\@@fc@ordstr\ \@thousandth}%
  \else
    \let\@@fc@ordstr#2\relax
    \edef#2{\@@fc@ordstr\ \@thousand}%
  \fi
\else
  \ifnum\@strctr>0\relax
    \let\@@fc@ordstr#2\relax
    \edef#2{\@@fc@ordstr\@unitstring{\@strctr}}%
    \@strctr=#1\relax \@modulo{\@strctr}{1000}%
    \let\@@fc@ordstr#2\relax
    \ifnum\@strctr=0\relax
      \edef#2{\@@fc@ordstr\ \@thousandth}%
    \else
      \edef#2{\@@fc@ordstr\ \@thousand}%
    \fi
  \fi
\fi
\@strctr=#1\relax \@modulo{\@strctr}{1000}%
\divide\@strctr by 100
\ifnum\@strctr>0\relax
  \ifnum#1>1000\relax
    \let\@@fc@ordstr#2\relax
    \edef#2{\@@fc@ordstr\ }%
  \fi
  \let\@@fc@ordstr#2\relax
  \edef#2{\@@fc@ordstr\@unitstring{\@strctr}}%
  \@strctr=#1\relax \@modulo{\@strctr}{100}%
  \let\@@fc@ordstr#2\relax
  \ifnum\@strctr=0\relax
    \edef#2{\@@fc@ordstr\ \@hundredth}%
  \else
    \edef#2{\@@fc@ordstr\ \@hundred}%
  \fi
\fi
\@strctr=#1\relax \@modulo{\@strctr}{100}%
\ifnum#1>100\relax
  \ifnum\@strctr>0\relax
    \let\@@fc@ordstr#2\relax
    \edef#2{\@@fc@ordstr\ \@andname\ }%
  \fi
\fi
\ifnum\@strctr>19\relax
  \@tmpstrctr=\@strctr
  \divide\@strctr by 10\relax
  \@modulo{\@tmpstrctr}{10}%
  \let\@@fc@ordstr#2\relax
  \ifnum\@tmpstrctr=0\relax
    \edef#2{\@@fc@ordstr\@tenthstring{\@strctr}}%
  \else
    \edef#2{\@@fc@ordstr\@tenstring{\@strctr}}%
  \fi
  \@strctr=#1\relax \@modulo{\@strctr}{10}%
  \ifnum\@strctr>0\relax
    \let\@@fc@ordstr#2\relax
    \edef#2{\@@fc@ordstr-\@unitthstring{\@strctr}}%
  \fi
\else
  \ifnum\@strctr<10\relax
    \ifnum\@strctr=0\relax
      \ifnum#1<100\relax
        \let\@@fc@ordstr#2\relax
        \edef#2{\@@fc@ordstr\@unitthstring{\@strctr}}%
      \fi
    \else
      \let\@@fc@ordstr#2\relax
      \edef#2{\@@fc@ordstr\@unitthstring{\@strctr}}%
    \fi
  \else
    \@modulo{\@strctr}{10}%
    \let\@@fc@ordstr#2\relax
    \edef#2{\@@fc@ordstr\@teenthstring{\@strctr}}%
  \fi
\fi
}
%    \end{macrocode}
% All lower case version. Again, the second argument must be a
% control sequence in which the resulting text is stored.
%    \begin{macrocode}
\DeclareRobustCommand{\@ordinalstringMenglish}[2]{%
\let\@unitthstring=\@@unitthstringenglish 
\let\@teenthstring=\@@teenthstringenglish 
\let\@tenthstring=\@@tenthstringenglish
\let\@unitstring=\@@unitstringenglish 
\let\@teenstring=\@@teenstringenglish
\let\@tenstring=\@@tenstringenglish
\def\@andname{and}%
\def\@hundred{hundred}\def\@thousand{thousand}%
\def\@hundredth{hundredth}\def\@thousandth{thousandth}%
\@@ordinalstringenglish{#1}{#2}}
%    \end{macrocode}
% No gender in English, so make feminine and neuter same as masculine:
%    \begin{macrocode}
\let\@ordinalstringFenglish=\@ordinalstringMenglish
\let\@ordinalstringNenglish=\@ordinalstringMenglish
%    \end{macrocode}
% First letter of each word in upper case:
%    \begin{macrocode}
\DeclareRobustCommand{\@OrdinalstringMenglish}[2]{%
\let\@unitthstring=\@@Unitthstringenglish
\let\@teenthstring=\@@Teenthstringenglish
\let\@tenthstring=\@@Tenthstringenglish
\let\@unitstring=\@@Unitstringenglish
\let\@teenstring=\@@Teenstringenglish
\let\@tenstring=\@@Tenstringenglish
\def\@andname{and}%
\def\@hundred{Hundred}\def\@thousand{Thousand}%
\def\@hundredth{Hundredth}\def\@thousandth{Thousandth}%
\@@ordinalstringenglish{#1}{#2}}
%    \end{macrocode}
% No gender in English, so make feminine and neuter same as masculine:
%    \begin{macrocode}
\let\@OrdinalstringFenglish=\@OrdinalstringMenglish
\let\@OrdinalstringNenglish=\@OrdinalstringMenglish
%    \end{macrocode}
%\iffalse
%    \begin{macrocode}
%</fc-english.def>
%    \end{macrocode}
%\fi
%\iffalse
%    \begin{macrocode}
%<*fc-french.def>
%    \end{macrocode}
%\fi
% \subsection{fc-french.def}
% French definitions
%    \begin{macrocode}
\ProvidesFile{fc-french.def}[2007/05/26]
%    \end{macrocode}
% Define macro that converts a number or count register (first
% argument) to an ordinal, and store the result in the second
% argument, which must be a control sequence. Masculine:
%    \begin{macrocode}
\newcommand*{\@ordinalMfrench}[2]{%
\iffmtord@abbrv
  \edef#2{\number#1\relax\noexpand\fmtord{e}}%
\else
  \ifnum#1=1\relax
    \edef#2{\number#1\relax\noexpand\fmtord{er}}%
  \else
    \edef#2{\number#1\relax\noexpand\fmtord{eme}}%
  \fi
\fi}
%    \end{macrocode}
% Feminine:
%    \begin{macrocode}
\newcommand*{\@ordinalFfrench}[2]{%
\iffmtord@abbrv
  \edef#2{\number#1\relax\noexpand\fmtord{e}}%
\else
  \ifnum#1=1\relax
     \edef#2{\number#1\relax\noexpand\fmtord{ere}}%
  \else
     \edef#2{\number#1\relax\noexpand\fmtord{eme}}%
  \fi
\fi}
%    \end{macrocode}
% Make neuter same as masculine:
%    \begin{macrocode}
\let\@ordinalNfrench\@ordinalMfrench
%    \end{macrocode}
% Textual representation of a number. To make it easier break it
% into units, tens and teens. First the units:
%   \begin{macrocode}
\newcommand*{\@@unitstringfrench}[1]{%
\ifcase#1\relax
zero%
\or un%
\or deux%
\or trois%
\or quatre%
\or cinq%
\or six%
\or sept%
\or huit%
\or neuf%
\fi
}
%    \end{macrocode}
% Feminine only changes for 1:
%    \begin{macrocode}
\newcommand*{\@@unitstringFfrench}[1]{%
\ifnum#1=1\relax
une%
\else\@@unitstringfrench{#1}%
\fi
}
%    \end{macrocode}
% Tens (this includes the Belgian and Swiss variants, special
% cases employed lower down.)
%    \begin{macrocode}
\newcommand*{\@@tenstringfrench}[1]{%
\ifcase#1\relax
\or dix%
\or vingt%
\or trente%
\or quarante%
\or cinquante%
\or soixante%
\or septente%
\or huitante%
\or nonente%
\or cent%
\fi
}
%    \end{macrocode}
% Teens:
%    \begin{macrocode}
\newcommand*{\@@teenstringfrench}[1]{%
\ifcase#1\relax
dix%
\or onze%
\or douze%
\or treize%
\or quatorze%
\or quinze%
\or seize%
\or dix-sept%
\or dix-huit%
\or dix-neuf%
\fi
}
%    \end{macrocode}
% Seventies are a special case, depending on dialect:
%    \begin{macrocode}
\newcommand*{\@@seventiesfrench}[1]{%
\@tenstring{6}%
\ifnum#1=1\relax
\ \@andname\ 
\else
-%
\fi
\@teenstring{#1}%
}
%    \end{macrocode}
% Eighties are a special case, depending on dialect:
%    \begin{macrocode}
\newcommand*{\@@eightiesfrench}[1]{%
\@unitstring{4}-\@tenstring{2}%
\ifnum#1>0
-\@unitstring{#1}%
\else
s%
\fi
}
%    \end{macrocode}
% Nineties are a special case, depending on dialect:
%    \begin{macrocode}
\newcommand*{\@@ninetiesfrench}[1]{%
\@unitstring{4}-\@tenstring{2}-\@teenstring{#1}%
}
%    \end{macrocode}
% Swiss seventies:
%    \begin{macrocode}
\newcommand*{\@@seventiesfrenchswiss}[1]{%
\@tenstring{7}%
\ifnum#1=1\ \@andname\ \fi
\ifnum#1>1-\fi
\ifnum#1>0\@unitstring{#1}\fi
}
%    \end{macrocode}
% Swiss eighties:
%    \begin{macrocode}
\newcommand*{\@@eightiesfrenchswiss}[1]{%
\@tenstring{8}%
\ifnum#1=1\ \@andname\ \fi
\ifnum#1>1-\fi
\ifnum#1>0\@unitstring{#1}\fi
}
%    \end{macrocode}
% Swiss nineties:
%    \begin{macrocode}
\newcommand*{\@@ninetiesfrenchswiss}[1]{%
\@tenstring{9}%
\ifnum#1=1\ \@andname\ \fi
\ifnum#1>1-\fi
\ifnum#1>0\@unitstring{#1}\fi
}
%    \end{macrocode}
% Units with initial letter in upper case:
%    \begin{macrocode}
\newcommand*{\@@Unitstringfrench}[1]{%
\ifcase#1\relax
Zero%
\or Un%
\or Deux%
\or Trois%
\or Quatre%
\or Cinq%
\or Six%
\or Sept%
\or Huit%
\or Neuf%
\fi
}
%    \end{macrocode}
% As above, but feminine:
%    \begin{macrocode}
\newcommand*{\@@UnitstringFfrench}[1]{%
\ifnum#1=1\relax
Une%
\else \@@Unitstringfrench{#1}%
\fi
}
%    \end{macrocode}
% Tens, with initial letter in upper case (includes Swiss and
% Belgian variants):
%    \begin{macrocode}
\newcommand*{\@@Tenstringfrench}[1]{%
\ifcase#1\relax
\or Dix%
\or Vingt%
\or Trente%
\or Quarante%
\or Cinquante%
\or Soixante%
\or Septente%
\or Huitante%
\or Nonente%
\or Cent%
\fi
}
%    \end{macrocode}
% Teens, with initial letter in upper case:
%    \begin{macrocode}
\newcommand*{\@@Teenstringfrench}[1]{%
\ifcase#1\relax
Dix%
\or Onze%
\or Douze%
\or Treize%
\or Quatorze%
\or Quinze%
\or Seize%
\or Dix-Sept%
\or Dix-Huit%
\or Dix-Neuf%
\fi
}
%    \end{macrocode}
% This has changed in version 1.09, so that it now stores the
% result in the second argument, but doesn't display anything.
% Since it only affects internal macros, it shouldn't affect
% documents created with older versions. (These internal macros
% are not defined for use in documents.) Firstly, the Swiss
% version:
%    \begin{macrocode}
\DeclareRobustCommand{\@numberstringMfrenchswiss}[2]{%
\let\@unitstring=\@@unitstringfrench
\let\@teenstring=\@@teenstringfrench
\let\@tenstring=\@@tenstringfrench
\let\@seventies=\@@seventiesfrenchswiss
\let\@eighties=\@@eightiesfrenchswiss
\let\@nineties=\@@ninetiesfrenchswiss
\def\@hundred{cent}\def\@thousand{mille}%
\def\@andname{et}%
\@@numberstringfrench{#1}{#2}}
%    \end{macrocode}
% Same as above, but for French as spoken in France:
%    \begin{macrocode}
\DeclareRobustCommand{\@numberstringMfrenchfrance}[2]{%
\let\@unitstring=\@@unitstringfrench
\let\@teenstring=\@@teenstringfrench
\let\@tenstring=\@@tenstringfrench
\let\@seventies=\@@seventiesfrench
\let\@eighties=\@@eightiesfrench
\let\@nineties=\@@ninetiesfrench
\def\@hundred{cent}\def\@thousand{mille}%
\def\@andname{et}%
\@@numberstringfrench{#1}{#2}}
%    \end{macrocode}
% Same as above, but for Belgian dialect:
%    \begin{macrocode}
\DeclareRobustCommand{\@numberstringMfrenchbelgian}[2]{%
\let\@unitstring=\@@unitstringfrench
\let\@teenstring=\@@teenstringfrench
\let\@tenstring=\@@tenstringfrench
\let\@seventies=\@@seventiesfrenchswiss
\let\@eighties=\@@eightiesfrench
\let\@nineties=\@@ninetiesfrench
\def\@hundred{cent}\def\@thousand{mille}%
\def\@andname{et}%
\@@numberstringfrench{#1}{#2}}
%    \end{macrocode}
% Set default dialect:
%    \begin{macrocode}
\let\@numberstringMfrench=\@numberstringMfrenchfrance
%    \end{macrocode}
% As above, but for feminine version. Swiss:
%    \begin{macrocode}
\DeclareRobustCommand{\@numberstringFfrenchswiss}[2]{%
\let\@unitstring=\@@unitstringFfrench
\let\@teenstring=\@@teenstringfrench
\let\@tenstring=\@@tenstringfrench
\let\@seventies=\@@seventiesfrenchswiss
\let\@eighties=\@@eightiesfrenchswiss
\let\@nineties=\@@ninetiesfrenchswiss
\def\@hundred{cent}\def\@thousand{mille}%
\def\@andname{et}%
\@@numberstringfrench{#1}{#2}}
%    \end{macrocode}
% French:
%    \begin{macrocode}
\DeclareRobustCommand{\@numberstringFfrenchfrance}[2]{%
\let\@unitstring=\@@unitstringFfrench
\let\@teenstring=\@@teenstringfrench
\let\@tenstring=\@@tenstringfrench
\let\@seventies=\@@seventiesfrench
\let\@eighties=\@@eightiesfrench
\let\@nineties=\@@ninetiesfrench
\def\@hundred{cent}\def\@thousand{mille}%
\def\@andname{et}%
\@@numberstringfrench{#1}{#2}}
%    \end{macrocode}
% Belgian:
%    \begin{macrocode}
\DeclareRobustCommand{\@numberstringFfrenchbelgian}[2]{%
\let\@unitstring=\@@unitstringFfrench
\let\@teenstring=\@@teenstringfrench
\let\@tenstring=\@@tenstringfrench
\let\@seventies=\@@seventiesfrenchswiss
\let\@eighties=\@@eightiesfrench
\let\@nineties=\@@ninetiesfrench
\def\@hundred{cent}\def\@thousand{mille}%
\def\@andname{et}%
\@@numberstringfrench{#1}{#2}}
%    \end{macrocode}
% Set default dialect:
%    \begin{macrocode}
\let\@numberstringFfrench=\@numberstringFfrenchfrance
%    \end{macrocode}
% Make neuter same as masculine:
%    \begin{macrocode}
\let\@ordinalstringNfrench\@ordinalstringMfrench
%    \end{macrocode}
% As above, but with initial letter in upper case. Swiss (masculine):
%    \begin{macrocode}
\DeclareRobustCommand{\@NumberstringMfrenchswiss}[2]{%
\let\@unitstring=\@@Unitstringfrench
\let\@teenstring=\@@Teenstringfrench
\let\@tenstring=\@@Tenstringfrench
\let\@seventies=\@@seventiesfrenchswiss
\let\@eighties=\@@eightiesfrenchswiss
\let\@nineties=\@@ninetiesfrenchswiss
\def\@hundred{Cent}\def\@thousand{Mille}%
\def\@andname{et}%
\@@numberstringfrench{#1}{#2}}
%    \end{macrocode}
% French:
%    \begin{macrocode}
\DeclareRobustCommand{\@NumberstringMfrenchfrance}[2]{%
\let\@unitstring=\@@Unitstringfrench
\let\@teenstring=\@@Teenstringfrench
\let\@tenstring=\@@Tenstringfrench
\let\@seventies=\@@seventiesfrench
\let\@eighties=\@@eightiesfrench
\let\@nineties=\@@ninetiesfrench
\def\@hundred{Cent}\def\@thousand{Mille}%
\def\@andname{et}%
\@@numberstringfrench{#1}{#2}}
%    \end{macrocode}
% Belgian:
%    \begin{macrocode}
\DeclareRobustCommand{\@NumberstringMfrenchbelgian}[2]{%
\let\@unitstring=\@@Unitstringfrench
\let\@teenstring=\@@Teenstringfrench
\let\@tenstring=\@@Tenstringfrench
\let\@seventies=\@@seventiesfrenchswiss
\let\@eighties=\@@eightiesfrench
\let\@nineties=\@@ninetiesfrench
\def\@hundred{Cent}\def\@thousand{Mille}%
\def\@andname{et}%
\@@numberstringfrench{#1}{#2}}
%    \end{macrocode}
% Set default dialect:
%    \begin{macrocode}
\let\@NumberstringMfrench=\@NumberstringMfrenchfrance
%    \end{macrocode}
% As above, but feminine. Swiss:
%    \begin{macrocode}
\DeclareRobustCommand{\@NumberstringFfrenchswiss}[2]{%
\let\@unitstring=\@@UnitstringFfrench
\let\@teenstring=\@@Teenstringfrench
\let\@tenstring=\@@Tenstringfrench
\let\@seventies=\@@seventiesfrenchswiss
\let\@eighties=\@@eightiesfrenchswiss
\let\@nineties=\@@ninetiesfrenchswiss
\def\@hundred{Cent}\def\@thousand{Mille}%
\def\@andname{et}%
\@@numberstringfrench{#1}{#2}}
%    \end{macrocode}
% French (feminine):
%    \begin{macrocode}
\DeclareRobustCommand{\@NumberstringFfrenchfrance}[2]{%
\let\@unitstring=\@@UnitstringFfrench
\let\@teenstring=\@@Teenstringfrench
\let\@tenstring=\@@Tenstringfrench
\let\@seventies=\@@seventiesfrench
\let\@eighties=\@@eightiesfrench
\let\@nineties=\@@ninetiesfrench
\def\@hundred{Cent}\def\@thousand{Mille}%
\def\@andname{et}%
\@@numberstringfrench{#1}{#2}}
%    \end{macrocode}
% Belgian (feminine):
%    \begin{macrocode}
\DeclareRobustCommand{\@NumberstringFfrenchbelgian}[2]{%
\let\@unitstring=\@@UnitstringFfrench
\let\@teenstring=\@@Teenstringfrench
\let\@tenstring=\@@Tenstringfrench
\let\@seventies=\@@seventiesfrenchswiss
\let\@eighties=\@@eightiesfrench
\let\@nineties=\@@ninetiesfrench
\def\@hundred{Cent}\def\@thousand{Mille}%
\def\@andname{et}%
\@@numberstringfrench{#1}{#2}}
%    \end{macrocode}
% Set default dialect:
%    \begin{macrocode}
\let\@NumberstringFfrench=\@NumberstringFfrenchfrance
%    \end{macrocode}
% Make neuter same as masculine:
%    \begin{macrocode}
\let\@NumberstringNfrench\@NumberstringMfrench
%    \end{macrocode}
% Again, as from version 1.09, this has been changed to take
% two arguments, where the second argument is a control
% sequence, and nothing is displayed. Store textual representation
% of an ordinal in the given control sequence. Swiss dialect (masculine):
%    \begin{macrocode}
\DeclareRobustCommand{\@ordinalstringMfrenchswiss}[2]{%
\ifnum#1=1\relax
\def#2{premier}%
\else
\let\@unitthstring=\@@unitthstringfrench
\let\@unitstring=\@@unitstringfrench
\let\@teenthstring=\@@teenthstringfrench
\let\@teenstring=\@@teenstringfrench
\let\@tenthstring=\@@tenthstringfrench
\let\@tenstring=\@@tenstringfrench
\let\@seventieths=\@@seventiethsfrenchswiss
\let\@eightieths=\@@eightiethsfrenchswiss
\let\@ninetieths=\@@ninetiethsfrenchswiss
\let\@seventies=\@@seventiesfrenchswiss
\let\@eighties=\@@eightiesfrenchswiss
\let\@nineties=\@@ninetiesfrenchswiss
\def\@hundredth{centi\`eme}\def\@hundred{cent}%
\def\@thousandth{mili\`eme}\def\@thousand{mille}%
\def\@andname{et}%
\@@ordinalstringfrench{#1}{#2}%
\fi}
%    \end{macrocode}
% French (masculine):
%    \begin{macrocode}
\DeclareRobustCommand{\@ordinalstringMfrenchfrance}[2]{%
\ifnum#1=1\relax
\def#2{premier}%
\else
\let\@unitthstring=\@@unitthstringfrench
\let\@unitstring=\@@unitstringfrench
\let\@teenthstring=\@@teenthstringfrench
\let\@teenstring=\@@teenstringfrench
\let\@tenthstring=\@@tenthstringfrench
\let\@tenstring=\@@tenstringfrench
\let\@seventieths=\@@seventiethsfrench
\let\@eightieths=\@@eightiethsfrench
\let\@ninetieths=\@@ninetiethsfrench
\let\@seventies=\@@seventiesfrench
\let\@eighties=\@@eightiesfrench
\let\@nineties=\@@ninetiesfrench
\let\@teenstring=\@@teenstringfrench
\def\@hundredth{centi\`eme}\def\@hundred{cent}%
\def\@thousandth{mili\`eme}\def\@thousand{mille}%
\def\@andname{et}%
\@@ordinalstringfrench{#1}{#2}%
\fi}
%    \end{macrocode}
% Belgian dialect (masculine):
%    \begin{macrocode}
\DeclareRobustCommand{\@ordinalstringMfrenchbelgian}[2]{%
\ifnum#1=1\relax
\def#2{premier}%
\else
\let\@unitthstring=\@@unitthstringfrench
\let\@unitstring=\@@unitstringfrench
\let\@teenthstring=\@@teenthstringfrench
\let\@teenstring=\@@teenstringfrench
\let\@tenthstring=\@@tenthstringfrench
\let\@tenstring=\@@tenstringfrench
\let\@seventieths=\@@seventiethsfrenchswiss
\let\@eightieths=\@@eightiethsfrench
\let\@ninetieths=\@@ninetiethsfrenchswiss
\let\@seventies=\@@seventiesfrench
\let\@eighties=\@@eightiesfrench
\let\@nineties=\@@ninetiesfrench
\let\@teenstring=\@@teenstringfrench
\def\@hundredth{centi\`eme}\def\@hundred{cent}%
\def\@thousandth{mili\`eme}\def\@thousand{mille}%
\def\@andname{et}%
\@@ordinalstringfrench{#1}{#2}%
\fi}
%    \end{macrocode}
% Set up default dialect:
%    \begin{macrocode}
\let\@ordinalstringMfrench=\@ordinalstringMfrenchfrance
%    \end{macrocode}
% As above, but feminine. Swiss:
%    \begin{macrocode}
\DeclareRobustCommand{\@ordinalstringFfrenchswiss}[2]{%
\ifnum#1=1\relax
\def#2{premi\`ere}%
\else
\let\@unitthstring=\@@unitthstringfrench
\let\@unitstring=\@@unitstringFfrench
\let\@teenthstring=\@@teenthstringfrench
\let\@teenstring=\@@teenstringfrench
\let\@tenthstring=\@@tenthstringfrench
\let\@tenstring=\@@tenstringfrench
\let\@seventieths=\@@seventiethsfrenchswiss
\let\@eightieths=\@@eightiethsfrenchswiss
\let\@ninetieths=\@@ninetiethsfrenchswiss
\let\@seventies=\@@seventiesfrenchswiss
\let\@eighties=\@@eightiesfrenchswiss
\let\@nineties=\@@ninetiesfrenchswiss
\def\@hundredth{centi\`eme}\def\@hundred{cent}%
\def\@thousandth{mili\`eme}\def\@thousand{mille}%
\def\@andname{et}%
\@@ordinalstringfrench{#1}{#2}%
\fi}
%    \end{macrocode}
% French (feminine):
%    \begin{macrocode}
\DeclareRobustCommand{\@ordinalstringFfrenchfrance}[2]{%
\ifnum#1=1\relax
\def#2{premi\`ere}%
\else
\let\@unitthstring=\@@unitthstringfrench
\let\@unitstring=\@@unitstringFfrench
\let\@teenthstring=\@@teenthstringfrench
\let\@teenstring=\@@teenstringfrench
\let\@tenthstring=\@@tenthstringfrench
\let\@tenstring=\@@tenstringfrench
\let\@seventieths=\@@seventiethsfrench
\let\@eightieths=\@@eightiethsfrench
\let\@ninetieths=\@@ninetiethsfrench
\let\@seventies=\@@seventiesfrench
\let\@eighties=\@@eightiesfrench
\let\@nineties=\@@ninetiesfrench
\let\@teenstring=\@@teenstringfrench
\def\@hundredth{centi\`eme}\def\@hundred{cent}%
\def\@thousandth{mili\`eme}\def\@thousand{mille}%
\def\@andname{et}%
\@@ordinalstringfrench{#1}{#2}%
\fi}
%    \end{macrocode}
% Belgian (feminine):
%    \begin{macrocode}
\DeclareRobustCommand{\@ordinalstringFfrenchbelgian}[2]{%
\ifnum#1=1\relax
\def#2{premi\`ere}%
\else
\let\@unitthstring=\@@unitthstringfrench
\let\@unitstring=\@@unitstringFfrench
\let\@teenthstring=\@@teenthstringfrench
\let\@teenstring=\@@teenstringfrench
\let\@tenthstring=\@@tenthstringfrench
\let\@tenstring=\@@tenstringfrench
\let\@seventieths=\@@seventiethsfrenchswiss
\let\@eightieths=\@@eightiethsfrench
\let\@ninetieths=\@@ninetiethsfrench
\let\@seventies=\@@seventiesfrench
\let\@eighties=\@@eightiesfrench
\let\@nineties=\@@ninetiesfrench
\let\@teenstring=\@@teenstringfrench
\def\@hundredth{centi\`eme}\def\@hundred{cent}%
\def\@thousandth{mili\`eme}\def\@thousand{mille}%
\def\@andname{et}%
\@@ordinalstringfrench{#1}{#2}%
\fi}
%    \end{macrocode}
% Set up default dialect:
%    \begin{macrocode}
\let\@ordinalstringFfrench=\@ordinalstringFfrenchfrance
%    \end{macrocode}
% Make neuter same as masculine:
%    \begin{macrocode}
\let\@ordinalstringNfrench\@ordinalstringMfrench
%    \end{macrocode}
% As above, but with initial letters in upper case. Swiss (masculine):
%    \begin{macrocode}
\DeclareRobustCommand{\@OrdinalstringMfrenchswiss}[2]{%
\ifnum#1=1\relax
\def#2{Premi\`ere}%
\else
\let\@unitthstring=\@@Unitthstringfrench
\let\@unitstring=\@@Unitstringfrench
\let\@teenthstring=\@@Teenthstringfrench
\let\@teenstring=\@@Teenstringfrench
\let\@tenthstring=\@@Tenthstringfrench
\let\@tenstring=\@@Tenstringfrench
\let\@seventieths=\@@seventiethsfrenchswiss
\let\@eightieths=\@@eightiethsfrenchswiss
\let\@ninetieths=\@@ninetiethsfrenchswiss
\let\@seventies=\@@seventiesfrenchswiss
\let\@eighties=\@@eightiesfrenchswiss
\let\@nineties=\@@ninetiesfrenchswiss
\def\@hundredth{Centi\`eme}\def\@hundred{Cent}%
\def\@thousandth{Mili\`eme}\def\@thousand{Mille}%
\def\@andname{et}%
\@@ordinalstringfrench{#1}{#2}%
\fi}
%    \end{macrocode}
% French (masculine):
%    \begin{macrocode}
\DeclareRobustCommand{\@OrdinalstringMfrenchfrance}[2]{%
\ifnum#1=1\relax
\def#2{Premi\`ere}%
\else
\let\@unitthstring=\@@Unitthstringfrench
\let\@unitstring=\@@Unitstringfrench
\let\@teenthstring=\@@Teenthstringfrench
\let\@teenstring=\@@Teenstringfrench
\let\@tenthstring=\@@Tenthstringfrench
\let\@tenstring=\@@Tenstringfrench
\let\@seventieths=\@@seventiethsfrench
\let\@eightieths=\@@eightiethsfrench
\let\@ninetieths=\@@ninetiethsfrench
\let\@seventies=\@@seventiesfrench
\let\@eighties=\@@eightiesfrench
\let\@nineties=\@@ninetiesfrench
\let\@teenstring=\@@Teenstringfrench
\def\@hundredth{Centi\`eme}\def\@hundred{Cent}%
\def\@thousandth{Mili\`eme}\def\@thousand{Mille}%
\def\@andname{et}%
\@@ordinalstringfrench{#1}{#2}%
\fi}
%    \end{macrocode}
% Belgian (masculine):
%    \begin{macrocode}
\DeclareRobustCommand{\@OrdinalstringMfrenchbelgian}[2]{%
\ifnum#1=1\relax
\def#2{Premi\`ere}%
\else
\let\@unitthstring=\@@Unitthstringfrench
\let\@unitstring=\@@Unitstringfrench
\let\@teenthstring=\@@Teenthstringfrench
\let\@teenstring=\@@Teenstringfrench
\let\@tenthstring=\@@Tenthstringfrench
\let\@tenstring=\@@Tenstringfrench
\let\@seventieths=\@@seventiethsfrenchswiss
\let\@eightieths=\@@eightiethsfrench
\let\@ninetieths=\@@ninetiethsfrench
\let\@seventies=\@@seventiesfrench
\let\@eighties=\@@eightiesfrench
\let\@nineties=\@@ninetiesfrench
\let\@teenstring=\@@Teenstringfrench
\def\@hundredth{Centi\`eme}\def\@hundred{Cent}%
\def\@thousandth{Mili\`eme}\def\@thousand{Mille}%
\def\@andname{et}%
\@@ordinalstringfrench{#1}{#2}%
\fi}
%    \end{macrocode}
% Set up default dialect:
%    \begin{macrocode}
\let\@OrdinalstringMfrench=\@OrdinalstringMfrenchfrance
%    \end{macrocode}
% As above, but feminine form. Swiss:
%    \begin{macrocode}
\DeclareRobustCommand{\@OrdinalstringFfrenchswiss}[2]{%
\ifnum#1=1\relax
\def#2{Premi\`ere}%
\else
\let\@unitthstring=\@@Unitthstringfrench
\let\@unitstring=\@@UnitstringFfrench
\let\@teenthstring=\@@Teenthstringfrench
\let\@teenstring=\@@Teenstringfrench
\let\@tenthstring=\@@Tenthstringfrench
\let\@tenstring=\@@Tenstringfrench
\let\@seventieths=\@@seventiethsfrenchswiss
\let\@eightieths=\@@eightiethsfrenchswiss
\let\@ninetieths=\@@ninetiethsfrenchswiss
\let\@seventies=\@@seventiesfrenchswiss
\let\@eighties=\@@eightiesfrenchswiss
\let\@nineties=\@@ninetiesfrenchswiss
\def\@hundredth{Centi\`eme}\def\@hundred{Cent}%
\def\@thousandth{Mili\`eme}\def\@thousand{Mille}%
\def\@andname{et}%
\@@ordinalstringfrench{#1}{#2}%
\fi}
%    \end{macrocode}
% French (feminine):
%    \begin{macrocode}
\DeclareRobustCommand{\@OrdinalstringFfrenchfrance}[2]{%
\ifnum#1=1\relax
\def#2{Premi\`ere}%
\else
\let\@unitthstring=\@@Unitthstringfrench
\let\@unitstring=\@@UnitstringFfrench
\let\@teenthstring=\@@Teenthstringfrench
\let\@teenstring=\@@Teenstringfrench
\let\@tenthstring=\@@Tenthstringfrench
\let\@tenstring=\@@Tenstringfrench
\let\@seventieths=\@@seventiethsfrench
\let\@eightieths=\@@eightiethsfrench
\let\@ninetieths=\@@ninetiethsfrench
\let\@seventies=\@@seventiesfrench
\let\@eighties=\@@eightiesfrench
\let\@nineties=\@@ninetiesfrench
\let\@teenstring=\@@Teenstringfrench
\def\@hundredth{Centi\`eme}\def\@hundred{Cent}%
\def\@thousandth{Mili\`eme}\def\@thousand{Mille}%
\def\@andname{et}%
\@@ordinalstringfrench{#1}{#2}%
\fi}
%    \end{macrocode}
% Belgian (feminine):
%    \begin{macrocode}
\DeclareRobustCommand{\@OrdinalstringFfrenchbelgian}[2]{%
\ifnum#1=1\relax
\def#2{Premi\`ere}%
\else
\let\@unitthstring=\@@Unitthstringfrench
\let\@unitstring=\@@UnitstringFfrench
\let\@teenthstring=\@@Teenthstringfrench
\let\@teenstring=\@@Teenstringfrench
\let\@tenthstring=\@@Tenthstringfrench
\let\@tenstring=\@@Tenstringfrench
\let\@seventieths=\@@seventiethsfrenchswiss
\let\@eightieths=\@@eightiethsfrench
\let\@ninetieths=\@@ninetiethsfrench
\let\@seventies=\@@seventiesfrench
\let\@eighties=\@@eightiesfrench
\let\@nineties=\@@ninetiesfrench
\let\@teenstring=\@@Teenstringfrench
\def\@hundredth{Centi\`eme}\def\@hundred{Cent}%
\def\@thousandth{Mili\`eme}\def\@thousand{Mille}%
\def\@andname{et}%
\@@ordinalstringfrench{#1}{#2}%
\fi}
%    \end{macrocode}
% Set up default dialect:
%    \begin{macrocode}
\let\@OrdinalstringFfrench=\@OrdinalstringFfrenchfrance
%    \end{macrocode}
% Make neuter same as masculine:
%    \begin{macrocode}
\let\@OrdinalstringNfrench\@OrdinalstringMfrench
%    \end{macrocode}
% In order to convert numbers into textual ordinals, need
% to break it up into units, tens and teens. First the units.
% The argument must be a number or count register between 0
% and 9.
%    \begin{macrocode}
\newcommand*{\@@unitthstringfrench}[1]{%
\ifcase#1\relax
zero%
\or uni\`eme%
\or deuxi\`eme%
\or troisi\`eme%
\or quatri\`eme%
\or cinqui\`eme%
\or sixi\`eme%
\or septi\`eme%
\or huiti\`eme%
\or neuvi\`eme%
\fi
}
%    \end{macrocode}
% Tens (includes Swiss and Belgian variants, special cases are
% dealt with later.)
%    \begin{macrocode}
\newcommand*{\@@tenthstringfrench}[1]{%
\ifcase#1\relax
\or dixi\`eme%
\or vingti\`eme%
\or trentri\`eme%
\or quaranti\`eme%
\or cinquanti\`eme%
\or soixanti\`eme%
\or septenti\`eme%
\or huitanti\`eme%
\or nonenti\`eme%
\fi
}
%    \end{macrocode}
% Teens:
%    \begin{macrocode}
\newcommand*{\@@teenthstringfrench}[1]{%
\ifcase#1\relax
dixi\`eme%
\or onzi\`eme%
\or douzi\`eme%
\or treizi\`eme%
\or quatorzi\`eme%
\or quinzi\`eme%
\or seizi\`eme%
\or dix-septi\`eme%
\or dix-huiti\`eme%
\or dix-neuvi\`eme%
\fi
}
%    \end{macrocode}
% Seventies vary depending on dialect. Swiss:
%    \begin{macrocode}
\newcommand*{\@@seventiethsfrenchswiss}[1]{%
\ifcase#1\relax
\@tenthstring{7}%
\or
\@tenstring{7} \@andname\ \@unitthstring{1}%
\else
\@tenstring{7}-\@unitthstring{#1}%
\fi}
%    \end{macrocode}
% Eighties vary depending on dialect. Swiss:
%    \begin{macrocode}
\newcommand*{\@@eightiethsfrenchswiss}[1]{%
\ifcase#1\relax
\@tenthstring{8}%
\or
\@tenstring{8} \@andname\ \@unitthstring{1}%
\else
\@tenstring{8}-\@unitthstring{#1}%
\fi}
%    \end{macrocode}
% Nineties vary depending on dialect. Swiss:
%    \begin{macrocode}
\newcommand*{\@@ninetiethsfrenchswiss}[1]{%
\ifcase#1\relax
\@tenthstring{9}%
\or
\@tenstring{9} \@andname\ \@unitthstring{1}%
\else
\@tenstring{9}-\@unitthstring{#1}%
\fi}
%    \end{macrocode}
% French (as spoken in France) version:
%    \begin{macrocode}
\newcommand*{\@@seventiethsfrench}[1]{%
\ifnum#1=0\relax
\@tenstring{6}%
-%
\else
\@tenstring{6}%
\ \@andname\ 
\fi
\@teenthstring{#1}%
}
%    \end{macrocode}
% Eighties (as spoken in France):
%    \begin{macrocode}
\newcommand*{\@@eightiethsfrench}[1]{%
\ifnum#1>0\relax
\@unitstring{4}-\@tenstring{2}%
-\@unitthstring{#1}%
\else
\@unitstring{4}-\@tenthstring{2}%
\fi
}
%    \end{macrocode}
% Nineties (as spoken in France):
%    \begin{macrocode}
\newcommand*{\@@ninetiethsfrench}[1]{%
\@unitstring{4}-\@tenstring{2}-\@teenthstring{#1}%
}
%    \end{macrocode}
% As above, but with initial letter in upper case. Units:
%    \begin{macrocode}
\newcommand*{\@@Unitthstringfrench}[1]{%
\ifcase#1\relax
Zero%
\or Uni\`eme%
\or Deuxi\`eme%
\or Troisi\`eme%
\or Quatri\`eme%
\or Cinqui\`eme%
\or Sixi\`eme%
\or Septi\`eme%
\or Huiti\`eme%
\or Neuvi\`eme%
\fi
}
%    \end{macrocode}
% Tens (includes Belgian and Swiss variants):
%    \begin{macrocode}
\newcommand*{\@@Tenthstringfrench}[1]{%
\ifcase#1\relax
\or Dixi\`eme%
\or Vingti\`eme%
\or Trentri\`eme%
\or Quaranti\`eme%
\or Cinquanti\`eme%
\or Soixanti\`eme%
\or Septenti\`eme%
\or Huitanti\`eme%
\or Nonenti\`eme%
\fi
}
%    \end{macrocode}
% Teens:
%    \begin{macrocode}
\newcommand*{\@@Teenthstringfrench}[1]{%
\ifcase#1\relax
Dixi\`eme%
\or Onzi\`eme%
\or Douzi\`eme%
\or Treizi\`eme%
\or Quatorzi\`eme%
\or Quinzi\`eme%
\or Seizi\`eme%
\or Dix-Septi\`eme%
\or Dix-Huiti\`eme%
\or Dix-Neuvi\`eme%
\fi
}
%    \end{macrocode}
% Store textual representation of number (first argument) in given control
% sequence (second argument).
%    \begin{macrocode}
\newcommand*{\@@numberstringfrench}[2]{%
\ifnum#1>99999
\PackageError{fmtcount}{Out of range}%
{This macro only works for values less than 100000}%
\else
\ifnum#1<0
\PackageError{fmtcount}{Negative numbers not permitted}%
{This macro does not work for negative numbers, however
you can try typing "minus" first, and then pass the modulus of
this number}%
\fi
\fi
\def#2{}%
\@strctr=#1\relax \divide\@strctr by 1000\relax
\ifnum\@strctr>9\relax
% #1 is greater or equal to 10000
  \@tmpstrctr=\@strctr
  \divide\@strctr by 10\relax
  \ifnum\@strctr>1\relax
    \ifthenelse{\(\@strctr>6\)\and\(\@strctr<10\)}{%
      \@modulo{\@tmpstrctr}{10}%
      \ifnum\@strctr<8\relax
        \let\@@fc@numstr#2\relax
        \edef#2{\@@fc@numstr\@seventies{\@tmpstrctr}}%
      \else
        \ifnum\@strctr<9\relax
          \let\@@fc@numstr#2\relax
          \edef#2{\@@fc@numstr\@eighties{\@tmpstrctr}}%
        \else
          \ifnum\@strctr<10\relax
             \let\@@fc@numstr#2\relax
             \edef#2{\@@fc@numstr\@nineties{\@tmpstrctr}}%
          \fi
        \fi
      \fi
    }{%
      \let\@@fc@numstr#2\relax
      \edef#2{\@@fc@numstr\@tenstring{\@strctr}}%
      \@strctr=#1\relax
      \divide\@strctr by 1000\relax
      \@modulo{\@strctr}{10}%
      \ifnum\@strctr>0\relax
        \let\@@fc@numstr#2\relax
        \edef#2{\@@fc@numstr\ \@unitstring{\@strctr}}%
      \fi
    }%
  \else
    \@strctr=#1\relax
    \divide\@strctr by 1000
    \@modulo{\@strctr}{10}%
    \let\@@fc@numstr#2\relax
    \edef#2{\@@fc@numstr\@teenstring{\@strctr}}%
  \fi
  \let\@@fc@numstr#2\relax
  \edef#2{\@@fc@numstr\ \@thousand}%
\else
  \ifnum\@strctr>0\relax 
    \ifnum\@strctr>1\relax
      \let\@@fc@numstr#2\relax
      \edef#2{\@@fc@numstr\@unitstring{\@strctr}\ }%
    \fi
    \let\@@fc@numstr#2\relax
    \edef#2{\@@fc@numstr\@thousand}%
  \fi
\fi
\@strctr=#1\relax \@modulo{\@strctr}{1000}%
\divide\@strctr by 100
\ifnum\@strctr>0\relax
  \ifnum#1>1000\relax
    \let\@@fc@numstr#2\relax
    \edef#2{\@@fc@numstr\ }%
  \fi
  \@tmpstrctr=#1\relax
  \@modulo{\@tmpstrctr}{1000}\relax
  \ifnum\@tmpstrctr=100\relax
    \let\@@fc@numstr#2\relax
    \edef#2{\@@fc@numstr\@tenstring{10}}%
  \else
    \ifnum\@strctr>1\relax
      \let\@@fc@numstr#2\relax
      \edef#2{\@@fc@numstr\@unitstring{\@strctr}\ }%
    \fi
    \let\@@fc@numstr#2\relax
    \edef#2{\@@fc@numstr\@hundred}%
  \fi
\fi
\@strctr=#1\relax \@modulo{\@strctr}{100}%
%\@tmpstrctr=#1\relax
%\divide\@tmpstrctr by 100\relax
\ifnum#1>100\relax
  \ifnum\@strctr>0\relax
    \let\@@fc@numstr#2\relax
    \edef#2{\@@fc@numstr\ }%
  \else
    \ifnum\@tmpstrctr>0\relax
       \let\@@fc@numstr#2\relax
       \edef#2{\@@fc@numstr s}%
    \fi%
  \fi
\fi
\ifnum\@strctr>19\relax
  \@tmpstrctr=\@strctr
  \divide\@strctr by 10\relax
  \ifthenelse{\@strctr>6}{%
    \@modulo{\@tmpstrctr}{10}%
    \ifnum\@strctr<8\relax
      \let\@@fc@numstr#2\relax
      \edef#2{\@@fc@numstr\@seventies{\@tmpstrctr}}%
    \else
      \ifnum\@strctr<9\relax
        \let\@@fc@numstr#2\relax
        \edef#2{\@@fc@numstr\@eighties{\@tmpstrctr}}%
      \else
        \let\@@fc@numstr#2\relax
        \edef#2{\@@fc@numstr\@nineties{\@tmpstrctr}}%
      \fi
    \fi
  }{%
    \let\@@fc@numstr#2\relax
    \edef#2{\@@fc@numstr\@tenstring{\@strctr}}%
    \@strctr=#1\relax \@modulo{\@strctr}{10}%
    \ifnum\@strctr>0\relax
      \let\@@fc@numstr#2\relax
      \ifnum\@strctr=1\relax
         \edef#2{\@@fc@numstr\ \@andname\ }%
      \else
         \edef#2{\@@fc@numstr-}%
      \fi
      \let\@@fc@numstr#2\relax
      \edef#2{\@@fc@numstr\@unitstring{\@strctr}}%
    \fi
  }%
\else
  \ifnum\@strctr<10\relax
    \ifnum\@strctr=0\relax
      \ifnum#1<100\relax
        \let\@@fc@numstr#2\relax
        \edef#2{\@@fc@numstr\@unitstring{\@strctr}}%
      \fi
    \else%(>0,<10)
      \let\@@fc@numstr#2\relax
      \edef#2{\@@fc@numstr\@unitstring{\@strctr}}%
    \fi
  \else%>10
    \@modulo{\@strctr}{10}%
    \let\@@fc@numstr#2\relax
    \edef#2{\@@fc@numstr\@teenstring{\@strctr}}%
  \fi
\fi
}
%    \end{macrocode}
% Store textual representation of an ordinal (from number 
% specified in first argument) in given control
% sequence (second argument).
%    \begin{macrocode}
\newcommand*{\@@ordinalstringfrench}[2]{%
\ifnum#1>99999
\PackageError{fmtcount}{Out of range}%
{This macro only works for values less than 100000}%
\else
\ifnum#1<0
\PackageError{fmtcount}{Negative numbers not permitted}%
{This macro does not work for negative numbers, however
you can try typing "minus" first, and then pass the modulus of
this number}%
\fi
\fi
\def#2{}%
\@strctr=#1\relax \divide\@strctr by 1000\relax
\ifnum\@strctr>9
% #1 is greater or equal to 10000
  \@tmpstrctr=\@strctr
  \divide\@strctr by 10\relax
  \ifnum\@strctr>1\relax
    \ifthenelse{\@strctr>6}{%
      \@modulo{\@tmpstrctr}{10}%
      \ifnum\@strctr=7\relax
        \let\@@fc@ordstr#2\relax
        \edef#2{\@@fc@ordstr\@seventies{\@tmpstrctr}}%
      \else
        \ifnum\@strctr=8\relax
          \let\@@fc@ordstr#2\relax
          \edef#2{\@@fc@ordstr\@eighties{\@tmpstrctr}}%
        \else
          \let\@@fc@ordstr#2\relax
          \edef#2{\@@fc@ordstr\@nineties{\@tmpstrctr}}%
        \fi
      \fi
    }{%
      \let\@@fc@ordstr#2\relax
      \edef#2{\@@fc@ordstr\@tenstring{\@strctr}}%
      \@strctr=#1\relax
      \divide\@strctr by 1000\relax
      \@modulo{\@strctr}{10}%
      \ifnum\@strctr=1\relax
         \let\@@fc@ordstr#2\relax
         \edef#2{\@@fc@ordstr\ \@andname}%
      \fi
      \ifnum\@strctr>0\relax
         \let\@@fc@ordstr#2\relax
         \edef#2{\@@fc@ordstr\ \@unitstring{\@strctr}}%
      \fi
    }%
  \else
    \@strctr=#1\relax
    \divide\@strctr by 1000\relax
    \@modulo{\@strctr}{10}%
    \let\@@fc@ordstr#2\relax
    \edef#2{\@@fc@ordstr\@teenstring{\@strctr}}%
  \fi
  \@strctr=#1\relax \@modulo{\@strctr}{1000}%
  \ifnum\@strctr=0\relax
    \let\@@fc@ordstr#2\relax
    \edef#2{\@@fc@ordstr\ \@thousandth}%
  \else
    \let\@@fc@ordstr#2\relax
    \edef#2{\@@fc@ordstr\ \@thousand}%
  \fi
\else
  \ifnum\@strctr>0\relax
    \let\@@fc@ordstr#2\relax
    \edef#2{\@@fc@ordstr\@unitstring{\@strctr}}%
    \@strctr=#1\relax \@modulo{\@strctr}{1000}%
    \ifnum\@strctr=0\relax
      \let\@@fc@ordstr#2\relax
      \edef#2{\@@fc@ordstr\ \@thousandth}%
    \else
      \let\@@fc@ordstr#2\relax
      \edef#2{\@@fc@ordstr\ \@thousand}%
    \fi
  \fi
\fi
\@strctr=#1\relax \@modulo{\@strctr}{1000}%
\divide\@strctr by 100\relax
\ifnum\@strctr>0\relax
  \ifnum#1>1000\relax
    \let\@@fc@ordstr#2\relax
    \edef#2{\@@fc@ordstr\ }%
  \fi
  \let\@@fc@ordstr#2\relax
  \edef#2{\@@fc@ordstr\@unitstring{\@strctr}}%
  \@strctr=#1\relax \@modulo{\@strctr}{100}%
  \let\@@fc@ordstr#2\relax
  \ifnum\@strctr=0\relax
    \edef#2{\@@fc@ordstr\ \@hundredth}%
  \else
    \edef#2{\@@fc@ordstr\ \@hundred}%
  \fi
\fi
\@tmpstrctr=\@strctr
\@strctr=#1\relax \@modulo{\@strctr}{100}%
\ifnum#1>100\relax
  \ifnum\@strctr>0\relax
    \let\@@fc@ordstr#2\relax
    \edef#2{\@@fc@ordstr\ \@andname\ }%
  \fi
\fi
\ifnum\@strctr>19\relax
  \@tmpstrctr=\@strctr
  \divide\@strctr by 10\relax
  \@modulo{\@tmpstrctr}{10}%
  \ifthenelse{\@strctr>6}{%
    \ifnum\@strctr=7\relax
      \let\@@fc@ordstr#2\relax
      \edef#2{\@@fc@ordstr\@seventieths{\@tmpstrctr}}%
    \else
      \ifnum\@strctr=8\relax
        \let\@@fc@ordstr#2\relax
        \edef#2{\@@fc@ordstr\@eightieths{\@tmpstrctr}}%
      \else
        \let\@@fc@ordstr#2\relax
        \edef#2{\@@fc@ordstr\@ninetieths{\@tmpstrctr}}%
      \fi
    \fi
  }{%
    \ifnum\@tmpstrctr=0\relax
      \let\@@fc@ordstr#2\relax
      \edef#2{\@@fc@ordstr\@tenthstring{\@strctr}}%
    \else 
      \let\@@fc@ordstr#2\relax
      \edef#2{\@@fc@ordstr\@tenstring{\@strctr}}%
    \fi
    \@strctr=#1\relax \@modulo{\@strctr}{10}%
    \ifnum\@strctr=1\relax
      \let\@@fc@ordstr#2\relax
      \edef#2{\@@fc@ordstr\ \@andname}%
    \fi
    \ifnum\@strctr>0\relax
      \let\@@fc@ordstr#2\relax
      \edef#2{\@@fc@ordstr\ \@unitthstring{\@strctr}}%
    \fi
  }%
\else
  \ifnum\@strctr<10\relax
    \ifnum\@strctr=0\relax
      \ifnum#1<100\relax
        \let\@@fc@ordstr#2\relax
        \edef#2{\@@fc@ordstr\@unitthstring{\@strctr}}%
      \fi
    \else
      \let\@@fc@ordstr#2\relax
      \edef#2{\@@fc@ordstr\@unitthstring{\@strctr}}%
    \fi
  \else
    \@modulo{\@strctr}{10}%
    \let\@@fc@ordstr#2\relax
    \edef#2{\@@fc@ordstr\@teenthstring{\@strctr}}%
  \fi
\fi
}
%    \end{macrocode}
%\iffalse
%    \begin{macrocode}
%</fc-french.def>
%    \end{macrocode}
%\fi
%\iffalse
%    \begin{macrocode}
%<*fc-german.def>
%    \end{macrocode}
%\fi
% \subsection{fc-german.def}
% German definitions (thank you to K. H. Fricke for supplying
% this information)
%    \begin{macrocode}
\ProvidesFile{fc-german.def}[2007/06/14]
%    \end{macrocode}
% Define macro that converts a number or count register (first
% argument) to an ordinal, and stores the result in the
% second argument, which must be a control sequence.
% Masculine:
%    \begin{macrocode}
\newcommand{\@ordinalMgerman}[2]{%
\edef#2{\number#1\relax.}}
%    \end{macrocode}
% Feminine:
%    \begin{macrocode}
\newcommand{\@ordinalFgerman}[2]{%
\edef#2{\number#1\relax.}}
%    \end{macrocode}
% Neuter:
%    \begin{macrocode}
\newcommand{\@ordinalNgerman}[2]{%
\edef#2{\number#1\relax.}}
%    \end{macrocode}
% Convert a number to text. The easiest way to do this is to
% break it up into units, tens and teens.
% Units (argument must be a number from 0 to 9, 1 on its own (eins)
% is dealt with separately):
%    \begin{macrocode}
\newcommand{\@@unitstringgerman}[1]{%
\ifcase#1%
null%
\or ein%
\or zwei%
\or drei%
\or vier%
\or f\"unf%
\or sechs%
\or sieben%
\or acht%
\or neun%
\fi
}
%    \end{macrocode}
% Tens (argument must go from 1 to 10):
%    \begin{macrocode}
\newcommand{\@@tenstringgerman}[1]{%
\ifcase#1%
\or zehn%
\or zwanzig%
\or drei{\ss}ig%
\or vierzig%
\or f\"unfzig%
\or sechzig%
\or siebzig%
\or achtzig%
\or neunzig%
\or einhundert%
\fi
}
%    \end{macrocode}
% |\einhundert| is set to |einhundert| by default, user can
% redefine this command to just |hundert| if required, similarly
% for |\eintausend|.
%    \begin{macrocode}
\providecommand*{\einhundert}{einhundert}
\providecommand*{\eintausend}{eintausend}
%    \end{macrocode}
% Teens:
%    \begin{macrocode}
\newcommand{\@@teenstringgerman}[1]{%
\ifcase#1%
zehn%
\or elf%
\or zw\"olf%
\or dreizehn%
\or vierzehn%
\or f\"unfzehn%
\or sechzehn%
\or siebzehn%
\or achtzehn%
\or neunzehn%
\fi
}
%    \end{macrocode}
% The results are stored in the second argument, but doesn't 
% display anything.
%    \begin{macrocode}
\DeclareRobustCommand{\@numberstringMgerman}[2]{%
\let\@unitstring=\@@unitstringgerman
\let\@teenstring=\@@teenstringgerman
\let\@tenstring=\@@tenstringgerman
\@@numberstringgerman{#1}{#2}}
%    \end{macrocode}
% Feminine and neuter forms:
%    \begin{macrocode}
\let\@numberstringFgerman=\@numberstringMgerman
\let\@numberstringNgerman=\@numberstringMgerman
%    \end{macrocode}
% As above, but initial letters in upper case:
%    \begin{macrocode}
\DeclareRobustCommand{\@NumberstringMgerman}[2]{%
\@numberstringMgerman{#1}{\@@num@str}%
\edef#2{\noexpand\MakeUppercase\@@num@str}}
%    \end{macrocode}
% Feminine and neuter form:
%    \begin{macrocode}
\let\@NumberstringFgerman=\@NumberstringMgerman
\let\@NumberstringNgerman=\@NumberstringMgerman
%    \end{macrocode}
% As above, but for ordinals.
%    \begin{macrocode}
\DeclareRobustCommand{\@ordinalstringMgerman}[2]{%
\let\@unitthstring=\@@unitthstringMgerman
\let\@teenthstring=\@@teenthstringMgerman
\let\@tenthstring=\@@tenthstringMgerman
\let\@unitstring=\@@unitstringgerman
\let\@teenstring=\@@teenstringgerman
\let\@tenstring=\@@tenstringgerman
\def\@thousandth{tausendster}%
\def\@hundredth{hundertster}%
\@@ordinalstringgerman{#1}{#2}}
%    \end{macrocode}
% Feminine form:
%    \begin{macrocode}
\DeclareRobustCommand{\@ordinalstringFgerman}[2]{%
\let\@unitthstring=\@@unitthstringFgerman
\let\@teenthstring=\@@teenthstringFgerman
\let\@tenthstring=\@@tenthstringFgerman
\let\@unitstring=\@@unitstringgerman
\let\@teenstring=\@@teenstringgerman
\let\@tenstring=\@@tenstringgerman
\def\@thousandth{tausendste}%
\def\@hundredth{hundertste}%
\@@ordinalstringgerman{#1}{#2}}
%    \end{macrocode}
% Neuter form:
%    \begin{macrocode}
\DeclareRobustCommand{\@ordinalstringNgerman}[2]{%
\let\@unitthstring=\@@unitthstringNgerman
\let\@teenthstring=\@@teenthstringNgerman
\let\@tenthstring=\@@tenthstringNgerman
\let\@unitstring=\@@unitstringgerman
\let\@teenstring=\@@teenstringgerman
\let\@tenstring=\@@tenstringgerman
\def\@thousandth{tausendstes}%
\def\@hundredth{hunderstes}%
\@@ordinalstringgerman{#1}{#2}}
%    \end{macrocode}
% As above, but with initial letters in upper case.
%    \begin{macrocode}
\DeclareRobustCommand{\@OrdinalstringMgerman}[2]{%
\@ordinalstringMgerman{#1}{\@@num@str}%
\edef#2{\protect\MakeUppercase\@@num@str}}
%    \end{macrocode}
% Feminine form:
%    \begin{macrocode}
\DeclareRobustCommand{\@OrdinalstringFgerman}[2]{%
\@ordinalstringFgerman{#1}{\@@num@str}%
\edef#2{\protect\MakeUppercase\@@num@str}}
%    \end{macrocode}
% Neuter form:
%    \begin{macrocode}
\DeclareRobustCommand{\@OrdinalstringNgerman}[2]{%
\@ordinalstringNgerman{#1}{\@@num@str}%
\edef#2{\protect\MakeUppercase\@@num@str}}
%    \end{macrocode}
% Code for converting numbers into textual ordinals. As before,
% it is easier to split it into units, tens and teens.
% Units:
%    \begin{macrocode}
\newcommand{\@@unitthstringMgerman}[1]{%
\ifcase#1%
nullter%
\or erster%
\or zweiter%
\or dritter%
\or vierter%
\or f\"unter%
\or sechster%
\or siebter%
\or achter%
\or neunter%
\fi
}
%    \end{macrocode}
% Tens:
%    \begin{macrocode}
\newcommand{\@@tenthstringMgerman}[1]{%
\ifcase#1%
\or zehnter%
\or zwanzigster%
\or drei{\ss}igster%
\or vierzigster%
\or f\"unfzigster%
\or sechzigster%
\or siebzigster%
\or achtzigster%
\or neunzigster%
\fi
}
%    \end{macrocode}
% Teens:
%    \begin{macrocode}
\newcommand{\@@teenthstringMgerman}[1]{%
\ifcase#1%
zehnter%
\or elfter%
\or zw\"olfter%
\or dreizehnter%
\or vierzehnter%
\or f\"unfzehnter%
\or sechzehnter%
\or siebzehnter%
\or achtzehnter%
\or neunzehnter%
\fi
}
%    \end{macrocode}
% Units (feminine):
%    \begin{macrocode}
\newcommand{\@@unitthstringFgerman}[1]{%
\ifcase#1%
nullte%
\or erste%
\or zweite%
\or dritte%
\or vierte%
\or f\"unfte%
\or sechste%
\or siebte%
\or achte%
\or neunte%
\fi
}
%    \end{macrocode}
% Tens (feminine):
%    \begin{macrocode}
\newcommand{\@@tenthstringFgerman}[1]{%
\ifcase#1%
\or zehnte%
\or zwanzigste%
\or drei{\ss}igste%
\or vierzigste%
\or f\"unfzigste%
\or sechzigste%
\or siebzigste%
\or achtzigste%
\or neunzigste%
\fi
}
%    \end{macrocode}
% Teens (feminine)
%    \begin{macrocode}
\newcommand{\@@teenthstringFgerman}[1]{%
\ifcase#1%
zehnte%
\or elfte%
\or zw\"olfte%
\or dreizehnte%
\or vierzehnte%
\or f\"unfzehnte%
\or sechzehnte%
\or siebzehnte%
\or achtzehnte%
\or neunzehnte%
\fi
}
%    \end{macrocode}
% Units (neuter):
%    \begin{macrocode}
\newcommand{\@@unitthstringNgerman}[1]{%
\ifcase#1%
nulltes%
\or erstes%
\or zweites%
\or drittes%
\or viertes%
\or f\"unte%
\or sechstes%
\or siebtes%
\or achtes%
\or neuntes%
\fi
}
%    \end{macrocode}
% Tens (neuter):
%    \begin{macrocode}
\newcommand{\@@tenthstringNgerman}[1]{%
\ifcase#1%
\or zehntes%
\or zwanzigstes%
\or drei{\ss}igstes%
\or vierzigstes%
\or f\"unfzigstes%
\or sechzigstes%
\or siebzigstes%
\or achtzigstes%
\or neunzigstes%
\fi
}
%    \end{macrocode}
% Teens (neuter)
%    \begin{macrocode}
\newcommand{\@@teenthstringNgerman}[1]{%
\ifcase#1%
zehntes%
\or elftes%
\or zw\"olftes%
\or dreizehntes%
\or vierzehntes%
\or f\"unfzehntes%
\or sechzehntes%
\or siebzehntes%
\or achtzehntes%
\or neunzehntes%
\fi
}
%    \end{macrocode}
% This appends the results to |#2| for number |#2| (in range 0 to 100.)
% null and eins are dealt with separately in |\@@numberstringgerman|.
%    \begin{macrocode}
\newcommand{\@@numberunderhundredgerman}[2]{%
\ifnum#1<10\relax
  \ifnum#1>0\relax
    \let\@@fc@numstr#2\relax
    \edef#2{\@@fc@numstr\@unitstring{#1}}%
  \fi
\else
  \@tmpstrctr=#1\relax
  \@modulo{\@tmpstrctr}{10}%
  \ifnum#1<20\relax
    \let\@@fc@numstr#2\relax
    \edef#2{\@@fc@numstr\@teenstring{\@tmpstrctr}}%
  \else
    \ifnum\@tmpstrctr=0\relax
    \else
      \let\@@fc@numstr#2\relax
      \edef#2{\@@fc@numstr\@unitstring{\@tmpstrctr}und}%
    \fi
    \@tmpstrctr=#1\relax
    \divide\@tmpstrctr by 10\relax
    \let\@@fc@numstr#2\relax
    \edef#2{\@@fc@numstr\@tenstring{\@tmpstrctr}}%
  \fi
\fi
}
%    \end{macrocode}
% This stores the results in the second argument 
% (which must be a control
% sequence), but it doesn't display anything.
%    \begin{macrocode}
\newcommand{\@@numberstringgerman}[2]{%
\ifnum#1>99999\relax
  \PackageError{fmtcount}{Out of range}%
  {This macro only works for values less than 100000}%
\else
  \ifnum#1<0\relax
    \PackageError{fmtcount}{Negative numbers not permitted}%
    {This macro does not work for negative numbers, however
    you can try typing "minus" first, and then pass the modulus of
    this number}%
  \fi
\fi
\def#2{}%
\@strctr=#1\relax \divide\@strctr by 1000\relax
\ifnum\@strctr>1\relax
% #1 is >= 2000, \@strctr now contains the number of thousands
\@@numberunderhundredgerman{\@strctr}{#2}%
  \let\@@fc@numstr#2\relax
  \edef#2{\@@fc@numstr tausend}%
\else
% #1 lies in range [1000,1999]
  \ifnum\@strctr=1\relax
    \let\@@fc@numstr#2\relax
    \edef#2{\@@fc@numstr\eintausend}%
  \fi
\fi
\@strctr=#1\relax
\@modulo{\@strctr}{1000}%
\divide\@strctr by 100\relax
\ifnum\@strctr>1\relax
% now dealing with number in range [200,999]
  \let\@@fc@numstr#2\relax
  \edef#2{\@@fc@numstr\@unitstring{\@strctr}hundert}%
\else
   \ifnum\@strctr=1\relax
% dealing with number in range [100,199]
     \ifnum#1>1000\relax
% if orginal number > 1000, use einhundert
        \let\@@fc@numstr#2\relax
        \edef#2{\@@fc@numstr einhundert}%
     \else
% otherwise use \einhundert
        \let\@@fc@numstr#2\relax
        \edef#2{\@@fc@numstr\einhundert}%
      \fi
   \fi
\fi
\@strctr=#1\relax
\@modulo{\@strctr}{100}%
\ifnum#1=0\relax
  \def#2{null}%
\else
  \ifnum\@strctr=1\relax
    \let\@@fc@numstr#2\relax
    \edef#2{\@@fc@numstr eins}%
  \else
    \@@numberunderhundredgerman{\@strctr}{#2}%
  \fi
\fi
}
%    \end{macrocode}
% As above, but for ordinals
%    \begin{macrocode}
\newcommand{\@@numberunderhundredthgerman}[2]{%
\ifnum#1<10\relax
 \let\@@fc@numstr#2\relax
 \edef#2{\@@fc@numstr\@unitthstring{#1}}%
\else
  \@tmpstrctr=#1\relax
  \@modulo{\@tmpstrctr}{10}%
  \ifnum#1<20\relax
    \let\@@fc@numstr#2\relax
    \edef#2{\@@fc@numstr\@teenthstring{\@tmpstrctr}}%
  \else
    \ifnum\@tmpstrctr=0\relax
    \else
      \let\@@fc@numstr#2\relax
      \edef#2{\@@fc@numstr\@unitstring{\@tmpstrctr}und}%
    \fi
    \@tmpstrctr=#1\relax
    \divide\@tmpstrctr by 10\relax
    \let\@@fc@numstr#2\relax
    \edef#2{\@@fc@numstr\@tenthstring{\@tmpstrctr}}%
  \fi
\fi
}
%    \end{macrocode}
%    \begin{macrocode}
\newcommand{\@@ordinalstringgerman}[2]{%
\ifnum#1>99999\relax
  \PackageError{fmtcount}{Out of range}%
  {This macro only works for values less than 100000}%
\else
  \ifnum#1<0\relax
    \PackageError{fmtcount}{Negative numbers not permitted}%
    {This macro does not work for negative numbers, however
    you can try typing "minus" first, and then pass the modulus of
    this number}%
  \fi
\fi
\def#2{}%
\@strctr=#1\relax \divide\@strctr by 1000\relax
\ifnum\@strctr>1\relax
% #1 is >= 2000, \@strctr now contains the number of thousands
\@@numberunderhundredgerman{\@strctr}{#2}%
  \let\@@fc@numstr#2\relax
  % is that it, or is there more?
  \@tmpstrctr=#1\relax \@modulo{\@tmpstrctr}{1000}%
  \ifnum\@tmpstrctr=0\relax
    \edef#2{\@@fc@numstr\@thousandth}%
  \else
    \edef#2{\@@fc@numstr tausend}%
  \fi
\else
% #1 lies in range [1000,1999]
  \ifnum\@strctr=1\relax
    \ifnum#1=1000\relax
      \let\@@fc@numstr#2\relax
      \edef#2{\@@fc@numstr\@thousandth}%
    \else
      \let\@@fc@numstr#2\relax
      \edef#2{\@@fc@numstr\eintausend}%
    \fi
  \fi
\fi
\@strctr=#1\relax
\@modulo{\@strctr}{1000}%
\divide\@strctr by 100\relax
\ifnum\@strctr>1\relax
% now dealing with number in range [200,999]
  \let\@@fc@numstr#2\relax
  % is that it, or is there more?
  \@tmpstrctr=#1\relax \@modulo{\@tmpstrctr}{100}%
  \ifnum\@tmpstrctr=0\relax
     \ifnum\@strctr=1\relax
       \edef#2{\@@fc@numstr\@hundredth}%
     \else
       \edef#2{\@@fc@numstr\@unitstring{\@strctr}\@hundredth}%
     \fi
  \else
     \edef#2{\@@fc@numstr\@unitstring{\@strctr}hundert}%
  \fi
\else
   \ifnum\@strctr=1\relax
% dealing with number in range [100,199]
% is that it, or is there more?
     \@tmpstrctr=#1\relax \@modulo{\@tmpstrctr}{100}%
     \ifnum\@tmpstrctr=0\relax
        \let\@@fc@numstr#2\relax
        \edef#2{\@@fc@numstr\@hundredth}%
     \else
     \ifnum#1>1000\relax
        \let\@@fc@numstr#2\relax
        \edef#2{\@@fc@numstr einhundert}%
     \else
        \let\@@fc@numstr#2\relax
        \edef#2{\@@fc@numstr\einhundert}%
     \fi
     \fi
   \fi
\fi
\@strctr=#1\relax
\@modulo{\@strctr}{100}%
\ifthenelse{\@strctr=0 \and #1>0}{}{%
\@@numberunderhundredthgerman{\@strctr}{#2}%
}%
}
%    \end{macrocode}
% Set |ngerman| to be equivalent to |german|. Is it okay to do
% this? (I don't know the difference between the two.)
%    \begin{macrocode}
\let\@ordinalMngerman=\@ordinalMgerman
\let\@ordinalFngerman=\@ordinalFgerman
\let\@ordinalNngerman=\@ordinalNgerman
\let\@numberstringMngerman=\@numberstringMgerman
\let\@numberstringFngerman=\@numberstringFgerman
\let\@numberstringNngerman=\@numberstringNgerman
\let\@NumberstringMngerman=\@NumberstringMgerman
\let\@NumberstringFngerman=\@NumberstringFgerman
\let\@NumberstringNngerman=\@NumberstringNgerman
\let\@ordinalstringMngerman=\@ordinalstringMgerman
\let\@ordinalstringFngerman=\@ordinalstringFgerman
\let\@ordinalstringNngerman=\@ordinalstringNgerman
\let\@OrdinalstringMngerman=\@OrdinalstringMgerman
\let\@OrdinalstringFngerman=\@OrdinalstringFgerman
\let\@OrdinalstringNngerman=\@OrdinalstringNgerman
%    \end{macrocode}
%\iffalse
%    \begin{macrocode}
%</fc-german.def>
%    \end{macrocode}
%\fi
%\iffalse
%    \begin{macrocode}
%<*fc-portuges.def>
%    \end{macrocode}
%\fi
% \subsection{fc-portuges.def}
% Portuguse definitions
%    \begin{macrocode}
\ProvidesFile{fc-portuges.def}[2007/05/26]
%    \end{macrocode}
% Define macro that converts a number or count register (first
% argument) to an ordinal, and stores the result in the second
% argument, which should be a control sequence. Masculine:
%    \begin{macrocode}
\newcommand*{\@ordinalMportuges}[2]{%
\ifnum#1=0\relax
  \edef#2{\number#1}%
\else
  \edef#2{\number#1\relax\noexpand\fmtord{o}}%
\fi}
%    \end{macrocode}
% Feminine:
%    \begin{macrocode}
\newcommand*{\@ordinalFportuges}[2]{%
\ifnum#1=0\relax
  \edef#2{\number#1}%
\else
  \edef#2{\number#1\relax\noexpand\fmtord{a}}%
\fi}
%    \end{macrocode}
% Make neuter same as masculine:
%    \begin{macrocode}
\let\@ordinalNportuges\@ordinalMportuges
%    \end{macrocode}
% Convert a number to a textual representation. To make it easier,
% split it up into units, tens, teens and hundreds. Units (argument must
% be a number from 0 to 9):
%    \begin{macrocode}
\newcommand*{\@@unitstringportuges}[1]{%
\ifcase#1\relax
zero%
\or um%
\or dois%
\or tr\^es%
\or quatro%
\or cinco%
\or seis%
\or sete%
\or oito%
\or nove%
\fi
}
%   \end{macrocode}
% As above, but for feminine:
%   \begin{macrocode}
\newcommand*{\@@unitstringFportuges}[1]{%
\ifcase#1\relax
zero%
\or uma%
\or duas%
\or tr\^es%
\or quatro%
\or cinco%
\or seis%
\or sete%
\or oito%
\or nove%
\fi
}
%    \end{macrocode}
% Tens (argument must be a number from 0 to 10):
%    \begin{macrocode}
\newcommand*{\@@tenstringportuges}[1]{%
\ifcase#1\relax
\or dez%
\or vinte%
\or trinta%
\or quarenta%
\or cinq\"uenta%
\or sessenta%
\or setenta%
\or oitenta%
\or noventa%
\or cem%
\fi
}
%    \end{macrocode}
% Teens (argument must be a number from 0 to 9):
%    \begin{macrocode}
\newcommand*{\@@teenstringportuges}[1]{%
\ifcase#1\relax
dez%
\or onze%
\or doze%
\or treze%
\or quatorze%
\or quinze%
\or dezesseis%
\or dezessete%
\or dezoito%
\or dezenove%
\fi
}
%    \end{macrocode}
% Hundreds:
%    \begin{macrocode}
\newcommand*{\@@hundredstringportuges}[1]{%
\ifcase#1\relax
\or cento%
\or duzentos%
\or trezentos%
\or quatrocentos%
\or quinhentos%
\or seiscentos%
\or setecentos%
\or oitocentos%
\or novecentos%
\fi}
%    \end{macrocode}
% Hundreds (feminine):
%    \begin{macrocode}
\newcommand*{\@@hundredstringFportuges}[1]{%
\ifcase#1\relax
\or cento%
\or duzentas%
\or trezentas%
\or quatrocentas%
\or quinhentas%
\or seiscentas%
\or setecentas%
\or oitocentas%
\or novecentas%
\fi}
%    \end{macrocode}
% Units (initial letter in upper case):
%    \begin{macrocode}
\newcommand*{\@@Unitstringportuges}[1]{%
\ifcase#1\relax
Zero%
\or Um%
\or Dois%
\or Tr\^es%
\or Quatro%
\or Cinco%
\or Seis%
\or Sete%
\or Oito%
\or Nove%
\fi
}
%    \end{macrocode}
% As above, but feminine:
%    \begin{macrocode}
\newcommand*{\@@UnitstringFportuges}[1]{%
\ifcase#1\relax
Zera%
\or Uma%
\or Duas%
\or Tr\^es%
\or Quatro%
\or Cinco%
\or Seis%
\or Sete%
\or Oito%
\or Nove%
\fi
}
%    \end{macrocode}
% Tens (with initial letter in upper case):
%    \begin{macrocode}
\newcommand*{\@@Tenstringportuges}[1]{%
\ifcase#1\relax
\or Dez%
\or Vinte%
\or Trinta%
\or Quarenta%
\or Cinq\"uenta%
\or Sessenta%
\or Setenta%
\or Oitenta%
\or Noventa%
\or Cem%
\fi
}
%    \end{macrocode}
% Teens (with initial letter in upper case):
%    \begin{macrocode}
\newcommand*{\@@Teenstringportuges}[1]{%
\ifcase#1\relax
Dez%
\or Onze%
\or Doze%
\or Treze%
\or Quatorze%
\or Quinze%
\or Dezesseis%
\or Dezessete%
\or Dezoito%
\or Dezenove%
\fi
}
%    \end{macrocode}
% Hundreds (with initial letter in upper case):
%    \begin{macrocode}
\newcommand*{\@@Hundredstringportuges}[1]{%
\ifcase#1\relax
\or Cento%
\or Duzentos%
\or Trezentos%
\or Quatrocentos%
\or Quinhentos%
\or Seiscentos%
\or Setecentos%
\or Oitocentos%
\or Novecentos%
\fi}
%    \end{macrocode}
% As above, but feminine:
%    \begin{macrocode}
\newcommand*{\@@HundredstringFportuges}[1]{%
\ifcase#1\relax
\or Cento%
\or Duzentas%
\or Trezentas%
\or Quatrocentas%
\or Quinhentas%
\or Seiscentas%
\or Setecentas%
\or Oitocentas%
\or Novecentas%
\fi}
%    \end{macrocode}
% This has changed in version 1.08, so that it now stores
% the result in the second argument, but doesn't display
% anything. Since it only affects internal macros, it shouldn't
% affect documents created with older versions. (These internal
% macros are not meant for use in documents.)
%    \begin{macrocode}
\DeclareRobustCommand{\@numberstringMportuges}[2]{%
\let\@unitstring=\@@unitstringportuges
\let\@teenstring=\@@teenstringportuges
\let\@tenstring=\@@tenstringportuges
\let\@hundredstring=\@@hundredstringportuges
\def\@hundred{cem}\def\@thousand{mil}%
\def\@andname{e}%
\@@numberstringportuges{#1}{#2}}
%    \end{macrocode}
% As above, but feminine form:
%    \begin{macrocode}
\DeclareRobustCommand{\@numberstringFportuges}[2]{%
\let\@unitstring=\@@unitstringFportuges
\let\@teenstring=\@@teenstringportuges
\let\@tenstring=\@@tenstringportuges
\let\@hundredstring=\@@hundredstringFportuges
\def\@hundred{cem}\def\@thousand{mil}%
\def\@andname{e}%
\@@numberstringportuges{#1}{#2}}
%    \end{macrocode}
% Make neuter same as masculine:
%    \begin{macrocode}
\let\@numberstringNportuges\@numberstringMportuges
%    \end{macrocode}
% As above, but initial letters in upper case:
%    \begin{macrocode}
\DeclareRobustCommand{\@NumberstringMportuges}[2]{%
\let\@unitstring=\@@Unitstringportuges
\let\@teenstring=\@@Teenstringportuges
\let\@tenstring=\@@Tenstringportuges
\let\@hundredstring=\@@Hundredstringportuges
\def\@hundred{Cem}\def\@thousand{Mil}%
\def\@andname{e}%
\@@numberstringportuges{#1}{#2}}
%    \end{macrocode}
% As above, but feminine form:
%    \begin{macrocode}
\DeclareRobustCommand{\@NumberstringFportuges}[2]{%
\let\@unitstring=\@@UnitstringFportuges
\let\@teenstring=\@@Teenstringportuges
\let\@tenstring=\@@Tenstringportuges
\let\@hundredstring=\@@HundredstringFportuges
\def\@hundred{Cem}\def\@thousand{Mil}%
\def\@andname{e}%
\@@numberstringportuges{#1}{#2}}
%    \end{macrocode}
% Make neuter same as masculine:
%    \begin{macrocode}
\let\@NumberstringNportuges\@NumberstringMportuges
%    \end{macrocode}
% As above, but for ordinals.
%    \begin{macrocode}
\DeclareRobustCommand{\@ordinalstringMportuges}[2]{%
\let\@unitthstring=\@@unitthstringportuges
\let\@unitstring=\@@unitstringportuges
\let\@teenthstring=\@@teenthstringportuges
\let\@tenthstring=\@@tenthstringportuges
\let\@hundredthstring=\@@hundredthstringportuges
\def\@thousandth{mil\'esimo}%
\@@ordinalstringportuges{#1}{#2}}
%    \end{macrocode}
% Feminine form:
%    \begin{macrocode}
\DeclareRobustCommand{\@ordinalstringFportuges}[2]{%
\let\@unitthstring=\@@unitthstringFportuges
\let\@unitstring=\@@unitstringFportuges
\let\@teenthstring=\@@teenthstringportuges
\let\@tenthstring=\@@tenthstringFportuges
\let\@hundredthstring=\@@hundredthstringFportuges
\def\@thousandth{mil\'esima}%
\@@ordinalstringportuges{#1}{#2}}
%    \end{macrocode}
% Make neuter same as masculine:
%    \begin{macrocode}
\let\@ordinalstringNportuges\@ordinalstringMportuges
%    \end{macrocode}
% As above, but initial letters in upper case (masculine):
%    \begin{macrocode}
\DeclareRobustCommand{\@OrdinalstringMportuges}[2]{%
\let\@unitthstring=\@@Unitthstringportuges
\let\@unitstring=\@@Unitstringportuges
\let\@teenthstring=\@@teenthstringportuges
\let\@tenthstring=\@@Tenthstringportuges
\let\@hundredthstring=\@@Hundredthstringportuges
\def\@thousandth{Mil\'esimo}%
\@@ordinalstringportuges{#1}{#2}}
%    \end{macrocode}
% Feminine form:
%    \begin{macrocode}
\DeclareRobustCommand{\@OrdinalstringFportuges}[2]{%
\let\@unitthstring=\@@UnitthstringFportuges
\let\@unitstring=\@@UnitstringFportuges
\let\@teenthstring=\@@teenthstringportuges
\let\@tenthstring=\@@TenthstringFportuges
\let\@hundredthstring=\@@HundredthstringFportuges
\def\@thousandth{Mil\'esima}%
\@@ordinalstringportuges{#1}{#2}}
%    \end{macrocode}
% Make neuter same as masculine:
%    \begin{macrocode}
\let\@OrdinalstringNportuges\@OrdinalstringMportuges
%    \end{macrocode}
% In order to do the ordinals, split into units, teens, tens
% and hundreds. Units:
%    \begin{macrocode}
\newcommand*{\@@unitthstringportuges}[1]{%
\ifcase#1\relax
zero%
\or primeiro%
\or segundo%
\or terceiro%
\or quarto%
\or quinto%
\or sexto%
\or s\'etimo%
\or oitavo%
\or nono%
\fi
}
%    \end{macrocode}
% Tens:
%    \begin{macrocode}
\newcommand*{\@@tenthstringportuges}[1]{%
\ifcase#1\relax
\or d\'ecimo%
\or vig\'esimo%
\or trig\'esimo%
\or quadrag\'esimo%
\or q\"uinquag\'esimo%
\or sexag\'esimo%
\or setuag\'esimo%
\or octog\'esimo%
\or nonag\'esimo%
\fi
}
%    \end{macrocode}
% Teens:
%    \begin{macrocode}
\newcommand*{\@@teenthstringportuges}[1]{%
\@tenthstring{1}%
\ifnum#1>0\relax
-\@unitthstring{#1}%
\fi}
%    \end{macrocode}
% Hundreds:
%    \begin{macrocode}
\newcommand*{\@@hundredthstringportuges}[1]{%
\ifcase#1\relax
\or cent\'esimo%
\or ducent\'esimo%
\or trecent\'esimo%
\or quadringent\'esimo%
\or q\"uingent\'esimo%
\or seiscent\'esimo%
\or setingent\'esimo%
\or octingent\'esimo%
\or nongent\'esimo%
\fi}
%    \end{macrocode}
% Units (feminine):
%    \begin{macrocode}
\newcommand*{\@@unitthstringFportuges}[1]{%
\ifcase#1\relax
zero%
\or primeira%
\or segunda%
\or terceira%
\or quarta%
\or quinta%
\or sexta%
\or s\'etima%
\or oitava%
\or nona%
\fi
}
%    \end{macrocode}
% Tens (feminine):
%    \begin{macrocode}
\newcommand*{\@@tenthstringFportuges}[1]{%
\ifcase#1\relax
\or d\'ecima%
\or vig\'esima%
\or trig\'esima%
\or quadrag\'esima%
\or q\"uinquag\'esima%
\or sexag\'esima%
\or setuag\'esima%
\or octog\'esima%
\or nonag\'esima%
\fi
}
%    \end{macrocode}
% Hundreds (feminine):
%    \begin{macrocode}
\newcommand*{\@@hundredthstringFportuges}[1]{%
\ifcase#1\relax
\or cent\'esima%
\or ducent\'esima%
\or trecent\'esima%
\or quadringent\'esima%
\or q\"uingent\'esima%
\or seiscent\'esima%
\or setingent\'esima%
\or octingent\'esima%
\or nongent\'esima%
\fi}
%    \end{macrocode}
% As above, but with initial letter in upper case. Units:
%    \begin{macrocode}
\newcommand*{\@@Unitthstringportuges}[1]{%
\ifcase#1\relax
Zero%
\or Primeiro%
\or Segundo%
\or Terceiro%
\or Quarto%
\or Quinto%
\or Sexto%
\or S\'etimo%
\or Oitavo%
\or Nono%
\fi
}
%    \end{macrocode}
% Tens:
%    \begin{macrocode}
\newcommand*{\@@Tenthstringportuges}[1]{%
\ifcase#1\relax
\or D\'ecimo%
\or Vig\'esimo%
\or Trig\'esimo%
\or Quadrag\'esimo%
\or Q\"uinquag\'esimo%
\or Sexag\'esimo%
\or Setuag\'esimo%
\or Octog\'esimo%
\or Nonag\'esimo%
\fi
}
%    \end{macrocode}
% Hundreds:
%    \begin{macrocode}
\newcommand*{\@@Hundredthstringportuges}[1]{%
\ifcase#1\relax
\or Cent\'esimo%
\or Ducent\'esimo%
\or Trecent\'esimo%
\or Quadringent\'esimo%
\or Q\"uingent\'esimo%
\or Seiscent\'esimo%
\or Setingent\'esimo%
\or Octingent\'esimo%
\or Nongent\'esimo%
\fi}
%    \end{macrocode}
% As above, but feminine. Units:
%    \begin{macrocode}
\newcommand*{\@@UnitthstringFportuges}[1]{%
\ifcase#1\relax
Zera%
\or Primeira%
\or Segunda%
\or Terceira%
\or Quarta%
\or Quinta%
\or Sexta%
\or S\'etima%
\or Oitava%
\or Nona%
\fi
}
%    \end{macrocode}
% Tens (feminine);
%    \begin{macrocode}
\newcommand*{\@@TenthstringFportuges}[1]{%
\ifcase#1\relax
\or D\'ecima%
\or Vig\'esima%
\or Trig\'esima%
\or Quadrag\'esima%
\or Q\"uinquag\'esima%
\or Sexag\'esima%
\or Setuag\'esima%
\or Octog\'esima%
\or Nonag\'esima%
\fi
}
%    \end{macrocode}
% Hundreds (feminine):
%    \begin{macrocode}
\newcommand*{\@@HundredthstringFportuges}[1]{%
\ifcase#1\relax
\or Cent\'esima%
\or Ducent\'esima%
\or Trecent\'esima%
\or Quadringent\'esima%
\or Q\"uingent\'esima%
\or Seiscent\'esima%
\or Setingent\'esima%
\or Octingent\'esima%
\or Nongent\'esima%
\fi}
%    \end{macrocode}
% This has changed in version 1.09, so that it now stores
% the result in the second argument (a control sequence), but it
% doesn't display anything. Since it only affects internal macros,
% it shouldn't affect documents created with older versions.
% (These internal macros are not meant for use in documents.)
%    \begin{macrocode}
\newcommand*{\@@numberstringportuges}[2]{%
\ifnum#1>99999
\PackageError{fmtcount}{Out of range}%
{This macro only works for values less than 100000}%
\else
\ifnum#1<0
\PackageError{fmtcount}{Negative numbers not permitted}%
{This macro does not work for negative numbers, however
you can try typing "minus" first, and then pass the modulus of
this number}%
\fi
\fi
\def#2{}%
\@strctr=#1\relax \divide\@strctr by 1000\relax
\ifnum\@strctr>9
% #1 is greater or equal to 10000
  \divide\@strctr by 10
  \ifnum\@strctr>1\relax
    \let\@@fc@numstr#2\relax
    \edef#2{\@@fc@numstr\@tenstring{\@strctr}}%
    \@strctr=#1 \divide\@strctr by 1000\relax
    \@modulo{\@strctr}{10}%
    \ifnum\@strctr>0
      \ifnum\@strctr=1\relax
        \let\@@fc@numstr#2\relax
        \edef#2{\@@fc@numstr\ \@andname}%
      \fi
      \let\@@fc@numstr#2\relax
      \edef#2{\@@fc@numstr\ \@unitstring{\@strctr}}%
    \fi
  \else
    \@strctr=#1\relax
    \divide\@strctr by 1000\relax
    \@modulo{\@strctr}{10}%
    \let\@@fc@numstr#2\relax
    \edef#2{\@@fc@numstr\@teenstring{\@strctr}}%
  \fi
  \let\@@fc@numstr#2\relax
  \edef#2{\@@fc@numstr\ \@thousand}%
\else
  \ifnum\@strctr>0\relax 
    \ifnum\@strctr>1\relax
      \let\@@fc@numstr#2\relax
      \edef#2{\@@fc@numstr\@unitstring{\@strctr}\ }%
    \fi
    \let\@@fc@numstr#2\relax
    \edef#2{\@@fc@numstr\@thousand}%
  \fi
\fi
\@strctr=#1\relax \@modulo{\@strctr}{1000}%
\divide\@strctr by 100\relax
\ifnum\@strctr>0\relax
  \ifnum#1>1000 \relax
    \let\@@fc@numstr#2\relax
    \edef#2{\@@fc@numstr\ }%
  \fi
  \@tmpstrctr=#1\relax
  \@modulo{\@tmpstrctr}{1000}%
  \let\@@fc@numstr#2\relax
  \ifnum\@tmpstrctr=100\relax
    \edef#2{\@@fc@numstr\@tenstring{10}}%
  \else
    \edef#2{\@@fc@numstr\@hundredstring{\@strctr}}%
  \fi%
\fi
\@strctr=#1\relax \@modulo{\@strctr}{100}%
\ifnum#1>100\relax
  \ifnum\@strctr>0\relax
    \let\@@fc@numstr#2\relax
    \edef#2{\@@fc@numstr\ \@andname\ }%
  \fi
\fi
\ifnum\@strctr>19\relax
  \divide\@strctr by 10\relax
  \let\@@fc@numstr#2\relax
  \edef#2{\@@fc@numstr\@tenstring{\@strctr}}%
  \@strctr=#1\relax \@modulo{\@strctr}{10}%
  \ifnum\@strctr>0
    \ifnum\@strctr=1\relax
      \let\@@fc@numstr#2\relax
      \edef#2{\@@fc@numstr\ \@andname}%
    \else
      \ifnum#1>100\relax
        \let\@@fc@numstr#2\relax
        \edef#2{\@@fc@numstr\ \@andname}%
      \fi
    \fi 
    \let\@@fc@numstr#2\relax
    \edef#2{\@@fc@numstr\ \@unitstring{\@strctr}}%
  \fi
\else
  \ifnum\@strctr<10\relax
    \ifnum\@strctr=0\relax
      \ifnum#1<100\relax
        \let\@@fc@numstr#2\relax
        \edef#2{\@@fc@numstr\@unitstring{\@strctr}}%
      \fi
    \else%(>0,<10)
      \let\@@fc@numstr#2\relax
      \edef#2{\@@fc@numstr\@unitstring{\@strctr}}%
    \fi
  \else%>10
    \@modulo{\@strctr}{10}%
    \let\@@fc@numstr#2\relax
    \edef#2{\@@fc@numstr\@teenstring{\@strctr}}%
  \fi
\fi
}
%    \end{macrocode}
% As above, but for ordinals.
%    \begin{macrocode}
\newcommand*{\@@ordinalstringportuges}[2]{%
\@strctr=#1\relax
\ifnum#1>99999
\PackageError{fmtcount}{Out of range}%
{This macro only works for values less than 100000}%
\else
\ifnum#1<0
\PackageError{fmtcount}{Negative numbers not permitted}%
{This macro does not work for negative numbers, however
you can try typing "minus" first, and then pass the modulus of
this number}%
\else
\def#2{}%
\ifnum\@strctr>999\relax
  \divide\@strctr by 1000\relax
  \ifnum\@strctr>1\relax
    \ifnum\@strctr>9\relax
      \@tmpstrctr=\@strctr
      \ifnum\@strctr<20
        \@modulo{\@tmpstrctr}{10}%
        \let\@@fc@ordstr#2\relax
        \edef#2{\@@fc@ordstr\@teenthstring{\@tmpstrctr}}%
      \else
        \divide\@tmpstrctr by 10\relax
        \let\@@fc@ordstr#2\relax
        \edef#2{\@@fc@ordstr\@tenthstring{\@tmpstrctr}}%
        \@tmpstrctr=\@strctr
        \@modulo{\@tmpstrctr}{10}%
        \ifnum\@tmpstrctr>0\relax
          \let\@@fc@ordstr#2\relax
          \edef#2{\@@fc@ordstr\@unitthstring{\@tmpstrctr}}%
        \fi
      \fi
    \else
      \let\@@fc@ordstr#2\relax
      \edef#2{\@@fc@ordstr\@unitstring{\@strctr}}%
    \fi
  \fi
  \let\@@fc@ordstr#2\relax
  \edef#2{\@@fc@ordstr\@thousandth}%
\fi
\@strctr=#1\relax
\@modulo{\@strctr}{1000}%
\ifnum\@strctr>99\relax
  \@tmpstrctr=\@strctr
  \divide\@tmpstrctr by 100\relax
  \ifnum#1>1000\relax
    \let\@@fc@ordstr#2\relax
    \edef#2{\@@fc@ordstr-}%
  \fi
  \let\@@fc@ordstr#2\relax
  \edef#2{\@@fc@ordstr\@hundredthstring{\@tmpstrctr}}%
\fi
\@modulo{\@strctr}{100}%
\ifnum#1>99\relax
  \ifnum\@strctr>0\relax
    \let\@@fc@ordstr#2\relax
    \edef#2{\@@fc@ordstr-}%
  \fi
\fi
\ifnum\@strctr>9\relax
  \@tmpstrctr=\@strctr
  \divide\@tmpstrctr by 10\relax
  \let\@@fc@ordstr#2\relax
  \edef#2{\@@fc@ordstr\@tenthstring{\@tmpstrctr}}%
  \@tmpstrctr=\@strctr
  \@modulo{\@tmpstrctr}{10}%
  \ifnum\@tmpstrctr>0\relax
    \let\@@fc@ordstr#2\relax
    \edef#2{\@@fc@ordstr-\@unitthstring{\@tmpstrctr}}%
  \fi
\else
  \ifnum\@strctr=0\relax
    \ifnum#1=0\relax
      \let\@@fc@ordstr#2\relax
      \edef#2{\@@fc@ordstr\@unitstring{0}}%
    \fi
  \else
    \let\@@fc@ordstr#2\relax
    \edef#2{\@@fc@ordstr\@unitthstring{\@strctr}}%
  \fi
\fi
\fi
\fi
}
%    \end{macrocode}
%\iffalse
%    \begin{macrocode}
%</fc-portuges.def>
%    \end{macrocode}
%\fi
%\iffalse
%    \begin{macrocode}
%<*fc-spanish.def>
%    \end{macrocode}
%\fi
% \subsection{fc-spanish.def}
% Spanish definitions
%    \begin{macrocode}
\ProvidesFile{fc-spanish.def}[2007/05/26]
%    \end{macrocode}
% Define macro that converts a number or count register (first
% argument) to an ordinal, and stores the result in the
% second argument, which must be a control sequence.
% Masculine:
%    \begin{macrocode}
\newcommand{\@ordinalMspanish}[2]{%
\edef#2{\number#1\relax\noexpand\fmtord{o}}}
%    \end{macrocode}
% Feminine:
%    \begin{macrocode}
\newcommand{\@ordinalFspanish}[2]{%
\edef#2{\number#1\relax\noexpand\fmtord{a}}}
%    \end{macrocode}
% Make neuter same as masculine:
%    \begin{macrocode}
\let\@ordinalNspanish\@ordinalMspanish
%    \end{macrocode}
% Convert a number to text. The easiest way to do this is to
% break it up into units, tens, teens, twenties and hundreds.
% Units (argument must be a number from 0 to 9):
%    \begin{macrocode}
\newcommand{\@@unitstringspanish}[1]{%
\ifcase#1\relax
cero%
\or uno%
\or dos%
\or tres%
\or cuatro%
\or cinco%
\or seis%
\or siete%
\or ocho%
\or nueve%
\fi
}
%    \end{macrocode}
% Feminine:
%    \begin{macrocode}
\newcommand{\@@unitstringFspanish}[1]{%
\ifcase#1\relax
cera%
\or una%
\or dos%
\or tres%
\or cuatro%
\or cinco%
\or seis%
\or siete%
\or ocho%
\or nueve%
\fi
}
%    \end{macrocode}
% Tens (argument must go from 1 to 10):
%    \begin{macrocode}
\newcommand{\@@tenstringspanish}[1]{%
\ifcase#1\relax
\or diez%
\or viente%
\or treinta%
\or cuarenta%
\or cincuenta%
\or sesenta%
\or setenta%
\or ochenta%
\or noventa%
\or cien%
\fi
}
%    \end{macrocode}
% Teens:
%    \begin{macrocode}
\newcommand{\@@teenstringspanish}[1]{%
\ifcase#1\relax
diez%
\or once%
\or doce%
\or trece%
\or catorce%
\or quince%
\or diecis\'eis%
\or diecisiete%
\or dieciocho%
\or diecinueve%
\fi
}
%    \end{macrocode}
% Twenties:
%    \begin{macrocode}
\newcommand{\@@twentystringspanish}[1]{%
\ifcase#1\relax
veinte%
\or veintiuno%
\or veintid\'os%
\or veintitr\'es%
\or veinticuatro%
\or veinticinco%
\or veintis\'eis%
\or veintisiete%
\or veintiocho%
\or veintinueve%
\fi}
%    \end{macrocode}
% Feminine form:
%    \begin{macrocode}
\newcommand{\@@twentystringFspanish}[1]{%
\ifcase#1\relax
veinte%
\or veintiuna%
\or veintid\'os%
\or veintitr\'es%
\or veinticuatro%
\or veinticinco%
\or veintis\'eis%
\or veintisiete%
\or veintiocho%
\or veintinueve%
\fi}
%    \end{macrocode}
% Hundreds:
%    \begin{macrocode}
\newcommand{\@@hundredstringspanish}[1]{%
\ifcase#1\relax
\or ciento%
\or doscientos%
\or trescientos%
\or cuatrocientos%
\or quinientos%
\or seiscientos%
\or setecientos%
\or ochocientos%
\or novecientos%
\fi}
%    \end{macrocode}
% Feminine form:
%    \begin{macrocode}
\newcommand{\@@hundredstringFspanish}[1]{%
\ifcase#1\relax
\or cienta%
\or doscientas%
\or trescientas%
\or cuatrocientas%
\or quinientas%
\or seiscientas%
\or setecientas%
\or ochocientas%
\or novecientas%
\fi}
%    \end{macrocode}
% As above, but with initial letter uppercase:
%    \begin{macrocode}
\newcommand{\@@Unitstringspanish}[1]{%
\ifcase#1\relax
Cero%
\or Uno%
\or Dos%
\or Tres%
\or Cuatro%
\or Cinco%
\or Seis%
\or Siete%
\or Ocho%
\or Nueve%
\fi
}
%    \end{macrocode}
% Feminine form:
%    \begin{macrocode}
\newcommand{\@@UnitstringFspanish}[1]{%
\ifcase#1\relax
Cera%
\or Una%
\or Dos%
\or Tres%
\or Cuatro%
\or Cinco%
\or Seis%
\or Siete%
\or Ocho%
\or Nueve%
\fi
}
%    \end{macrocode}
% Tens:
%    \begin{macrocode}
\newcommand{\@@Tenstringspanish}[1]{%
\ifcase#1\relax
\or Diez%
\or Viente%
\or Treinta%
\or Cuarenta%
\or Cincuenta%
\or Sesenta%
\or Setenta%
\or Ochenta%
\or Noventa%
\or Cien%
\fi
}
%    \end{macrocode}
% Teens:
%    \begin{macrocode}
\newcommand{\@@Teenstringspanish}[1]{%
\ifcase#1\relax
Diez%
\or Once%
\or Doce%
\or Trece%
\or Catorce%
\or Quince%
\or Diecis\'eis%
\or Diecisiete%
\or Dieciocho%
\or Diecinueve%
\fi
}
%    \end{macrocode}
% Twenties:
%    \begin{macrocode}
\newcommand{\@@Twentystringspanish}[1]{%
\ifcase#1\relax
Veinte%
\or Veintiuno%
\or Veintid\'os%
\or Veintitr\'es%
\or Veinticuatro%
\or Veinticinco%
\or Veintis\'eis%
\or Veintisiete%
\or Veintiocho%
\or Veintinueve%
\fi}
%    \end{macrocode}
% Feminine form:
%    \begin{macrocode}
\newcommand{\@@TwentystringFspanish}[1]{%
\ifcase#1\relax
Veinte%
\or Veintiuna%
\or Veintid\'os%
\or Veintitr\'es%
\or Veinticuatro%
\or Veinticinco%
\or Veintis\'eis%
\or Veintisiete%
\or Veintiocho%
\or Veintinueve%
\fi}
%    \end{macrocode}
% Hundreds:
%    \begin{macrocode}
\newcommand{\@@Hundredstringspanish}[1]{%
\ifcase#1\relax
\or Ciento%
\or Doscientos%
\or Trescientos%
\or Cuatrocientos%
\or Quinientos%
\or Seiscientos%
\or Setecientos%
\or Ochocientos%
\or Novecientos%
\fi}
%    \end{macrocode}
% Feminine form:
%    \begin{macrocode}
\newcommand{\@@HundredstringFspanish}[1]{%
\ifcase#1\relax
\or Cienta%
\or Doscientas%
\or Trescientas%
\or Cuatrocientas%
\or Quinientas%
\or Seiscientas%
\or Setecientas%
\or Ochocientas%
\or Novecientas%
\fi}
%    \end{macrocode}
% This has changed in version 1.09, so that it now stores the
% result in the second argument, but doesn't display anything.
% Since it only affects internal macros, it shouldn't affect
% documents created with older versions. (These internal macros
% are not meant for use in documents.)
%    \begin{macrocode}
\DeclareRobustCommand{\@numberstringMspanish}[2]{%
\let\@unitstring=\@@unitstringspanish
\let\@teenstring=\@@teenstringspanish
\let\@tenstring=\@@tenstringspanish
\let\@twentystring=\@@twentystringspanish
\let\@hundredstring=\@@hundredstringspanish
\def\@hundred{cien}\def\@thousand{mil}%
\def\@andname{y}%
\@@numberstringspanish{#1}{#2}}
%    \end{macrocode}
% Feminine form:
%    \begin{macrocode}
\DeclareRobustCommand{\@numberstringFspanish}[2]{%
\let\@unitstring=\@@unitstringFspanish
\let\@teenstring=\@@teenstringspanish
\let\@tenstring=\@@tenstringspanish
\let\@twentystring=\@@twentystringFspanish
\let\@hundredstring=\@@hundredstringFspanish
\def\@hundred{cien}\def\@thousand{mil}%
\def\@andname{y}%
\@@numberstringspanish{#1}{#2}}
%    \end{macrocode}
% Make neuter same as masculine:
%    \begin{macrocode}
\let\@numberstringNspanish\@numberstringMspanish
%    \end{macrocode}
% As above, but initial letters in upper case:
%    \begin{macrocode}
\DeclareRobustCommand{\@NumberstringMspanish}[2]{%
\let\@unitstring=\@@Unitstringspanish
\let\@teenstring=\@@Teenstringspanish
\let\@tenstring=\@@Tenstringspanish
\let\@twentystring=\@@Twentystringspanish
\let\@hundredstring=\@@Hundredstringspanish
\def\@andname{y}%
\def\@hundred{Cien}\def\@thousand{Mil}%
\@@numberstringspanish{#1}{#2}}
%    \end{macrocode}
% Feminine form:
%    \begin{macrocode}
\DeclareRobustCommand{\@NumberstringFspanish}[2]{%
\let\@unitstring=\@@UnitstringFspanish
\let\@teenstring=\@@Teenstringspanish
\let\@tenstring=\@@Tenstringspanish
\let\@twentystring=\@@TwentystringFspanish
\let\@hundredstring=\@@HundredstringFspanish
\def\@andname{y}%
\def\@hundred{Cien}\def\@thousand{Mil}%
\@@numberstringspanish{#1}{#2}}
%    \end{macrocode}
% Make neuter same as masculine:
%    \begin{macrocode}
\let\@NumberstringNspanish\@NumberstringMspanish
%    \end{macrocode}
% As above, but for ordinals.
%    \begin{macrocode}
\DeclareRobustCommand{\@ordinalstringMspanish}[2]{%
\let\@unitthstring=\@@unitthstringspanish
\let\@unitstring=\@@unitstringspanish
\let\@teenthstring=\@@teenthstringspanish
\let\@tenthstring=\@@tenthstringspanish
\let\@hundredthstring=\@@hundredthstringspanish
\def\@thousandth{mil\'esimo}%
\@@ordinalstringspanish{#1}{#2}}
%    \end{macrocode}
% Feminine form:
%    \begin{macrocode}
\DeclareRobustCommand{\@ordinalstringFspanish}[2]{%
\let\@unitthstring=\@@unitthstringFspanish
\let\@unitstring=\@@unitstringFspanish
\let\@teenthstring=\@@teenthstringFspanish
\let\@tenthstring=\@@tenthstringFspanish
\let\@hundredthstring=\@@hundredthstringFspanish
\def\@thousandth{mil\'esima}%
\@@ordinalstringspanish{#1}{#2}}
%    \end{macrocode}
% Make neuter same as masculine:
%    \begin{macrocode}
\let\@ordinalstringNspanish\@ordinalstringMspanish
%    \end{macrocode}
% As above, but with initial letters in upper case.
%    \begin{macrocode}
\DeclareRobustCommand{\@OrdinalstringMspanish}[2]{%
\let\@unitthstring=\@@Unitthstringspanish
\let\@unitstring=\@@Unitstringspanish
\let\@teenthstring=\@@Teenthstringspanish
\let\@tenthstring=\@@Tenthstringspanish
\let\@hundredthstring=\@@Hundredthstringspanish
\def\@thousandth{Mil\'esimo}%
\@@ordinalstringspanish{#1}{#2}}
%    \end{macrocode}
% Feminine form:
%    \begin{macrocode}
\DeclareRobustCommand{\@OrdinalstringFspanish}[2]{%
\let\@unitthstring=\@@UnitthstringFspanish
\let\@unitstring=\@@UnitstringFspanish
\let\@teenthstring=\@@TeenthstringFspanish
\let\@tenthstring=\@@TenthstringFspanish
\let\@hundredthstring=\@@HundredthstringFspanish
\def\@thousandth{Mil\'esima}%
\@@ordinalstringspanish{#1}{#2}}
%    \end{macrocode}
% Make neuter same as masculine:
%    \begin{macrocode}
\let\@OrdinalstringNspanish\@OrdinalstringMspanish
%    \end{macrocode}
% Code for convert numbers into textual ordinals. As before,
% it is easier to split it into units, tens, teens and hundreds.
% Units:
%    \begin{macrocode}
\newcommand{\@@unitthstringspanish}[1]{%
\ifcase#1\relax
cero%
\or primero%
\or segundo%
\or tercero%
\or cuarto%
\or quinto%
\or sexto%
\or s\'eptimo%
\or octavo%
\or noveno%
\fi
}
%    \end{macrocode}
% Tens:
%    \begin{macrocode}
\newcommand{\@@tenthstringspanish}[1]{%
\ifcase#1\relax
\or d\'ecimo%
\or vig\'esimo%
\or trig\'esimo%
\or cuadrag\'esimo%
\or quincuag\'esimo%
\or sexag\'esimo%
\or septuag\'esimo%
\or octog\'esimo%
\or nonag\'esimo%
\fi
}
%    \end{macrocode}
% Teens:
%    \begin{macrocode}
\newcommand{\@@teenthstringspanish}[1]{%
\ifcase#1\relax
d\'ecimo%
\or und\'ecimo%
\or duod\'ecimo%
\or decimotercero%
\or decimocuarto%
\or decimoquinto%
\or decimosexto%
\or decimos\'eptimo%
\or decimoctavo%
\or decimonoveno%
\fi
}
%    \end{macrocode}
% Hundreds:
%    \begin{macrocode}
\newcommand{\@@hundredthstringspanish}[1]{%
\ifcase#1\relax
\or cent\'esimo%
\or ducent\'esimo%
\or tricent\'esimo%
\or cuadringent\'esimo%
\or quingent\'esimo%
\or sexcent\'esimo%
\or septing\'esimo%
\or octingent\'esimo%
\or noningent\'esimo%
\fi}
%    \end{macrocode}
% Units (feminine):
%    \begin{macrocode}
\newcommand{\@@unitthstringFspanish}[1]{%
\ifcase#1\relax
cera%
\or primera%
\or segunda%
\or tercera%
\or cuarta%
\or quinta%
\or sexta%
\or s\'eptima%
\or octava%
\or novena%
\fi
}
%    \end{macrocode}
% Tens (feminine):
%    \begin{macrocode}
\newcommand{\@@tenthstringFspanish}[1]{%
\ifcase#1\relax
\or d\'ecima%
\or vig\'esima%
\or trig\'esima%
\or cuadrag\'esima%
\or quincuag\'esima%
\or sexag\'esima%
\or septuag\'esima%
\or octog\'esima%
\or nonag\'esima%
\fi
}
%    \end{macrocode}
% Teens (feminine)
%    \begin{macrocode}
\newcommand{\@@teenthstringFspanish}[1]{%
\ifcase#1\relax
d\'ecima%
\or und\'ecima%
\or duod\'ecima%
\or decimotercera%
\or decimocuarta%
\or decimoquinta%
\or decimosexta%
\or decimos\'eptima%
\or decimoctava%
\or decimonovena%
\fi
}
%    \end{macrocode}
% Hundreds (feminine)
%    \begin{macrocode}
\newcommand{\@@hundredthstringFspanish}[1]{%
\ifcase#1\relax
\or cent\'esima%
\or ducent\'esima%
\or tricent\'esima%
\or cuadringent\'esima%
\or quingent\'esima%
\or sexcent\'esima%
\or septing\'esima%
\or octingent\'esima%
\or noningent\'esima%
\fi}
%    \end{macrocode}
% As above, but with initial letters in upper case
%    \begin{macrocode}
\newcommand{\@@Unitthstringspanish}[1]{%
\ifcase#1\relax
Cero%
\or Primero%
\or Segundo%
\or Tercero%
\or Cuarto%
\or Quinto%
\or Sexto%
\or S\'eptimo%
\or Octavo%
\or Noveno%
\fi
}
%    \end{macrocode}
% Tens:
%    \begin{macrocode}
\newcommand{\@@Tenthstringspanish}[1]{%
\ifcase#1\relax
\or D\'ecimo%
\or Vig\'esimo%
\or Trig\'esimo%
\or Cuadrag\'esimo%
\or Quincuag\'esimo%
\or Sexag\'esimo%
\or Septuag\'esimo%
\or Octog\'esimo%
\or Nonag\'esimo%
\fi
}
%    \end{macrocode}
% Teens:
%    \begin{macrocode}
\newcommand{\@@Teenthstringspanish}[1]{%
\ifcase#1\relax
D\'ecimo%
\or Und\'ecimo%
\or Duod\'ecimo%
\or Decimotercero%
\or Decimocuarto%
\or Decimoquinto%
\or Decimosexto%
\or Decimos\'eptimo%
\or Decimoctavo%
\or Decimonoveno%
\fi
}
%    \end{macrocode}
% Hundreds
%    \begin{macrocode}
\newcommand{\@@Hundredthstringspanish}[1]{%
\ifcase#1\relax
\or Cent\'esimo%
\or Ducent\'esimo%
\or Tricent\'esimo%
\or Cuadringent\'esimo%
\or Quingent\'esimo%
\or Sexcent\'esimo%
\or Septing\'esimo%
\or Octingent\'esimo%
\or Noningent\'esimo%
\fi}
%    \end{macrocode}
% As above, but feminine.
%    \begin{macrocode}
\newcommand{\@@UnitthstringFspanish}[1]{%
\ifcase#1\relax
Cera%
\or Primera%
\or Segunda%
\or Tercera%
\or Cuarta%
\or Quinta%
\or Sexta%
\or S\'eptima%
\or Octava%
\or Novena%
\fi
}
%    \end{macrocode}
% Tens (feminine)
%    \begin{macrocode}
\newcommand{\@@TenthstringFspanish}[1]{%
\ifcase#1\relax
\or D\'ecima%
\or Vig\'esima%
\or Trig\'esima%
\or Cuadrag\'esima%
\or Quincuag\'esima%
\or Sexag\'esima%
\or Septuag\'esima%
\or Octog\'esima%
\or Nonag\'esima%
\fi
}
%    \end{macrocode}
% Teens (feminine):
%    \begin{macrocode}
\newcommand{\@@TeenthstringFspanish}[1]{%
\ifcase#1\relax
D\'ecima%
\or Und\'ecima%
\or Duod\'ecima%
\or Decimotercera%
\or Decimocuarta%
\or Decimoquinta%
\or Decimosexta%
\or Decimos\'eptima%
\or Decimoctava%
\or Decimonovena%
\fi
}
%    \end{macrocode}
% Hundreds (feminine):
%    \begin{macrocode}
\newcommand{\@@HundredthstringFspanish}[1]{%
\ifcase#1\relax
\or Cent\'esima%
\or Ducent\'esima%
\or Tricent\'esima%
\or Cuadringent\'esima%
\or Quingent\'esima%
\or Sexcent\'esima%
\or Septing\'esima%
\or Octingent\'esima%
\or Noningent\'esima%
\fi}

%    \end{macrocode}
% This has changed in version 1.09, so that it now stores the
% results in the second argument (which must be a control
% sequence), but it doesn't display anything. Since it only
% affects internal macros, it shouldn't affect documnets created
% with older versions. (These internal macros are not meant for
% use in documents.)
%    \begin{macrocode}
\newcommand{\@@numberstringspanish}[2]{%
\ifnum#1>99999
\PackageError{fmtcount}{Out of range}%
{This macro only works for values less than 100000}%
\else
\ifnum#1<0
\PackageError{fmtcount}{Negative numbers not permitted}%
{This macro does not work for negative numbers, however
you can try typing "minus" first, and then pass the modulus of
this number}%
\fi
\fi
\def#2{}%
\@strctr=#1\relax \divide\@strctr by 1000\relax
\ifnum\@strctr>9
% #1 is greater or equal to 10000
  \divide\@strctr by 10
  \ifnum\@strctr>1
    \let\@@fc@numstr#2\relax
    \edef#2{\@@fc@numstr\@tenstring{\@strctr}}%
    \@strctr=#1 \divide\@strctr by 1000\relax
    \@modulo{\@strctr}{10}%
    \ifnum\@strctr>0\relax
       \let\@@fc@numstr#2\relax
       \edef#2{\@@fc@numstr\ \@andname\ \@unitstring{\@strctr}}%
    \fi
  \else
    \@strctr=#1\relax
    \divide\@strctr by 1000\relax
    \@modulo{\@strctr}{10}%
    \let\@@fc@numstr#2\relax
    \edef#2{\@@fc@numstr\@teenstring{\@strctr}}%
  \fi
  \let\@@fc@numstr#2\relax
  \edef#2{\@@fc@numstr\ \@thousand}%
\else
  \ifnum\@strctr>0\relax 
    \ifnum\@strctr>1\relax
       \let\@@fc@numstr#2\relax
       \edef#2{\@@fc@numstr\@unitstring{\@strctr}\ }%
    \fi
    \let\@@fc@numstr#2\relax
    \edef#2{\@@fc@numstr\@thousand}%
  \fi
\fi
\@strctr=#1\relax \@modulo{\@strctr}{1000}%
\divide\@strctr by 100\relax
\ifnum\@strctr>0\relax
  \ifnum#1>1000\relax
    \let\@@fc@numstr#2\relax
    \edef#2{\@@fc@numstr\ }%
  \fi
  \@tmpstrctr=#1\relax
  \@modulo{\@tmpstrctr}{1000}%
  \ifnum\@tmpstrctr=100\relax
    \let\@@fc@numstr#2\relax
    \edef#2{\@@fc@numstr\@tenstring{10}}%
  \else
    \let\@@fc@numstr#2\relax
    \edef#2{\@@fc@numstr\@hundredstring{\@strctr}}%
  \fi
\fi
\@strctr=#1\relax \@modulo{\@strctr}{100}%
\ifnum#1>100\relax
  \ifnum\@strctr>0\relax
    \let\@@fc@numstr#2\relax
    \edef#2{\@@fc@numstr\ \@andname\ }%
  \fi
\fi
\ifnum\@strctr>29\relax
  \divide\@strctr by 10\relax
  \let\@@fc@numstr#2\relax
  \edef#2{\@@fc@numstr\@tenstring{\@strctr}}%
  \@strctr=#1\relax \@modulo{\@strctr}{10}%
  \ifnum\@strctr>0\relax
    \let\@@fc@numstr#2\relax
    \edef#2{\@@fc@numstr\ \@andname\ \@unitstring{\@strctr}}%
  \fi
\else
  \ifnum\@strctr<10\relax
    \ifnum\@strctr=0\relax
      \ifnum#1<100\relax
        \let\@@fc@numstr#2\relax
        \edef#2{\@@fc@numstr\@unitstring{\@strctr}}%
      \fi
    \else
      \let\@@fc@numstr#2\relax
      \edef#2{\@@fc@numstr\@unitstring{\@strctr}}%
    \fi
  \else
    \ifnum\@strctr>19\relax
      \@modulo{\@strctr}{10}%
      \let\@@fc@numstr#2\relax
      \edef#2{\@@fc@numstr\@twentystring{\@strctr}}%
    \else
      \@modulo{\@strctr}{10}%
      \let\@@fc@numstr#2\relax
      \edef#2{\@@fc@numstr\@teenstring{\@strctr}}%
    \fi
  \fi
\fi
}
%    \end{macrocode}
% As above, but for ordinals
%    \begin{macrocode}
\newcommand{\@@ordinalstringspanish}[2]{%
\@strctr=#1\relax
\ifnum#1>99999
\PackageError{fmtcount}{Out of range}%
{This macro only works for values less than 100000}%
\else
\ifnum#1<0
\PackageError{fmtcount}{Negative numbers not permitted}%
{This macro does not work for negative numbers, however
you can try typing "minus" first, and then pass the modulus of
this number}%
\else
\def#2{}%
\ifnum\@strctr>999\relax
  \divide\@strctr by 1000\relax
  \ifnum\@strctr>1\relax
    \ifnum\@strctr>9\relax
      \@tmpstrctr=\@strctr
      \ifnum\@strctr<20
        \@modulo{\@tmpstrctr}{10}%
        \let\@@fc@ordstr#2\relax
        \edef#2{\@@fc@ordstr\@teenthstring{\@tmpstrctr}}%
      \else
        \divide\@tmpstrctr by 10\relax
        \let\@@fc@ordstr#2\relax
        \edef#2{\@@fc@ordstr\@tenthstring{\@tmpstrctr}}%
        \@tmpstrctr=\@strctr
        \@modulo{\@tmpstrctr}{10}%
        \ifnum\@tmpstrctr>0\relax
          \let\@@fc@ordstr#2\relax
          \edef#2{\@@fc@ordstr\@unitthstring{\@tmpstrctr}}%
        \fi
      \fi
    \else
       \let\@@fc@ordstr#2\relax
       \edef#2{\@@fc@ordstr\@unitstring{\@strctr}}%
    \fi
  \fi
  \let\@@fc@ordstr#2\relax
  \edef#2{\@@fc@ordstr\@thousandth}%
\fi
\@strctr=#1\relax
\@modulo{\@strctr}{1000}%
\ifnum\@strctr>99\relax
  \@tmpstrctr=\@strctr
  \divide\@tmpstrctr by 100\relax
  \ifnum#1>1000\relax
    \let\@@fc@ordstr#2\relax
    \edef#2{\@@fc@ordstr\ }%
  \fi
  \let\@@fc@ordstr#2\relax
  \edef#2{\@@fc@ordstr\@hundredthstring{\@tmpstrctr}}%
\fi
\@modulo{\@strctr}{100}%
\ifnum#1>99\relax
  \ifnum\@strctr>0\relax
    \let\@@fc@ordstr#2\relax
    \edef#2{\@@fc@ordstr\ }%
  \fi
\fi
\ifnum\@strctr>19\relax
  \@tmpstrctr=\@strctr
  \divide\@tmpstrctr by 10\relax
  \let\@@fc@ordstr#2\relax
  \edef#2{\@@fc@ordstr\@tenthstring{\@tmpstrctr}}%
  \@tmpstrctr=\@strctr
  \@modulo{\@tmpstrctr}{10}%
  \ifnum\@tmpstrctr>0\relax
    \let\@@fc@ordstr#2\relax
    \edef#2{\@@fc@ordstr\ \@unitthstring{\@tmpstrctr}}%
  \fi
\else
  \ifnum\@strctr>9\relax
    \@modulo{\@strctr}{10}%
    \let\@@fc@ordstr#2\relax
    \edef#2{\@@fc@ordstr\@teenthstring{\@strctr}}%
  \else
    \ifnum\@strctr=0\relax
      \ifnum#1=0\relax
        \let\@@fc@ordstr#2\relax
        \edef#2{\@@fc@ordstr\@unitstring{0}}%
      \fi
    \else
      \let\@@fc@ordstr#2\relax
      \edef#2{\@@fc@ordstr\@unitthstring{\@strctr}}%
    \fi
  \fi
\fi
\fi
\fi
}
%    \end{macrocode}
%\iffalse
%    \begin{macrocode}
%</fc-spanish.def>
%    \end{macrocode}
%\fi
%\iffalse
%    \begin{macrocode}
%<*fc-UKenglish.def>
%    \end{macrocode}
%\fi
% \subsection{fc-UKenglish.def}
% UK English definitions
%    \begin{macrocode}
\ProvidesFile{fc-UKenglish}[2007/06/14]
%    \end{macrocode}
% Check that fc-english.def has been loaded
%    \begin{macrocode}
\@ifundefined{@ordinalMenglish}{\input{fc-english.def}}{}
%    \end{macrocode}
% These are all just synonyms for the commands provided by
% fc-english.def.
%    \begin{macrocode}
\let\@ordinalMUKenglish\@ordinalMenglish
\let\@ordinalFUKenglish\@ordinalMenglish
\let\@ordinalNUKenglish\@ordinalMenglish
\let\@numberstringMUKenglish\@numberstringMenglish
\let\@numberstringFUKenglish\@numberstringMenglish
\let\@numberstringNUKenglish\@numberstringMenglish
\let\@NumberstringMUKenglish\@NumberstringMenglish
\let\@NumberstringFUKenglish\@NumberstringMenglish
\let\@NumberstringNUKenglish\@NumberstringMenglish
\let\@ordinalstringMUKenglish\@ordinalstringMenglish
\let\@ordinalstringFUKenglish\@ordinalstringMenglish
\let\@ordinalstringNUKenglish\@ordinalstringMenglish
\let\@OrdinalstringMUKenglish\@OrdinalstringMenglish
\let\@OrdinalstringFUKenglish\@OrdinalstringMenglish
\let\@OrdinalstringNUKenglish\@OrdinalstringMenglish
%    \end{macrocode}
%\iffalse
%    \begin{macrocode}
%</fc-UKenglish.def>
%    \end{macrocode}
%\fi
%\iffalse
%    \begin{macrocode}
%<*fc-USenglish.def>
%    \end{macrocode}
%\fi
% \subsection{fc-USenglish.def}
% US English definitions
%    \begin{macrocode}
\ProvidesFile{fc-USenglish}[2007/06/14]
%    \end{macrocode}
% Check that fc-english.def has been loaded
%    \begin{macrocode}
\@ifundefined{@ordinalMenglish}{\input{fc-english.def}}{}
%    \end{macrocode}
% These are all just synonyms for the commands provided by
% fc-english.def.
%    \begin{macrocode}
\let\@ordinalMUSenglish\@ordinalMenglish
\let\@ordinalFUSenglish\@ordinalMenglish
\let\@ordinalNUSenglish\@ordinalMenglish
\let\@numberstringMUSenglish\@numberstringMenglish
\let\@numberstringFUSenglish\@numberstringMenglish
\let\@numberstringNUSenglish\@numberstringMenglish
\let\@NumberstringMUSenglish\@NumberstringMenglish
\let\@NumberstringFUSenglish\@NumberstringMenglish
\let\@NumberstringNUSenglish\@NumberstringMenglish
\let\@ordinalstringMUSenglish\@ordinalstringMenglish
\let\@ordinalstringFUSenglish\@ordinalstringMenglish
\let\@ordinalstringNUSenglish\@ordinalstringMenglish
\let\@OrdinalstringMUSenglish\@OrdinalstringMenglish
\let\@OrdinalstringFUSenglish\@OrdinalstringMenglish
\let\@OrdinalstringNUSenglish\@OrdinalstringMenglish
%    \end{macrocode}
%\iffalse
%    \begin{macrocode}
%</fc-USenglish.def>
%    \end{macrocode}
%\fi
%\iffalse
%    \begin{macrocode}
%<*fmtcount.sty>
%    \end{macrocode}
%\fi
%\subsection{fmtcount.sty}
% This section deals with the code for |fmtcount.sty|
%    \begin{macrocode}
\NeedsTeXFormat{LaTeX2e}
\ProvidesPackage{fmtcount}[2007/06/22 v1.2]
\RequirePackage{ifthen}
\RequirePackage{keyval}
%    \end{macrocode}
% As from version 1.2, now load xspace package:
%    \begin{macrocode}
\RequirePackage{xspace}
%    \end{macrocode}
% These commands need to be defined before the
% configuration file is loaded.
%
% Define the macro to format the |st|, |nd|, |rd| or |th| of an 
% ordinal.
%    \begin{macrocode}
\providecommand{\fmtord}[1]{\textsuperscript{#1}}
%    \end{macrocode}
% Define |\padzeroes| to specify how many digits should be 
% displayed.
%    \begin{macrocode}
\newcount\c@padzeroesN
\c@padzeroesN=1\relax
\providecommand{\padzeroes}[1][17]{\c@padzeroesN=#1}
%    \end{macrocode}
% Load appropriate language definition files (I don't
% know if there is a standard way of detecting which
% languages are defined, so I'm just going to check
% if \verb"\date"\meta{language} is defined):
%\changes{v1.1}{14 June 2007}{added check for UKenglish,
% british and USenglish babel settings}
%    \begin{macrocode}
\@ifundefined{dateenglish}{}{\input{fc-english.def}}
\@ifundefined{l@UKenglish}{}{\input{fc-UKenglish.def}}
\@ifundefined{l@british}{}{\input{fc-british.def}}
\@ifundefined{l@USenglish}{}{\input{fc-USenglish.def}}
\@ifundefined{datespanish}{}{\input{fc-spanish.def}}
\@ifundefined{dateportuges}{}{\input{fc-portuges.def}}
\@ifundefined{datefrench}{}{\input{fc-french.def}}
\@ifundefined{dategerman}{%
\@ifundefined{datengerman}{}{\input{fc-german.def}}}{%
\input{fc-german.def}}
%    \end{macrocode}
% Define keys for use with |\fmtcountsetoptions|.
% Key to switch French dialects (Does babel store
%this kind of information?)
%    \begin{macrocode}
\def\fmtcount@french{france}
\define@key{fmtcount}{french}[france]{%
\@ifundefined{datefrench}{%
\PackageError{fmtcount}{Language `french' not defined}{You need
to load babel before loading fmtcount}}{
\ifthenelse{\equal{#1}{france}
         \or\equal{#1}{swiss}
         \or\equal{#1}{belgian}}{%
         \def\fmtcount@french{#1}}{%
\PackageError{fmtcount}{Invalid value `#1' to french key}
{Option `french' can only take the values `france', 
`belgian' or `swiss'}}
}}
%    \end{macrocode}
% Key to determine how to display the ordinal
%    \begin{macrocode}
\define@key{fmtcount}{fmtord}{%
\ifthenelse{\equal{#1}{level}
          \or\equal{#1}{raise}
          \or\equal{#1}{user}}{
          \def\fmtcount@fmtord{#1}}{%
\PackageError{fmtcount}{Invalid value `#1' to fmtord key}
{Option `fmtord' can only take the values `level', `raise' 
or `user'}}}
%    \end{macrocode}
% Key to determine whether the ordinal should be abbreviated
% (language dependent, currently only affects French ordinals.)
%    \begin{macrocode}
\newif\iffmtord@abbrv
\fmtord@abbrvfalse
\define@key{fmtcount}{abbrv}[true]{%
\ifthenelse{\equal{#1}{true}\or\equal{#1}{false}}{
          \csname fmtord@abbrv#1\endcsname}{%
\PackageError{fmtcount}{Invalid value `#1' to fmtord key}
{Option `fmtord' can only take the values `true' or
`false'}}}
%    \end{macrocode}
% Define command to set options.
%    \begin{macrocode}
\newcommand{\fmtcountsetoptions}[1]{%
\def\fmtcount@fmtord{}%
\setkeys{fmtcount}{#1}%
\@ifundefined{datefrench}{}{%
\edef\@ordinalstringMfrench{\noexpand
\csname @ordinalstringMfrench\fmtcount@french\noexpand\endcsname}%
\edef\@ordinalstringFfrench{\noexpand
\csname @ordinalstringFfrench\fmtcount@french\noexpand\endcsname}%
\edef\@OrdinalstringMfrench{\noexpand
\csname @OrdinalstringMfrench\fmtcount@french\noexpand\endcsname}%
\edef\@OrdinalstringFfrench{\noexpand
\csname @OrdinalstringFfrench\fmtcount@french\noexpand\endcsname}%
\edef\@numberstringMfrench{\noexpand
\csname @numberstringMfrench\fmtcount@french\noexpand\endcsname}%
\edef\@numberstringFfrench{\noexpand
\csname @numberstringFfrench\fmtcount@french\noexpand\endcsname}%
\edef\@NumberstringMfrench{\noexpand
\csname @NumberstringMfrench\fmtcount@french\noexpand\endcsname}%
\edef\@NumberstringFfrench{\noexpand
\csname @NumberstringFfrench\fmtcount@french\noexpand\endcsname}%
}%
%
\ifthenelse{\equal{\fmtcount@fmtord}{level}}{%
\renewcommand{\fmtord}[1]{##1}}{%
\ifthenelse{\equal{\fmtcount@fmtord}{raise}}{%
\renewcommand{\fmtord}[1]{\textsuperscript{##1}}}{%
}}
}
%    \end{macrocode}
% Load confguration file if it exists.  This needs to be done
% before the package options, to allow the user to override
% the settings in the configuration file.
%    \begin{macrocode}
\InputIfFileExists{fmtcount.cfg}{%
\typeout{Using configuration file fmtcount.cfg}}{%
\typeout{No configuration file fmtcount.cfg found.}}
%    \end{macrocode}
%Declare options
%    \begin{macrocode}
\DeclareOption{level}{\def\fmtcount@fmtord{level}%
\def\fmtord#1{#1}}
\DeclareOption{raise}{\def\fmtcount@fmtord{raise}%
\def\fmtord#1{\textsuperscript{#1}}}
%    \end{macrocode}
% Process package options
%    \begin{macrocode}
\ProcessOptions
%    \end{macrocode}
% Define macro that performs modulo arthmetic. This is used for the
% date, time, ordinal and numberstring commands. (The fmtcount
% package was originally part of the datetime package.)
%    \begin{macrocode}
\newcount\@DT@modctr
\def\@modulo#1#2{%
\@DT@modctr=#1\relax
\divide \@DT@modctr by #2\relax
\multiply \@DT@modctr by #2\relax
\advance #1 by -\@DT@modctr}
%    \end{macrocode}
% The following registers are needed by |\@ordinal| etc
%    \begin{macrocode}
\newcount\@ordinalctr
\newcount\@orgargctr
\newcount\@strctr
\newcount\@tmpstrctr
%    \end{macrocode}
%Define commands that display numbers in different bases.
% Define counters and conditionals needed.
%    \begin{macrocode}
\newif\if@DT@padzeroes
\newcount\@DT@loopN
\newcount\@DT@X
%    \end{macrocode}
% Binary
%    \begin{macrocode}
\newcommand{\@binary}[1]{%
\@DT@padzeroestrue
\@DT@loopN=17\relax
\@strctr=\@DT@loopN
\whiledo{\@strctr<\c@padzeroesN}{0\advance\@strctr by 1}%
\@strctr=65536\relax
\@DT@X=#1\relax
\loop
\@DT@modctr=\@DT@X
\divide\@DT@modctr by \@strctr
\ifthenelse{\boolean{@DT@padzeroes} \and \(\@DT@modctr=0\) \and \(\@DT@loopN>\c@padzeroesN\)}{}{\the\@DT@modctr}%
\ifnum\@DT@modctr=0\else\@DT@padzeroesfalse\fi
\multiply\@DT@modctr by \@strctr
\advance\@DT@X by -\@DT@modctr
\divide\@strctr by 2\relax
\advance\@DT@loopN by -1\relax
\ifnum\@strctr>1
\repeat
\the\@DT@X}

\let\binarynum=\@binary
%    \end{macrocode}
% Octal
%    \begin{macrocode}
\newcommand{\@octal}[1]{%
\ifnum#1>32768
\PackageError{fmtcount}{Value of counter too large for \protect\@octal}{Maximum value 32768}
\else
\@DT@padzeroestrue
\@DT@loopN=6\relax
\@strctr=\@DT@loopN
\whiledo{\@strctr<\c@padzeroesN}{0\advance\@strctr by 1}%
\@strctr=32768\relax
\@DT@X=#1\relax
\loop
\@DT@modctr=\@DT@X
\divide\@DT@modctr by \@strctr
\ifthenelse{\boolean{@DT@padzeroes} \and \(\@DT@modctr=0\) \and \(\@DT@loopN>\c@padzeroesN\)}{}{\the\@DT@modctr}%
\ifnum\@DT@modctr=0\else\@DT@padzeroesfalse\fi
\multiply\@DT@modctr by \@strctr
\advance\@DT@X by -\@DT@modctr
\divide\@strctr by 8\relax
\advance\@DT@loopN by -1\relax
\ifnum\@strctr>1
\repeat
\the\@DT@X
\fi}
\let\octalnum=\@octal
%    \end{macrocode}
% Lowercase hexadecimal
%    \begin{macrocode}
\newcommand{\@@hexadecimal}[1]{\ifcase#10\or1\or2\or3\or4\or5\or6\or7\or8\or9\or a\or b\or c\or d\or e\or f\fi}

\newcommand{\@hexadecimal}[1]{%
\@DT@padzeroestrue
\@DT@loopN=5\relax
\@strctr=\@DT@loopN
\whiledo{\@strctr<\c@padzeroesN}{0\advance\@strctr by 1}%
\@strctr=65536\relax
\@DT@X=#1\relax
\loop
\@DT@modctr=\@DT@X
\divide\@DT@modctr by \@strctr
\ifthenelse{\boolean{@DT@padzeroes} \and \(\@DT@modctr=0\) \and \(\@DT@loopN>\c@padzeroesN\)}{}{\@@hexadecimal\@DT@modctr}%
\ifnum\@DT@modctr=0\else\@DT@padzeroesfalse\fi
\multiply\@DT@modctr by \@strctr
\advance\@DT@X by -\@DT@modctr
\divide\@strctr by 16\relax
\advance\@DT@loopN by -1\relax
\ifnum\@strctr>1
\repeat
\@@hexadecimal\@DT@X}

\let\hexadecimalnum=\@hexadecimal
%    \end{macrocode}
% Uppercase hexadecimal
%    \begin{macrocode}
\newcommand{\@@Hexadecimal}[1]{\ifcase#10\or1\or2\or3\or4\or5\or6\or
7\or8\or9\or A\or B\or C\or D\or E\or F\fi}

\newcommand{\@Hexadecimal}[1]{%
\@DT@padzeroestrue
\@DT@loopN=5\relax
\@strctr=\@DT@loopN
\whiledo{\@strctr<\c@padzeroesN}{0\advance\@strctr by 1}%
\@strctr=65536\relax
\@DT@X=#1\relax
\loop
\@DT@modctr=\@DT@X
\divide\@DT@modctr by \@strctr
\ifthenelse{\boolean{@DT@padzeroes} \and \(\@DT@modctr=0\) \and \(\@DT@loopN>\c@padzeroesN\)}{}{\@@Hexadecimal\@DT@modctr}%
\ifnum\@DT@modctr=0\else\@DT@padzeroesfalse\fi
\multiply\@DT@modctr by \@strctr
\advance\@DT@X by -\@DT@modctr
\divide\@strctr by 16\relax
\advance\@DT@loopN by -1\relax
\ifnum\@strctr>1
\repeat
\@@Hexadecimal\@DT@X}

\let\Hexadecimalnum=\@Hexadecimal
%    \end{macrocode}
% Uppercase alphabetical representation (a \ldots\ z aa \ldots\ zz)
%    \begin{macrocode}
\newcommand{\@aaalph}[1]{%
\@DT@loopN=#1\relax
\advance\@DT@loopN by -1\relax
\divide\@DT@loopN by 26\relax
\@DT@modctr=\@DT@loopN
\multiply\@DT@modctr by 26\relax
\@DT@X=#1\relax
\advance\@DT@X by -1\relax
\advance\@DT@X by -\@DT@modctr
\advance\@DT@loopN by 1\relax
\advance\@DT@X by 1\relax
\loop
\@alph\@DT@X
\advance\@DT@loopN by -1\relax
\ifnum\@DT@loopN>0
\repeat
}

\let\aaalphnum=\@aaalph
%    \end{macrocode}
% Uppercase alphabetical representation (a \ldots\ z aa \ldots\ zz)
%    \begin{macrocode}
\newcommand{\@AAAlph}[1]{%
\@DT@loopN=#1\relax
\advance\@DT@loopN by -1\relax
\divide\@DT@loopN by 26\relax
\@DT@modctr=\@DT@loopN
\multiply\@DT@modctr by 26\relax
\@DT@X=#1\relax
\advance\@DT@X by -1\relax
\advance\@DT@X by -\@DT@modctr
\advance\@DT@loopN by 1\relax
\advance\@DT@X by 1\relax
\loop
\@Alph\@DT@X
\advance\@DT@loopN by -1\relax
\ifnum\@DT@loopN>0
\repeat
}

\let\AAAlphnum=\@AAAlph
%    \end{macrocode}
% Lowercase alphabetical representation
%    \begin{macrocode}
\newcommand{\@abalph}[1]{%
\ifnum#1>17576
\PackageError{fmtcount}{Value of counter too large for \protect\@abalph}{Maximum value 17576}
\else
\@DT@padzeroestrue
\@strctr=17576\relax
\@DT@X=#1\relax
\advance\@DT@X by -1\relax
\loop
\@DT@modctr=\@DT@X
\divide\@DT@modctr by \@strctr
\ifthenelse{\boolean{@DT@padzeroes} \and \(\@DT@modctr=1\)}{}{\@alph\@DT@modctr}%
\ifnum\@DT@modctr=1\else\@DT@padzeroesfalse\fi
\multiply\@DT@modctr by \@strctr
\advance\@DT@X by -\@DT@modctr
\divide\@strctr by 26\relax
\ifnum\@strctr>1
\repeat
\advance\@DT@X by 1\relax
\@alph\@DT@X
\fi}

\let\abalphnum=\@abalph
%    \end{macrocode}
% Uppercase alphabetical representation
%    \begin{macrocode}
\newcommand{\@ABAlph}[1]{%
\ifnum#1>17576
\PackageError{fmtcount}{Value of counter too large for \protect\@ABAlph}{Maximum value 17576}
\else
\@DT@padzeroestrue
\@strctr=17576\relax
\@DT@X=#1\relax
\advance\@DT@X by -1\relax
\loop
\@DT@modctr=\@DT@X
\divide\@DT@modctr by \@strctr
\ifthenelse{\boolean{@DT@padzeroes} \and \(\@DT@modctr=1\)}{}{\@Alph\@DT@modctr}%
\ifnum\@DT@modctr=1\else\@DT@padzeroesfalse\fi
\multiply\@DT@modctr by \@strctr
\advance\@DT@X by -\@DT@modctr
\divide\@strctr by 26\relax
\ifnum\@strctr>1
\repeat
\advance\@DT@X by 1\relax
\@Alph\@DT@X
\fi}

\let\ABAlphnum=\@ABAlph
%    \end{macrocode}
% Recursive command to count number of characters in argument.
% |\@strctr| should be set to zero before calling it.
%    \begin{macrocode}
\def\@fmtc@count#1#2\relax{%
\if\relax#1
\else
\advance\@strctr by 1\relax
\@fmtc@count#2\relax
\fi}
%    \end{macrocode}
% Internal decimal macro:
%    \begin{macrocode}
\newcommand{\@decimal}[1]{%
\@strctr=0\relax
\expandafter\@fmtc@count\number#1\relax
\@DT@loopN=\c@padzeroesN
\advance\@DT@loopN by -\@strctr
\ifnum\@DT@loopN>0\relax
\@strctr=0\relax
\whiledo{\@strctr < \@DT@loopN}{0\advance\@strctr by 1}%
\fi
\number#1\relax
}

\let\decimalnum=\@decimal
%    \end{macrocode}
% This is a bit cumbersome.  Previously \verb"\@ordinal"
% was defined in a similar way to \verb"\abalph" etc.
% This ensured that the actual value of the counter was
% written in the new label stuff in the .aux file. However
% adding in an optional argument to determine the gender
% for multilingual compatibility messed things up somewhat.
% This was the only work around I could get to keep the
% the cross-referencing stuff working, which is why
% the optional argument comes \emph{after} the compulsory
% argument, instead of the usual manner of placing it before.
% Version 1.04 changed \verb"\ordinal" to \verb"\FCordinal"
% to prevent it clashing with the memoir class. 
%    \begin{macrocode}
\newcommand{\FCordinal}[1]{%
\expandafter\protect\expandafter\ordinalnum{%
\expandafter\the\csname c@#1\endcsname}}
%    \end{macrocode}
% If \verb"\ordinal" isn't defined make \verb"\ordinal" a synonym
% for \verb"\FCordinal" to maintain compatibility with previous
% versions.
%    \begin{macrocode}
\@ifundefined{ordinal}{\let\ordinal\FCordinal}{%
\PackageWarning{fmtcount}{\string\ordinal
\space already defined use \string\FCordinal \space instead.}}
%    \end{macrocode}
% Display ordinal where value is given as a number or 
% count register instead of a counter:
%    \begin{macrocode}
\newcommand{\ordinalnum}[1]{\@ifnextchar[{\@ordinalnum{#1}}{%
\@ordinalnum{#1}[m]}}
%    \end{macrocode}
% Display ordinal according to gender (neuter added in v1.1,
% \cmdname{xspace} added in v1.2):
%    \begin{macrocode}
\def\@ordinalnum#1[#2]{{%
\ifthenelse{\equal{#2}{f}}{%
\protect\@ordinalF{#1}{\@fc@ordstr}}{%
\ifthenelse{\equal{#2}{n}}{%
\protect\@ordinalN{#1}{\@fc@ordstr}}{%
\ifthenelse{\equal{#2}{m}}{}{%
\PackageError{fmtcount}{Invalid gender option `#2'}{%
Available options are m, f or n}}%
\protect\@ordinalM{#1}{\@fc@ordstr}}}\@fc@ordstr}\xspace}
%    \end{macrocode}
% Store the ordinal (first argument
% is identifying name, second argument is a counter.)
%    \begin{macrocode}
\newcommand*{\storeordinal}[2]{%
\expandafter\protect\expandafter\storeordinalnum{#1}{%
\expandafter\the\csname c@#2\endcsname}}
%    \end{macrocode}
% Store ordinal (first argument
% is identifying name, second argument is a number or
% count register.)
%    \begin{macrocode}
\newcommand*{\storeordinalnum}[2]{%
\@ifnextchar[{\@storeordinalnum{#1}{#2}}{%
\@storeordinalnum{#1}{#2}[m]}}
%    \end{macrocode}
% Store ordinal according to gender:
%    \begin{macrocode}
\def\@storeordinalnum#1#2[#3]{%
\ifthenelse{\equal{#3}{f}}{%
\protect\@ordinalF{#2}{\@fc@ord}}{%
\ifthenelse{\equal{#3}{n}}{%
\protect\@ordinalN{#2}{\@fc@ord}}{%
\ifthenelse{\equal{#3}{m}}{}{%
\PackageError{fmtcount}{Invalid gender option `#3'}{%
Available options are m or f}}%
\protect\@ordinalM{#2}{\@fc@ord}}}%
\expandafter\let\csname @fcs@#1\endcsname\@fc@ord}
%    \end{macrocode}
% Get stored information:
%    \begin{macrocode}
\newcommand*{\FMCuse}[1]{\csname @fcs@#1\endcsname}
%    \end{macrocode}
% Display ordinal as a string (argument is a counter)
%    \begin{macrocode}
\newcommand{\ordinalstring}[1]{%
\expandafter\protect\expandafter\ordinalstringnum{%
\expandafter\the\csname c@#1\endcsname}}
%    \end{macrocode}
% Display ordinal as a string (argument is a count register or
% number.)
%    \begin{macrocode}
\newcommand{\ordinalstringnum}[1]{%
\@ifnextchar[{\@ordinal@string{#1}}{\@ordinal@string{#1}[m]}}
%    \end{macrocode}
% Display ordinal as a string according to gender (\cmdname{xspace}
% added in version 1.2).
%    \begin{macrocode}
\def\@ordinal@string#1[#2]{{%
\ifthenelse{\equal{#2}{f}}{%
\protect\@ordinalstringF{#1}{\@fc@ordstr}}{%
\ifthenelse{\equal{#2}{n}}{%
\protect\@ordinalstringN{#1}{\@fc@ordstr}}{%
\ifthenelse{\equal{#2}{m}}{}{%
\PackageError{fmtcount}{Invalid gender option `#2' to 
\string\ordinalstring}{Available options are m, f or f}}%
\protect\@ordinalstringM{#1}{\@fc@ordstr}}}\@fc@ordstr}\xspace}
%    \end{macrocode}
% Store textual representation of number. First argument is 
% identifying name, second argument is the counter set to the 
% required number.
%    \begin{macrocode}
\newcommand{\storeordinalstring}[2]{%
\expandafter\protect\expandafter\storeordinalstringnum{#1}{%
\expandafter\the\csname c@#2\endcsname}}
%    \end{macrocode}
% Store textual representation of number. First argument is 
% identifying name, second argument is a count register or number.
%    \begin{macrocode}
\newcommand{\storeordinalstringnum}[2]{%
\@ifnextchar[{\@store@ordinal@string{#1}{#2}}{%
\@store@ordinal@string{#1}{#2}[m]}}
%    \end{macrocode}
% Store textual representation of number according to gender.
%    \begin{macrocode}
\def\@store@ordinal@string#1#2[#3]{%
\ifthenelse{\equal{#3}{f}}{%
\protect\@ordinalstringF{#2}{\@fc@ordstr}}{%
\ifthenelse{\equal{#3}{n}}{%
\protect\@ordinalstringN{#2}{\@fc@ordstr}}{%
\ifthenelse{\equal{#3}{m}}{}{%
\PackageError{fmtcount}{Invalid gender option `#3' to 
\string\ordinalstring}{Available options are m, f or n}}%
\protect\@ordinalstringM{#2}{\@fc@ordstr}}}%
\expandafter\let\csname @fcs@#1\endcsname\@fc@ordstr}
%    \end{macrocode}
% Display ordinal as a string with initial letters in upper case
% (argument is a counter)
%    \begin{macrocode}
\newcommand{\Ordinalstring}[1]{%
\expandafter\protect\expandafter\Ordinalstringnum{%
\expandafter\the\csname c@#1\endcsname}}
%    \end{macrocode}
% Display ordinal as a string with initial letters in upper case
% (argument is a number or count register)
%    \begin{macrocode}
\newcommand{\Ordinalstringnum}[1]{%
\@ifnextchar[{\@Ordinal@string{#1}}{\@Ordinal@string{#1}[m]}}
%    \end{macrocode}
% Display ordinal as a string with initial letters in upper case
% according to gender
%    \begin{macrocode}
\def\@Ordinal@string#1[#2]{{%
\ifthenelse{\equal{#2}{f}}{%
\protect\@OrdinalstringF{#1}{\@fc@ordstr}}{%
\ifthenelse{\equal{#2}{n}}{%
\protect\@OrdinalstringN{#1}{\@fc@ordstr}}{%
\ifthenelse{\equal{#2}{m}}{}{%
\PackageError{fmtcount}{Invalid gender option `#2'}{%
Available options are m, f or n}}%
\protect\@OrdinalstringM{#1}{\@fc@ordstr}}}\@fc@ordstr}\xspace}
%    \end{macrocode}
% Store textual representation of number, with initial letters in 
% upper case. First argument is identifying name, second argument 
% is the counter set to the 
% required number.
%    \begin{macrocode}
\newcommand{\storeOrdinalstring}[2]{%
\expandafter\protect\expandafter\storeOrdinalstringnum{#1}{%
\expandafter\the\csname c@#2\endcsname}}
%    \end{macrocode}
% Store textual representation of number, with initial letters in 
% upper case. First argument is identifying name, second argument 
% is a count register or number.
%    \begin{macrocode}
\newcommand{\storeOrdinalstringnum}[2]{%
\@ifnextchar[{\@store@Ordinal@string{#1}{#2}}{%
\@store@Ordinal@string{#1}{#2}[m]}}
%    \end{macrocode}
% Store textual representation of number according to gender, 
% with initial letters in upper case.
%    \begin{macrocode}
\def\@store@Ordinal@string#1#2[#3]{%
\ifthenelse{\equal{#3}{f}}{%
\protect\@OrdinalstringF{#2}{\@fc@ordstr}}{%
\ifthenelse{\equal{#3}{n}}{%
\protect\@OrdinalstringN{#2}{\@fc@ordstr}}{%
\ifthenelse{\equal{#3}{m}}{}{%
\PackageError{fmtcount}{Invalid gender option `#3'}{%
Available options are m or f}}%
\protect\@OrdinalstringM{#2}{\@fc@ordstr}}}%
\expandafter\let\csname @fcs@#1\endcsname\@fc@ordstr}
%    \end{macrocode}
% Store upper case textual representation of ordinal. The first 
% argument is identifying name, the second argument is a counter.
%    \begin{macrocode}
\newcommand{\storeORDINALstring}[2]{%
\expandafter\protect\expandafter\storeORDINALstringnum{#1}{%
\expandafter\the\csname c@#2\endcsname}}
%    \end{macrocode}
% As above, but the second argument is a count register or a
% number.
%    \begin{macrocode}
\newcommand{\storeORDINALstringnum}[2]{%
\@ifnextchar[{\@store@ORDINAL@string{#1}{#2}}{%
\@store@ORDINAL@string{#1}{#2}[m]}}
%    \end{macrocode}
% Gender is specified as an optional argument at the end.
%    \begin{macrocode}
\def\@store@ORDINAL@string#1#2[#3]{%
\ifthenelse{\equal{#3}{f}}{%
\protect\@ordinalstringF{#2}{\@fc@ordstr}}{%
\ifthenelse{\equal{#3}{n}}{%
\protect\@ordinalstringN{#2}{\@fc@ordstr}}{%
\ifthenelse{\equal{#3}{m}}{}{%
\PackageError{fmtcount}{Invalid gender option `#3'}{%
Available options are m or f}}%
\protect\@ordinalstringM{#2}{\@fc@ordstr}}}%
\expandafter\edef\csname @fcs@#1\endcsname{%
\noexpand\MakeUppercase{\@fc@ordstr}}}
%    \end{macrocode}
% Display upper case textual representation of an ordinal. The
% argument must be a counter.
%    \begin{macrocode}
\newcommand{\ORDINALstring}[1]{%
\expandafter\protect\expandafter\ORDINALstringnum{%
\expandafter\the\csname c@#1\endcsname}}
%    \end{macrocode}
% As above, but the argument is a count register or a number.
%    \begin{macrocode}
\newcommand{\ORDINALstringnum}[1]{%
\@ifnextchar[{\@ORDINAL@string{#1}}{\@ORDINAL@string{#1}[m]}}
%    \end{macrocode}
% Gender is specified as an optional argument at the end.
%    \begin{macrocode}
\def\@ORDINAL@string#1[#2]{{%
\ifthenelse{\equal{#2}{f}}{%
\protect\@ordinalstringF{#1}{\@fc@ordstr}}{%
\ifthenelse{\equal{#2}{n}}{%
\protect\@ordinalstringN{#1}{\@fc@ordstr}}{%
\ifthenelse{\equal{#2}{m}}{}{%
\PackageError{fmtcount}{Invalid gender option `#2'}{%
Available options are m, f or n}}%
\protect\@ordinalstringM{#1}{\@fc@ordstr}}}%
\MakeUppercase{\@fc@ordstr}}\xspace}
%    \end{macrocode}
% Convert number to textual respresentation, and store. First 
% argument is the identifying name, second argument is a counter 
% containing the number.
%    \begin{macrocode}
\newcommand{\storenumberstring}[2]{%
\expandafter\protect\expandafter\storenumberstringnum{#1}{%
\expandafter\the\csname c@#2\endcsname}}
%    \end{macrocode}
% As above, but second argument is a number or count register.
%    \begin{macrocode}
\newcommand{\storenumberstringnum}[2]{%
\@ifnextchar[{\@store@number@string{#1}{#2}}{%
\@store@number@string{#1}{#2}[m]}}
%    \end{macrocode}
% Gender is given as optional argument, \emph{at the end}.
%    \begin{macrocode}
\def\@store@number@string#1#2[#3]{%
\ifthenelse{\equal{#3}{f}}{%
\protect\@numberstringF{#2}{\@fc@numstr}}{%
\ifthenelse{\equal{#3}{n}}{%
\protect\@numberstringN{#2}{\@fc@numstr}}{%
\ifthenelse{\equal{#3}{m}}{}{%
\PackageError{fmtcount}{Invalid gender option `#3'}{%
Available options are m, f or n}}%
\protect\@numberstringM{#2}{\@fc@numstr}}}%
\expandafter\let\csname @fcs@#1\endcsname\@fc@numstr}
%    \end{macrocode}
% Display textual representation of a number. The argument
% must be a counter.
%    \begin{macrocode}
\newcommand{\numberstring}[1]{%
\expandafter\protect\expandafter\numberstringnum{%
\expandafter\the\csname c@#1\endcsname}}
%    \end{macrocode}
% As above, but the argument is a count register or a number.
%    \begin{macrocode}
\newcommand{\numberstringnum}[1]{%
\@ifnextchar[{\@number@string{#1}}{\@number@string{#1}[m]}}
%    \end{macrocode}
% Gender is specified as an optional argument \emph{at the end}.
%    \begin{macrocode}
\def\@number@string#1[#2]{{%
\ifthenelse{\equal{#2}{f}}{%
\protect\@numberstringF{#1}{\@fc@numstr}}{%
\ifthenelse{\equal{#2}{n}}{%
\protect\@numberstringN{#1}{\@fc@numstr}}{%
\ifthenelse{\equal{#2}{m}}{}{%
\PackageError{fmtcount}{Invalid gender option `#2'}{%
Available options are m, f or n}}%
\protect\@numberstringM{#1}{\@fc@numstr}}}\@fc@numstr}\xspace}
%    \end{macrocode}
% Store textual representation of number. First argument is 
% identifying name, second argument is a counter.
%    \begin{macrocode}
\newcommand{\storeNumberstring}[2]{%
\expandafter\protect\expandafter\storeNumberstringnum{#1}{%
\expandafter\the\csname c@#2\endcsname}}
%    \end{macrocode}
% As above, but second argument is a count register or number.
%    \begin{macrocode}
\newcommand{\storeNumberstringnum}[2]{%
\@ifnextchar[{\@store@Number@string{#1}{#2}}{%
\@store@Number@string{#1}{#2}[m]}}
%    \end{macrocode}
% Gender is specified as an optional argument \emph{at the end}:
%    \begin{macrocode}
\def\@store@Number@string#1#2[#3]{%
\ifthenelse{\equal{#3}{f}}{%
\protect\@NumberstringF{#2}{\@fc@numstr}}{%
\ifthenelse{\equal{#3}{n}}{%
\protect\@NumberstringN{#2}{\@fc@numstr}}{%
\ifthenelse{\equal{#3}{m}}{}{%
\PackageError{fmtcount}{Invalid gender option `#3'}{%
Available options are m, f or n}}%
\protect\@NumberstringM{#2}{\@fc@numstr}}}%
\expandafter\let\csname @fcs@#1\endcsname\@fc@numstr}
%    \end{macrocode}
% Display textual representation of number. The argument must be
% a counter. 
%    \begin{macrocode}
\newcommand{\Numberstring}[1]{%
\expandafter\protect\expandafter\Numberstringnum{%
\expandafter\the\csname c@#1\endcsname}}
%    \end{macrocode}
% As above, but the argument is a count register or number.
%    \begin{macrocode}
\newcommand{\Numberstringnum}[1]{%
\@ifnextchar[{\@Number@string{#1}}{\@Number@string{#1}[m]}}
%    \end{macrocode}
% Gender is specified as an optional argument at the end.
%    \begin{macrocode}
\def\@Number@string#1[#2]{{%
\ifthenelse{\equal{#2}{f}}{%
\protect\@NumberstringF{#1}{\@fc@numstr}}{%
\ifthenelse{\equal{#2}{n}}{%
\protect\@NumberstringN{#1}{\@fc@numstr}}{%
\ifthenelse{\equal{#2}{m}}{}{%
\PackageError{fmtcount}{Invalid gender option `#2'}{%
Available options are m, f or n}}%
\protect\@NumberstringM{#1}{\@fc@numstr}}}\@fc@numstr}\xspace}
%    \end{macrocode}
% Store upper case textual representation of number. The first 
% argument is identifying name, the second argument is a counter.
%    \begin{macrocode}
\newcommand{\storeNUMBERstring}[2]{%
\expandafter\protect\expandafter\storeNUMBERstringnum{#1}{%
\expandafter\the\csname c@#2\endcsname}}
%    \end{macrocode}
% As above, but the second argument is a count register or a
% number.
%    \begin{macrocode}
\newcommand{\storeNUMBERstringnum}[2]{%
\@ifnextchar[{\@store@NUMBER@string{#1}{#2}}{%
\@store@NUMBER@string{#1}{#2}[m]}}
%    \end{macrocode}
% Gender is specified as an optional argument at the end.
%    \begin{macrocode}
\def\@store@NUMBER@string#1#2[#3]{%
\ifthenelse{\equal{#3}{f}}{%
\protect\@numberstringF{#2}{\@fc@numstr}}{%
\ifthenelse{\equal{#3}{n}}{%
\protect\@numberstringN{#2}{\@fc@numstr}}{%
\ifthenelse{\equal{#3}{m}}{}{%
\PackageError{fmtcount}{Invalid gender option `#3'}{%
Available options are m or f}}%
\protect\@numberstringM{#2}{\@fc@numstr}}}%
\expandafter\edef\csname @fcs@#1\endcsname{%
\noexpand\MakeUppercase{\@fc@numstr}}}
%    \end{macrocode}
% Display upper case textual representation of a number. The
% argument must be a counter.
%    \begin{macrocode}
\newcommand{\NUMBERstring}[1]{%
\expandafter\protect\expandafter\NUMBERstringnum{%
\expandafter\the\csname c@#1\endcsname}}
%    \end{macrocode}
% As above, but the argument is a count register or a number.
%    \begin{macrocode}
\newcommand{\NUMBERstringnum}[1]{%
\@ifnextchar[{\@NUMBER@string{#1}}{\@NUMBER@string{#1}[m]}}
%    \end{macrocode}
% Gender is specified as an optional argument at the end.
%    \begin{macrocode}
\def\@NUMBER@string#1[#2]{{%
\ifthenelse{\equal{#2}{f}}{%
\protect\@numberstringF{#1}{\@fc@numstr}}{%
\ifthenelse{\equal{#2}{n}}{%
\protect\@numberstringN{#1}{\@fc@numstr}}{%
\ifthenelse{\equal{#2}{m}}{}{%
\PackageError{fmtcount}{Invalid gender option `#2'}{%
Available options are m, f or n}}%
\protect\@numberstringM{#1}{\@fc@numstr}}}%
\MakeUppercase{\@fc@numstr}}\xspace}
%    \end{macrocode}
% Number representations in other bases. Binary:
%    \begin{macrocode}
\providecommand{\binary}[1]{%
\expandafter\protect\expandafter\@binary{%
\expandafter\the\csname c@#1\endcsname}}
%    \end{macrocode}
% Like \verb"\alph", but goes beyond 26. (a \ldots\ z aa \ldots zz \ldots)
%    \begin{macrocode}
\providecommand{\aaalph}[1]{%
\expandafter\protect\expandafter\@aaalph{%
\expandafter\the\csname c@#1\endcsname}}
%    \end{macrocode}
% As before, but upper case.
%    \begin{macrocode}
\providecommand{\AAAlph}[1]{%
\expandafter\protect\expandafter\@AAAlph{%
\expandafter\the\csname c@#1\endcsname}}
%    \end{macrocode}
% Like \verb"\alph", but goes beyond 26. (a \ldots\ z ab \ldots az \ldots)
%    \begin{macrocode}
\providecommand{\abalph}[1]{%
\expandafter\protect\expandafter\@abalph{%
\expandafter\the\csname c@#1\endcsname}}
%    \end{macrocode}
% As above, but upper case.
%    \begin{macrocode}
\providecommand{\ABAlph}[1]{%
\expandafter\protect\expandafter\@ABAlph{%
\expandafter\the\csname c@#1\endcsname}}
%    \end{macrocode}
% Hexadecimal:
%    \begin{macrocode}
\providecommand{\hexadecimal}[1]{%
\expandafter\protect\expandafter\@hexadecimal{%
\expandafter\the\csname c@#1\endcsname}}
%    \end{macrocode}
% As above, but in upper case.
%    \begin{macrocode}
\providecommand{\Hexadecimal}[1]{%
\expandafter\protect\expandafter\@Hexadecimal{%
\expandafter\the\csname c@#1\endcsname}}
%    \end{macrocode}
% Octal:
%    \begin{macrocode}
\providecommand{\octal}[1]{%
\expandafter\protect\expandafter\@octal{%
\expandafter\the\csname c@#1\endcsname}}
%    \end{macrocode}
% Decimal:
%    \begin{macrocode}
\providecommand{\decimal}[1]{%
\expandafter\protect\expandafter\@decimal{%
\expandafter\the\csname c@#1\endcsname}}
%    \end{macrocode}
%\subsubsection{Multilinguage Definitions}
% If multilingual support is provided, make \verb"\@numberstring" 
% etc use the correct language (if defined).
% Otherwise use English definitions. "\@setdef@ultfmtcount"
% sets the macros to use English.
%    \begin{macrocode}
\def\@setdef@ultfmtcount{
\@ifundefined{@ordinalMenglish}{\input{fc-english.def}}{}
\def\@ordinalstringM{\@ordinalstringMenglish}
\let\@ordinalstringF=\@ordinalstringMenglish
\let\@ordinalstringN=\@ordinalstringMenglish
\def\@OrdinalstringM{\@OrdinalstringMenglish}
\let\@OrdinalstringF=\@OrdinalstringMenglish
\let\@OrdinalstringN=\@OrdinalstringMenglish
\def\@numberstringM{\@numberstringMenglish}
\let\@numberstringF=\@numberstringMenglish
\let\@numberstringN=\@numberstringMenglish
\def\@NumberstringM{\@NumberstringMenglish}
\let\@NumberstringF=\@NumberstringMenglish
\let\@NumberstringN=\@NumberstringMenglish
\def\@ordinalM{\@ordinalMenglish}
\let\@ordinalF=\@ordinalM
\let\@ordinalN=\@ordinalM
}
%    \end{macrocode}
% Define a command to set macros to use "languagename":
%    \begin{macrocode}
\def\@set@mulitling@fmtcount{%
%
\def\@numberstringM{\@ifundefined{@numberstringM\languagename}{%
\PackageError{fmtcount}{No support for language '\languagename'}{%
The fmtcount package currently does not support language 
'\languagename' for command \string\@numberstringM}}{%
\csname @numberstringM\languagename\endcsname}}%
%
\def\@numberstringF{\@ifundefined{@numberstringF\languagename}{%
\PackageError{fmtcount}{No support for language '\languagename'}{%
The fmtcount package currently does not support language 
'\languagename' for command \string\@numberstringF}}{%
\csname @numberstringF\languagename\endcsname}}%
%
\def\@numberstringN{\@ifundefined{@numberstringN\languagename}{%
\PackageError{fmtcount}{No support for language '\languagename'}{%
The fmtcount package currently does not support language 
'\languagename' for command \string\@numberstringN}}{%
\csname @numberstringN\languagename\endcsname}}%
%
\def\@NumberstringM{\@ifundefined{@NumberstringM\languagename}{%
\PackageError{fmtcount}{No support for language '\languagename'}{%
The fmtcount package currently does not support language 
'\languagename' for command \string\@NumberstringM}}{%
\csname @NumberstringM\languagename\endcsname}}%
%
\def\@NumberstringF{\@ifundefined{@NumberstringF\languagename}{%
\PackageError{fmtcount}{No support for language '\languagename'}{%
The fmtcount package currently does not support language 
'\languagename' for command \string\@NumberstringF}}{%
\csname @NumberstringF\languagename\endcsname}}%
%
\def\@NumberstringN{\@ifundefined{@NumberstringN\languagename}{%
\PackageError{fmtcount}{No support for language '\languagename'}{%
The fmtcount package currently does not support language 
'\languagename' for command \string\@NumberstringN}}{%
\csname @NumberstringN\languagename\endcsname}}%
%
\def\@ordinalM{\@ifundefined{@ordinalM\languagename}{%
\PackageError{fmtcount}{No support for language '\languagename'}{%
The fmtcount package currently does not support language 
'\languagename' for command \string\@ordinalM}}{%
\csname @ordinalM\languagename\endcsname}}%
%
\def\@ordinalF{\@ifundefined{@ordinalF\languagename}{%
\PackageError{fmtcount}{No support for language '\languagename'}{%
The fmtcount package currently does not support language 
'\languagename' for command \string\@ordinalF}}{%
\csname @ordinalF\languagename\endcsname}}%
%
\def\@ordinalN{\@ifundefined{@ordinalN\languagename}{%
\PackageError{fmtcount}{No support for language '\languagename'}{%
The fmtcount package currently does not support language 
'\languagename' for command \string\@ordinalN}}{%
\csname @ordinalN\languagename\endcsname}}%
%
\def\@ordinalstringM{\@ifundefined{@ordinalstringM\languagename}{%
\PackageError{fmtcount}{No support for language '\languagename'}{%
The fmtcount package currently does not support language 
'\languagename' for command \string\@ordinalstringM}}{%
\csname @ordinalstringM\languagename\endcsname}}%
%
\def\@ordinalstringF{\@ifundefined{@ordinalstringF\languagename}{%
\PackageError{fmtcount}{No support for language '\languagename'}{%
The fmtcount package currently does not support language 
'\languagename' for command \string\@ordinalstringF}}{%
\csname @ordinalstringF\languagename\endcsname}}%
%
\def\@ordinalstringN{\@ifundefined{@ordinalstringN\languagename}{%
\PackageError{fmtcount}{No support for language '\languagename'}{%
The fmtcount package currently does not support language 
'\languagename' for command \string\@ordinalstringN}}{%
\csname @ordinalstringN\languagename\endcsname}}%
%
\def\@OrdinalstringM{\@ifundefined{@OrdinalstringM\languagename}{%
\PackageError{fmtcount}{No support for language '\languagename'}{%
The fmtcount package currently does not support language 
'\languagename' for command \string\@OrdinalstringM}}{%
\csname @OrdinalstringM\languagename\endcsname}}%
%
\def\@OrdinalstringF{\@ifundefined{@OrdinalstringF\languagename}{%
\PackageError{fmtcount}{No support for language '\languagename'}{%
The fmtcount package currently does not support language 
'\languagename' for command \string\@OrdinalstringF}}{%
\csname @OrdinalstringF\languagename\endcsname}}%
%
\def\@OrdinalstringN{\@ifundefined{@OrdinalstringN\languagename}{%
\PackageError{fmtcount}{No support for language '\languagename'}{%
The fmtcount package currently does not support language 
'\languagename' for command \string\@OrdinalstringN}}{%
\csname @OrdinalstringN\languagename\endcsname}}
}
%    \end{macrocode}
% Check to see if babel or ngerman packages have been loaded.
%    \begin{macrocode}
\@ifpackageloaded{babel}{%
\ifthenelse{\equal{\languagename}{nohyphenation}\or
\equal{languagename}{english}}{\@setdef@ultfmtcount}{%
\@set@mulitling@fmtcount}
}{%
\@ifpackageloaded{ngerman}{%
\@ifundefined{@numberstringMgerman}{%
\input{fc-german.def}}{}\@set@mulitling@fmtcount}{%
\@setdef@ultfmtcount}}
%    \end{macrocode}
% Backwards compatibility:
%    \begin{macrocode}
\let\@ordinal=\@ordinalM
\let\@ordinalstring=\@ordinalstringM
\let\@Ordinalstring=\@OrdinalstringM
\let\@numberstring=\@numberstringM
\let\@Numberstring=\@NumberstringM
%    \end{macrocode}
%\iffalse
%    \begin{macrocode}
%</fmtcount.sty>
%    \end{macrocode}
%\fi
%\Finale
\endinput
}
%\end{verbatim}
%This, I agree, is an unpleasant cludge.
%
%\end{itemize}
%
%\section{Acknowledgements}
%
%I would like to thank my mother for the French and Portuguese
%support and my Spanish dictionary for the Spanish support.
%Thank you to K. H. Fricke for providing me with the German
%translations.
%
%\section{Troubleshooting}
%
%There is a FAQ available at: \url{http://theoval.cmp.uea.ac.uk/~nlct/latex/packages/faq/}.
%
% \section{Contact Details}
% Dr Nicola Talbot\\
% School of Computing Sciences\\
% University of East Anglia\\
% Norwich.  NR4 7TJ.\\
% United Kingdom.\\
% \url{http://theoval.cmp.uea.ac.uk/~nlct/}
%
%
%\StopEventually{}
%\section{The Code}
%\iffalse
%    \begin{macrocode}
%<*fc-british.def>
%    \end{macrocode}
%\fi
% \subsection{fc-british.def}
% British definitions
%    \begin{macrocode}
\ProvidesFile{fc-british}[2007/06/14]
%    \end{macrocode}
% Check that fc-english.def has been loaded
%    \begin{macrocode}
\@ifundefined{@ordinalMenglish}{\input{fc-english.def}}{}
%    \end{macrocode}
% These are all just synonyms for the commands provided by
% fc-english.def.
%    \begin{macrocode}
\let\@ordinalMbritish\@ordinalMenglish
\let\@ordinalFbritish\@ordinalMenglish
\let\@ordinalNbritish\@ordinalMenglish
\let\@numberstringMbritish\@numberstringMenglish
\let\@numberstringFbritish\@numberstringMenglish
\let\@numberstringNbritish\@numberstringMenglish
\let\@NumberstringMbritish\@NumberstringMenglish
\let\@NumberstringFbritish\@NumberstringMenglish
\let\@NumberstringNbritish\@NumberstringMenglish
\let\@ordinalstringMbritish\@ordinalstringMenglish
\let\@ordinalstringFbritish\@ordinalstringMenglish
\let\@ordinalstringNbritish\@ordinalstringMenglish
\let\@OrdinalstringMbritish\@OrdinalstringMenglish
\let\@OrdinalstringFbritish\@OrdinalstringMenglish
\let\@OrdinalstringNbritish\@OrdinalstringMenglish
%    \end{macrocode}
%\iffalse
%    \begin{macrocode}
%</fc-british.def>
%    \end{macrocode}
%\fi
%\iffalse
%    \begin{macrocode}
%<*fc-english.def>
%    \end{macrocode}
%\fi
% \subsection{fc-english.def}
% English definitions
%    \begin{macrocode}
\ProvidesFile{fc-english}[2007/05/26]
%    \end{macrocode}
% Define macro that converts a number or count register (first 
% argument) to an ordinal, and stores the result in the 
% second argument, which should be a control sequence.
%    \begin{macrocode}
\newcommand*{\@ordinalMenglish}[2]{%
\def\@fc@ord{}%
\@orgargctr=#1\relax
\@ordinalctr=#1%
\@modulo{\@ordinalctr}{100}%
\ifnum\@ordinalctr=11\relax
  \def\@fc@ord{th}%
\else
  \ifnum\@ordinalctr=12\relax
    \def\@fc@ord{th}%
  \else
    \ifnum\@ordinalctr=13\relax
      \def\@fc@ord{th}%
    \else
      \@modulo{\@ordinalctr}{10}%
      \ifcase\@ordinalctr
        \def\@fc@ord{th}%      case 0
        \or \def\@fc@ord{st}%  case 1
        \or \def\@fc@ord{nd}%  case 2
        \or \def\@fc@ord{rd}%  case 3
      \else
        \def\@fc@ord{th}%      default case
      \fi
    \fi
  \fi
\fi
\edef#2{\number#1\relax\noexpand\fmtord{\@fc@ord}}%
}
%    \end{macrocode}
% There is no gender difference in English, so make feminine and
% neuter the same as the masculine.
%    \begin{macrocode}
\let\@ordinalFenglish=\@ordinalMenglish
\let\@ordinalNenglish=\@ordinalMenglish
%    \end{macrocode}
% Define the macro that prints the value of a \TeX\ count register
% as text. To make it easier, break it up into units, teens and
% tens. First, the units: the argument should be between 0 and 9
% inclusive.
%    \begin{macrocode}
\newcommand*{\@@unitstringenglish}[1]{%
\ifcase#1\relax
zero%
\or one%
\or two%
\or three%
\or four%
\or five%
\or six%
\or seven%
\or eight%
\or nine%
\fi
}
%    \end{macrocode}
% Next the tens, again the argument should be between 0 and 9
% inclusive.
%    \begin{macrocode}
\newcommand*{\@@tenstringenglish}[1]{%
\ifcase#1\relax
\or ten%
\or twenty%
\or thirty%
\or forty%
\or fifty%
\or sixty%
\or seventy%
\or eighty%
\or ninety%
\fi
}
%    \end{macrocode}
% Finally the teens, again the argument should be between 0 and 9
% inclusive.
%    \begin{macrocode}
\newcommand*{\@@teenstringenglish}[1]{%
\ifcase#1\relax
ten%
\or eleven%
\or twelve%
\or thirteen%
\or fourteen%
\or fifteen%
\or sixteen%
\or seventeen%
\or eighteen%
\or nineteen%
\fi
}
%    \end{macrocode}
% As above, but with the initial letter in uppercase. The units:
%    \begin{macrocode}
\newcommand*{\@@Unitstringenglish}[1]{%
\ifcase#1\relax
Zero%
\or One%
\or Two%
\or Three%
\or Four%
\or Five%
\or Six%
\or Seven%
\or Eight%
\or Nine%
\fi
}
%    \end{macrocode}
% The tens:
%    \begin{macrocode}
\newcommand*{\@@Tenstringenglish}[1]{%
\ifcase#1\relax
\or Ten%
\or Twenty%
\or Thirty%
\or Forty%
\or Fifty%
\or Sixty%
\or Seventy%
\or Eighty%
\or Ninety%
\fi
}
%    \end{macrocode}
% The teens:
%    \begin{macrocode}
\newcommand*{\@@Teenstringenglish}[1]{%
\ifcase#1\relax
Ten%
\or Eleven%
\or Twelve%
\or Thirteen%
\or Fourteen%
\or Fifteen%
\or Sixteen%
\or Seventeen%
\or Eighteen%
\or Nineteen%
\fi
}
%    \end{macrocode}
% This has changed in version 1.09, so that it now stores
% the result in the second argument, but doesn't display anything.
% Since it only affects internal macros, it shouldn't affect
% documents created with older versions. (These internal macros are
% not meant for use in documents.)
%    \begin{macrocode}
\newcommand*{\@@numberstringenglish}[2]{%
\ifnum#1>99999
\PackageError{fmtcount}{Out of range}%
{This macro only works for values less than 100000}%
\else
\ifnum#1<0
\PackageError{fmtcount}{Negative numbers not permitted}%
{This macro does not work for negative numbers, however
you can try typing "minus" first, and then pass the modulus of
this number}%
\fi
\fi
\def#2{}%
\@strctr=#1\relax \divide\@strctr by 1000\relax
\ifnum\@strctr>9
% #1 is greater or equal to 10000
  \divide\@strctr by 10
  \ifnum\@strctr>1\relax
    \let\@@fc@numstr#2\relax
    \edef#2{\@@fc@numstr\@tenstring{\@strctr}}%
    \@strctr=#1 \divide\@strctr by 1000\relax
    \@modulo{\@strctr}{10}%
    \ifnum\@strctr>0\relax
      \let\@@fc@numstr#2\relax
      \edef#2{\@@fc@numstr-\@unitstring{\@strctr}}%
    \fi
  \else
    \@strctr=#1\relax
    \divide\@strctr by 1000\relax
    \@modulo{\@strctr}{10}%
    \let\@@fc@numstr#2\relax
    \edef#2{\@@fc@numstr\@teenstring{\@strctr}}%
  \fi
  \let\@@fc@numstr#2\relax
  \edef#2{\@@fc@numstr\ \@thousand}%
\else
  \ifnum\@strctr>0\relax
    \let\@@fc@numstr#2\relax
    \edef#2{\@@fc@numstr\@unitstring{\@strctr}\ \@thousand}%
  \fi
\fi
\@strctr=#1\relax \@modulo{\@strctr}{1000}%
\divide\@strctr by 100
\ifnum\@strctr>0\relax
   \ifnum#1>1000\relax
      \let\@@fc@numstr#2\relax
      \edef#2{\@@fc@numstr\ }%
   \fi
   \let\@@fc@numstr#2\relax
   \edef#2{\@@fc@numstr\@unitstring{\@strctr}\ \@hundred}%
\fi
\@strctr=#1\relax \@modulo{\@strctr}{100}%
\ifnum#1>100\relax
  \ifnum\@strctr>0\relax
    \let\@@fc@numstr#2\relax
    \edef#2{\@@fc@numstr\ \@andname\ }%
  \fi
\fi
\ifnum\@strctr>19\relax
  \divide\@strctr by 10\relax
  \let\@@fc@numstr#2\relax
  \edef#2{\@@fc@numstr\@tenstring{\@strctr}}%
  \@strctr=#1\relax \@modulo{\@strctr}{10}%
  \ifnum\@strctr>0\relax
    \let\@@fc@numstr#2\relax
    \edef#2{\@@fc@numstr-\@unitstring{\@strctr}}%
  \fi
\else
  \ifnum\@strctr<10\relax
    \ifnum\@strctr=0\relax
       \ifnum#1<100\relax
          \let\@@fc@numstr#2\relax
          \edef#2{\@@fc@numstr\@unitstring{\@strctr}}%
       \fi
    \else
      \let\@@fc@numstr#2\relax
      \edef#2{\@@fc@numstr\@unitstring{\@strctr}}%
    \fi
  \else
    \@modulo{\@strctr}{10}%
    \let\@@fc@numstr#2\relax
    \edef#2{\@@fc@numstr\@teenstring{\@strctr}}%
  \fi
\fi
}
%    \end{macrocode}
% All lower case version, the second argument must be a 
% control sequence.
%    \begin{macrocode}
\DeclareRobustCommand{\@numberstringMenglish}[2]{%
\let\@unitstring=\@@unitstringenglish 
\let\@teenstring=\@@teenstringenglish 
\let\@tenstring=\@@tenstringenglish
\def\@hundred{hundred}\def\@thousand{thousand}%
\def\@andname{and}%
\@@numberstringenglish{#1}{#2}%
}
%    \end{macrocode}
% There is no gender in English, so make feminine and neuter the same
% as the masculine.
%    \begin{macrocode}
\let\@numberstringFenglish=\@numberstringMenglish
\let\@numberstringNenglish=\@numberstringMenglish
%    \end{macrocode}
% This version makes the first letter of each word an uppercase
% character (except ``and''). The second argument must be a control 
% sequence.
%    \begin{macrocode}
\newcommand*{\@NumberstringMenglish}[2]{%
\let\@unitstring=\@@Unitstringenglish 
\let\@teenstring=\@@Teenstringenglish 
\let\@tenstring=\@@Tenstringenglish
\def\@hundred{Hundred}\def\@thousand{Thousand}%
\def\@andname{and}%
\@@numberstringenglish{#1}{#2}}
%    \end{macrocode}
% There is no gender in English, so make feminine and neuter the same
% as the masculine.
%    \begin{macrocode}
\let\@NumberstringFenglish=\@NumberstringMenglish
\let\@NumberstringNenglish=\@NumberstringMenglish
%    \end{macrocode}
% Define a macro that produces an ordinal as a string. Again, break
% it up into units, teens and tens. First the units:
%    \begin{macrocode}
\newcommand*{\@@unitthstringenglish}[1]{%
\ifcase#1\relax
zeroth%
\or first%
\or second%
\or third%
\or fourth%
\or fifth%
\or sixth%
\or seventh%
\or eighth%
\or ninth%
\fi
}
%    \end{macrocode}
% Next the tens:
%    \begin{macrocode}
\newcommand*{\@@tenthstringenglish}[1]{%
\ifcase#1\relax
\or tenth%
\or twentieth%
\or thirtieth%
\or fortieth%
\or fiftieth%
\or sixtieth%
\or seventieth%
\or eightieth%
\or ninetieth%
\fi
}
%   \end{macrocode}
% The teens:
%   \begin{macrocode}
\newcommand*{\@@teenthstringenglish}[1]{%
\ifcase#1\relax
tenth%
\or eleventh%
\or twelfth%
\or thirteenth%
\or fourteenth%
\or fifteenth%
\or sixteenth%
\or seventeenth%
\or eighteenth%
\or nineteenth%
\fi
}
%   \end{macrocode}
% As before, but with the first letter in upper case. The units:
%   \begin{macrocode}
\newcommand*{\@@Unitthstringenglish}[1]{%
\ifcase#1\relax
Zeroth%
\or First%
\or Second%
\or Third%
\or Fourth%
\or Fifth%
\or Sixth%
\or Seventh%
\or Eighth%
\or Ninth%
\fi
}
%    \end{macrocode}
% The tens:
%    \begin{macrocode}
\newcommand*{\@@Tenthstringenglish}[1]{%
\ifcase#1\relax
\or Tenth%
\or Twentieth%
\or Thirtieth%
\or Fortieth%
\or Fiftieth%
\or Sixtieth%
\or Seventieth%
\or Eightieth%
\or Ninetieth%
\fi
}
%    \end{macrocode}
% The teens:
%    \begin{macrocode}
\newcommand*{\@@Teenthstringenglish}[1]{%
\ifcase#1\relax
Tenth%
\or Eleventh%
\or Twelfth%
\or Thirteenth%
\or Fourteenth%
\or Fifteenth%
\or Sixteenth%
\or Seventeenth%
\or Eighteenth%
\or Nineteenth%
\fi
}
%    \end{macrocode}
% Again, as from version 1.09, this has been changed to take two
% arguments, where the second argument is a control sequence.
% The resulting text is stored in the control sequence, and nothing
% is displayed.
%    \begin{macrocode}
\newcommand*{\@@ordinalstringenglish}[2]{%
\@strctr=#1\relax
\ifnum#1>99999
\PackageError{fmtcount}{Out of range}%
{This macro only works for values less than 100000 (value given: \number\@strctr)}%
\else
\ifnum#1<0
\PackageError{fmtcount}{Negative numbers not permitted}%
{This macro does not work for negative numbers, however
you can try typing "minus" first, and then pass the modulus of
this number}%
\fi
\def#2{}%
\fi
\@strctr=#1\relax \divide\@strctr by 1000\relax
\ifnum\@strctr>9\relax
% #1 is greater or equal to 10000
  \divide\@strctr by 10
  \ifnum\@strctr>1\relax
    \let\@@fc@ordstr#2\relax
    \edef#2{\@@fc@ordstr\@tenstring{\@strctr}}%
    \@strctr=#1\relax
    \divide\@strctr by 1000\relax
    \@modulo{\@strctr}{10}%
    \ifnum\@strctr>0\relax
      \let\@@fc@ordstr#2\relax
      \edef#2{\@@fc@ordstr-\@unitstring{\@strctr}}%
    \fi
  \else
    \@strctr=#1\relax \divide\@strctr by 1000\relax
    \@modulo{\@strctr}{10}%
    \let\@@fc@ordstr#2\relax
    \edef#2{\@@fc@ordstr\@teenstring{\@strctr}}%
  \fi
  \@strctr=#1\relax \@modulo{\@strctr}{1000}%
  \ifnum\@strctr=0\relax
    \let\@@fc@ordstr#2\relax
    \edef#2{\@@fc@ordstr\ \@thousandth}%
  \else
    \let\@@fc@ordstr#2\relax
    \edef#2{\@@fc@ordstr\ \@thousand}%
  \fi
\else
  \ifnum\@strctr>0\relax
    \let\@@fc@ordstr#2\relax
    \edef#2{\@@fc@ordstr\@unitstring{\@strctr}}%
    \@strctr=#1\relax \@modulo{\@strctr}{1000}%
    \let\@@fc@ordstr#2\relax
    \ifnum\@strctr=0\relax
      \edef#2{\@@fc@ordstr\ \@thousandth}%
    \else
      \edef#2{\@@fc@ordstr\ \@thousand}%
    \fi
  \fi
\fi
\@strctr=#1\relax \@modulo{\@strctr}{1000}%
\divide\@strctr by 100
\ifnum\@strctr>0\relax
  \ifnum#1>1000\relax
    \let\@@fc@ordstr#2\relax
    \edef#2{\@@fc@ordstr\ }%
  \fi
  \let\@@fc@ordstr#2\relax
  \edef#2{\@@fc@ordstr\@unitstring{\@strctr}}%
  \@strctr=#1\relax \@modulo{\@strctr}{100}%
  \let\@@fc@ordstr#2\relax
  \ifnum\@strctr=0\relax
    \edef#2{\@@fc@ordstr\ \@hundredth}%
  \else
    \edef#2{\@@fc@ordstr\ \@hundred}%
  \fi
\fi
\@strctr=#1\relax \@modulo{\@strctr}{100}%
\ifnum#1>100\relax
  \ifnum\@strctr>0\relax
    \let\@@fc@ordstr#2\relax
    \edef#2{\@@fc@ordstr\ \@andname\ }%
  \fi
\fi
\ifnum\@strctr>19\relax
  \@tmpstrctr=\@strctr
  \divide\@strctr by 10\relax
  \@modulo{\@tmpstrctr}{10}%
  \let\@@fc@ordstr#2\relax
  \ifnum\@tmpstrctr=0\relax
    \edef#2{\@@fc@ordstr\@tenthstring{\@strctr}}%
  \else
    \edef#2{\@@fc@ordstr\@tenstring{\@strctr}}%
  \fi
  \@strctr=#1\relax \@modulo{\@strctr}{10}%
  \ifnum\@strctr>0\relax
    \let\@@fc@ordstr#2\relax
    \edef#2{\@@fc@ordstr-\@unitthstring{\@strctr}}%
  \fi
\else
  \ifnum\@strctr<10\relax
    \ifnum\@strctr=0\relax
      \ifnum#1<100\relax
        \let\@@fc@ordstr#2\relax
        \edef#2{\@@fc@ordstr\@unitthstring{\@strctr}}%
      \fi
    \else
      \let\@@fc@ordstr#2\relax
      \edef#2{\@@fc@ordstr\@unitthstring{\@strctr}}%
    \fi
  \else
    \@modulo{\@strctr}{10}%
    \let\@@fc@ordstr#2\relax
    \edef#2{\@@fc@ordstr\@teenthstring{\@strctr}}%
  \fi
\fi
}
%    \end{macrocode}
% All lower case version. Again, the second argument must be a
% control sequence in which the resulting text is stored.
%    \begin{macrocode}
\DeclareRobustCommand{\@ordinalstringMenglish}[2]{%
\let\@unitthstring=\@@unitthstringenglish 
\let\@teenthstring=\@@teenthstringenglish 
\let\@tenthstring=\@@tenthstringenglish
\let\@unitstring=\@@unitstringenglish 
\let\@teenstring=\@@teenstringenglish
\let\@tenstring=\@@tenstringenglish
\def\@andname{and}%
\def\@hundred{hundred}\def\@thousand{thousand}%
\def\@hundredth{hundredth}\def\@thousandth{thousandth}%
\@@ordinalstringenglish{#1}{#2}}
%    \end{macrocode}
% No gender in English, so make feminine and neuter same as masculine:
%    \begin{macrocode}
\let\@ordinalstringFenglish=\@ordinalstringMenglish
\let\@ordinalstringNenglish=\@ordinalstringMenglish
%    \end{macrocode}
% First letter of each word in upper case:
%    \begin{macrocode}
\DeclareRobustCommand{\@OrdinalstringMenglish}[2]{%
\let\@unitthstring=\@@Unitthstringenglish
\let\@teenthstring=\@@Teenthstringenglish
\let\@tenthstring=\@@Tenthstringenglish
\let\@unitstring=\@@Unitstringenglish
\let\@teenstring=\@@Teenstringenglish
\let\@tenstring=\@@Tenstringenglish
\def\@andname{and}%
\def\@hundred{Hundred}\def\@thousand{Thousand}%
\def\@hundredth{Hundredth}\def\@thousandth{Thousandth}%
\@@ordinalstringenglish{#1}{#2}}
%    \end{macrocode}
% No gender in English, so make feminine and neuter same as masculine:
%    \begin{macrocode}
\let\@OrdinalstringFenglish=\@OrdinalstringMenglish
\let\@OrdinalstringNenglish=\@OrdinalstringMenglish
%    \end{macrocode}
%\iffalse
%    \begin{macrocode}
%</fc-english.def>
%    \end{macrocode}
%\fi
%\iffalse
%    \begin{macrocode}
%<*fc-french.def>
%    \end{macrocode}
%\fi
% \subsection{fc-french.def}
% French definitions
%    \begin{macrocode}
\ProvidesFile{fc-french.def}[2007/05/26]
%    \end{macrocode}
% Define macro that converts a number or count register (first
% argument) to an ordinal, and store the result in the second
% argument, which must be a control sequence. Masculine:
%    \begin{macrocode}
\newcommand*{\@ordinalMfrench}[2]{%
\iffmtord@abbrv
  \edef#2{\number#1\relax\noexpand\fmtord{e}}%
\else
  \ifnum#1=1\relax
    \edef#2{\number#1\relax\noexpand\fmtord{er}}%
  \else
    \edef#2{\number#1\relax\noexpand\fmtord{eme}}%
  \fi
\fi}
%    \end{macrocode}
% Feminine:
%    \begin{macrocode}
\newcommand*{\@ordinalFfrench}[2]{%
\iffmtord@abbrv
  \edef#2{\number#1\relax\noexpand\fmtord{e}}%
\else
  \ifnum#1=1\relax
     \edef#2{\number#1\relax\noexpand\fmtord{ere}}%
  \else
     \edef#2{\number#1\relax\noexpand\fmtord{eme}}%
  \fi
\fi}
%    \end{macrocode}
% Make neuter same as masculine:
%    \begin{macrocode}
\let\@ordinalNfrench\@ordinalMfrench
%    \end{macrocode}
% Textual representation of a number. To make it easier break it
% into units, tens and teens. First the units:
%   \begin{macrocode}
\newcommand*{\@@unitstringfrench}[1]{%
\ifcase#1\relax
zero%
\or un%
\or deux%
\or trois%
\or quatre%
\or cinq%
\or six%
\or sept%
\or huit%
\or neuf%
\fi
}
%    \end{macrocode}
% Feminine only changes for 1:
%    \begin{macrocode}
\newcommand*{\@@unitstringFfrench}[1]{%
\ifnum#1=1\relax
une%
\else\@@unitstringfrench{#1}%
\fi
}
%    \end{macrocode}
% Tens (this includes the Belgian and Swiss variants, special
% cases employed lower down.)
%    \begin{macrocode}
\newcommand*{\@@tenstringfrench}[1]{%
\ifcase#1\relax
\or dix%
\or vingt%
\or trente%
\or quarante%
\or cinquante%
\or soixante%
\or septente%
\or huitante%
\or nonente%
\or cent%
\fi
}
%    \end{macrocode}
% Teens:
%    \begin{macrocode}
\newcommand*{\@@teenstringfrench}[1]{%
\ifcase#1\relax
dix%
\or onze%
\or douze%
\or treize%
\or quatorze%
\or quinze%
\or seize%
\or dix-sept%
\or dix-huit%
\or dix-neuf%
\fi
}
%    \end{macrocode}
% Seventies are a special case, depending on dialect:
%    \begin{macrocode}
\newcommand*{\@@seventiesfrench}[1]{%
\@tenstring{6}%
\ifnum#1=1\relax
\ \@andname\ 
\else
-%
\fi
\@teenstring{#1}%
}
%    \end{macrocode}
% Eighties are a special case, depending on dialect:
%    \begin{macrocode}
\newcommand*{\@@eightiesfrench}[1]{%
\@unitstring{4}-\@tenstring{2}%
\ifnum#1>0
-\@unitstring{#1}%
\else
s%
\fi
}
%    \end{macrocode}
% Nineties are a special case, depending on dialect:
%    \begin{macrocode}
\newcommand*{\@@ninetiesfrench}[1]{%
\@unitstring{4}-\@tenstring{2}-\@teenstring{#1}%
}
%    \end{macrocode}
% Swiss seventies:
%    \begin{macrocode}
\newcommand*{\@@seventiesfrenchswiss}[1]{%
\@tenstring{7}%
\ifnum#1=1\ \@andname\ \fi
\ifnum#1>1-\fi
\ifnum#1>0\@unitstring{#1}\fi
}
%    \end{macrocode}
% Swiss eighties:
%    \begin{macrocode}
\newcommand*{\@@eightiesfrenchswiss}[1]{%
\@tenstring{8}%
\ifnum#1=1\ \@andname\ \fi
\ifnum#1>1-\fi
\ifnum#1>0\@unitstring{#1}\fi
}
%    \end{macrocode}
% Swiss nineties:
%    \begin{macrocode}
\newcommand*{\@@ninetiesfrenchswiss}[1]{%
\@tenstring{9}%
\ifnum#1=1\ \@andname\ \fi
\ifnum#1>1-\fi
\ifnum#1>0\@unitstring{#1}\fi
}
%    \end{macrocode}
% Units with initial letter in upper case:
%    \begin{macrocode}
\newcommand*{\@@Unitstringfrench}[1]{%
\ifcase#1\relax
Zero%
\or Un%
\or Deux%
\or Trois%
\or Quatre%
\or Cinq%
\or Six%
\or Sept%
\or Huit%
\or Neuf%
\fi
}
%    \end{macrocode}
% As above, but feminine:
%    \begin{macrocode}
\newcommand*{\@@UnitstringFfrench}[1]{%
\ifnum#1=1\relax
Une%
\else \@@Unitstringfrench{#1}%
\fi
}
%    \end{macrocode}
% Tens, with initial letter in upper case (includes Swiss and
% Belgian variants):
%    \begin{macrocode}
\newcommand*{\@@Tenstringfrench}[1]{%
\ifcase#1\relax
\or Dix%
\or Vingt%
\or Trente%
\or Quarante%
\or Cinquante%
\or Soixante%
\or Septente%
\or Huitante%
\or Nonente%
\or Cent%
\fi
}
%    \end{macrocode}
% Teens, with initial letter in upper case:
%    \begin{macrocode}
\newcommand*{\@@Teenstringfrench}[1]{%
\ifcase#1\relax
Dix%
\or Onze%
\or Douze%
\or Treize%
\or Quatorze%
\or Quinze%
\or Seize%
\or Dix-Sept%
\or Dix-Huit%
\or Dix-Neuf%
\fi
}
%    \end{macrocode}
% This has changed in version 1.09, so that it now stores the
% result in the second argument, but doesn't display anything.
% Since it only affects internal macros, it shouldn't affect
% documents created with older versions. (These internal macros
% are not defined for use in documents.) Firstly, the Swiss
% version:
%    \begin{macrocode}
\DeclareRobustCommand{\@numberstringMfrenchswiss}[2]{%
\let\@unitstring=\@@unitstringfrench
\let\@teenstring=\@@teenstringfrench
\let\@tenstring=\@@tenstringfrench
\let\@seventies=\@@seventiesfrenchswiss
\let\@eighties=\@@eightiesfrenchswiss
\let\@nineties=\@@ninetiesfrenchswiss
\def\@hundred{cent}\def\@thousand{mille}%
\def\@andname{et}%
\@@numberstringfrench{#1}{#2}}
%    \end{macrocode}
% Same as above, but for French as spoken in France:
%    \begin{macrocode}
\DeclareRobustCommand{\@numberstringMfrenchfrance}[2]{%
\let\@unitstring=\@@unitstringfrench
\let\@teenstring=\@@teenstringfrench
\let\@tenstring=\@@tenstringfrench
\let\@seventies=\@@seventiesfrench
\let\@eighties=\@@eightiesfrench
\let\@nineties=\@@ninetiesfrench
\def\@hundred{cent}\def\@thousand{mille}%
\def\@andname{et}%
\@@numberstringfrench{#1}{#2}}
%    \end{macrocode}
% Same as above, but for Belgian dialect:
%    \begin{macrocode}
\DeclareRobustCommand{\@numberstringMfrenchbelgian}[2]{%
\let\@unitstring=\@@unitstringfrench
\let\@teenstring=\@@teenstringfrench
\let\@tenstring=\@@tenstringfrench
\let\@seventies=\@@seventiesfrenchswiss
\let\@eighties=\@@eightiesfrench
\let\@nineties=\@@ninetiesfrench
\def\@hundred{cent}\def\@thousand{mille}%
\def\@andname{et}%
\@@numberstringfrench{#1}{#2}}
%    \end{macrocode}
% Set default dialect:
%    \begin{macrocode}
\let\@numberstringMfrench=\@numberstringMfrenchfrance
%    \end{macrocode}
% As above, but for feminine version. Swiss:
%    \begin{macrocode}
\DeclareRobustCommand{\@numberstringFfrenchswiss}[2]{%
\let\@unitstring=\@@unitstringFfrench
\let\@teenstring=\@@teenstringfrench
\let\@tenstring=\@@tenstringfrench
\let\@seventies=\@@seventiesfrenchswiss
\let\@eighties=\@@eightiesfrenchswiss
\let\@nineties=\@@ninetiesfrenchswiss
\def\@hundred{cent}\def\@thousand{mille}%
\def\@andname{et}%
\@@numberstringfrench{#1}{#2}}
%    \end{macrocode}
% French:
%    \begin{macrocode}
\DeclareRobustCommand{\@numberstringFfrenchfrance}[2]{%
\let\@unitstring=\@@unitstringFfrench
\let\@teenstring=\@@teenstringfrench
\let\@tenstring=\@@tenstringfrench
\let\@seventies=\@@seventiesfrench
\let\@eighties=\@@eightiesfrench
\let\@nineties=\@@ninetiesfrench
\def\@hundred{cent}\def\@thousand{mille}%
\def\@andname{et}%
\@@numberstringfrench{#1}{#2}}
%    \end{macrocode}
% Belgian:
%    \begin{macrocode}
\DeclareRobustCommand{\@numberstringFfrenchbelgian}[2]{%
\let\@unitstring=\@@unitstringFfrench
\let\@teenstring=\@@teenstringfrench
\let\@tenstring=\@@tenstringfrench
\let\@seventies=\@@seventiesfrenchswiss
\let\@eighties=\@@eightiesfrench
\let\@nineties=\@@ninetiesfrench
\def\@hundred{cent}\def\@thousand{mille}%
\def\@andname{et}%
\@@numberstringfrench{#1}{#2}}
%    \end{macrocode}
% Set default dialect:
%    \begin{macrocode}
\let\@numberstringFfrench=\@numberstringFfrenchfrance
%    \end{macrocode}
% Make neuter same as masculine:
%    \begin{macrocode}
\let\@ordinalstringNfrench\@ordinalstringMfrench
%    \end{macrocode}
% As above, but with initial letter in upper case. Swiss (masculine):
%    \begin{macrocode}
\DeclareRobustCommand{\@NumberstringMfrenchswiss}[2]{%
\let\@unitstring=\@@Unitstringfrench
\let\@teenstring=\@@Teenstringfrench
\let\@tenstring=\@@Tenstringfrench
\let\@seventies=\@@seventiesfrenchswiss
\let\@eighties=\@@eightiesfrenchswiss
\let\@nineties=\@@ninetiesfrenchswiss
\def\@hundred{Cent}\def\@thousand{Mille}%
\def\@andname{et}%
\@@numberstringfrench{#1}{#2}}
%    \end{macrocode}
% French:
%    \begin{macrocode}
\DeclareRobustCommand{\@NumberstringMfrenchfrance}[2]{%
\let\@unitstring=\@@Unitstringfrench
\let\@teenstring=\@@Teenstringfrench
\let\@tenstring=\@@Tenstringfrench
\let\@seventies=\@@seventiesfrench
\let\@eighties=\@@eightiesfrench
\let\@nineties=\@@ninetiesfrench
\def\@hundred{Cent}\def\@thousand{Mille}%
\def\@andname{et}%
\@@numberstringfrench{#1}{#2}}
%    \end{macrocode}
% Belgian:
%    \begin{macrocode}
\DeclareRobustCommand{\@NumberstringMfrenchbelgian}[2]{%
\let\@unitstring=\@@Unitstringfrench
\let\@teenstring=\@@Teenstringfrench
\let\@tenstring=\@@Tenstringfrench
\let\@seventies=\@@seventiesfrenchswiss
\let\@eighties=\@@eightiesfrench
\let\@nineties=\@@ninetiesfrench
\def\@hundred{Cent}\def\@thousand{Mille}%
\def\@andname{et}%
\@@numberstringfrench{#1}{#2}}
%    \end{macrocode}
% Set default dialect:
%    \begin{macrocode}
\let\@NumberstringMfrench=\@NumberstringMfrenchfrance
%    \end{macrocode}
% As above, but feminine. Swiss:
%    \begin{macrocode}
\DeclareRobustCommand{\@NumberstringFfrenchswiss}[2]{%
\let\@unitstring=\@@UnitstringFfrench
\let\@teenstring=\@@Teenstringfrench
\let\@tenstring=\@@Tenstringfrench
\let\@seventies=\@@seventiesfrenchswiss
\let\@eighties=\@@eightiesfrenchswiss
\let\@nineties=\@@ninetiesfrenchswiss
\def\@hundred{Cent}\def\@thousand{Mille}%
\def\@andname{et}%
\@@numberstringfrench{#1}{#2}}
%    \end{macrocode}
% French (feminine):
%    \begin{macrocode}
\DeclareRobustCommand{\@NumberstringFfrenchfrance}[2]{%
\let\@unitstring=\@@UnitstringFfrench
\let\@teenstring=\@@Teenstringfrench
\let\@tenstring=\@@Tenstringfrench
\let\@seventies=\@@seventiesfrench
\let\@eighties=\@@eightiesfrench
\let\@nineties=\@@ninetiesfrench
\def\@hundred{Cent}\def\@thousand{Mille}%
\def\@andname{et}%
\@@numberstringfrench{#1}{#2}}
%    \end{macrocode}
% Belgian (feminine):
%    \begin{macrocode}
\DeclareRobustCommand{\@NumberstringFfrenchbelgian}[2]{%
\let\@unitstring=\@@UnitstringFfrench
\let\@teenstring=\@@Teenstringfrench
\let\@tenstring=\@@Tenstringfrench
\let\@seventies=\@@seventiesfrenchswiss
\let\@eighties=\@@eightiesfrench
\let\@nineties=\@@ninetiesfrench
\def\@hundred{Cent}\def\@thousand{Mille}%
\def\@andname{et}%
\@@numberstringfrench{#1}{#2}}
%    \end{macrocode}
% Set default dialect:
%    \begin{macrocode}
\let\@NumberstringFfrench=\@NumberstringFfrenchfrance
%    \end{macrocode}
% Make neuter same as masculine:
%    \begin{macrocode}
\let\@NumberstringNfrench\@NumberstringMfrench
%    \end{macrocode}
% Again, as from version 1.09, this has been changed to take
% two arguments, where the second argument is a control
% sequence, and nothing is displayed. Store textual representation
% of an ordinal in the given control sequence. Swiss dialect (masculine):
%    \begin{macrocode}
\DeclareRobustCommand{\@ordinalstringMfrenchswiss}[2]{%
\ifnum#1=1\relax
\def#2{premier}%
\else
\let\@unitthstring=\@@unitthstringfrench
\let\@unitstring=\@@unitstringfrench
\let\@teenthstring=\@@teenthstringfrench
\let\@teenstring=\@@teenstringfrench
\let\@tenthstring=\@@tenthstringfrench
\let\@tenstring=\@@tenstringfrench
\let\@seventieths=\@@seventiethsfrenchswiss
\let\@eightieths=\@@eightiethsfrenchswiss
\let\@ninetieths=\@@ninetiethsfrenchswiss
\let\@seventies=\@@seventiesfrenchswiss
\let\@eighties=\@@eightiesfrenchswiss
\let\@nineties=\@@ninetiesfrenchswiss
\def\@hundredth{centi\`eme}\def\@hundred{cent}%
\def\@thousandth{mili\`eme}\def\@thousand{mille}%
\def\@andname{et}%
\@@ordinalstringfrench{#1}{#2}%
\fi}
%    \end{macrocode}
% French (masculine):
%    \begin{macrocode}
\DeclareRobustCommand{\@ordinalstringMfrenchfrance}[2]{%
\ifnum#1=1\relax
\def#2{premier}%
\else
\let\@unitthstring=\@@unitthstringfrench
\let\@unitstring=\@@unitstringfrench
\let\@teenthstring=\@@teenthstringfrench
\let\@teenstring=\@@teenstringfrench
\let\@tenthstring=\@@tenthstringfrench
\let\@tenstring=\@@tenstringfrench
\let\@seventieths=\@@seventiethsfrench
\let\@eightieths=\@@eightiethsfrench
\let\@ninetieths=\@@ninetiethsfrench
\let\@seventies=\@@seventiesfrench
\let\@eighties=\@@eightiesfrench
\let\@nineties=\@@ninetiesfrench
\let\@teenstring=\@@teenstringfrench
\def\@hundredth{centi\`eme}\def\@hundred{cent}%
\def\@thousandth{mili\`eme}\def\@thousand{mille}%
\def\@andname{et}%
\@@ordinalstringfrench{#1}{#2}%
\fi}
%    \end{macrocode}
% Belgian dialect (masculine):
%    \begin{macrocode}
\DeclareRobustCommand{\@ordinalstringMfrenchbelgian}[2]{%
\ifnum#1=1\relax
\def#2{premier}%
\else
\let\@unitthstring=\@@unitthstringfrench
\let\@unitstring=\@@unitstringfrench
\let\@teenthstring=\@@teenthstringfrench
\let\@teenstring=\@@teenstringfrench
\let\@tenthstring=\@@tenthstringfrench
\let\@tenstring=\@@tenstringfrench
\let\@seventieths=\@@seventiethsfrenchswiss
\let\@eightieths=\@@eightiethsfrench
\let\@ninetieths=\@@ninetiethsfrenchswiss
\let\@seventies=\@@seventiesfrench
\let\@eighties=\@@eightiesfrench
\let\@nineties=\@@ninetiesfrench
\let\@teenstring=\@@teenstringfrench
\def\@hundredth{centi\`eme}\def\@hundred{cent}%
\def\@thousandth{mili\`eme}\def\@thousand{mille}%
\def\@andname{et}%
\@@ordinalstringfrench{#1}{#2}%
\fi}
%    \end{macrocode}
% Set up default dialect:
%    \begin{macrocode}
\let\@ordinalstringMfrench=\@ordinalstringMfrenchfrance
%    \end{macrocode}
% As above, but feminine. Swiss:
%    \begin{macrocode}
\DeclareRobustCommand{\@ordinalstringFfrenchswiss}[2]{%
\ifnum#1=1\relax
\def#2{premi\`ere}%
\else
\let\@unitthstring=\@@unitthstringfrench
\let\@unitstring=\@@unitstringFfrench
\let\@teenthstring=\@@teenthstringfrench
\let\@teenstring=\@@teenstringfrench
\let\@tenthstring=\@@tenthstringfrench
\let\@tenstring=\@@tenstringfrench
\let\@seventieths=\@@seventiethsfrenchswiss
\let\@eightieths=\@@eightiethsfrenchswiss
\let\@ninetieths=\@@ninetiethsfrenchswiss
\let\@seventies=\@@seventiesfrenchswiss
\let\@eighties=\@@eightiesfrenchswiss
\let\@nineties=\@@ninetiesfrenchswiss
\def\@hundredth{centi\`eme}\def\@hundred{cent}%
\def\@thousandth{mili\`eme}\def\@thousand{mille}%
\def\@andname{et}%
\@@ordinalstringfrench{#1}{#2}%
\fi}
%    \end{macrocode}
% French (feminine):
%    \begin{macrocode}
\DeclareRobustCommand{\@ordinalstringFfrenchfrance}[2]{%
\ifnum#1=1\relax
\def#2{premi\`ere}%
\else
\let\@unitthstring=\@@unitthstringfrench
\let\@unitstring=\@@unitstringFfrench
\let\@teenthstring=\@@teenthstringfrench
\let\@teenstring=\@@teenstringfrench
\let\@tenthstring=\@@tenthstringfrench
\let\@tenstring=\@@tenstringfrench
\let\@seventieths=\@@seventiethsfrench
\let\@eightieths=\@@eightiethsfrench
\let\@ninetieths=\@@ninetiethsfrench
\let\@seventies=\@@seventiesfrench
\let\@eighties=\@@eightiesfrench
\let\@nineties=\@@ninetiesfrench
\let\@teenstring=\@@teenstringfrench
\def\@hundredth{centi\`eme}\def\@hundred{cent}%
\def\@thousandth{mili\`eme}\def\@thousand{mille}%
\def\@andname{et}%
\@@ordinalstringfrench{#1}{#2}%
\fi}
%    \end{macrocode}
% Belgian (feminine):
%    \begin{macrocode}
\DeclareRobustCommand{\@ordinalstringFfrenchbelgian}[2]{%
\ifnum#1=1\relax
\def#2{premi\`ere}%
\else
\let\@unitthstring=\@@unitthstringfrench
\let\@unitstring=\@@unitstringFfrench
\let\@teenthstring=\@@teenthstringfrench
\let\@teenstring=\@@teenstringfrench
\let\@tenthstring=\@@tenthstringfrench
\let\@tenstring=\@@tenstringfrench
\let\@seventieths=\@@seventiethsfrenchswiss
\let\@eightieths=\@@eightiethsfrench
\let\@ninetieths=\@@ninetiethsfrench
\let\@seventies=\@@seventiesfrench
\let\@eighties=\@@eightiesfrench
\let\@nineties=\@@ninetiesfrench
\let\@teenstring=\@@teenstringfrench
\def\@hundredth{centi\`eme}\def\@hundred{cent}%
\def\@thousandth{mili\`eme}\def\@thousand{mille}%
\def\@andname{et}%
\@@ordinalstringfrench{#1}{#2}%
\fi}
%    \end{macrocode}
% Set up default dialect:
%    \begin{macrocode}
\let\@ordinalstringFfrench=\@ordinalstringFfrenchfrance
%    \end{macrocode}
% Make neuter same as masculine:
%    \begin{macrocode}
\let\@ordinalstringNfrench\@ordinalstringMfrench
%    \end{macrocode}
% As above, but with initial letters in upper case. Swiss (masculine):
%    \begin{macrocode}
\DeclareRobustCommand{\@OrdinalstringMfrenchswiss}[2]{%
\ifnum#1=1\relax
\def#2{Premi\`ere}%
\else
\let\@unitthstring=\@@Unitthstringfrench
\let\@unitstring=\@@Unitstringfrench
\let\@teenthstring=\@@Teenthstringfrench
\let\@teenstring=\@@Teenstringfrench
\let\@tenthstring=\@@Tenthstringfrench
\let\@tenstring=\@@Tenstringfrench
\let\@seventieths=\@@seventiethsfrenchswiss
\let\@eightieths=\@@eightiethsfrenchswiss
\let\@ninetieths=\@@ninetiethsfrenchswiss
\let\@seventies=\@@seventiesfrenchswiss
\let\@eighties=\@@eightiesfrenchswiss
\let\@nineties=\@@ninetiesfrenchswiss
\def\@hundredth{Centi\`eme}\def\@hundred{Cent}%
\def\@thousandth{Mili\`eme}\def\@thousand{Mille}%
\def\@andname{et}%
\@@ordinalstringfrench{#1}{#2}%
\fi}
%    \end{macrocode}
% French (masculine):
%    \begin{macrocode}
\DeclareRobustCommand{\@OrdinalstringMfrenchfrance}[2]{%
\ifnum#1=1\relax
\def#2{Premi\`ere}%
\else
\let\@unitthstring=\@@Unitthstringfrench
\let\@unitstring=\@@Unitstringfrench
\let\@teenthstring=\@@Teenthstringfrench
\let\@teenstring=\@@Teenstringfrench
\let\@tenthstring=\@@Tenthstringfrench
\let\@tenstring=\@@Tenstringfrench
\let\@seventieths=\@@seventiethsfrench
\let\@eightieths=\@@eightiethsfrench
\let\@ninetieths=\@@ninetiethsfrench
\let\@seventies=\@@seventiesfrench
\let\@eighties=\@@eightiesfrench
\let\@nineties=\@@ninetiesfrench
\let\@teenstring=\@@Teenstringfrench
\def\@hundredth{Centi\`eme}\def\@hundred{Cent}%
\def\@thousandth{Mili\`eme}\def\@thousand{Mille}%
\def\@andname{et}%
\@@ordinalstringfrench{#1}{#2}%
\fi}
%    \end{macrocode}
% Belgian (masculine):
%    \begin{macrocode}
\DeclareRobustCommand{\@OrdinalstringMfrenchbelgian}[2]{%
\ifnum#1=1\relax
\def#2{Premi\`ere}%
\else
\let\@unitthstring=\@@Unitthstringfrench
\let\@unitstring=\@@Unitstringfrench
\let\@teenthstring=\@@Teenthstringfrench
\let\@teenstring=\@@Teenstringfrench
\let\@tenthstring=\@@Tenthstringfrench
\let\@tenstring=\@@Tenstringfrench
\let\@seventieths=\@@seventiethsfrenchswiss
\let\@eightieths=\@@eightiethsfrench
\let\@ninetieths=\@@ninetiethsfrench
\let\@seventies=\@@seventiesfrench
\let\@eighties=\@@eightiesfrench
\let\@nineties=\@@ninetiesfrench
\let\@teenstring=\@@Teenstringfrench
\def\@hundredth{Centi\`eme}\def\@hundred{Cent}%
\def\@thousandth{Mili\`eme}\def\@thousand{Mille}%
\def\@andname{et}%
\@@ordinalstringfrench{#1}{#2}%
\fi}
%    \end{macrocode}
% Set up default dialect:
%    \begin{macrocode}
\let\@OrdinalstringMfrench=\@OrdinalstringMfrenchfrance
%    \end{macrocode}
% As above, but feminine form. Swiss:
%    \begin{macrocode}
\DeclareRobustCommand{\@OrdinalstringFfrenchswiss}[2]{%
\ifnum#1=1\relax
\def#2{Premi\`ere}%
\else
\let\@unitthstring=\@@Unitthstringfrench
\let\@unitstring=\@@UnitstringFfrench
\let\@teenthstring=\@@Teenthstringfrench
\let\@teenstring=\@@Teenstringfrench
\let\@tenthstring=\@@Tenthstringfrench
\let\@tenstring=\@@Tenstringfrench
\let\@seventieths=\@@seventiethsfrenchswiss
\let\@eightieths=\@@eightiethsfrenchswiss
\let\@ninetieths=\@@ninetiethsfrenchswiss
\let\@seventies=\@@seventiesfrenchswiss
\let\@eighties=\@@eightiesfrenchswiss
\let\@nineties=\@@ninetiesfrenchswiss
\def\@hundredth{Centi\`eme}\def\@hundred{Cent}%
\def\@thousandth{Mili\`eme}\def\@thousand{Mille}%
\def\@andname{et}%
\@@ordinalstringfrench{#1}{#2}%
\fi}
%    \end{macrocode}
% French (feminine):
%    \begin{macrocode}
\DeclareRobustCommand{\@OrdinalstringFfrenchfrance}[2]{%
\ifnum#1=1\relax
\def#2{Premi\`ere}%
\else
\let\@unitthstring=\@@Unitthstringfrench
\let\@unitstring=\@@UnitstringFfrench
\let\@teenthstring=\@@Teenthstringfrench
\let\@teenstring=\@@Teenstringfrench
\let\@tenthstring=\@@Tenthstringfrench
\let\@tenstring=\@@Tenstringfrench
\let\@seventieths=\@@seventiethsfrench
\let\@eightieths=\@@eightiethsfrench
\let\@ninetieths=\@@ninetiethsfrench
\let\@seventies=\@@seventiesfrench
\let\@eighties=\@@eightiesfrench
\let\@nineties=\@@ninetiesfrench
\let\@teenstring=\@@Teenstringfrench
\def\@hundredth{Centi\`eme}\def\@hundred{Cent}%
\def\@thousandth{Mili\`eme}\def\@thousand{Mille}%
\def\@andname{et}%
\@@ordinalstringfrench{#1}{#2}%
\fi}
%    \end{macrocode}
% Belgian (feminine):
%    \begin{macrocode}
\DeclareRobustCommand{\@OrdinalstringFfrenchbelgian}[2]{%
\ifnum#1=1\relax
\def#2{Premi\`ere}%
\else
\let\@unitthstring=\@@Unitthstringfrench
\let\@unitstring=\@@UnitstringFfrench
\let\@teenthstring=\@@Teenthstringfrench
\let\@teenstring=\@@Teenstringfrench
\let\@tenthstring=\@@Tenthstringfrench
\let\@tenstring=\@@Tenstringfrench
\let\@seventieths=\@@seventiethsfrenchswiss
\let\@eightieths=\@@eightiethsfrench
\let\@ninetieths=\@@ninetiethsfrench
\let\@seventies=\@@seventiesfrench
\let\@eighties=\@@eightiesfrench
\let\@nineties=\@@ninetiesfrench
\let\@teenstring=\@@Teenstringfrench
\def\@hundredth{Centi\`eme}\def\@hundred{Cent}%
\def\@thousandth{Mili\`eme}\def\@thousand{Mille}%
\def\@andname{et}%
\@@ordinalstringfrench{#1}{#2}%
\fi}
%    \end{macrocode}
% Set up default dialect:
%    \begin{macrocode}
\let\@OrdinalstringFfrench=\@OrdinalstringFfrenchfrance
%    \end{macrocode}
% Make neuter same as masculine:
%    \begin{macrocode}
\let\@OrdinalstringNfrench\@OrdinalstringMfrench
%    \end{macrocode}
% In order to convert numbers into textual ordinals, need
% to break it up into units, tens and teens. First the units.
% The argument must be a number or count register between 0
% and 9.
%    \begin{macrocode}
\newcommand*{\@@unitthstringfrench}[1]{%
\ifcase#1\relax
zero%
\or uni\`eme%
\or deuxi\`eme%
\or troisi\`eme%
\or quatri\`eme%
\or cinqui\`eme%
\or sixi\`eme%
\or septi\`eme%
\or huiti\`eme%
\or neuvi\`eme%
\fi
}
%    \end{macrocode}
% Tens (includes Swiss and Belgian variants, special cases are
% dealt with later.)
%    \begin{macrocode}
\newcommand*{\@@tenthstringfrench}[1]{%
\ifcase#1\relax
\or dixi\`eme%
\or vingti\`eme%
\or trentri\`eme%
\or quaranti\`eme%
\or cinquanti\`eme%
\or soixanti\`eme%
\or septenti\`eme%
\or huitanti\`eme%
\or nonenti\`eme%
\fi
}
%    \end{macrocode}
% Teens:
%    \begin{macrocode}
\newcommand*{\@@teenthstringfrench}[1]{%
\ifcase#1\relax
dixi\`eme%
\or onzi\`eme%
\or douzi\`eme%
\or treizi\`eme%
\or quatorzi\`eme%
\or quinzi\`eme%
\or seizi\`eme%
\or dix-septi\`eme%
\or dix-huiti\`eme%
\or dix-neuvi\`eme%
\fi
}
%    \end{macrocode}
% Seventies vary depending on dialect. Swiss:
%    \begin{macrocode}
\newcommand*{\@@seventiethsfrenchswiss}[1]{%
\ifcase#1\relax
\@tenthstring{7}%
\or
\@tenstring{7} \@andname\ \@unitthstring{1}%
\else
\@tenstring{7}-\@unitthstring{#1}%
\fi}
%    \end{macrocode}
% Eighties vary depending on dialect. Swiss:
%    \begin{macrocode}
\newcommand*{\@@eightiethsfrenchswiss}[1]{%
\ifcase#1\relax
\@tenthstring{8}%
\or
\@tenstring{8} \@andname\ \@unitthstring{1}%
\else
\@tenstring{8}-\@unitthstring{#1}%
\fi}
%    \end{macrocode}
% Nineties vary depending on dialect. Swiss:
%    \begin{macrocode}
\newcommand*{\@@ninetiethsfrenchswiss}[1]{%
\ifcase#1\relax
\@tenthstring{9}%
\or
\@tenstring{9} \@andname\ \@unitthstring{1}%
\else
\@tenstring{9}-\@unitthstring{#1}%
\fi}
%    \end{macrocode}
% French (as spoken in France) version:
%    \begin{macrocode}
\newcommand*{\@@seventiethsfrench}[1]{%
\ifnum#1=0\relax
\@tenstring{6}%
-%
\else
\@tenstring{6}%
\ \@andname\ 
\fi
\@teenthstring{#1}%
}
%    \end{macrocode}
% Eighties (as spoken in France):
%    \begin{macrocode}
\newcommand*{\@@eightiethsfrench}[1]{%
\ifnum#1>0\relax
\@unitstring{4}-\@tenstring{2}%
-\@unitthstring{#1}%
\else
\@unitstring{4}-\@tenthstring{2}%
\fi
}
%    \end{macrocode}
% Nineties (as spoken in France):
%    \begin{macrocode}
\newcommand*{\@@ninetiethsfrench}[1]{%
\@unitstring{4}-\@tenstring{2}-\@teenthstring{#1}%
}
%    \end{macrocode}
% As above, but with initial letter in upper case. Units:
%    \begin{macrocode}
\newcommand*{\@@Unitthstringfrench}[1]{%
\ifcase#1\relax
Zero%
\or Uni\`eme%
\or Deuxi\`eme%
\or Troisi\`eme%
\or Quatri\`eme%
\or Cinqui\`eme%
\or Sixi\`eme%
\or Septi\`eme%
\or Huiti\`eme%
\or Neuvi\`eme%
\fi
}
%    \end{macrocode}
% Tens (includes Belgian and Swiss variants):
%    \begin{macrocode}
\newcommand*{\@@Tenthstringfrench}[1]{%
\ifcase#1\relax
\or Dixi\`eme%
\or Vingti\`eme%
\or Trentri\`eme%
\or Quaranti\`eme%
\or Cinquanti\`eme%
\or Soixanti\`eme%
\or Septenti\`eme%
\or Huitanti\`eme%
\or Nonenti\`eme%
\fi
}
%    \end{macrocode}
% Teens:
%    \begin{macrocode}
\newcommand*{\@@Teenthstringfrench}[1]{%
\ifcase#1\relax
Dixi\`eme%
\or Onzi\`eme%
\or Douzi\`eme%
\or Treizi\`eme%
\or Quatorzi\`eme%
\or Quinzi\`eme%
\or Seizi\`eme%
\or Dix-Septi\`eme%
\or Dix-Huiti\`eme%
\or Dix-Neuvi\`eme%
\fi
}
%    \end{macrocode}
% Store textual representation of number (first argument) in given control
% sequence (second argument).
%    \begin{macrocode}
\newcommand*{\@@numberstringfrench}[2]{%
\ifnum#1>99999
\PackageError{fmtcount}{Out of range}%
{This macro only works for values less than 100000}%
\else
\ifnum#1<0
\PackageError{fmtcount}{Negative numbers not permitted}%
{This macro does not work for negative numbers, however
you can try typing "minus" first, and then pass the modulus of
this number}%
\fi
\fi
\def#2{}%
\@strctr=#1\relax \divide\@strctr by 1000\relax
\ifnum\@strctr>9\relax
% #1 is greater or equal to 10000
  \@tmpstrctr=\@strctr
  \divide\@strctr by 10\relax
  \ifnum\@strctr>1\relax
    \ifthenelse{\(\@strctr>6\)\and\(\@strctr<10\)}{%
      \@modulo{\@tmpstrctr}{10}%
      \ifnum\@strctr<8\relax
        \let\@@fc@numstr#2\relax
        \edef#2{\@@fc@numstr\@seventies{\@tmpstrctr}}%
      \else
        \ifnum\@strctr<9\relax
          \let\@@fc@numstr#2\relax
          \edef#2{\@@fc@numstr\@eighties{\@tmpstrctr}}%
        \else
          \ifnum\@strctr<10\relax
             \let\@@fc@numstr#2\relax
             \edef#2{\@@fc@numstr\@nineties{\@tmpstrctr}}%
          \fi
        \fi
      \fi
    }{%
      \let\@@fc@numstr#2\relax
      \edef#2{\@@fc@numstr\@tenstring{\@strctr}}%
      \@strctr=#1\relax
      \divide\@strctr by 1000\relax
      \@modulo{\@strctr}{10}%
      \ifnum\@strctr>0\relax
        \let\@@fc@numstr#2\relax
        \edef#2{\@@fc@numstr\ \@unitstring{\@strctr}}%
      \fi
    }%
  \else
    \@strctr=#1\relax
    \divide\@strctr by 1000
    \@modulo{\@strctr}{10}%
    \let\@@fc@numstr#2\relax
    \edef#2{\@@fc@numstr\@teenstring{\@strctr}}%
  \fi
  \let\@@fc@numstr#2\relax
  \edef#2{\@@fc@numstr\ \@thousand}%
\else
  \ifnum\@strctr>0\relax 
    \ifnum\@strctr>1\relax
      \let\@@fc@numstr#2\relax
      \edef#2{\@@fc@numstr\@unitstring{\@strctr}\ }%
    \fi
    \let\@@fc@numstr#2\relax
    \edef#2{\@@fc@numstr\@thousand}%
  \fi
\fi
\@strctr=#1\relax \@modulo{\@strctr}{1000}%
\divide\@strctr by 100
\ifnum\@strctr>0\relax
  \ifnum#1>1000\relax
    \let\@@fc@numstr#2\relax
    \edef#2{\@@fc@numstr\ }%
  \fi
  \@tmpstrctr=#1\relax
  \@modulo{\@tmpstrctr}{1000}\relax
  \ifnum\@tmpstrctr=100\relax
    \let\@@fc@numstr#2\relax
    \edef#2{\@@fc@numstr\@tenstring{10}}%
  \else
    \ifnum\@strctr>1\relax
      \let\@@fc@numstr#2\relax
      \edef#2{\@@fc@numstr\@unitstring{\@strctr}\ }%
    \fi
    \let\@@fc@numstr#2\relax
    \edef#2{\@@fc@numstr\@hundred}%
  \fi
\fi
\@strctr=#1\relax \@modulo{\@strctr}{100}%
%\@tmpstrctr=#1\relax
%\divide\@tmpstrctr by 100\relax
\ifnum#1>100\relax
  \ifnum\@strctr>0\relax
    \let\@@fc@numstr#2\relax
    \edef#2{\@@fc@numstr\ }%
  \else
    \ifnum\@tmpstrctr>0\relax
       \let\@@fc@numstr#2\relax
       \edef#2{\@@fc@numstr s}%
    \fi%
  \fi
\fi
\ifnum\@strctr>19\relax
  \@tmpstrctr=\@strctr
  \divide\@strctr by 10\relax
  \ifthenelse{\@strctr>6}{%
    \@modulo{\@tmpstrctr}{10}%
    \ifnum\@strctr<8\relax
      \let\@@fc@numstr#2\relax
      \edef#2{\@@fc@numstr\@seventies{\@tmpstrctr}}%
    \else
      \ifnum\@strctr<9\relax
        \let\@@fc@numstr#2\relax
        \edef#2{\@@fc@numstr\@eighties{\@tmpstrctr}}%
      \else
        \let\@@fc@numstr#2\relax
        \edef#2{\@@fc@numstr\@nineties{\@tmpstrctr}}%
      \fi
    \fi
  }{%
    \let\@@fc@numstr#2\relax
    \edef#2{\@@fc@numstr\@tenstring{\@strctr}}%
    \@strctr=#1\relax \@modulo{\@strctr}{10}%
    \ifnum\@strctr>0\relax
      \let\@@fc@numstr#2\relax
      \ifnum\@strctr=1\relax
         \edef#2{\@@fc@numstr\ \@andname\ }%
      \else
         \edef#2{\@@fc@numstr-}%
      \fi
      \let\@@fc@numstr#2\relax
      \edef#2{\@@fc@numstr\@unitstring{\@strctr}}%
    \fi
  }%
\else
  \ifnum\@strctr<10\relax
    \ifnum\@strctr=0\relax
      \ifnum#1<100\relax
        \let\@@fc@numstr#2\relax
        \edef#2{\@@fc@numstr\@unitstring{\@strctr}}%
      \fi
    \else%(>0,<10)
      \let\@@fc@numstr#2\relax
      \edef#2{\@@fc@numstr\@unitstring{\@strctr}}%
    \fi
  \else%>10
    \@modulo{\@strctr}{10}%
    \let\@@fc@numstr#2\relax
    \edef#2{\@@fc@numstr\@teenstring{\@strctr}}%
  \fi
\fi
}
%    \end{macrocode}
% Store textual representation of an ordinal (from number 
% specified in first argument) in given control
% sequence (second argument).
%    \begin{macrocode}
\newcommand*{\@@ordinalstringfrench}[2]{%
\ifnum#1>99999
\PackageError{fmtcount}{Out of range}%
{This macro only works for values less than 100000}%
\else
\ifnum#1<0
\PackageError{fmtcount}{Negative numbers not permitted}%
{This macro does not work for negative numbers, however
you can try typing "minus" first, and then pass the modulus of
this number}%
\fi
\fi
\def#2{}%
\@strctr=#1\relax \divide\@strctr by 1000\relax
\ifnum\@strctr>9
% #1 is greater or equal to 10000
  \@tmpstrctr=\@strctr
  \divide\@strctr by 10\relax
  \ifnum\@strctr>1\relax
    \ifthenelse{\@strctr>6}{%
      \@modulo{\@tmpstrctr}{10}%
      \ifnum\@strctr=7\relax
        \let\@@fc@ordstr#2\relax
        \edef#2{\@@fc@ordstr\@seventies{\@tmpstrctr}}%
      \else
        \ifnum\@strctr=8\relax
          \let\@@fc@ordstr#2\relax
          \edef#2{\@@fc@ordstr\@eighties{\@tmpstrctr}}%
        \else
          \let\@@fc@ordstr#2\relax
          \edef#2{\@@fc@ordstr\@nineties{\@tmpstrctr}}%
        \fi
      \fi
    }{%
      \let\@@fc@ordstr#2\relax
      \edef#2{\@@fc@ordstr\@tenstring{\@strctr}}%
      \@strctr=#1\relax
      \divide\@strctr by 1000\relax
      \@modulo{\@strctr}{10}%
      \ifnum\@strctr=1\relax
         \let\@@fc@ordstr#2\relax
         \edef#2{\@@fc@ordstr\ \@andname}%
      \fi
      \ifnum\@strctr>0\relax
         \let\@@fc@ordstr#2\relax
         \edef#2{\@@fc@ordstr\ \@unitstring{\@strctr}}%
      \fi
    }%
  \else
    \@strctr=#1\relax
    \divide\@strctr by 1000\relax
    \@modulo{\@strctr}{10}%
    \let\@@fc@ordstr#2\relax
    \edef#2{\@@fc@ordstr\@teenstring{\@strctr}}%
  \fi
  \@strctr=#1\relax \@modulo{\@strctr}{1000}%
  \ifnum\@strctr=0\relax
    \let\@@fc@ordstr#2\relax
    \edef#2{\@@fc@ordstr\ \@thousandth}%
  \else
    \let\@@fc@ordstr#2\relax
    \edef#2{\@@fc@ordstr\ \@thousand}%
  \fi
\else
  \ifnum\@strctr>0\relax
    \let\@@fc@ordstr#2\relax
    \edef#2{\@@fc@ordstr\@unitstring{\@strctr}}%
    \@strctr=#1\relax \@modulo{\@strctr}{1000}%
    \ifnum\@strctr=0\relax
      \let\@@fc@ordstr#2\relax
      \edef#2{\@@fc@ordstr\ \@thousandth}%
    \else
      \let\@@fc@ordstr#2\relax
      \edef#2{\@@fc@ordstr\ \@thousand}%
    \fi
  \fi
\fi
\@strctr=#1\relax \@modulo{\@strctr}{1000}%
\divide\@strctr by 100\relax
\ifnum\@strctr>0\relax
  \ifnum#1>1000\relax
    \let\@@fc@ordstr#2\relax
    \edef#2{\@@fc@ordstr\ }%
  \fi
  \let\@@fc@ordstr#2\relax
  \edef#2{\@@fc@ordstr\@unitstring{\@strctr}}%
  \@strctr=#1\relax \@modulo{\@strctr}{100}%
  \let\@@fc@ordstr#2\relax
  \ifnum\@strctr=0\relax
    \edef#2{\@@fc@ordstr\ \@hundredth}%
  \else
    \edef#2{\@@fc@ordstr\ \@hundred}%
  \fi
\fi
\@tmpstrctr=\@strctr
\@strctr=#1\relax \@modulo{\@strctr}{100}%
\ifnum#1>100\relax
  \ifnum\@strctr>0\relax
    \let\@@fc@ordstr#2\relax
    \edef#2{\@@fc@ordstr\ \@andname\ }%
  \fi
\fi
\ifnum\@strctr>19\relax
  \@tmpstrctr=\@strctr
  \divide\@strctr by 10\relax
  \@modulo{\@tmpstrctr}{10}%
  \ifthenelse{\@strctr>6}{%
    \ifnum\@strctr=7\relax
      \let\@@fc@ordstr#2\relax
      \edef#2{\@@fc@ordstr\@seventieths{\@tmpstrctr}}%
    \else
      \ifnum\@strctr=8\relax
        \let\@@fc@ordstr#2\relax
        \edef#2{\@@fc@ordstr\@eightieths{\@tmpstrctr}}%
      \else
        \let\@@fc@ordstr#2\relax
        \edef#2{\@@fc@ordstr\@ninetieths{\@tmpstrctr}}%
      \fi
    \fi
  }{%
    \ifnum\@tmpstrctr=0\relax
      \let\@@fc@ordstr#2\relax
      \edef#2{\@@fc@ordstr\@tenthstring{\@strctr}}%
    \else 
      \let\@@fc@ordstr#2\relax
      \edef#2{\@@fc@ordstr\@tenstring{\@strctr}}%
    \fi
    \@strctr=#1\relax \@modulo{\@strctr}{10}%
    \ifnum\@strctr=1\relax
      \let\@@fc@ordstr#2\relax
      \edef#2{\@@fc@ordstr\ \@andname}%
    \fi
    \ifnum\@strctr>0\relax
      \let\@@fc@ordstr#2\relax
      \edef#2{\@@fc@ordstr\ \@unitthstring{\@strctr}}%
    \fi
  }%
\else
  \ifnum\@strctr<10\relax
    \ifnum\@strctr=0\relax
      \ifnum#1<100\relax
        \let\@@fc@ordstr#2\relax
        \edef#2{\@@fc@ordstr\@unitthstring{\@strctr}}%
      \fi
    \else
      \let\@@fc@ordstr#2\relax
      \edef#2{\@@fc@ordstr\@unitthstring{\@strctr}}%
    \fi
  \else
    \@modulo{\@strctr}{10}%
    \let\@@fc@ordstr#2\relax
    \edef#2{\@@fc@ordstr\@teenthstring{\@strctr}}%
  \fi
\fi
}
%    \end{macrocode}
%\iffalse
%    \begin{macrocode}
%</fc-french.def>
%    \end{macrocode}
%\fi
%\iffalse
%    \begin{macrocode}
%<*fc-german.def>
%    \end{macrocode}
%\fi
% \subsection{fc-german.def}
% German definitions (thank you to K. H. Fricke for supplying
% this information)
%    \begin{macrocode}
\ProvidesFile{fc-german.def}[2007/06/14]
%    \end{macrocode}
% Define macro that converts a number or count register (first
% argument) to an ordinal, and stores the result in the
% second argument, which must be a control sequence.
% Masculine:
%    \begin{macrocode}
\newcommand{\@ordinalMgerman}[2]{%
\edef#2{\number#1\relax.}}
%    \end{macrocode}
% Feminine:
%    \begin{macrocode}
\newcommand{\@ordinalFgerman}[2]{%
\edef#2{\number#1\relax.}}
%    \end{macrocode}
% Neuter:
%    \begin{macrocode}
\newcommand{\@ordinalNgerman}[2]{%
\edef#2{\number#1\relax.}}
%    \end{macrocode}
% Convert a number to text. The easiest way to do this is to
% break it up into units, tens and teens.
% Units (argument must be a number from 0 to 9, 1 on its own (eins)
% is dealt with separately):
%    \begin{macrocode}
\newcommand{\@@unitstringgerman}[1]{%
\ifcase#1%
null%
\or ein%
\or zwei%
\or drei%
\or vier%
\or f\"unf%
\or sechs%
\or sieben%
\or acht%
\or neun%
\fi
}
%    \end{macrocode}
% Tens (argument must go from 1 to 10):
%    \begin{macrocode}
\newcommand{\@@tenstringgerman}[1]{%
\ifcase#1%
\or zehn%
\or zwanzig%
\or drei{\ss}ig%
\or vierzig%
\or f\"unfzig%
\or sechzig%
\or siebzig%
\or achtzig%
\or neunzig%
\or einhundert%
\fi
}
%    \end{macrocode}
% |\einhundert| is set to |einhundert| by default, user can
% redefine this command to just |hundert| if required, similarly
% for |\eintausend|.
%    \begin{macrocode}
\providecommand*{\einhundert}{einhundert}
\providecommand*{\eintausend}{eintausend}
%    \end{macrocode}
% Teens:
%    \begin{macrocode}
\newcommand{\@@teenstringgerman}[1]{%
\ifcase#1%
zehn%
\or elf%
\or zw\"olf%
\or dreizehn%
\or vierzehn%
\or f\"unfzehn%
\or sechzehn%
\or siebzehn%
\or achtzehn%
\or neunzehn%
\fi
}
%    \end{macrocode}
% The results are stored in the second argument, but doesn't 
% display anything.
%    \begin{macrocode}
\DeclareRobustCommand{\@numberstringMgerman}[2]{%
\let\@unitstring=\@@unitstringgerman
\let\@teenstring=\@@teenstringgerman
\let\@tenstring=\@@tenstringgerman
\@@numberstringgerman{#1}{#2}}
%    \end{macrocode}
% Feminine and neuter forms:
%    \begin{macrocode}
\let\@numberstringFgerman=\@numberstringMgerman
\let\@numberstringNgerman=\@numberstringMgerman
%    \end{macrocode}
% As above, but initial letters in upper case:
%    \begin{macrocode}
\DeclareRobustCommand{\@NumberstringMgerman}[2]{%
\@numberstringMgerman{#1}{\@@num@str}%
\edef#2{\noexpand\MakeUppercase\@@num@str}}
%    \end{macrocode}
% Feminine and neuter form:
%    \begin{macrocode}
\let\@NumberstringFgerman=\@NumberstringMgerman
\let\@NumberstringNgerman=\@NumberstringMgerman
%    \end{macrocode}
% As above, but for ordinals.
%    \begin{macrocode}
\DeclareRobustCommand{\@ordinalstringMgerman}[2]{%
\let\@unitthstring=\@@unitthstringMgerman
\let\@teenthstring=\@@teenthstringMgerman
\let\@tenthstring=\@@tenthstringMgerman
\let\@unitstring=\@@unitstringgerman
\let\@teenstring=\@@teenstringgerman
\let\@tenstring=\@@tenstringgerman
\def\@thousandth{tausendster}%
\def\@hundredth{hundertster}%
\@@ordinalstringgerman{#1}{#2}}
%    \end{macrocode}
% Feminine form:
%    \begin{macrocode}
\DeclareRobustCommand{\@ordinalstringFgerman}[2]{%
\let\@unitthstring=\@@unitthstringFgerman
\let\@teenthstring=\@@teenthstringFgerman
\let\@tenthstring=\@@tenthstringFgerman
\let\@unitstring=\@@unitstringgerman
\let\@teenstring=\@@teenstringgerman
\let\@tenstring=\@@tenstringgerman
\def\@thousandth{tausendste}%
\def\@hundredth{hundertste}%
\@@ordinalstringgerman{#1}{#2}}
%    \end{macrocode}
% Neuter form:
%    \begin{macrocode}
\DeclareRobustCommand{\@ordinalstringNgerman}[2]{%
\let\@unitthstring=\@@unitthstringNgerman
\let\@teenthstring=\@@teenthstringNgerman
\let\@tenthstring=\@@tenthstringNgerman
\let\@unitstring=\@@unitstringgerman
\let\@teenstring=\@@teenstringgerman
\let\@tenstring=\@@tenstringgerman
\def\@thousandth{tausendstes}%
\def\@hundredth{hunderstes}%
\@@ordinalstringgerman{#1}{#2}}
%    \end{macrocode}
% As above, but with initial letters in upper case.
%    \begin{macrocode}
\DeclareRobustCommand{\@OrdinalstringMgerman}[2]{%
\@ordinalstringMgerman{#1}{\@@num@str}%
\edef#2{\protect\MakeUppercase\@@num@str}}
%    \end{macrocode}
% Feminine form:
%    \begin{macrocode}
\DeclareRobustCommand{\@OrdinalstringFgerman}[2]{%
\@ordinalstringFgerman{#1}{\@@num@str}%
\edef#2{\protect\MakeUppercase\@@num@str}}
%    \end{macrocode}
% Neuter form:
%    \begin{macrocode}
\DeclareRobustCommand{\@OrdinalstringNgerman}[2]{%
\@ordinalstringNgerman{#1}{\@@num@str}%
\edef#2{\protect\MakeUppercase\@@num@str}}
%    \end{macrocode}
% Code for converting numbers into textual ordinals. As before,
% it is easier to split it into units, tens and teens.
% Units:
%    \begin{macrocode}
\newcommand{\@@unitthstringMgerman}[1]{%
\ifcase#1%
nullter%
\or erster%
\or zweiter%
\or dritter%
\or vierter%
\or f\"unter%
\or sechster%
\or siebter%
\or achter%
\or neunter%
\fi
}
%    \end{macrocode}
% Tens:
%    \begin{macrocode}
\newcommand{\@@tenthstringMgerman}[1]{%
\ifcase#1%
\or zehnter%
\or zwanzigster%
\or drei{\ss}igster%
\or vierzigster%
\or f\"unfzigster%
\or sechzigster%
\or siebzigster%
\or achtzigster%
\or neunzigster%
\fi
}
%    \end{macrocode}
% Teens:
%    \begin{macrocode}
\newcommand{\@@teenthstringMgerman}[1]{%
\ifcase#1%
zehnter%
\or elfter%
\or zw\"olfter%
\or dreizehnter%
\or vierzehnter%
\or f\"unfzehnter%
\or sechzehnter%
\or siebzehnter%
\or achtzehnter%
\or neunzehnter%
\fi
}
%    \end{macrocode}
% Units (feminine):
%    \begin{macrocode}
\newcommand{\@@unitthstringFgerman}[1]{%
\ifcase#1%
nullte%
\or erste%
\or zweite%
\or dritte%
\or vierte%
\or f\"unfte%
\or sechste%
\or siebte%
\or achte%
\or neunte%
\fi
}
%    \end{macrocode}
% Tens (feminine):
%    \begin{macrocode}
\newcommand{\@@tenthstringFgerman}[1]{%
\ifcase#1%
\or zehnte%
\or zwanzigste%
\or drei{\ss}igste%
\or vierzigste%
\or f\"unfzigste%
\or sechzigste%
\or siebzigste%
\or achtzigste%
\or neunzigste%
\fi
}
%    \end{macrocode}
% Teens (feminine)
%    \begin{macrocode}
\newcommand{\@@teenthstringFgerman}[1]{%
\ifcase#1%
zehnte%
\or elfte%
\or zw\"olfte%
\or dreizehnte%
\or vierzehnte%
\or f\"unfzehnte%
\or sechzehnte%
\or siebzehnte%
\or achtzehnte%
\or neunzehnte%
\fi
}
%    \end{macrocode}
% Units (neuter):
%    \begin{macrocode}
\newcommand{\@@unitthstringNgerman}[1]{%
\ifcase#1%
nulltes%
\or erstes%
\or zweites%
\or drittes%
\or viertes%
\or f\"unte%
\or sechstes%
\or siebtes%
\or achtes%
\or neuntes%
\fi
}
%    \end{macrocode}
% Tens (neuter):
%    \begin{macrocode}
\newcommand{\@@tenthstringNgerman}[1]{%
\ifcase#1%
\or zehntes%
\or zwanzigstes%
\or drei{\ss}igstes%
\or vierzigstes%
\or f\"unfzigstes%
\or sechzigstes%
\or siebzigstes%
\or achtzigstes%
\or neunzigstes%
\fi
}
%    \end{macrocode}
% Teens (neuter)
%    \begin{macrocode}
\newcommand{\@@teenthstringNgerman}[1]{%
\ifcase#1%
zehntes%
\or elftes%
\or zw\"olftes%
\or dreizehntes%
\or vierzehntes%
\or f\"unfzehntes%
\or sechzehntes%
\or siebzehntes%
\or achtzehntes%
\or neunzehntes%
\fi
}
%    \end{macrocode}
% This appends the results to |#2| for number |#2| (in range 0 to 100.)
% null and eins are dealt with separately in |\@@numberstringgerman|.
%    \begin{macrocode}
\newcommand{\@@numberunderhundredgerman}[2]{%
\ifnum#1<10\relax
  \ifnum#1>0\relax
    \let\@@fc@numstr#2\relax
    \edef#2{\@@fc@numstr\@unitstring{#1}}%
  \fi
\else
  \@tmpstrctr=#1\relax
  \@modulo{\@tmpstrctr}{10}%
  \ifnum#1<20\relax
    \let\@@fc@numstr#2\relax
    \edef#2{\@@fc@numstr\@teenstring{\@tmpstrctr}}%
  \else
    \ifnum\@tmpstrctr=0\relax
    \else
      \let\@@fc@numstr#2\relax
      \edef#2{\@@fc@numstr\@unitstring{\@tmpstrctr}und}%
    \fi
    \@tmpstrctr=#1\relax
    \divide\@tmpstrctr by 10\relax
    \let\@@fc@numstr#2\relax
    \edef#2{\@@fc@numstr\@tenstring{\@tmpstrctr}}%
  \fi
\fi
}
%    \end{macrocode}
% This stores the results in the second argument 
% (which must be a control
% sequence), but it doesn't display anything.
%    \begin{macrocode}
\newcommand{\@@numberstringgerman}[2]{%
\ifnum#1>99999\relax
  \PackageError{fmtcount}{Out of range}%
  {This macro only works for values less than 100000}%
\else
  \ifnum#1<0\relax
    \PackageError{fmtcount}{Negative numbers not permitted}%
    {This macro does not work for negative numbers, however
    you can try typing "minus" first, and then pass the modulus of
    this number}%
  \fi
\fi
\def#2{}%
\@strctr=#1\relax \divide\@strctr by 1000\relax
\ifnum\@strctr>1\relax
% #1 is >= 2000, \@strctr now contains the number of thousands
\@@numberunderhundredgerman{\@strctr}{#2}%
  \let\@@fc@numstr#2\relax
  \edef#2{\@@fc@numstr tausend}%
\else
% #1 lies in range [1000,1999]
  \ifnum\@strctr=1\relax
    \let\@@fc@numstr#2\relax
    \edef#2{\@@fc@numstr\eintausend}%
  \fi
\fi
\@strctr=#1\relax
\@modulo{\@strctr}{1000}%
\divide\@strctr by 100\relax
\ifnum\@strctr>1\relax
% now dealing with number in range [200,999]
  \let\@@fc@numstr#2\relax
  \edef#2{\@@fc@numstr\@unitstring{\@strctr}hundert}%
\else
   \ifnum\@strctr=1\relax
% dealing with number in range [100,199]
     \ifnum#1>1000\relax
% if orginal number > 1000, use einhundert
        \let\@@fc@numstr#2\relax
        \edef#2{\@@fc@numstr einhundert}%
     \else
% otherwise use \einhundert
        \let\@@fc@numstr#2\relax
        \edef#2{\@@fc@numstr\einhundert}%
      \fi
   \fi
\fi
\@strctr=#1\relax
\@modulo{\@strctr}{100}%
\ifnum#1=0\relax
  \def#2{null}%
\else
  \ifnum\@strctr=1\relax
    \let\@@fc@numstr#2\relax
    \edef#2{\@@fc@numstr eins}%
  \else
    \@@numberunderhundredgerman{\@strctr}{#2}%
  \fi
\fi
}
%    \end{macrocode}
% As above, but for ordinals
%    \begin{macrocode}
\newcommand{\@@numberunderhundredthgerman}[2]{%
\ifnum#1<10\relax
 \let\@@fc@numstr#2\relax
 \edef#2{\@@fc@numstr\@unitthstring{#1}}%
\else
  \@tmpstrctr=#1\relax
  \@modulo{\@tmpstrctr}{10}%
  \ifnum#1<20\relax
    \let\@@fc@numstr#2\relax
    \edef#2{\@@fc@numstr\@teenthstring{\@tmpstrctr}}%
  \else
    \ifnum\@tmpstrctr=0\relax
    \else
      \let\@@fc@numstr#2\relax
      \edef#2{\@@fc@numstr\@unitstring{\@tmpstrctr}und}%
    \fi
    \@tmpstrctr=#1\relax
    \divide\@tmpstrctr by 10\relax
    \let\@@fc@numstr#2\relax
    \edef#2{\@@fc@numstr\@tenthstring{\@tmpstrctr}}%
  \fi
\fi
}
%    \end{macrocode}
%    \begin{macrocode}
\newcommand{\@@ordinalstringgerman}[2]{%
\ifnum#1>99999\relax
  \PackageError{fmtcount}{Out of range}%
  {This macro only works for values less than 100000}%
\else
  \ifnum#1<0\relax
    \PackageError{fmtcount}{Negative numbers not permitted}%
    {This macro does not work for negative numbers, however
    you can try typing "minus" first, and then pass the modulus of
    this number}%
  \fi
\fi
\def#2{}%
\@strctr=#1\relax \divide\@strctr by 1000\relax
\ifnum\@strctr>1\relax
% #1 is >= 2000, \@strctr now contains the number of thousands
\@@numberunderhundredgerman{\@strctr}{#2}%
  \let\@@fc@numstr#2\relax
  % is that it, or is there more?
  \@tmpstrctr=#1\relax \@modulo{\@tmpstrctr}{1000}%
  \ifnum\@tmpstrctr=0\relax
    \edef#2{\@@fc@numstr\@thousandth}%
  \else
    \edef#2{\@@fc@numstr tausend}%
  \fi
\else
% #1 lies in range [1000,1999]
  \ifnum\@strctr=1\relax
    \ifnum#1=1000\relax
      \let\@@fc@numstr#2\relax
      \edef#2{\@@fc@numstr\@thousandth}%
    \else
      \let\@@fc@numstr#2\relax
      \edef#2{\@@fc@numstr\eintausend}%
    \fi
  \fi
\fi
\@strctr=#1\relax
\@modulo{\@strctr}{1000}%
\divide\@strctr by 100\relax
\ifnum\@strctr>1\relax
% now dealing with number in range [200,999]
  \let\@@fc@numstr#2\relax
  % is that it, or is there more?
  \@tmpstrctr=#1\relax \@modulo{\@tmpstrctr}{100}%
  \ifnum\@tmpstrctr=0\relax
     \ifnum\@strctr=1\relax
       \edef#2{\@@fc@numstr\@hundredth}%
     \else
       \edef#2{\@@fc@numstr\@unitstring{\@strctr}\@hundredth}%
     \fi
  \else
     \edef#2{\@@fc@numstr\@unitstring{\@strctr}hundert}%
  \fi
\else
   \ifnum\@strctr=1\relax
% dealing with number in range [100,199]
% is that it, or is there more?
     \@tmpstrctr=#1\relax \@modulo{\@tmpstrctr}{100}%
     \ifnum\@tmpstrctr=0\relax
        \let\@@fc@numstr#2\relax
        \edef#2{\@@fc@numstr\@hundredth}%
     \else
     \ifnum#1>1000\relax
        \let\@@fc@numstr#2\relax
        \edef#2{\@@fc@numstr einhundert}%
     \else
        \let\@@fc@numstr#2\relax
        \edef#2{\@@fc@numstr\einhundert}%
     \fi
     \fi
   \fi
\fi
\@strctr=#1\relax
\@modulo{\@strctr}{100}%
\ifthenelse{\@strctr=0 \and #1>0}{}{%
\@@numberunderhundredthgerman{\@strctr}{#2}%
}%
}
%    \end{macrocode}
% Set |ngerman| to be equivalent to |german|. Is it okay to do
% this? (I don't know the difference between the two.)
%    \begin{macrocode}
\let\@ordinalMngerman=\@ordinalMgerman
\let\@ordinalFngerman=\@ordinalFgerman
\let\@ordinalNngerman=\@ordinalNgerman
\let\@numberstringMngerman=\@numberstringMgerman
\let\@numberstringFngerman=\@numberstringFgerman
\let\@numberstringNngerman=\@numberstringNgerman
\let\@NumberstringMngerman=\@NumberstringMgerman
\let\@NumberstringFngerman=\@NumberstringFgerman
\let\@NumberstringNngerman=\@NumberstringNgerman
\let\@ordinalstringMngerman=\@ordinalstringMgerman
\let\@ordinalstringFngerman=\@ordinalstringFgerman
\let\@ordinalstringNngerman=\@ordinalstringNgerman
\let\@OrdinalstringMngerman=\@OrdinalstringMgerman
\let\@OrdinalstringFngerman=\@OrdinalstringFgerman
\let\@OrdinalstringNngerman=\@OrdinalstringNgerman
%    \end{macrocode}
%\iffalse
%    \begin{macrocode}
%</fc-german.def>
%    \end{macrocode}
%\fi
%\iffalse
%    \begin{macrocode}
%<*fc-portuges.def>
%    \end{macrocode}
%\fi
% \subsection{fc-portuges.def}
% Portuguse definitions
%    \begin{macrocode}
\ProvidesFile{fc-portuges.def}[2007/05/26]
%    \end{macrocode}
% Define macro that converts a number or count register (first
% argument) to an ordinal, and stores the result in the second
% argument, which should be a control sequence. Masculine:
%    \begin{macrocode}
\newcommand*{\@ordinalMportuges}[2]{%
\ifnum#1=0\relax
  \edef#2{\number#1}%
\else
  \edef#2{\number#1\relax\noexpand\fmtord{o}}%
\fi}
%    \end{macrocode}
% Feminine:
%    \begin{macrocode}
\newcommand*{\@ordinalFportuges}[2]{%
\ifnum#1=0\relax
  \edef#2{\number#1}%
\else
  \edef#2{\number#1\relax\noexpand\fmtord{a}}%
\fi}
%    \end{macrocode}
% Make neuter same as masculine:
%    \begin{macrocode}
\let\@ordinalNportuges\@ordinalMportuges
%    \end{macrocode}
% Convert a number to a textual representation. To make it easier,
% split it up into units, tens, teens and hundreds. Units (argument must
% be a number from 0 to 9):
%    \begin{macrocode}
\newcommand*{\@@unitstringportuges}[1]{%
\ifcase#1\relax
zero%
\or um%
\or dois%
\or tr\^es%
\or quatro%
\or cinco%
\or seis%
\or sete%
\or oito%
\or nove%
\fi
}
%   \end{macrocode}
% As above, but for feminine:
%   \begin{macrocode}
\newcommand*{\@@unitstringFportuges}[1]{%
\ifcase#1\relax
zero%
\or uma%
\or duas%
\or tr\^es%
\or quatro%
\or cinco%
\or seis%
\or sete%
\or oito%
\or nove%
\fi
}
%    \end{macrocode}
% Tens (argument must be a number from 0 to 10):
%    \begin{macrocode}
\newcommand*{\@@tenstringportuges}[1]{%
\ifcase#1\relax
\or dez%
\or vinte%
\or trinta%
\or quarenta%
\or cinq\"uenta%
\or sessenta%
\or setenta%
\or oitenta%
\or noventa%
\or cem%
\fi
}
%    \end{macrocode}
% Teens (argument must be a number from 0 to 9):
%    \begin{macrocode}
\newcommand*{\@@teenstringportuges}[1]{%
\ifcase#1\relax
dez%
\or onze%
\or doze%
\or treze%
\or quatorze%
\or quinze%
\or dezesseis%
\or dezessete%
\or dezoito%
\or dezenove%
\fi
}
%    \end{macrocode}
% Hundreds:
%    \begin{macrocode}
\newcommand*{\@@hundredstringportuges}[1]{%
\ifcase#1\relax
\or cento%
\or duzentos%
\or trezentos%
\or quatrocentos%
\or quinhentos%
\or seiscentos%
\or setecentos%
\or oitocentos%
\or novecentos%
\fi}
%    \end{macrocode}
% Hundreds (feminine):
%    \begin{macrocode}
\newcommand*{\@@hundredstringFportuges}[1]{%
\ifcase#1\relax
\or cento%
\or duzentas%
\or trezentas%
\or quatrocentas%
\or quinhentas%
\or seiscentas%
\or setecentas%
\or oitocentas%
\or novecentas%
\fi}
%    \end{macrocode}
% Units (initial letter in upper case):
%    \begin{macrocode}
\newcommand*{\@@Unitstringportuges}[1]{%
\ifcase#1\relax
Zero%
\or Um%
\or Dois%
\or Tr\^es%
\or Quatro%
\or Cinco%
\or Seis%
\or Sete%
\or Oito%
\or Nove%
\fi
}
%    \end{macrocode}
% As above, but feminine:
%    \begin{macrocode}
\newcommand*{\@@UnitstringFportuges}[1]{%
\ifcase#1\relax
Zera%
\or Uma%
\or Duas%
\or Tr\^es%
\or Quatro%
\or Cinco%
\or Seis%
\or Sete%
\or Oito%
\or Nove%
\fi
}
%    \end{macrocode}
% Tens (with initial letter in upper case):
%    \begin{macrocode}
\newcommand*{\@@Tenstringportuges}[1]{%
\ifcase#1\relax
\or Dez%
\or Vinte%
\or Trinta%
\or Quarenta%
\or Cinq\"uenta%
\or Sessenta%
\or Setenta%
\or Oitenta%
\or Noventa%
\or Cem%
\fi
}
%    \end{macrocode}
% Teens (with initial letter in upper case):
%    \begin{macrocode}
\newcommand*{\@@Teenstringportuges}[1]{%
\ifcase#1\relax
Dez%
\or Onze%
\or Doze%
\or Treze%
\or Quatorze%
\or Quinze%
\or Dezesseis%
\or Dezessete%
\or Dezoito%
\or Dezenove%
\fi
}
%    \end{macrocode}
% Hundreds (with initial letter in upper case):
%    \begin{macrocode}
\newcommand*{\@@Hundredstringportuges}[1]{%
\ifcase#1\relax
\or Cento%
\or Duzentos%
\or Trezentos%
\or Quatrocentos%
\or Quinhentos%
\or Seiscentos%
\or Setecentos%
\or Oitocentos%
\or Novecentos%
\fi}
%    \end{macrocode}
% As above, but feminine:
%    \begin{macrocode}
\newcommand*{\@@HundredstringFportuges}[1]{%
\ifcase#1\relax
\or Cento%
\or Duzentas%
\or Trezentas%
\or Quatrocentas%
\or Quinhentas%
\or Seiscentas%
\or Setecentas%
\or Oitocentas%
\or Novecentas%
\fi}
%    \end{macrocode}
% This has changed in version 1.08, so that it now stores
% the result in the second argument, but doesn't display
% anything. Since it only affects internal macros, it shouldn't
% affect documents created with older versions. (These internal
% macros are not meant for use in documents.)
%    \begin{macrocode}
\DeclareRobustCommand{\@numberstringMportuges}[2]{%
\let\@unitstring=\@@unitstringportuges
\let\@teenstring=\@@teenstringportuges
\let\@tenstring=\@@tenstringportuges
\let\@hundredstring=\@@hundredstringportuges
\def\@hundred{cem}\def\@thousand{mil}%
\def\@andname{e}%
\@@numberstringportuges{#1}{#2}}
%    \end{macrocode}
% As above, but feminine form:
%    \begin{macrocode}
\DeclareRobustCommand{\@numberstringFportuges}[2]{%
\let\@unitstring=\@@unitstringFportuges
\let\@teenstring=\@@teenstringportuges
\let\@tenstring=\@@tenstringportuges
\let\@hundredstring=\@@hundredstringFportuges
\def\@hundred{cem}\def\@thousand{mil}%
\def\@andname{e}%
\@@numberstringportuges{#1}{#2}}
%    \end{macrocode}
% Make neuter same as masculine:
%    \begin{macrocode}
\let\@numberstringNportuges\@numberstringMportuges
%    \end{macrocode}
% As above, but initial letters in upper case:
%    \begin{macrocode}
\DeclareRobustCommand{\@NumberstringMportuges}[2]{%
\let\@unitstring=\@@Unitstringportuges
\let\@teenstring=\@@Teenstringportuges
\let\@tenstring=\@@Tenstringportuges
\let\@hundredstring=\@@Hundredstringportuges
\def\@hundred{Cem}\def\@thousand{Mil}%
\def\@andname{e}%
\@@numberstringportuges{#1}{#2}}
%    \end{macrocode}
% As above, but feminine form:
%    \begin{macrocode}
\DeclareRobustCommand{\@NumberstringFportuges}[2]{%
\let\@unitstring=\@@UnitstringFportuges
\let\@teenstring=\@@Teenstringportuges
\let\@tenstring=\@@Tenstringportuges
\let\@hundredstring=\@@HundredstringFportuges
\def\@hundred{Cem}\def\@thousand{Mil}%
\def\@andname{e}%
\@@numberstringportuges{#1}{#2}}
%    \end{macrocode}
% Make neuter same as masculine:
%    \begin{macrocode}
\let\@NumberstringNportuges\@NumberstringMportuges
%    \end{macrocode}
% As above, but for ordinals.
%    \begin{macrocode}
\DeclareRobustCommand{\@ordinalstringMportuges}[2]{%
\let\@unitthstring=\@@unitthstringportuges
\let\@unitstring=\@@unitstringportuges
\let\@teenthstring=\@@teenthstringportuges
\let\@tenthstring=\@@tenthstringportuges
\let\@hundredthstring=\@@hundredthstringportuges
\def\@thousandth{mil\'esimo}%
\@@ordinalstringportuges{#1}{#2}}
%    \end{macrocode}
% Feminine form:
%    \begin{macrocode}
\DeclareRobustCommand{\@ordinalstringFportuges}[2]{%
\let\@unitthstring=\@@unitthstringFportuges
\let\@unitstring=\@@unitstringFportuges
\let\@teenthstring=\@@teenthstringportuges
\let\@tenthstring=\@@tenthstringFportuges
\let\@hundredthstring=\@@hundredthstringFportuges
\def\@thousandth{mil\'esima}%
\@@ordinalstringportuges{#1}{#2}}
%    \end{macrocode}
% Make neuter same as masculine:
%    \begin{macrocode}
\let\@ordinalstringNportuges\@ordinalstringMportuges
%    \end{macrocode}
% As above, but initial letters in upper case (masculine):
%    \begin{macrocode}
\DeclareRobustCommand{\@OrdinalstringMportuges}[2]{%
\let\@unitthstring=\@@Unitthstringportuges
\let\@unitstring=\@@Unitstringportuges
\let\@teenthstring=\@@teenthstringportuges
\let\@tenthstring=\@@Tenthstringportuges
\let\@hundredthstring=\@@Hundredthstringportuges
\def\@thousandth{Mil\'esimo}%
\@@ordinalstringportuges{#1}{#2}}
%    \end{macrocode}
% Feminine form:
%    \begin{macrocode}
\DeclareRobustCommand{\@OrdinalstringFportuges}[2]{%
\let\@unitthstring=\@@UnitthstringFportuges
\let\@unitstring=\@@UnitstringFportuges
\let\@teenthstring=\@@teenthstringportuges
\let\@tenthstring=\@@TenthstringFportuges
\let\@hundredthstring=\@@HundredthstringFportuges
\def\@thousandth{Mil\'esima}%
\@@ordinalstringportuges{#1}{#2}}
%    \end{macrocode}
% Make neuter same as masculine:
%    \begin{macrocode}
\let\@OrdinalstringNportuges\@OrdinalstringMportuges
%    \end{macrocode}
% In order to do the ordinals, split into units, teens, tens
% and hundreds. Units:
%    \begin{macrocode}
\newcommand*{\@@unitthstringportuges}[1]{%
\ifcase#1\relax
zero%
\or primeiro%
\or segundo%
\or terceiro%
\or quarto%
\or quinto%
\or sexto%
\or s\'etimo%
\or oitavo%
\or nono%
\fi
}
%    \end{macrocode}
% Tens:
%    \begin{macrocode}
\newcommand*{\@@tenthstringportuges}[1]{%
\ifcase#1\relax
\or d\'ecimo%
\or vig\'esimo%
\or trig\'esimo%
\or quadrag\'esimo%
\or q\"uinquag\'esimo%
\or sexag\'esimo%
\or setuag\'esimo%
\or octog\'esimo%
\or nonag\'esimo%
\fi
}
%    \end{macrocode}
% Teens:
%    \begin{macrocode}
\newcommand*{\@@teenthstringportuges}[1]{%
\@tenthstring{1}%
\ifnum#1>0\relax
-\@unitthstring{#1}%
\fi}
%    \end{macrocode}
% Hundreds:
%    \begin{macrocode}
\newcommand*{\@@hundredthstringportuges}[1]{%
\ifcase#1\relax
\or cent\'esimo%
\or ducent\'esimo%
\or trecent\'esimo%
\or quadringent\'esimo%
\or q\"uingent\'esimo%
\or seiscent\'esimo%
\or setingent\'esimo%
\or octingent\'esimo%
\or nongent\'esimo%
\fi}
%    \end{macrocode}
% Units (feminine):
%    \begin{macrocode}
\newcommand*{\@@unitthstringFportuges}[1]{%
\ifcase#1\relax
zero%
\or primeira%
\or segunda%
\or terceira%
\or quarta%
\or quinta%
\or sexta%
\or s\'etima%
\or oitava%
\or nona%
\fi
}
%    \end{macrocode}
% Tens (feminine):
%    \begin{macrocode}
\newcommand*{\@@tenthstringFportuges}[1]{%
\ifcase#1\relax
\or d\'ecima%
\or vig\'esima%
\or trig\'esima%
\or quadrag\'esima%
\or q\"uinquag\'esima%
\or sexag\'esima%
\or setuag\'esima%
\or octog\'esima%
\or nonag\'esima%
\fi
}
%    \end{macrocode}
% Hundreds (feminine):
%    \begin{macrocode}
\newcommand*{\@@hundredthstringFportuges}[1]{%
\ifcase#1\relax
\or cent\'esima%
\or ducent\'esima%
\or trecent\'esima%
\or quadringent\'esima%
\or q\"uingent\'esima%
\or seiscent\'esima%
\or setingent\'esima%
\or octingent\'esima%
\or nongent\'esima%
\fi}
%    \end{macrocode}
% As above, but with initial letter in upper case. Units:
%    \begin{macrocode}
\newcommand*{\@@Unitthstringportuges}[1]{%
\ifcase#1\relax
Zero%
\or Primeiro%
\or Segundo%
\or Terceiro%
\or Quarto%
\or Quinto%
\or Sexto%
\or S\'etimo%
\or Oitavo%
\or Nono%
\fi
}
%    \end{macrocode}
% Tens:
%    \begin{macrocode}
\newcommand*{\@@Tenthstringportuges}[1]{%
\ifcase#1\relax
\or D\'ecimo%
\or Vig\'esimo%
\or Trig\'esimo%
\or Quadrag\'esimo%
\or Q\"uinquag\'esimo%
\or Sexag\'esimo%
\or Setuag\'esimo%
\or Octog\'esimo%
\or Nonag\'esimo%
\fi
}
%    \end{macrocode}
% Hundreds:
%    \begin{macrocode}
\newcommand*{\@@Hundredthstringportuges}[1]{%
\ifcase#1\relax
\or Cent\'esimo%
\or Ducent\'esimo%
\or Trecent\'esimo%
\or Quadringent\'esimo%
\or Q\"uingent\'esimo%
\or Seiscent\'esimo%
\or Setingent\'esimo%
\or Octingent\'esimo%
\or Nongent\'esimo%
\fi}
%    \end{macrocode}
% As above, but feminine. Units:
%    \begin{macrocode}
\newcommand*{\@@UnitthstringFportuges}[1]{%
\ifcase#1\relax
Zera%
\or Primeira%
\or Segunda%
\or Terceira%
\or Quarta%
\or Quinta%
\or Sexta%
\or S\'etima%
\or Oitava%
\or Nona%
\fi
}
%    \end{macrocode}
% Tens (feminine);
%    \begin{macrocode}
\newcommand*{\@@TenthstringFportuges}[1]{%
\ifcase#1\relax
\or D\'ecima%
\or Vig\'esima%
\or Trig\'esima%
\or Quadrag\'esima%
\or Q\"uinquag\'esima%
\or Sexag\'esima%
\or Setuag\'esima%
\or Octog\'esima%
\or Nonag\'esima%
\fi
}
%    \end{macrocode}
% Hundreds (feminine):
%    \begin{macrocode}
\newcommand*{\@@HundredthstringFportuges}[1]{%
\ifcase#1\relax
\or Cent\'esima%
\or Ducent\'esima%
\or Trecent\'esima%
\or Quadringent\'esima%
\or Q\"uingent\'esima%
\or Seiscent\'esima%
\or Setingent\'esima%
\or Octingent\'esima%
\or Nongent\'esima%
\fi}
%    \end{macrocode}
% This has changed in version 1.09, so that it now stores
% the result in the second argument (a control sequence), but it
% doesn't display anything. Since it only affects internal macros,
% it shouldn't affect documents created with older versions.
% (These internal macros are not meant for use in documents.)
%    \begin{macrocode}
\newcommand*{\@@numberstringportuges}[2]{%
\ifnum#1>99999
\PackageError{fmtcount}{Out of range}%
{This macro only works for values less than 100000}%
\else
\ifnum#1<0
\PackageError{fmtcount}{Negative numbers not permitted}%
{This macro does not work for negative numbers, however
you can try typing "minus" first, and then pass the modulus of
this number}%
\fi
\fi
\def#2{}%
\@strctr=#1\relax \divide\@strctr by 1000\relax
\ifnum\@strctr>9
% #1 is greater or equal to 10000
  \divide\@strctr by 10
  \ifnum\@strctr>1\relax
    \let\@@fc@numstr#2\relax
    \edef#2{\@@fc@numstr\@tenstring{\@strctr}}%
    \@strctr=#1 \divide\@strctr by 1000\relax
    \@modulo{\@strctr}{10}%
    \ifnum\@strctr>0
      \ifnum\@strctr=1\relax
        \let\@@fc@numstr#2\relax
        \edef#2{\@@fc@numstr\ \@andname}%
      \fi
      \let\@@fc@numstr#2\relax
      \edef#2{\@@fc@numstr\ \@unitstring{\@strctr}}%
    \fi
  \else
    \@strctr=#1\relax
    \divide\@strctr by 1000\relax
    \@modulo{\@strctr}{10}%
    \let\@@fc@numstr#2\relax
    \edef#2{\@@fc@numstr\@teenstring{\@strctr}}%
  \fi
  \let\@@fc@numstr#2\relax
  \edef#2{\@@fc@numstr\ \@thousand}%
\else
  \ifnum\@strctr>0\relax 
    \ifnum\@strctr>1\relax
      \let\@@fc@numstr#2\relax
      \edef#2{\@@fc@numstr\@unitstring{\@strctr}\ }%
    \fi
    \let\@@fc@numstr#2\relax
    \edef#2{\@@fc@numstr\@thousand}%
  \fi
\fi
\@strctr=#1\relax \@modulo{\@strctr}{1000}%
\divide\@strctr by 100\relax
\ifnum\@strctr>0\relax
  \ifnum#1>1000 \relax
    \let\@@fc@numstr#2\relax
    \edef#2{\@@fc@numstr\ }%
  \fi
  \@tmpstrctr=#1\relax
  \@modulo{\@tmpstrctr}{1000}%
  \let\@@fc@numstr#2\relax
  \ifnum\@tmpstrctr=100\relax
    \edef#2{\@@fc@numstr\@tenstring{10}}%
  \else
    \edef#2{\@@fc@numstr\@hundredstring{\@strctr}}%
  \fi%
\fi
\@strctr=#1\relax \@modulo{\@strctr}{100}%
\ifnum#1>100\relax
  \ifnum\@strctr>0\relax
    \let\@@fc@numstr#2\relax
    \edef#2{\@@fc@numstr\ \@andname\ }%
  \fi
\fi
\ifnum\@strctr>19\relax
  \divide\@strctr by 10\relax
  \let\@@fc@numstr#2\relax
  \edef#2{\@@fc@numstr\@tenstring{\@strctr}}%
  \@strctr=#1\relax \@modulo{\@strctr}{10}%
  \ifnum\@strctr>0
    \ifnum\@strctr=1\relax
      \let\@@fc@numstr#2\relax
      \edef#2{\@@fc@numstr\ \@andname}%
    \else
      \ifnum#1>100\relax
        \let\@@fc@numstr#2\relax
        \edef#2{\@@fc@numstr\ \@andname}%
      \fi
    \fi 
    \let\@@fc@numstr#2\relax
    \edef#2{\@@fc@numstr\ \@unitstring{\@strctr}}%
  \fi
\else
  \ifnum\@strctr<10\relax
    \ifnum\@strctr=0\relax
      \ifnum#1<100\relax
        \let\@@fc@numstr#2\relax
        \edef#2{\@@fc@numstr\@unitstring{\@strctr}}%
      \fi
    \else%(>0,<10)
      \let\@@fc@numstr#2\relax
      \edef#2{\@@fc@numstr\@unitstring{\@strctr}}%
    \fi
  \else%>10
    \@modulo{\@strctr}{10}%
    \let\@@fc@numstr#2\relax
    \edef#2{\@@fc@numstr\@teenstring{\@strctr}}%
  \fi
\fi
}
%    \end{macrocode}
% As above, but for ordinals.
%    \begin{macrocode}
\newcommand*{\@@ordinalstringportuges}[2]{%
\@strctr=#1\relax
\ifnum#1>99999
\PackageError{fmtcount}{Out of range}%
{This macro only works for values less than 100000}%
\else
\ifnum#1<0
\PackageError{fmtcount}{Negative numbers not permitted}%
{This macro does not work for negative numbers, however
you can try typing "minus" first, and then pass the modulus of
this number}%
\else
\def#2{}%
\ifnum\@strctr>999\relax
  \divide\@strctr by 1000\relax
  \ifnum\@strctr>1\relax
    \ifnum\@strctr>9\relax
      \@tmpstrctr=\@strctr
      \ifnum\@strctr<20
        \@modulo{\@tmpstrctr}{10}%
        \let\@@fc@ordstr#2\relax
        \edef#2{\@@fc@ordstr\@teenthstring{\@tmpstrctr}}%
      \else
        \divide\@tmpstrctr by 10\relax
        \let\@@fc@ordstr#2\relax
        \edef#2{\@@fc@ordstr\@tenthstring{\@tmpstrctr}}%
        \@tmpstrctr=\@strctr
        \@modulo{\@tmpstrctr}{10}%
        \ifnum\@tmpstrctr>0\relax
          \let\@@fc@ordstr#2\relax
          \edef#2{\@@fc@ordstr\@unitthstring{\@tmpstrctr}}%
        \fi
      \fi
    \else
      \let\@@fc@ordstr#2\relax
      \edef#2{\@@fc@ordstr\@unitstring{\@strctr}}%
    \fi
  \fi
  \let\@@fc@ordstr#2\relax
  \edef#2{\@@fc@ordstr\@thousandth}%
\fi
\@strctr=#1\relax
\@modulo{\@strctr}{1000}%
\ifnum\@strctr>99\relax
  \@tmpstrctr=\@strctr
  \divide\@tmpstrctr by 100\relax
  \ifnum#1>1000\relax
    \let\@@fc@ordstr#2\relax
    \edef#2{\@@fc@ordstr-}%
  \fi
  \let\@@fc@ordstr#2\relax
  \edef#2{\@@fc@ordstr\@hundredthstring{\@tmpstrctr}}%
\fi
\@modulo{\@strctr}{100}%
\ifnum#1>99\relax
  \ifnum\@strctr>0\relax
    \let\@@fc@ordstr#2\relax
    \edef#2{\@@fc@ordstr-}%
  \fi
\fi
\ifnum\@strctr>9\relax
  \@tmpstrctr=\@strctr
  \divide\@tmpstrctr by 10\relax
  \let\@@fc@ordstr#2\relax
  \edef#2{\@@fc@ordstr\@tenthstring{\@tmpstrctr}}%
  \@tmpstrctr=\@strctr
  \@modulo{\@tmpstrctr}{10}%
  \ifnum\@tmpstrctr>0\relax
    \let\@@fc@ordstr#2\relax
    \edef#2{\@@fc@ordstr-\@unitthstring{\@tmpstrctr}}%
  \fi
\else
  \ifnum\@strctr=0\relax
    \ifnum#1=0\relax
      \let\@@fc@ordstr#2\relax
      \edef#2{\@@fc@ordstr\@unitstring{0}}%
    \fi
  \else
    \let\@@fc@ordstr#2\relax
    \edef#2{\@@fc@ordstr\@unitthstring{\@strctr}}%
  \fi
\fi
\fi
\fi
}
%    \end{macrocode}
%\iffalse
%    \begin{macrocode}
%</fc-portuges.def>
%    \end{macrocode}
%\fi
%\iffalse
%    \begin{macrocode}
%<*fc-spanish.def>
%    \end{macrocode}
%\fi
% \subsection{fc-spanish.def}
% Spanish definitions
%    \begin{macrocode}
\ProvidesFile{fc-spanish.def}[2007/05/26]
%    \end{macrocode}
% Define macro that converts a number or count register (first
% argument) to an ordinal, and stores the result in the
% second argument, which must be a control sequence.
% Masculine:
%    \begin{macrocode}
\newcommand{\@ordinalMspanish}[2]{%
\edef#2{\number#1\relax\noexpand\fmtord{o}}}
%    \end{macrocode}
% Feminine:
%    \begin{macrocode}
\newcommand{\@ordinalFspanish}[2]{%
\edef#2{\number#1\relax\noexpand\fmtord{a}}}
%    \end{macrocode}
% Make neuter same as masculine:
%    \begin{macrocode}
\let\@ordinalNspanish\@ordinalMspanish
%    \end{macrocode}
% Convert a number to text. The easiest way to do this is to
% break it up into units, tens, teens, twenties and hundreds.
% Units (argument must be a number from 0 to 9):
%    \begin{macrocode}
\newcommand{\@@unitstringspanish}[1]{%
\ifcase#1\relax
cero%
\or uno%
\or dos%
\or tres%
\or cuatro%
\or cinco%
\or seis%
\or siete%
\or ocho%
\or nueve%
\fi
}
%    \end{macrocode}
% Feminine:
%    \begin{macrocode}
\newcommand{\@@unitstringFspanish}[1]{%
\ifcase#1\relax
cera%
\or una%
\or dos%
\or tres%
\or cuatro%
\or cinco%
\or seis%
\or siete%
\or ocho%
\or nueve%
\fi
}
%    \end{macrocode}
% Tens (argument must go from 1 to 10):
%    \begin{macrocode}
\newcommand{\@@tenstringspanish}[1]{%
\ifcase#1\relax
\or diez%
\or viente%
\or treinta%
\or cuarenta%
\or cincuenta%
\or sesenta%
\or setenta%
\or ochenta%
\or noventa%
\or cien%
\fi
}
%    \end{macrocode}
% Teens:
%    \begin{macrocode}
\newcommand{\@@teenstringspanish}[1]{%
\ifcase#1\relax
diez%
\or once%
\or doce%
\or trece%
\or catorce%
\or quince%
\or diecis\'eis%
\or diecisiete%
\or dieciocho%
\or diecinueve%
\fi
}
%    \end{macrocode}
% Twenties:
%    \begin{macrocode}
\newcommand{\@@twentystringspanish}[1]{%
\ifcase#1\relax
veinte%
\or veintiuno%
\or veintid\'os%
\or veintitr\'es%
\or veinticuatro%
\or veinticinco%
\or veintis\'eis%
\or veintisiete%
\or veintiocho%
\or veintinueve%
\fi}
%    \end{macrocode}
% Feminine form:
%    \begin{macrocode}
\newcommand{\@@twentystringFspanish}[1]{%
\ifcase#1\relax
veinte%
\or veintiuna%
\or veintid\'os%
\or veintitr\'es%
\or veinticuatro%
\or veinticinco%
\or veintis\'eis%
\or veintisiete%
\or veintiocho%
\or veintinueve%
\fi}
%    \end{macrocode}
% Hundreds:
%    \begin{macrocode}
\newcommand{\@@hundredstringspanish}[1]{%
\ifcase#1\relax
\or ciento%
\or doscientos%
\or trescientos%
\or cuatrocientos%
\or quinientos%
\or seiscientos%
\or setecientos%
\or ochocientos%
\or novecientos%
\fi}
%    \end{macrocode}
% Feminine form:
%    \begin{macrocode}
\newcommand{\@@hundredstringFspanish}[1]{%
\ifcase#1\relax
\or cienta%
\or doscientas%
\or trescientas%
\or cuatrocientas%
\or quinientas%
\or seiscientas%
\or setecientas%
\or ochocientas%
\or novecientas%
\fi}
%    \end{macrocode}
% As above, but with initial letter uppercase:
%    \begin{macrocode}
\newcommand{\@@Unitstringspanish}[1]{%
\ifcase#1\relax
Cero%
\or Uno%
\or Dos%
\or Tres%
\or Cuatro%
\or Cinco%
\or Seis%
\or Siete%
\or Ocho%
\or Nueve%
\fi
}
%    \end{macrocode}
% Feminine form:
%    \begin{macrocode}
\newcommand{\@@UnitstringFspanish}[1]{%
\ifcase#1\relax
Cera%
\or Una%
\or Dos%
\or Tres%
\or Cuatro%
\or Cinco%
\or Seis%
\or Siete%
\or Ocho%
\or Nueve%
\fi
}
%    \end{macrocode}
% Tens:
%    \begin{macrocode}
\newcommand{\@@Tenstringspanish}[1]{%
\ifcase#1\relax
\or Diez%
\or Viente%
\or Treinta%
\or Cuarenta%
\or Cincuenta%
\or Sesenta%
\or Setenta%
\or Ochenta%
\or Noventa%
\or Cien%
\fi
}
%    \end{macrocode}
% Teens:
%    \begin{macrocode}
\newcommand{\@@Teenstringspanish}[1]{%
\ifcase#1\relax
Diez%
\or Once%
\or Doce%
\or Trece%
\or Catorce%
\or Quince%
\or Diecis\'eis%
\or Diecisiete%
\or Dieciocho%
\or Diecinueve%
\fi
}
%    \end{macrocode}
% Twenties:
%    \begin{macrocode}
\newcommand{\@@Twentystringspanish}[1]{%
\ifcase#1\relax
Veinte%
\or Veintiuno%
\or Veintid\'os%
\or Veintitr\'es%
\or Veinticuatro%
\or Veinticinco%
\or Veintis\'eis%
\or Veintisiete%
\or Veintiocho%
\or Veintinueve%
\fi}
%    \end{macrocode}
% Feminine form:
%    \begin{macrocode}
\newcommand{\@@TwentystringFspanish}[1]{%
\ifcase#1\relax
Veinte%
\or Veintiuna%
\or Veintid\'os%
\or Veintitr\'es%
\or Veinticuatro%
\or Veinticinco%
\or Veintis\'eis%
\or Veintisiete%
\or Veintiocho%
\or Veintinueve%
\fi}
%    \end{macrocode}
% Hundreds:
%    \begin{macrocode}
\newcommand{\@@Hundredstringspanish}[1]{%
\ifcase#1\relax
\or Ciento%
\or Doscientos%
\or Trescientos%
\or Cuatrocientos%
\or Quinientos%
\or Seiscientos%
\or Setecientos%
\or Ochocientos%
\or Novecientos%
\fi}
%    \end{macrocode}
% Feminine form:
%    \begin{macrocode}
\newcommand{\@@HundredstringFspanish}[1]{%
\ifcase#1\relax
\or Cienta%
\or Doscientas%
\or Trescientas%
\or Cuatrocientas%
\or Quinientas%
\or Seiscientas%
\or Setecientas%
\or Ochocientas%
\or Novecientas%
\fi}
%    \end{macrocode}
% This has changed in version 1.09, so that it now stores the
% result in the second argument, but doesn't display anything.
% Since it only affects internal macros, it shouldn't affect
% documents created with older versions. (These internal macros
% are not meant for use in documents.)
%    \begin{macrocode}
\DeclareRobustCommand{\@numberstringMspanish}[2]{%
\let\@unitstring=\@@unitstringspanish
\let\@teenstring=\@@teenstringspanish
\let\@tenstring=\@@tenstringspanish
\let\@twentystring=\@@twentystringspanish
\let\@hundredstring=\@@hundredstringspanish
\def\@hundred{cien}\def\@thousand{mil}%
\def\@andname{y}%
\@@numberstringspanish{#1}{#2}}
%    \end{macrocode}
% Feminine form:
%    \begin{macrocode}
\DeclareRobustCommand{\@numberstringFspanish}[2]{%
\let\@unitstring=\@@unitstringFspanish
\let\@teenstring=\@@teenstringspanish
\let\@tenstring=\@@tenstringspanish
\let\@twentystring=\@@twentystringFspanish
\let\@hundredstring=\@@hundredstringFspanish
\def\@hundred{cien}\def\@thousand{mil}%
\def\@andname{y}%
\@@numberstringspanish{#1}{#2}}
%    \end{macrocode}
% Make neuter same as masculine:
%    \begin{macrocode}
\let\@numberstringNspanish\@numberstringMspanish
%    \end{macrocode}
% As above, but initial letters in upper case:
%    \begin{macrocode}
\DeclareRobustCommand{\@NumberstringMspanish}[2]{%
\let\@unitstring=\@@Unitstringspanish
\let\@teenstring=\@@Teenstringspanish
\let\@tenstring=\@@Tenstringspanish
\let\@twentystring=\@@Twentystringspanish
\let\@hundredstring=\@@Hundredstringspanish
\def\@andname{y}%
\def\@hundred{Cien}\def\@thousand{Mil}%
\@@numberstringspanish{#1}{#2}}
%    \end{macrocode}
% Feminine form:
%    \begin{macrocode}
\DeclareRobustCommand{\@NumberstringFspanish}[2]{%
\let\@unitstring=\@@UnitstringFspanish
\let\@teenstring=\@@Teenstringspanish
\let\@tenstring=\@@Tenstringspanish
\let\@twentystring=\@@TwentystringFspanish
\let\@hundredstring=\@@HundredstringFspanish
\def\@andname{y}%
\def\@hundred{Cien}\def\@thousand{Mil}%
\@@numberstringspanish{#1}{#2}}
%    \end{macrocode}
% Make neuter same as masculine:
%    \begin{macrocode}
\let\@NumberstringNspanish\@NumberstringMspanish
%    \end{macrocode}
% As above, but for ordinals.
%    \begin{macrocode}
\DeclareRobustCommand{\@ordinalstringMspanish}[2]{%
\let\@unitthstring=\@@unitthstringspanish
\let\@unitstring=\@@unitstringspanish
\let\@teenthstring=\@@teenthstringspanish
\let\@tenthstring=\@@tenthstringspanish
\let\@hundredthstring=\@@hundredthstringspanish
\def\@thousandth{mil\'esimo}%
\@@ordinalstringspanish{#1}{#2}}
%    \end{macrocode}
% Feminine form:
%    \begin{macrocode}
\DeclareRobustCommand{\@ordinalstringFspanish}[2]{%
\let\@unitthstring=\@@unitthstringFspanish
\let\@unitstring=\@@unitstringFspanish
\let\@teenthstring=\@@teenthstringFspanish
\let\@tenthstring=\@@tenthstringFspanish
\let\@hundredthstring=\@@hundredthstringFspanish
\def\@thousandth{mil\'esima}%
\@@ordinalstringspanish{#1}{#2}}
%    \end{macrocode}
% Make neuter same as masculine:
%    \begin{macrocode}
\let\@ordinalstringNspanish\@ordinalstringMspanish
%    \end{macrocode}
% As above, but with initial letters in upper case.
%    \begin{macrocode}
\DeclareRobustCommand{\@OrdinalstringMspanish}[2]{%
\let\@unitthstring=\@@Unitthstringspanish
\let\@unitstring=\@@Unitstringspanish
\let\@teenthstring=\@@Teenthstringspanish
\let\@tenthstring=\@@Tenthstringspanish
\let\@hundredthstring=\@@Hundredthstringspanish
\def\@thousandth{Mil\'esimo}%
\@@ordinalstringspanish{#1}{#2}}
%    \end{macrocode}
% Feminine form:
%    \begin{macrocode}
\DeclareRobustCommand{\@OrdinalstringFspanish}[2]{%
\let\@unitthstring=\@@UnitthstringFspanish
\let\@unitstring=\@@UnitstringFspanish
\let\@teenthstring=\@@TeenthstringFspanish
\let\@tenthstring=\@@TenthstringFspanish
\let\@hundredthstring=\@@HundredthstringFspanish
\def\@thousandth{Mil\'esima}%
\@@ordinalstringspanish{#1}{#2}}
%    \end{macrocode}
% Make neuter same as masculine:
%    \begin{macrocode}
\let\@OrdinalstringNspanish\@OrdinalstringMspanish
%    \end{macrocode}
% Code for convert numbers into textual ordinals. As before,
% it is easier to split it into units, tens, teens and hundreds.
% Units:
%    \begin{macrocode}
\newcommand{\@@unitthstringspanish}[1]{%
\ifcase#1\relax
cero%
\or primero%
\or segundo%
\or tercero%
\or cuarto%
\or quinto%
\or sexto%
\or s\'eptimo%
\or octavo%
\or noveno%
\fi
}
%    \end{macrocode}
% Tens:
%    \begin{macrocode}
\newcommand{\@@tenthstringspanish}[1]{%
\ifcase#1\relax
\or d\'ecimo%
\or vig\'esimo%
\or trig\'esimo%
\or cuadrag\'esimo%
\or quincuag\'esimo%
\or sexag\'esimo%
\or septuag\'esimo%
\or octog\'esimo%
\or nonag\'esimo%
\fi
}
%    \end{macrocode}
% Teens:
%    \begin{macrocode}
\newcommand{\@@teenthstringspanish}[1]{%
\ifcase#1\relax
d\'ecimo%
\or und\'ecimo%
\or duod\'ecimo%
\or decimotercero%
\or decimocuarto%
\or decimoquinto%
\or decimosexto%
\or decimos\'eptimo%
\or decimoctavo%
\or decimonoveno%
\fi
}
%    \end{macrocode}
% Hundreds:
%    \begin{macrocode}
\newcommand{\@@hundredthstringspanish}[1]{%
\ifcase#1\relax
\or cent\'esimo%
\or ducent\'esimo%
\or tricent\'esimo%
\or cuadringent\'esimo%
\or quingent\'esimo%
\or sexcent\'esimo%
\or septing\'esimo%
\or octingent\'esimo%
\or noningent\'esimo%
\fi}
%    \end{macrocode}
% Units (feminine):
%    \begin{macrocode}
\newcommand{\@@unitthstringFspanish}[1]{%
\ifcase#1\relax
cera%
\or primera%
\or segunda%
\or tercera%
\or cuarta%
\or quinta%
\or sexta%
\or s\'eptima%
\or octava%
\or novena%
\fi
}
%    \end{macrocode}
% Tens (feminine):
%    \begin{macrocode}
\newcommand{\@@tenthstringFspanish}[1]{%
\ifcase#1\relax
\or d\'ecima%
\or vig\'esima%
\or trig\'esima%
\or cuadrag\'esima%
\or quincuag\'esima%
\or sexag\'esima%
\or septuag\'esima%
\or octog\'esima%
\or nonag\'esima%
\fi
}
%    \end{macrocode}
% Teens (feminine)
%    \begin{macrocode}
\newcommand{\@@teenthstringFspanish}[1]{%
\ifcase#1\relax
d\'ecima%
\or und\'ecima%
\or duod\'ecima%
\or decimotercera%
\or decimocuarta%
\or decimoquinta%
\or decimosexta%
\or decimos\'eptima%
\or decimoctava%
\or decimonovena%
\fi
}
%    \end{macrocode}
% Hundreds (feminine)
%    \begin{macrocode}
\newcommand{\@@hundredthstringFspanish}[1]{%
\ifcase#1\relax
\or cent\'esima%
\or ducent\'esima%
\or tricent\'esima%
\or cuadringent\'esima%
\or quingent\'esima%
\or sexcent\'esima%
\or septing\'esima%
\or octingent\'esima%
\or noningent\'esima%
\fi}
%    \end{macrocode}
% As above, but with initial letters in upper case
%    \begin{macrocode}
\newcommand{\@@Unitthstringspanish}[1]{%
\ifcase#1\relax
Cero%
\or Primero%
\or Segundo%
\or Tercero%
\or Cuarto%
\or Quinto%
\or Sexto%
\or S\'eptimo%
\or Octavo%
\or Noveno%
\fi
}
%    \end{macrocode}
% Tens:
%    \begin{macrocode}
\newcommand{\@@Tenthstringspanish}[1]{%
\ifcase#1\relax
\or D\'ecimo%
\or Vig\'esimo%
\or Trig\'esimo%
\or Cuadrag\'esimo%
\or Quincuag\'esimo%
\or Sexag\'esimo%
\or Septuag\'esimo%
\or Octog\'esimo%
\or Nonag\'esimo%
\fi
}
%    \end{macrocode}
% Teens:
%    \begin{macrocode}
\newcommand{\@@Teenthstringspanish}[1]{%
\ifcase#1\relax
D\'ecimo%
\or Und\'ecimo%
\or Duod\'ecimo%
\or Decimotercero%
\or Decimocuarto%
\or Decimoquinto%
\or Decimosexto%
\or Decimos\'eptimo%
\or Decimoctavo%
\or Decimonoveno%
\fi
}
%    \end{macrocode}
% Hundreds
%    \begin{macrocode}
\newcommand{\@@Hundredthstringspanish}[1]{%
\ifcase#1\relax
\or Cent\'esimo%
\or Ducent\'esimo%
\or Tricent\'esimo%
\or Cuadringent\'esimo%
\or Quingent\'esimo%
\or Sexcent\'esimo%
\or Septing\'esimo%
\or Octingent\'esimo%
\or Noningent\'esimo%
\fi}
%    \end{macrocode}
% As above, but feminine.
%    \begin{macrocode}
\newcommand{\@@UnitthstringFspanish}[1]{%
\ifcase#1\relax
Cera%
\or Primera%
\or Segunda%
\or Tercera%
\or Cuarta%
\or Quinta%
\or Sexta%
\or S\'eptima%
\or Octava%
\or Novena%
\fi
}
%    \end{macrocode}
% Tens (feminine)
%    \begin{macrocode}
\newcommand{\@@TenthstringFspanish}[1]{%
\ifcase#1\relax
\or D\'ecima%
\or Vig\'esima%
\or Trig\'esima%
\or Cuadrag\'esima%
\or Quincuag\'esima%
\or Sexag\'esima%
\or Septuag\'esima%
\or Octog\'esima%
\or Nonag\'esima%
\fi
}
%    \end{macrocode}
% Teens (feminine):
%    \begin{macrocode}
\newcommand{\@@TeenthstringFspanish}[1]{%
\ifcase#1\relax
D\'ecima%
\or Und\'ecima%
\or Duod\'ecima%
\or Decimotercera%
\or Decimocuarta%
\or Decimoquinta%
\or Decimosexta%
\or Decimos\'eptima%
\or Decimoctava%
\or Decimonovena%
\fi
}
%    \end{macrocode}
% Hundreds (feminine):
%    \begin{macrocode}
\newcommand{\@@HundredthstringFspanish}[1]{%
\ifcase#1\relax
\or Cent\'esima%
\or Ducent\'esima%
\or Tricent\'esima%
\or Cuadringent\'esima%
\or Quingent\'esima%
\or Sexcent\'esima%
\or Septing\'esima%
\or Octingent\'esima%
\or Noningent\'esima%
\fi}

%    \end{macrocode}
% This has changed in version 1.09, so that it now stores the
% results in the second argument (which must be a control
% sequence), but it doesn't display anything. Since it only
% affects internal macros, it shouldn't affect documnets created
% with older versions. (These internal macros are not meant for
% use in documents.)
%    \begin{macrocode}
\newcommand{\@@numberstringspanish}[2]{%
\ifnum#1>99999
\PackageError{fmtcount}{Out of range}%
{This macro only works for values less than 100000}%
\else
\ifnum#1<0
\PackageError{fmtcount}{Negative numbers not permitted}%
{This macro does not work for negative numbers, however
you can try typing "minus" first, and then pass the modulus of
this number}%
\fi
\fi
\def#2{}%
\@strctr=#1\relax \divide\@strctr by 1000\relax
\ifnum\@strctr>9
% #1 is greater or equal to 10000
  \divide\@strctr by 10
  \ifnum\@strctr>1
    \let\@@fc@numstr#2\relax
    \edef#2{\@@fc@numstr\@tenstring{\@strctr}}%
    \@strctr=#1 \divide\@strctr by 1000\relax
    \@modulo{\@strctr}{10}%
    \ifnum\@strctr>0\relax
       \let\@@fc@numstr#2\relax
       \edef#2{\@@fc@numstr\ \@andname\ \@unitstring{\@strctr}}%
    \fi
  \else
    \@strctr=#1\relax
    \divide\@strctr by 1000\relax
    \@modulo{\@strctr}{10}%
    \let\@@fc@numstr#2\relax
    \edef#2{\@@fc@numstr\@teenstring{\@strctr}}%
  \fi
  \let\@@fc@numstr#2\relax
  \edef#2{\@@fc@numstr\ \@thousand}%
\else
  \ifnum\@strctr>0\relax 
    \ifnum\@strctr>1\relax
       \let\@@fc@numstr#2\relax
       \edef#2{\@@fc@numstr\@unitstring{\@strctr}\ }%
    \fi
    \let\@@fc@numstr#2\relax
    \edef#2{\@@fc@numstr\@thousand}%
  \fi
\fi
\@strctr=#1\relax \@modulo{\@strctr}{1000}%
\divide\@strctr by 100\relax
\ifnum\@strctr>0\relax
  \ifnum#1>1000\relax
    \let\@@fc@numstr#2\relax
    \edef#2{\@@fc@numstr\ }%
  \fi
  \@tmpstrctr=#1\relax
  \@modulo{\@tmpstrctr}{1000}%
  \ifnum\@tmpstrctr=100\relax
    \let\@@fc@numstr#2\relax
    \edef#2{\@@fc@numstr\@tenstring{10}}%
  \else
    \let\@@fc@numstr#2\relax
    \edef#2{\@@fc@numstr\@hundredstring{\@strctr}}%
  \fi
\fi
\@strctr=#1\relax \@modulo{\@strctr}{100}%
\ifnum#1>100\relax
  \ifnum\@strctr>0\relax
    \let\@@fc@numstr#2\relax
    \edef#2{\@@fc@numstr\ \@andname\ }%
  \fi
\fi
\ifnum\@strctr>29\relax
  \divide\@strctr by 10\relax
  \let\@@fc@numstr#2\relax
  \edef#2{\@@fc@numstr\@tenstring{\@strctr}}%
  \@strctr=#1\relax \@modulo{\@strctr}{10}%
  \ifnum\@strctr>0\relax
    \let\@@fc@numstr#2\relax
    \edef#2{\@@fc@numstr\ \@andname\ \@unitstring{\@strctr}}%
  \fi
\else
  \ifnum\@strctr<10\relax
    \ifnum\@strctr=0\relax
      \ifnum#1<100\relax
        \let\@@fc@numstr#2\relax
        \edef#2{\@@fc@numstr\@unitstring{\@strctr}}%
      \fi
    \else
      \let\@@fc@numstr#2\relax
      \edef#2{\@@fc@numstr\@unitstring{\@strctr}}%
    \fi
  \else
    \ifnum\@strctr>19\relax
      \@modulo{\@strctr}{10}%
      \let\@@fc@numstr#2\relax
      \edef#2{\@@fc@numstr\@twentystring{\@strctr}}%
    \else
      \@modulo{\@strctr}{10}%
      \let\@@fc@numstr#2\relax
      \edef#2{\@@fc@numstr\@teenstring{\@strctr}}%
    \fi
  \fi
\fi
}
%    \end{macrocode}
% As above, but for ordinals
%    \begin{macrocode}
\newcommand{\@@ordinalstringspanish}[2]{%
\@strctr=#1\relax
\ifnum#1>99999
\PackageError{fmtcount}{Out of range}%
{This macro only works for values less than 100000}%
\else
\ifnum#1<0
\PackageError{fmtcount}{Negative numbers not permitted}%
{This macro does not work for negative numbers, however
you can try typing "minus" first, and then pass the modulus of
this number}%
\else
\def#2{}%
\ifnum\@strctr>999\relax
  \divide\@strctr by 1000\relax
  \ifnum\@strctr>1\relax
    \ifnum\@strctr>9\relax
      \@tmpstrctr=\@strctr
      \ifnum\@strctr<20
        \@modulo{\@tmpstrctr}{10}%
        \let\@@fc@ordstr#2\relax
        \edef#2{\@@fc@ordstr\@teenthstring{\@tmpstrctr}}%
      \else
        \divide\@tmpstrctr by 10\relax
        \let\@@fc@ordstr#2\relax
        \edef#2{\@@fc@ordstr\@tenthstring{\@tmpstrctr}}%
        \@tmpstrctr=\@strctr
        \@modulo{\@tmpstrctr}{10}%
        \ifnum\@tmpstrctr>0\relax
          \let\@@fc@ordstr#2\relax
          \edef#2{\@@fc@ordstr\@unitthstring{\@tmpstrctr}}%
        \fi
      \fi
    \else
       \let\@@fc@ordstr#2\relax
       \edef#2{\@@fc@ordstr\@unitstring{\@strctr}}%
    \fi
  \fi
  \let\@@fc@ordstr#2\relax
  \edef#2{\@@fc@ordstr\@thousandth}%
\fi
\@strctr=#1\relax
\@modulo{\@strctr}{1000}%
\ifnum\@strctr>99\relax
  \@tmpstrctr=\@strctr
  \divide\@tmpstrctr by 100\relax
  \ifnum#1>1000\relax
    \let\@@fc@ordstr#2\relax
    \edef#2{\@@fc@ordstr\ }%
  \fi
  \let\@@fc@ordstr#2\relax
  \edef#2{\@@fc@ordstr\@hundredthstring{\@tmpstrctr}}%
\fi
\@modulo{\@strctr}{100}%
\ifnum#1>99\relax
  \ifnum\@strctr>0\relax
    \let\@@fc@ordstr#2\relax
    \edef#2{\@@fc@ordstr\ }%
  \fi
\fi
\ifnum\@strctr>19\relax
  \@tmpstrctr=\@strctr
  \divide\@tmpstrctr by 10\relax
  \let\@@fc@ordstr#2\relax
  \edef#2{\@@fc@ordstr\@tenthstring{\@tmpstrctr}}%
  \@tmpstrctr=\@strctr
  \@modulo{\@tmpstrctr}{10}%
  \ifnum\@tmpstrctr>0\relax
    \let\@@fc@ordstr#2\relax
    \edef#2{\@@fc@ordstr\ \@unitthstring{\@tmpstrctr}}%
  \fi
\else
  \ifnum\@strctr>9\relax
    \@modulo{\@strctr}{10}%
    \let\@@fc@ordstr#2\relax
    \edef#2{\@@fc@ordstr\@teenthstring{\@strctr}}%
  \else
    \ifnum\@strctr=0\relax
      \ifnum#1=0\relax
        \let\@@fc@ordstr#2\relax
        \edef#2{\@@fc@ordstr\@unitstring{0}}%
      \fi
    \else
      \let\@@fc@ordstr#2\relax
      \edef#2{\@@fc@ordstr\@unitthstring{\@strctr}}%
    \fi
  \fi
\fi
\fi
\fi
}
%    \end{macrocode}
%\iffalse
%    \begin{macrocode}
%</fc-spanish.def>
%    \end{macrocode}
%\fi
%\iffalse
%    \begin{macrocode}
%<*fc-UKenglish.def>
%    \end{macrocode}
%\fi
% \subsection{fc-UKenglish.def}
% UK English definitions
%    \begin{macrocode}
\ProvidesFile{fc-UKenglish}[2007/06/14]
%    \end{macrocode}
% Check that fc-english.def has been loaded
%    \begin{macrocode}
\@ifundefined{@ordinalMenglish}{\input{fc-english.def}}{}
%    \end{macrocode}
% These are all just synonyms for the commands provided by
% fc-english.def.
%    \begin{macrocode}
\let\@ordinalMUKenglish\@ordinalMenglish
\let\@ordinalFUKenglish\@ordinalMenglish
\let\@ordinalNUKenglish\@ordinalMenglish
\let\@numberstringMUKenglish\@numberstringMenglish
\let\@numberstringFUKenglish\@numberstringMenglish
\let\@numberstringNUKenglish\@numberstringMenglish
\let\@NumberstringMUKenglish\@NumberstringMenglish
\let\@NumberstringFUKenglish\@NumberstringMenglish
\let\@NumberstringNUKenglish\@NumberstringMenglish
\let\@ordinalstringMUKenglish\@ordinalstringMenglish
\let\@ordinalstringFUKenglish\@ordinalstringMenglish
\let\@ordinalstringNUKenglish\@ordinalstringMenglish
\let\@OrdinalstringMUKenglish\@OrdinalstringMenglish
\let\@OrdinalstringFUKenglish\@OrdinalstringMenglish
\let\@OrdinalstringNUKenglish\@OrdinalstringMenglish
%    \end{macrocode}
%\iffalse
%    \begin{macrocode}
%</fc-UKenglish.def>
%    \end{macrocode}
%\fi
%\iffalse
%    \begin{macrocode}
%<*fc-USenglish.def>
%    \end{macrocode}
%\fi
% \subsection{fc-USenglish.def}
% US English definitions
%    \begin{macrocode}
\ProvidesFile{fc-USenglish}[2007/06/14]
%    \end{macrocode}
% Check that fc-english.def has been loaded
%    \begin{macrocode}
\@ifundefined{@ordinalMenglish}{\input{fc-english.def}}{}
%    \end{macrocode}
% These are all just synonyms for the commands provided by
% fc-english.def.
%    \begin{macrocode}
\let\@ordinalMUSenglish\@ordinalMenglish
\let\@ordinalFUSenglish\@ordinalMenglish
\let\@ordinalNUSenglish\@ordinalMenglish
\let\@numberstringMUSenglish\@numberstringMenglish
\let\@numberstringFUSenglish\@numberstringMenglish
\let\@numberstringNUSenglish\@numberstringMenglish
\let\@NumberstringMUSenglish\@NumberstringMenglish
\let\@NumberstringFUSenglish\@NumberstringMenglish
\let\@NumberstringNUSenglish\@NumberstringMenglish
\let\@ordinalstringMUSenglish\@ordinalstringMenglish
\let\@ordinalstringFUSenglish\@ordinalstringMenglish
\let\@ordinalstringNUSenglish\@ordinalstringMenglish
\let\@OrdinalstringMUSenglish\@OrdinalstringMenglish
\let\@OrdinalstringFUSenglish\@OrdinalstringMenglish
\let\@OrdinalstringNUSenglish\@OrdinalstringMenglish
%    \end{macrocode}
%\iffalse
%    \begin{macrocode}
%</fc-USenglish.def>
%    \end{macrocode}
%\fi
%\iffalse
%    \begin{macrocode}
%<*fmtcount.sty>
%    \end{macrocode}
%\fi
%\subsection{fmtcount.sty}
% This section deals with the code for |fmtcount.sty|
%    \begin{macrocode}
\NeedsTeXFormat{LaTeX2e}
\ProvidesPackage{fmtcount}[2007/06/22 v1.2]
\RequirePackage{ifthen}
\RequirePackage{keyval}
%    \end{macrocode}
% As from version 1.2, now load xspace package:
%    \begin{macrocode}
\RequirePackage{xspace}
%    \end{macrocode}
% These commands need to be defined before the
% configuration file is loaded.
%
% Define the macro to format the |st|, |nd|, |rd| or |th| of an 
% ordinal.
%    \begin{macrocode}
\providecommand{\fmtord}[1]{\textsuperscript{#1}}
%    \end{macrocode}
% Define |\padzeroes| to specify how many digits should be 
% displayed.
%    \begin{macrocode}
\newcount\c@padzeroesN
\c@padzeroesN=1\relax
\providecommand{\padzeroes}[1][17]{\c@padzeroesN=#1}
%    \end{macrocode}
% Load appropriate language definition files (I don't
% know if there is a standard way of detecting which
% languages are defined, so I'm just going to check
% if \verb"\date"\meta{language} is defined):
%\changes{v1.1}{14 June 2007}{added check for UKenglish,
% british and USenglish babel settings}
%    \begin{macrocode}
\@ifundefined{dateenglish}{}{\input{fc-english.def}}
\@ifundefined{l@UKenglish}{}{\input{fc-UKenglish.def}}
\@ifundefined{l@british}{}{\input{fc-british.def}}
\@ifundefined{l@USenglish}{}{\input{fc-USenglish.def}}
\@ifundefined{datespanish}{}{\input{fc-spanish.def}}
\@ifundefined{dateportuges}{}{\input{fc-portuges.def}}
\@ifundefined{datefrench}{}{\input{fc-french.def}}
\@ifundefined{dategerman}{%
\@ifundefined{datengerman}{}{\input{fc-german.def}}}{%
\input{fc-german.def}}
%    \end{macrocode}
% Define keys for use with |\fmtcountsetoptions|.
% Key to switch French dialects (Does babel store
%this kind of information?)
%    \begin{macrocode}
\def\fmtcount@french{france}
\define@key{fmtcount}{french}[france]{%
\@ifundefined{datefrench}{%
\PackageError{fmtcount}{Language `french' not defined}{You need
to load babel before loading fmtcount}}{
\ifthenelse{\equal{#1}{france}
         \or\equal{#1}{swiss}
         \or\equal{#1}{belgian}}{%
         \def\fmtcount@french{#1}}{%
\PackageError{fmtcount}{Invalid value `#1' to french key}
{Option `french' can only take the values `france', 
`belgian' or `swiss'}}
}}
%    \end{macrocode}
% Key to determine how to display the ordinal
%    \begin{macrocode}
\define@key{fmtcount}{fmtord}{%
\ifthenelse{\equal{#1}{level}
          \or\equal{#1}{raise}
          \or\equal{#1}{user}}{
          \def\fmtcount@fmtord{#1}}{%
\PackageError{fmtcount}{Invalid value `#1' to fmtord key}
{Option `fmtord' can only take the values `level', `raise' 
or `user'}}}
%    \end{macrocode}
% Key to determine whether the ordinal should be abbreviated
% (language dependent, currently only affects French ordinals.)
%    \begin{macrocode}
\newif\iffmtord@abbrv
\fmtord@abbrvfalse
\define@key{fmtcount}{abbrv}[true]{%
\ifthenelse{\equal{#1}{true}\or\equal{#1}{false}}{
          \csname fmtord@abbrv#1\endcsname}{%
\PackageError{fmtcount}{Invalid value `#1' to fmtord key}
{Option `fmtord' can only take the values `true' or
`false'}}}
%    \end{macrocode}
% Define command to set options.
%    \begin{macrocode}
\newcommand{\fmtcountsetoptions}[1]{%
\def\fmtcount@fmtord{}%
\setkeys{fmtcount}{#1}%
\@ifundefined{datefrench}{}{%
\edef\@ordinalstringMfrench{\noexpand
\csname @ordinalstringMfrench\fmtcount@french\noexpand\endcsname}%
\edef\@ordinalstringFfrench{\noexpand
\csname @ordinalstringFfrench\fmtcount@french\noexpand\endcsname}%
\edef\@OrdinalstringMfrench{\noexpand
\csname @OrdinalstringMfrench\fmtcount@french\noexpand\endcsname}%
\edef\@OrdinalstringFfrench{\noexpand
\csname @OrdinalstringFfrench\fmtcount@french\noexpand\endcsname}%
\edef\@numberstringMfrench{\noexpand
\csname @numberstringMfrench\fmtcount@french\noexpand\endcsname}%
\edef\@numberstringFfrench{\noexpand
\csname @numberstringFfrench\fmtcount@french\noexpand\endcsname}%
\edef\@NumberstringMfrench{\noexpand
\csname @NumberstringMfrench\fmtcount@french\noexpand\endcsname}%
\edef\@NumberstringFfrench{\noexpand
\csname @NumberstringFfrench\fmtcount@french\noexpand\endcsname}%
}%
%
\ifthenelse{\equal{\fmtcount@fmtord}{level}}{%
\renewcommand{\fmtord}[1]{##1}}{%
\ifthenelse{\equal{\fmtcount@fmtord}{raise}}{%
\renewcommand{\fmtord}[1]{\textsuperscript{##1}}}{%
}}
}
%    \end{macrocode}
% Load confguration file if it exists.  This needs to be done
% before the package options, to allow the user to override
% the settings in the configuration file.
%    \begin{macrocode}
\InputIfFileExists{fmtcount.cfg}{%
\typeout{Using configuration file fmtcount.cfg}}{%
\typeout{No configuration file fmtcount.cfg found.}}
%    \end{macrocode}
%Declare options
%    \begin{macrocode}
\DeclareOption{level}{\def\fmtcount@fmtord{level}%
\def\fmtord#1{#1}}
\DeclareOption{raise}{\def\fmtcount@fmtord{raise}%
\def\fmtord#1{\textsuperscript{#1}}}
%    \end{macrocode}
% Process package options
%    \begin{macrocode}
\ProcessOptions
%    \end{macrocode}
% Define macro that performs modulo arthmetic. This is used for the
% date, time, ordinal and numberstring commands. (The fmtcount
% package was originally part of the datetime package.)
%    \begin{macrocode}
\newcount\@DT@modctr
\def\@modulo#1#2{%
\@DT@modctr=#1\relax
\divide \@DT@modctr by #2\relax
\multiply \@DT@modctr by #2\relax
\advance #1 by -\@DT@modctr}
%    \end{macrocode}
% The following registers are needed by |\@ordinal| etc
%    \begin{macrocode}
\newcount\@ordinalctr
\newcount\@orgargctr
\newcount\@strctr
\newcount\@tmpstrctr
%    \end{macrocode}
%Define commands that display numbers in different bases.
% Define counters and conditionals needed.
%    \begin{macrocode}
\newif\if@DT@padzeroes
\newcount\@DT@loopN
\newcount\@DT@X
%    \end{macrocode}
% Binary
%    \begin{macrocode}
\newcommand{\@binary}[1]{%
\@DT@padzeroestrue
\@DT@loopN=17\relax
\@strctr=\@DT@loopN
\whiledo{\@strctr<\c@padzeroesN}{0\advance\@strctr by 1}%
\@strctr=65536\relax
\@DT@X=#1\relax
\loop
\@DT@modctr=\@DT@X
\divide\@DT@modctr by \@strctr
\ifthenelse{\boolean{@DT@padzeroes} \and \(\@DT@modctr=0\) \and \(\@DT@loopN>\c@padzeroesN\)}{}{\the\@DT@modctr}%
\ifnum\@DT@modctr=0\else\@DT@padzeroesfalse\fi
\multiply\@DT@modctr by \@strctr
\advance\@DT@X by -\@DT@modctr
\divide\@strctr by 2\relax
\advance\@DT@loopN by -1\relax
\ifnum\@strctr>1
\repeat
\the\@DT@X}

\let\binarynum=\@binary
%    \end{macrocode}
% Octal
%    \begin{macrocode}
\newcommand{\@octal}[1]{%
\ifnum#1>32768
\PackageError{fmtcount}{Value of counter too large for \protect\@octal}{Maximum value 32768}
\else
\@DT@padzeroestrue
\@DT@loopN=6\relax
\@strctr=\@DT@loopN
\whiledo{\@strctr<\c@padzeroesN}{0\advance\@strctr by 1}%
\@strctr=32768\relax
\@DT@X=#1\relax
\loop
\@DT@modctr=\@DT@X
\divide\@DT@modctr by \@strctr
\ifthenelse{\boolean{@DT@padzeroes} \and \(\@DT@modctr=0\) \and \(\@DT@loopN>\c@padzeroesN\)}{}{\the\@DT@modctr}%
\ifnum\@DT@modctr=0\else\@DT@padzeroesfalse\fi
\multiply\@DT@modctr by \@strctr
\advance\@DT@X by -\@DT@modctr
\divide\@strctr by 8\relax
\advance\@DT@loopN by -1\relax
\ifnum\@strctr>1
\repeat
\the\@DT@X
\fi}
\let\octalnum=\@octal
%    \end{macrocode}
% Lowercase hexadecimal
%    \begin{macrocode}
\newcommand{\@@hexadecimal}[1]{\ifcase#10\or1\or2\or3\or4\or5\or6\or7\or8\or9\or a\or b\or c\or d\or e\or f\fi}

\newcommand{\@hexadecimal}[1]{%
\@DT@padzeroestrue
\@DT@loopN=5\relax
\@strctr=\@DT@loopN
\whiledo{\@strctr<\c@padzeroesN}{0\advance\@strctr by 1}%
\@strctr=65536\relax
\@DT@X=#1\relax
\loop
\@DT@modctr=\@DT@X
\divide\@DT@modctr by \@strctr
\ifthenelse{\boolean{@DT@padzeroes} \and \(\@DT@modctr=0\) \and \(\@DT@loopN>\c@padzeroesN\)}{}{\@@hexadecimal\@DT@modctr}%
\ifnum\@DT@modctr=0\else\@DT@padzeroesfalse\fi
\multiply\@DT@modctr by \@strctr
\advance\@DT@X by -\@DT@modctr
\divide\@strctr by 16\relax
\advance\@DT@loopN by -1\relax
\ifnum\@strctr>1
\repeat
\@@hexadecimal\@DT@X}

\let\hexadecimalnum=\@hexadecimal
%    \end{macrocode}
% Uppercase hexadecimal
%    \begin{macrocode}
\newcommand{\@@Hexadecimal}[1]{\ifcase#10\or1\or2\or3\or4\or5\or6\or
7\or8\or9\or A\or B\or C\or D\or E\or F\fi}

\newcommand{\@Hexadecimal}[1]{%
\@DT@padzeroestrue
\@DT@loopN=5\relax
\@strctr=\@DT@loopN
\whiledo{\@strctr<\c@padzeroesN}{0\advance\@strctr by 1}%
\@strctr=65536\relax
\@DT@X=#1\relax
\loop
\@DT@modctr=\@DT@X
\divide\@DT@modctr by \@strctr
\ifthenelse{\boolean{@DT@padzeroes} \and \(\@DT@modctr=0\) \and \(\@DT@loopN>\c@padzeroesN\)}{}{\@@Hexadecimal\@DT@modctr}%
\ifnum\@DT@modctr=0\else\@DT@padzeroesfalse\fi
\multiply\@DT@modctr by \@strctr
\advance\@DT@X by -\@DT@modctr
\divide\@strctr by 16\relax
\advance\@DT@loopN by -1\relax
\ifnum\@strctr>1
\repeat
\@@Hexadecimal\@DT@X}

\let\Hexadecimalnum=\@Hexadecimal
%    \end{macrocode}
% Uppercase alphabetical representation (a \ldots\ z aa \ldots\ zz)
%    \begin{macrocode}
\newcommand{\@aaalph}[1]{%
\@DT@loopN=#1\relax
\advance\@DT@loopN by -1\relax
\divide\@DT@loopN by 26\relax
\@DT@modctr=\@DT@loopN
\multiply\@DT@modctr by 26\relax
\@DT@X=#1\relax
\advance\@DT@X by -1\relax
\advance\@DT@X by -\@DT@modctr
\advance\@DT@loopN by 1\relax
\advance\@DT@X by 1\relax
\loop
\@alph\@DT@X
\advance\@DT@loopN by -1\relax
\ifnum\@DT@loopN>0
\repeat
}

\let\aaalphnum=\@aaalph
%    \end{macrocode}
% Uppercase alphabetical representation (a \ldots\ z aa \ldots\ zz)
%    \begin{macrocode}
\newcommand{\@AAAlph}[1]{%
\@DT@loopN=#1\relax
\advance\@DT@loopN by -1\relax
\divide\@DT@loopN by 26\relax
\@DT@modctr=\@DT@loopN
\multiply\@DT@modctr by 26\relax
\@DT@X=#1\relax
\advance\@DT@X by -1\relax
\advance\@DT@X by -\@DT@modctr
\advance\@DT@loopN by 1\relax
\advance\@DT@X by 1\relax
\loop
\@Alph\@DT@X
\advance\@DT@loopN by -1\relax
\ifnum\@DT@loopN>0
\repeat
}

\let\AAAlphnum=\@AAAlph
%    \end{macrocode}
% Lowercase alphabetical representation
%    \begin{macrocode}
\newcommand{\@abalph}[1]{%
\ifnum#1>17576
\PackageError{fmtcount}{Value of counter too large for \protect\@abalph}{Maximum value 17576}
\else
\@DT@padzeroestrue
\@strctr=17576\relax
\@DT@X=#1\relax
\advance\@DT@X by -1\relax
\loop
\@DT@modctr=\@DT@X
\divide\@DT@modctr by \@strctr
\ifthenelse{\boolean{@DT@padzeroes} \and \(\@DT@modctr=1\)}{}{\@alph\@DT@modctr}%
\ifnum\@DT@modctr=1\else\@DT@padzeroesfalse\fi
\multiply\@DT@modctr by \@strctr
\advance\@DT@X by -\@DT@modctr
\divide\@strctr by 26\relax
\ifnum\@strctr>1
\repeat
\advance\@DT@X by 1\relax
\@alph\@DT@X
\fi}

\let\abalphnum=\@abalph
%    \end{macrocode}
% Uppercase alphabetical representation
%    \begin{macrocode}
\newcommand{\@ABAlph}[1]{%
\ifnum#1>17576
\PackageError{fmtcount}{Value of counter too large for \protect\@ABAlph}{Maximum value 17576}
\else
\@DT@padzeroestrue
\@strctr=17576\relax
\@DT@X=#1\relax
\advance\@DT@X by -1\relax
\loop
\@DT@modctr=\@DT@X
\divide\@DT@modctr by \@strctr
\ifthenelse{\boolean{@DT@padzeroes} \and \(\@DT@modctr=1\)}{}{\@Alph\@DT@modctr}%
\ifnum\@DT@modctr=1\else\@DT@padzeroesfalse\fi
\multiply\@DT@modctr by \@strctr
\advance\@DT@X by -\@DT@modctr
\divide\@strctr by 26\relax
\ifnum\@strctr>1
\repeat
\advance\@DT@X by 1\relax
\@Alph\@DT@X
\fi}

\let\ABAlphnum=\@ABAlph
%    \end{macrocode}
% Recursive command to count number of characters in argument.
% |\@strctr| should be set to zero before calling it.
%    \begin{macrocode}
\def\@fmtc@count#1#2\relax{%
\if\relax#1
\else
\advance\@strctr by 1\relax
\@fmtc@count#2\relax
\fi}
%    \end{macrocode}
% Internal decimal macro:
%    \begin{macrocode}
\newcommand{\@decimal}[1]{%
\@strctr=0\relax
\expandafter\@fmtc@count\number#1\relax
\@DT@loopN=\c@padzeroesN
\advance\@DT@loopN by -\@strctr
\ifnum\@DT@loopN>0\relax
\@strctr=0\relax
\whiledo{\@strctr < \@DT@loopN}{0\advance\@strctr by 1}%
\fi
\number#1\relax
}

\let\decimalnum=\@decimal
%    \end{macrocode}
% This is a bit cumbersome.  Previously \verb"\@ordinal"
% was defined in a similar way to \verb"\abalph" etc.
% This ensured that the actual value of the counter was
% written in the new label stuff in the .aux file. However
% adding in an optional argument to determine the gender
% for multilingual compatibility messed things up somewhat.
% This was the only work around I could get to keep the
% the cross-referencing stuff working, which is why
% the optional argument comes \emph{after} the compulsory
% argument, instead of the usual manner of placing it before.
% Version 1.04 changed \verb"\ordinal" to \verb"\FCordinal"
% to prevent it clashing with the memoir class. 
%    \begin{macrocode}
\newcommand{\FCordinal}[1]{%
\expandafter\protect\expandafter\ordinalnum{%
\expandafter\the\csname c@#1\endcsname}}
%    \end{macrocode}
% If \verb"\ordinal" isn't defined make \verb"\ordinal" a synonym
% for \verb"\FCordinal" to maintain compatibility with previous
% versions.
%    \begin{macrocode}
\@ifundefined{ordinal}{\let\ordinal\FCordinal}{%
\PackageWarning{fmtcount}{\string\ordinal
\space already defined use \string\FCordinal \space instead.}}
%    \end{macrocode}
% Display ordinal where value is given as a number or 
% count register instead of a counter:
%    \begin{macrocode}
\newcommand{\ordinalnum}[1]{\@ifnextchar[{\@ordinalnum{#1}}{%
\@ordinalnum{#1}[m]}}
%    \end{macrocode}
% Display ordinal according to gender (neuter added in v1.1,
% \cmdname{xspace} added in v1.2):
%    \begin{macrocode}
\def\@ordinalnum#1[#2]{{%
\ifthenelse{\equal{#2}{f}}{%
\protect\@ordinalF{#1}{\@fc@ordstr}}{%
\ifthenelse{\equal{#2}{n}}{%
\protect\@ordinalN{#1}{\@fc@ordstr}}{%
\ifthenelse{\equal{#2}{m}}{}{%
\PackageError{fmtcount}{Invalid gender option `#2'}{%
Available options are m, f or n}}%
\protect\@ordinalM{#1}{\@fc@ordstr}}}\@fc@ordstr}\xspace}
%    \end{macrocode}
% Store the ordinal (first argument
% is identifying name, second argument is a counter.)
%    \begin{macrocode}
\newcommand*{\storeordinal}[2]{%
\expandafter\protect\expandafter\storeordinalnum{#1}{%
\expandafter\the\csname c@#2\endcsname}}
%    \end{macrocode}
% Store ordinal (first argument
% is identifying name, second argument is a number or
% count register.)
%    \begin{macrocode}
\newcommand*{\storeordinalnum}[2]{%
\@ifnextchar[{\@storeordinalnum{#1}{#2}}{%
\@storeordinalnum{#1}{#2}[m]}}
%    \end{macrocode}
% Store ordinal according to gender:
%    \begin{macrocode}
\def\@storeordinalnum#1#2[#3]{%
\ifthenelse{\equal{#3}{f}}{%
\protect\@ordinalF{#2}{\@fc@ord}}{%
\ifthenelse{\equal{#3}{n}}{%
\protect\@ordinalN{#2}{\@fc@ord}}{%
\ifthenelse{\equal{#3}{m}}{}{%
\PackageError{fmtcount}{Invalid gender option `#3'}{%
Available options are m or f}}%
\protect\@ordinalM{#2}{\@fc@ord}}}%
\expandafter\let\csname @fcs@#1\endcsname\@fc@ord}
%    \end{macrocode}
% Get stored information:
%    \begin{macrocode}
\newcommand*{\FMCuse}[1]{\csname @fcs@#1\endcsname}
%    \end{macrocode}
% Display ordinal as a string (argument is a counter)
%    \begin{macrocode}
\newcommand{\ordinalstring}[1]{%
\expandafter\protect\expandafter\ordinalstringnum{%
\expandafter\the\csname c@#1\endcsname}}
%    \end{macrocode}
% Display ordinal as a string (argument is a count register or
% number.)
%    \begin{macrocode}
\newcommand{\ordinalstringnum}[1]{%
\@ifnextchar[{\@ordinal@string{#1}}{\@ordinal@string{#1}[m]}}
%    \end{macrocode}
% Display ordinal as a string according to gender (\cmdname{xspace}
% added in version 1.2).
%    \begin{macrocode}
\def\@ordinal@string#1[#2]{{%
\ifthenelse{\equal{#2}{f}}{%
\protect\@ordinalstringF{#1}{\@fc@ordstr}}{%
\ifthenelse{\equal{#2}{n}}{%
\protect\@ordinalstringN{#1}{\@fc@ordstr}}{%
\ifthenelse{\equal{#2}{m}}{}{%
\PackageError{fmtcount}{Invalid gender option `#2' to 
\string\ordinalstring}{Available options are m, f or f}}%
\protect\@ordinalstringM{#1}{\@fc@ordstr}}}\@fc@ordstr}\xspace}
%    \end{macrocode}
% Store textual representation of number. First argument is 
% identifying name, second argument is the counter set to the 
% required number.
%    \begin{macrocode}
\newcommand{\storeordinalstring}[2]{%
\expandafter\protect\expandafter\storeordinalstringnum{#1}{%
\expandafter\the\csname c@#2\endcsname}}
%    \end{macrocode}
% Store textual representation of number. First argument is 
% identifying name, second argument is a count register or number.
%    \begin{macrocode}
\newcommand{\storeordinalstringnum}[2]{%
\@ifnextchar[{\@store@ordinal@string{#1}{#2}}{%
\@store@ordinal@string{#1}{#2}[m]}}
%    \end{macrocode}
% Store textual representation of number according to gender.
%    \begin{macrocode}
\def\@store@ordinal@string#1#2[#3]{%
\ifthenelse{\equal{#3}{f}}{%
\protect\@ordinalstringF{#2}{\@fc@ordstr}}{%
\ifthenelse{\equal{#3}{n}}{%
\protect\@ordinalstringN{#2}{\@fc@ordstr}}{%
\ifthenelse{\equal{#3}{m}}{}{%
\PackageError{fmtcount}{Invalid gender option `#3' to 
\string\ordinalstring}{Available options are m, f or n}}%
\protect\@ordinalstringM{#2}{\@fc@ordstr}}}%
\expandafter\let\csname @fcs@#1\endcsname\@fc@ordstr}
%    \end{macrocode}
% Display ordinal as a string with initial letters in upper case
% (argument is a counter)
%    \begin{macrocode}
\newcommand{\Ordinalstring}[1]{%
\expandafter\protect\expandafter\Ordinalstringnum{%
\expandafter\the\csname c@#1\endcsname}}
%    \end{macrocode}
% Display ordinal as a string with initial letters in upper case
% (argument is a number or count register)
%    \begin{macrocode}
\newcommand{\Ordinalstringnum}[1]{%
\@ifnextchar[{\@Ordinal@string{#1}}{\@Ordinal@string{#1}[m]}}
%    \end{macrocode}
% Display ordinal as a string with initial letters in upper case
% according to gender
%    \begin{macrocode}
\def\@Ordinal@string#1[#2]{{%
\ifthenelse{\equal{#2}{f}}{%
\protect\@OrdinalstringF{#1}{\@fc@ordstr}}{%
\ifthenelse{\equal{#2}{n}}{%
\protect\@OrdinalstringN{#1}{\@fc@ordstr}}{%
\ifthenelse{\equal{#2}{m}}{}{%
\PackageError{fmtcount}{Invalid gender option `#2'}{%
Available options are m, f or n}}%
\protect\@OrdinalstringM{#1}{\@fc@ordstr}}}\@fc@ordstr}\xspace}
%    \end{macrocode}
% Store textual representation of number, with initial letters in 
% upper case. First argument is identifying name, second argument 
% is the counter set to the 
% required number.
%    \begin{macrocode}
\newcommand{\storeOrdinalstring}[2]{%
\expandafter\protect\expandafter\storeOrdinalstringnum{#1}{%
\expandafter\the\csname c@#2\endcsname}}
%    \end{macrocode}
% Store textual representation of number, with initial letters in 
% upper case. First argument is identifying name, second argument 
% is a count register or number.
%    \begin{macrocode}
\newcommand{\storeOrdinalstringnum}[2]{%
\@ifnextchar[{\@store@Ordinal@string{#1}{#2}}{%
\@store@Ordinal@string{#1}{#2}[m]}}
%    \end{macrocode}
% Store textual representation of number according to gender, 
% with initial letters in upper case.
%    \begin{macrocode}
\def\@store@Ordinal@string#1#2[#3]{%
\ifthenelse{\equal{#3}{f}}{%
\protect\@OrdinalstringF{#2}{\@fc@ordstr}}{%
\ifthenelse{\equal{#3}{n}}{%
\protect\@OrdinalstringN{#2}{\@fc@ordstr}}{%
\ifthenelse{\equal{#3}{m}}{}{%
\PackageError{fmtcount}{Invalid gender option `#3'}{%
Available options are m or f}}%
\protect\@OrdinalstringM{#2}{\@fc@ordstr}}}%
\expandafter\let\csname @fcs@#1\endcsname\@fc@ordstr}
%    \end{macrocode}
% Store upper case textual representation of ordinal. The first 
% argument is identifying name, the second argument is a counter.
%    \begin{macrocode}
\newcommand{\storeORDINALstring}[2]{%
\expandafter\protect\expandafter\storeORDINALstringnum{#1}{%
\expandafter\the\csname c@#2\endcsname}}
%    \end{macrocode}
% As above, but the second argument is a count register or a
% number.
%    \begin{macrocode}
\newcommand{\storeORDINALstringnum}[2]{%
\@ifnextchar[{\@store@ORDINAL@string{#1}{#2}}{%
\@store@ORDINAL@string{#1}{#2}[m]}}
%    \end{macrocode}
% Gender is specified as an optional argument at the end.
%    \begin{macrocode}
\def\@store@ORDINAL@string#1#2[#3]{%
\ifthenelse{\equal{#3}{f}}{%
\protect\@ordinalstringF{#2}{\@fc@ordstr}}{%
\ifthenelse{\equal{#3}{n}}{%
\protect\@ordinalstringN{#2}{\@fc@ordstr}}{%
\ifthenelse{\equal{#3}{m}}{}{%
\PackageError{fmtcount}{Invalid gender option `#3'}{%
Available options are m or f}}%
\protect\@ordinalstringM{#2}{\@fc@ordstr}}}%
\expandafter\edef\csname @fcs@#1\endcsname{%
\noexpand\MakeUppercase{\@fc@ordstr}}}
%    \end{macrocode}
% Display upper case textual representation of an ordinal. The
% argument must be a counter.
%    \begin{macrocode}
\newcommand{\ORDINALstring}[1]{%
\expandafter\protect\expandafter\ORDINALstringnum{%
\expandafter\the\csname c@#1\endcsname}}
%    \end{macrocode}
% As above, but the argument is a count register or a number.
%    \begin{macrocode}
\newcommand{\ORDINALstringnum}[1]{%
\@ifnextchar[{\@ORDINAL@string{#1}}{\@ORDINAL@string{#1}[m]}}
%    \end{macrocode}
% Gender is specified as an optional argument at the end.
%    \begin{macrocode}
\def\@ORDINAL@string#1[#2]{{%
\ifthenelse{\equal{#2}{f}}{%
\protect\@ordinalstringF{#1}{\@fc@ordstr}}{%
\ifthenelse{\equal{#2}{n}}{%
\protect\@ordinalstringN{#1}{\@fc@ordstr}}{%
\ifthenelse{\equal{#2}{m}}{}{%
\PackageError{fmtcount}{Invalid gender option `#2'}{%
Available options are m, f or n}}%
\protect\@ordinalstringM{#1}{\@fc@ordstr}}}%
\MakeUppercase{\@fc@ordstr}}\xspace}
%    \end{macrocode}
% Convert number to textual respresentation, and store. First 
% argument is the identifying name, second argument is a counter 
% containing the number.
%    \begin{macrocode}
\newcommand{\storenumberstring}[2]{%
\expandafter\protect\expandafter\storenumberstringnum{#1}{%
\expandafter\the\csname c@#2\endcsname}}
%    \end{macrocode}
% As above, but second argument is a number or count register.
%    \begin{macrocode}
\newcommand{\storenumberstringnum}[2]{%
\@ifnextchar[{\@store@number@string{#1}{#2}}{%
\@store@number@string{#1}{#2}[m]}}
%    \end{macrocode}
% Gender is given as optional argument, \emph{at the end}.
%    \begin{macrocode}
\def\@store@number@string#1#2[#3]{%
\ifthenelse{\equal{#3}{f}}{%
\protect\@numberstringF{#2}{\@fc@numstr}}{%
\ifthenelse{\equal{#3}{n}}{%
\protect\@numberstringN{#2}{\@fc@numstr}}{%
\ifthenelse{\equal{#3}{m}}{}{%
\PackageError{fmtcount}{Invalid gender option `#3'}{%
Available options are m, f or n}}%
\protect\@numberstringM{#2}{\@fc@numstr}}}%
\expandafter\let\csname @fcs@#1\endcsname\@fc@numstr}
%    \end{macrocode}
% Display textual representation of a number. The argument
% must be a counter.
%    \begin{macrocode}
\newcommand{\numberstring}[1]{%
\expandafter\protect\expandafter\numberstringnum{%
\expandafter\the\csname c@#1\endcsname}}
%    \end{macrocode}
% As above, but the argument is a count register or a number.
%    \begin{macrocode}
\newcommand{\numberstringnum}[1]{%
\@ifnextchar[{\@number@string{#1}}{\@number@string{#1}[m]}}
%    \end{macrocode}
% Gender is specified as an optional argument \emph{at the end}.
%    \begin{macrocode}
\def\@number@string#1[#2]{{%
\ifthenelse{\equal{#2}{f}}{%
\protect\@numberstringF{#1}{\@fc@numstr}}{%
\ifthenelse{\equal{#2}{n}}{%
\protect\@numberstringN{#1}{\@fc@numstr}}{%
\ifthenelse{\equal{#2}{m}}{}{%
\PackageError{fmtcount}{Invalid gender option `#2'}{%
Available options are m, f or n}}%
\protect\@numberstringM{#1}{\@fc@numstr}}}\@fc@numstr}\xspace}
%    \end{macrocode}
% Store textual representation of number. First argument is 
% identifying name, second argument is a counter.
%    \begin{macrocode}
\newcommand{\storeNumberstring}[2]{%
\expandafter\protect\expandafter\storeNumberstringnum{#1}{%
\expandafter\the\csname c@#2\endcsname}}
%    \end{macrocode}
% As above, but second argument is a count register or number.
%    \begin{macrocode}
\newcommand{\storeNumberstringnum}[2]{%
\@ifnextchar[{\@store@Number@string{#1}{#2}}{%
\@store@Number@string{#1}{#2}[m]}}
%    \end{macrocode}
% Gender is specified as an optional argument \emph{at the end}:
%    \begin{macrocode}
\def\@store@Number@string#1#2[#3]{%
\ifthenelse{\equal{#3}{f}}{%
\protect\@NumberstringF{#2}{\@fc@numstr}}{%
\ifthenelse{\equal{#3}{n}}{%
\protect\@NumberstringN{#2}{\@fc@numstr}}{%
\ifthenelse{\equal{#3}{m}}{}{%
\PackageError{fmtcount}{Invalid gender option `#3'}{%
Available options are m, f or n}}%
\protect\@NumberstringM{#2}{\@fc@numstr}}}%
\expandafter\let\csname @fcs@#1\endcsname\@fc@numstr}
%    \end{macrocode}
% Display textual representation of number. The argument must be
% a counter. 
%    \begin{macrocode}
\newcommand{\Numberstring}[1]{%
\expandafter\protect\expandafter\Numberstringnum{%
\expandafter\the\csname c@#1\endcsname}}
%    \end{macrocode}
% As above, but the argument is a count register or number.
%    \begin{macrocode}
\newcommand{\Numberstringnum}[1]{%
\@ifnextchar[{\@Number@string{#1}}{\@Number@string{#1}[m]}}
%    \end{macrocode}
% Gender is specified as an optional argument at the end.
%    \begin{macrocode}
\def\@Number@string#1[#2]{{%
\ifthenelse{\equal{#2}{f}}{%
\protect\@NumberstringF{#1}{\@fc@numstr}}{%
\ifthenelse{\equal{#2}{n}}{%
\protect\@NumberstringN{#1}{\@fc@numstr}}{%
\ifthenelse{\equal{#2}{m}}{}{%
\PackageError{fmtcount}{Invalid gender option `#2'}{%
Available options are m, f or n}}%
\protect\@NumberstringM{#1}{\@fc@numstr}}}\@fc@numstr}\xspace}
%    \end{macrocode}
% Store upper case textual representation of number. The first 
% argument is identifying name, the second argument is a counter.
%    \begin{macrocode}
\newcommand{\storeNUMBERstring}[2]{%
\expandafter\protect\expandafter\storeNUMBERstringnum{#1}{%
\expandafter\the\csname c@#2\endcsname}}
%    \end{macrocode}
% As above, but the second argument is a count register or a
% number.
%    \begin{macrocode}
\newcommand{\storeNUMBERstringnum}[2]{%
\@ifnextchar[{\@store@NUMBER@string{#1}{#2}}{%
\@store@NUMBER@string{#1}{#2}[m]}}
%    \end{macrocode}
% Gender is specified as an optional argument at the end.
%    \begin{macrocode}
\def\@store@NUMBER@string#1#2[#3]{%
\ifthenelse{\equal{#3}{f}}{%
\protect\@numberstringF{#2}{\@fc@numstr}}{%
\ifthenelse{\equal{#3}{n}}{%
\protect\@numberstringN{#2}{\@fc@numstr}}{%
\ifthenelse{\equal{#3}{m}}{}{%
\PackageError{fmtcount}{Invalid gender option `#3'}{%
Available options are m or f}}%
\protect\@numberstringM{#2}{\@fc@numstr}}}%
\expandafter\edef\csname @fcs@#1\endcsname{%
\noexpand\MakeUppercase{\@fc@numstr}}}
%    \end{macrocode}
% Display upper case textual representation of a number. The
% argument must be a counter.
%    \begin{macrocode}
\newcommand{\NUMBERstring}[1]{%
\expandafter\protect\expandafter\NUMBERstringnum{%
\expandafter\the\csname c@#1\endcsname}}
%    \end{macrocode}
% As above, but the argument is a count register or a number.
%    \begin{macrocode}
\newcommand{\NUMBERstringnum}[1]{%
\@ifnextchar[{\@NUMBER@string{#1}}{\@NUMBER@string{#1}[m]}}
%    \end{macrocode}
% Gender is specified as an optional argument at the end.
%    \begin{macrocode}
\def\@NUMBER@string#1[#2]{{%
\ifthenelse{\equal{#2}{f}}{%
\protect\@numberstringF{#1}{\@fc@numstr}}{%
\ifthenelse{\equal{#2}{n}}{%
\protect\@numberstringN{#1}{\@fc@numstr}}{%
\ifthenelse{\equal{#2}{m}}{}{%
\PackageError{fmtcount}{Invalid gender option `#2'}{%
Available options are m, f or n}}%
\protect\@numberstringM{#1}{\@fc@numstr}}}%
\MakeUppercase{\@fc@numstr}}\xspace}
%    \end{macrocode}
% Number representations in other bases. Binary:
%    \begin{macrocode}
\providecommand{\binary}[1]{%
\expandafter\protect\expandafter\@binary{%
\expandafter\the\csname c@#1\endcsname}}
%    \end{macrocode}
% Like \verb"\alph", but goes beyond 26. (a \ldots\ z aa \ldots zz \ldots)
%    \begin{macrocode}
\providecommand{\aaalph}[1]{%
\expandafter\protect\expandafter\@aaalph{%
\expandafter\the\csname c@#1\endcsname}}
%    \end{macrocode}
% As before, but upper case.
%    \begin{macrocode}
\providecommand{\AAAlph}[1]{%
\expandafter\protect\expandafter\@AAAlph{%
\expandafter\the\csname c@#1\endcsname}}
%    \end{macrocode}
% Like \verb"\alph", but goes beyond 26. (a \ldots\ z ab \ldots az \ldots)
%    \begin{macrocode}
\providecommand{\abalph}[1]{%
\expandafter\protect\expandafter\@abalph{%
\expandafter\the\csname c@#1\endcsname}}
%    \end{macrocode}
% As above, but upper case.
%    \begin{macrocode}
\providecommand{\ABAlph}[1]{%
\expandafter\protect\expandafter\@ABAlph{%
\expandafter\the\csname c@#1\endcsname}}
%    \end{macrocode}
% Hexadecimal:
%    \begin{macrocode}
\providecommand{\hexadecimal}[1]{%
\expandafter\protect\expandafter\@hexadecimal{%
\expandafter\the\csname c@#1\endcsname}}
%    \end{macrocode}
% As above, but in upper case.
%    \begin{macrocode}
\providecommand{\Hexadecimal}[1]{%
\expandafter\protect\expandafter\@Hexadecimal{%
\expandafter\the\csname c@#1\endcsname}}
%    \end{macrocode}
% Octal:
%    \begin{macrocode}
\providecommand{\octal}[1]{%
\expandafter\protect\expandafter\@octal{%
\expandafter\the\csname c@#1\endcsname}}
%    \end{macrocode}
% Decimal:
%    \begin{macrocode}
\providecommand{\decimal}[1]{%
\expandafter\protect\expandafter\@decimal{%
\expandafter\the\csname c@#1\endcsname}}
%    \end{macrocode}
%\subsubsection{Multilinguage Definitions}
% If multilingual support is provided, make \verb"\@numberstring" 
% etc use the correct language (if defined).
% Otherwise use English definitions. "\@setdef@ultfmtcount"
% sets the macros to use English.
%    \begin{macrocode}
\def\@setdef@ultfmtcount{
\@ifundefined{@ordinalMenglish}{\input{fc-english.def}}{}
\def\@ordinalstringM{\@ordinalstringMenglish}
\let\@ordinalstringF=\@ordinalstringMenglish
\let\@ordinalstringN=\@ordinalstringMenglish
\def\@OrdinalstringM{\@OrdinalstringMenglish}
\let\@OrdinalstringF=\@OrdinalstringMenglish
\let\@OrdinalstringN=\@OrdinalstringMenglish
\def\@numberstringM{\@numberstringMenglish}
\let\@numberstringF=\@numberstringMenglish
\let\@numberstringN=\@numberstringMenglish
\def\@NumberstringM{\@NumberstringMenglish}
\let\@NumberstringF=\@NumberstringMenglish
\let\@NumberstringN=\@NumberstringMenglish
\def\@ordinalM{\@ordinalMenglish}
\let\@ordinalF=\@ordinalM
\let\@ordinalN=\@ordinalM
}
%    \end{macrocode}
% Define a command to set macros to use "languagename":
%    \begin{macrocode}
\def\@set@mulitling@fmtcount{%
%
\def\@numberstringM{\@ifundefined{@numberstringM\languagename}{%
\PackageError{fmtcount}{No support for language '\languagename'}{%
The fmtcount package currently does not support language 
'\languagename' for command \string\@numberstringM}}{%
\csname @numberstringM\languagename\endcsname}}%
%
\def\@numberstringF{\@ifundefined{@numberstringF\languagename}{%
\PackageError{fmtcount}{No support for language '\languagename'}{%
The fmtcount package currently does not support language 
'\languagename' for command \string\@numberstringF}}{%
\csname @numberstringF\languagename\endcsname}}%
%
\def\@numberstringN{\@ifundefined{@numberstringN\languagename}{%
\PackageError{fmtcount}{No support for language '\languagename'}{%
The fmtcount package currently does not support language 
'\languagename' for command \string\@numberstringN}}{%
\csname @numberstringN\languagename\endcsname}}%
%
\def\@NumberstringM{\@ifundefined{@NumberstringM\languagename}{%
\PackageError{fmtcount}{No support for language '\languagename'}{%
The fmtcount package currently does not support language 
'\languagename' for command \string\@NumberstringM}}{%
\csname @NumberstringM\languagename\endcsname}}%
%
\def\@NumberstringF{\@ifundefined{@NumberstringF\languagename}{%
\PackageError{fmtcount}{No support for language '\languagename'}{%
The fmtcount package currently does not support language 
'\languagename' for command \string\@NumberstringF}}{%
\csname @NumberstringF\languagename\endcsname}}%
%
\def\@NumberstringN{\@ifundefined{@NumberstringN\languagename}{%
\PackageError{fmtcount}{No support for language '\languagename'}{%
The fmtcount package currently does not support language 
'\languagename' for command \string\@NumberstringN}}{%
\csname @NumberstringN\languagename\endcsname}}%
%
\def\@ordinalM{\@ifundefined{@ordinalM\languagename}{%
\PackageError{fmtcount}{No support for language '\languagename'}{%
The fmtcount package currently does not support language 
'\languagename' for command \string\@ordinalM}}{%
\csname @ordinalM\languagename\endcsname}}%
%
\def\@ordinalF{\@ifundefined{@ordinalF\languagename}{%
\PackageError{fmtcount}{No support for language '\languagename'}{%
The fmtcount package currently does not support language 
'\languagename' for command \string\@ordinalF}}{%
\csname @ordinalF\languagename\endcsname}}%
%
\def\@ordinalN{\@ifundefined{@ordinalN\languagename}{%
\PackageError{fmtcount}{No support for language '\languagename'}{%
The fmtcount package currently does not support language 
'\languagename' for command \string\@ordinalN}}{%
\csname @ordinalN\languagename\endcsname}}%
%
\def\@ordinalstringM{\@ifundefined{@ordinalstringM\languagename}{%
\PackageError{fmtcount}{No support for language '\languagename'}{%
The fmtcount package currently does not support language 
'\languagename' for command \string\@ordinalstringM}}{%
\csname @ordinalstringM\languagename\endcsname}}%
%
\def\@ordinalstringF{\@ifundefined{@ordinalstringF\languagename}{%
\PackageError{fmtcount}{No support for language '\languagename'}{%
The fmtcount package currently does not support language 
'\languagename' for command \string\@ordinalstringF}}{%
\csname @ordinalstringF\languagename\endcsname}}%
%
\def\@ordinalstringN{\@ifundefined{@ordinalstringN\languagename}{%
\PackageError{fmtcount}{No support for language '\languagename'}{%
The fmtcount package currently does not support language 
'\languagename' for command \string\@ordinalstringN}}{%
\csname @ordinalstringN\languagename\endcsname}}%
%
\def\@OrdinalstringM{\@ifundefined{@OrdinalstringM\languagename}{%
\PackageError{fmtcount}{No support for language '\languagename'}{%
The fmtcount package currently does not support language 
'\languagename' for command \string\@OrdinalstringM}}{%
\csname @OrdinalstringM\languagename\endcsname}}%
%
\def\@OrdinalstringF{\@ifundefined{@OrdinalstringF\languagename}{%
\PackageError{fmtcount}{No support for language '\languagename'}{%
The fmtcount package currently does not support language 
'\languagename' for command \string\@OrdinalstringF}}{%
\csname @OrdinalstringF\languagename\endcsname}}%
%
\def\@OrdinalstringN{\@ifundefined{@OrdinalstringN\languagename}{%
\PackageError{fmtcount}{No support for language '\languagename'}{%
The fmtcount package currently does not support language 
'\languagename' for command \string\@OrdinalstringN}}{%
\csname @OrdinalstringN\languagename\endcsname}}
}
%    \end{macrocode}
% Check to see if babel or ngerman packages have been loaded.
%    \begin{macrocode}
\@ifpackageloaded{babel}{%
\ifthenelse{\equal{\languagename}{nohyphenation}\or
\equal{languagename}{english}}{\@setdef@ultfmtcount}{%
\@set@mulitling@fmtcount}
}{%
\@ifpackageloaded{ngerman}{%
\@ifundefined{@numberstringMgerman}{%
\input{fc-german.def}}{}\@set@mulitling@fmtcount}{%
\@setdef@ultfmtcount}}
%    \end{macrocode}
% Backwards compatibility:
%    \begin{macrocode}
\let\@ordinal=\@ordinalM
\let\@ordinalstring=\@ordinalstringM
\let\@Ordinalstring=\@OrdinalstringM
\let\@numberstring=\@numberstringM
\let\@Numberstring=\@NumberstringM
%    \end{macrocode}
%\iffalse
%    \begin{macrocode}
%</fmtcount.sty>
%    \end{macrocode}
%\fi
%\Finale
\endinput
}
%\end{verbatim}
%This, I agree, is an unpleasant cludge.
%
%\end{itemize}
%
%\section{Acknowledgements}
%
%I would like to thank my mother for the French and Portuguese
%support and my Spanish dictionary for the Spanish support.
%Thank you to K. H. Fricke for providing me with the German
%translations.
%
%\section{Troubleshooting}
%
%There is a FAQ available at: \url{http://theoval.cmp.uea.ac.uk/~nlct/latex/packages/faq/}.
%
% \section{Contact Details}
% Dr Nicola Talbot\\
% School of Computing Sciences\\
% University of East Anglia\\
% Norwich.  NR4 7TJ.\\
% United Kingdom.\\
% \url{http://theoval.cmp.uea.ac.uk/~nlct/}
%
%
%\StopEventually{}
%\section{The Code}
%\iffalse
%    \begin{macrocode}
%<*fc-british.def>
%    \end{macrocode}
%\fi
% \subsection{fc-british.def}
% British definitions
%    \begin{macrocode}
\ProvidesFile{fc-british}[2007/06/14]
%    \end{macrocode}
% Check that fc-english.def has been loaded
%    \begin{macrocode}
\@ifundefined{@ordinalMenglish}{\input{fc-english.def}}{}
%    \end{macrocode}
% These are all just synonyms for the commands provided by
% fc-english.def.
%    \begin{macrocode}
\let\@ordinalMbritish\@ordinalMenglish
\let\@ordinalFbritish\@ordinalMenglish
\let\@ordinalNbritish\@ordinalMenglish
\let\@numberstringMbritish\@numberstringMenglish
\let\@numberstringFbritish\@numberstringMenglish
\let\@numberstringNbritish\@numberstringMenglish
\let\@NumberstringMbritish\@NumberstringMenglish
\let\@NumberstringFbritish\@NumberstringMenglish
\let\@NumberstringNbritish\@NumberstringMenglish
\let\@ordinalstringMbritish\@ordinalstringMenglish
\let\@ordinalstringFbritish\@ordinalstringMenglish
\let\@ordinalstringNbritish\@ordinalstringMenglish
\let\@OrdinalstringMbritish\@OrdinalstringMenglish
\let\@OrdinalstringFbritish\@OrdinalstringMenglish
\let\@OrdinalstringNbritish\@OrdinalstringMenglish
%    \end{macrocode}
%\iffalse
%    \begin{macrocode}
%</fc-british.def>
%    \end{macrocode}
%\fi
%\iffalse
%    \begin{macrocode}
%<*fc-english.def>
%    \end{macrocode}
%\fi
% \subsection{fc-english.def}
% English definitions
%    \begin{macrocode}
\ProvidesFile{fc-english}[2007/05/26]
%    \end{macrocode}
% Define macro that converts a number or count register (first 
% argument) to an ordinal, and stores the result in the 
% second argument, which should be a control sequence.
%    \begin{macrocode}
\newcommand*{\@ordinalMenglish}[2]{%
\def\@fc@ord{}%
\@orgargctr=#1\relax
\@ordinalctr=#1%
\@modulo{\@ordinalctr}{100}%
\ifnum\@ordinalctr=11\relax
  \def\@fc@ord{th}%
\else
  \ifnum\@ordinalctr=12\relax
    \def\@fc@ord{th}%
  \else
    \ifnum\@ordinalctr=13\relax
      \def\@fc@ord{th}%
    \else
      \@modulo{\@ordinalctr}{10}%
      \ifcase\@ordinalctr
        \def\@fc@ord{th}%      case 0
        \or \def\@fc@ord{st}%  case 1
        \or \def\@fc@ord{nd}%  case 2
        \or \def\@fc@ord{rd}%  case 3
      \else
        \def\@fc@ord{th}%      default case
      \fi
    \fi
  \fi
\fi
\edef#2{\number#1\relax\noexpand\fmtord{\@fc@ord}}%
}
%    \end{macrocode}
% There is no gender difference in English, so make feminine and
% neuter the same as the masculine.
%    \begin{macrocode}
\let\@ordinalFenglish=\@ordinalMenglish
\let\@ordinalNenglish=\@ordinalMenglish
%    \end{macrocode}
% Define the macro that prints the value of a \TeX\ count register
% as text. To make it easier, break it up into units, teens and
% tens. First, the units: the argument should be between 0 and 9
% inclusive.
%    \begin{macrocode}
\newcommand*{\@@unitstringenglish}[1]{%
\ifcase#1\relax
zero%
\or one%
\or two%
\or three%
\or four%
\or five%
\or six%
\or seven%
\or eight%
\or nine%
\fi
}
%    \end{macrocode}
% Next the tens, again the argument should be between 0 and 9
% inclusive.
%    \begin{macrocode}
\newcommand*{\@@tenstringenglish}[1]{%
\ifcase#1\relax
\or ten%
\or twenty%
\or thirty%
\or forty%
\or fifty%
\or sixty%
\or seventy%
\or eighty%
\or ninety%
\fi
}
%    \end{macrocode}
% Finally the teens, again the argument should be between 0 and 9
% inclusive.
%    \begin{macrocode}
\newcommand*{\@@teenstringenglish}[1]{%
\ifcase#1\relax
ten%
\or eleven%
\or twelve%
\or thirteen%
\or fourteen%
\or fifteen%
\or sixteen%
\or seventeen%
\or eighteen%
\or nineteen%
\fi
}
%    \end{macrocode}
% As above, but with the initial letter in uppercase. The units:
%    \begin{macrocode}
\newcommand*{\@@Unitstringenglish}[1]{%
\ifcase#1\relax
Zero%
\or One%
\or Two%
\or Three%
\or Four%
\or Five%
\or Six%
\or Seven%
\or Eight%
\or Nine%
\fi
}
%    \end{macrocode}
% The tens:
%    \begin{macrocode}
\newcommand*{\@@Tenstringenglish}[1]{%
\ifcase#1\relax
\or Ten%
\or Twenty%
\or Thirty%
\or Forty%
\or Fifty%
\or Sixty%
\or Seventy%
\or Eighty%
\or Ninety%
\fi
}
%    \end{macrocode}
% The teens:
%    \begin{macrocode}
\newcommand*{\@@Teenstringenglish}[1]{%
\ifcase#1\relax
Ten%
\or Eleven%
\or Twelve%
\or Thirteen%
\or Fourteen%
\or Fifteen%
\or Sixteen%
\or Seventeen%
\or Eighteen%
\or Nineteen%
\fi
}
%    \end{macrocode}
% This has changed in version 1.09, so that it now stores
% the result in the second argument, but doesn't display anything.
% Since it only affects internal macros, it shouldn't affect
% documents created with older versions. (These internal macros are
% not meant for use in documents.)
%    \begin{macrocode}
\newcommand*{\@@numberstringenglish}[2]{%
\ifnum#1>99999
\PackageError{fmtcount}{Out of range}%
{This macro only works for values less than 100000}%
\else
\ifnum#1<0
\PackageError{fmtcount}{Negative numbers not permitted}%
{This macro does not work for negative numbers, however
you can try typing "minus" first, and then pass the modulus of
this number}%
\fi
\fi
\def#2{}%
\@strctr=#1\relax \divide\@strctr by 1000\relax
\ifnum\@strctr>9
% #1 is greater or equal to 10000
  \divide\@strctr by 10
  \ifnum\@strctr>1\relax
    \let\@@fc@numstr#2\relax
    \edef#2{\@@fc@numstr\@tenstring{\@strctr}}%
    \@strctr=#1 \divide\@strctr by 1000\relax
    \@modulo{\@strctr}{10}%
    \ifnum\@strctr>0\relax
      \let\@@fc@numstr#2\relax
      \edef#2{\@@fc@numstr-\@unitstring{\@strctr}}%
    \fi
  \else
    \@strctr=#1\relax
    \divide\@strctr by 1000\relax
    \@modulo{\@strctr}{10}%
    \let\@@fc@numstr#2\relax
    \edef#2{\@@fc@numstr\@teenstring{\@strctr}}%
  \fi
  \let\@@fc@numstr#2\relax
  \edef#2{\@@fc@numstr\ \@thousand}%
\else
  \ifnum\@strctr>0\relax
    \let\@@fc@numstr#2\relax
    \edef#2{\@@fc@numstr\@unitstring{\@strctr}\ \@thousand}%
  \fi
\fi
\@strctr=#1\relax \@modulo{\@strctr}{1000}%
\divide\@strctr by 100
\ifnum\@strctr>0\relax
   \ifnum#1>1000\relax
      \let\@@fc@numstr#2\relax
      \edef#2{\@@fc@numstr\ }%
   \fi
   \let\@@fc@numstr#2\relax
   \edef#2{\@@fc@numstr\@unitstring{\@strctr}\ \@hundred}%
\fi
\@strctr=#1\relax \@modulo{\@strctr}{100}%
\ifnum#1>100\relax
  \ifnum\@strctr>0\relax
    \let\@@fc@numstr#2\relax
    \edef#2{\@@fc@numstr\ \@andname\ }%
  \fi
\fi
\ifnum\@strctr>19\relax
  \divide\@strctr by 10\relax
  \let\@@fc@numstr#2\relax
  \edef#2{\@@fc@numstr\@tenstring{\@strctr}}%
  \@strctr=#1\relax \@modulo{\@strctr}{10}%
  \ifnum\@strctr>0\relax
    \let\@@fc@numstr#2\relax
    \edef#2{\@@fc@numstr-\@unitstring{\@strctr}}%
  \fi
\else
  \ifnum\@strctr<10\relax
    \ifnum\@strctr=0\relax
       \ifnum#1<100\relax
          \let\@@fc@numstr#2\relax
          \edef#2{\@@fc@numstr\@unitstring{\@strctr}}%
       \fi
    \else
      \let\@@fc@numstr#2\relax
      \edef#2{\@@fc@numstr\@unitstring{\@strctr}}%
    \fi
  \else
    \@modulo{\@strctr}{10}%
    \let\@@fc@numstr#2\relax
    \edef#2{\@@fc@numstr\@teenstring{\@strctr}}%
  \fi
\fi
}
%    \end{macrocode}
% All lower case version, the second argument must be a 
% control sequence.
%    \begin{macrocode}
\DeclareRobustCommand{\@numberstringMenglish}[2]{%
\let\@unitstring=\@@unitstringenglish 
\let\@teenstring=\@@teenstringenglish 
\let\@tenstring=\@@tenstringenglish
\def\@hundred{hundred}\def\@thousand{thousand}%
\def\@andname{and}%
\@@numberstringenglish{#1}{#2}%
}
%    \end{macrocode}
% There is no gender in English, so make feminine and neuter the same
% as the masculine.
%    \begin{macrocode}
\let\@numberstringFenglish=\@numberstringMenglish
\let\@numberstringNenglish=\@numberstringMenglish
%    \end{macrocode}
% This version makes the first letter of each word an uppercase
% character (except ``and''). The second argument must be a control 
% sequence.
%    \begin{macrocode}
\newcommand*{\@NumberstringMenglish}[2]{%
\let\@unitstring=\@@Unitstringenglish 
\let\@teenstring=\@@Teenstringenglish 
\let\@tenstring=\@@Tenstringenglish
\def\@hundred{Hundred}\def\@thousand{Thousand}%
\def\@andname{and}%
\@@numberstringenglish{#1}{#2}}
%    \end{macrocode}
% There is no gender in English, so make feminine and neuter the same
% as the masculine.
%    \begin{macrocode}
\let\@NumberstringFenglish=\@NumberstringMenglish
\let\@NumberstringNenglish=\@NumberstringMenglish
%    \end{macrocode}
% Define a macro that produces an ordinal as a string. Again, break
% it up into units, teens and tens. First the units:
%    \begin{macrocode}
\newcommand*{\@@unitthstringenglish}[1]{%
\ifcase#1\relax
zeroth%
\or first%
\or second%
\or third%
\or fourth%
\or fifth%
\or sixth%
\or seventh%
\or eighth%
\or ninth%
\fi
}
%    \end{macrocode}
% Next the tens:
%    \begin{macrocode}
\newcommand*{\@@tenthstringenglish}[1]{%
\ifcase#1\relax
\or tenth%
\or twentieth%
\or thirtieth%
\or fortieth%
\or fiftieth%
\or sixtieth%
\or seventieth%
\or eightieth%
\or ninetieth%
\fi
}
%   \end{macrocode}
% The teens:
%   \begin{macrocode}
\newcommand*{\@@teenthstringenglish}[1]{%
\ifcase#1\relax
tenth%
\or eleventh%
\or twelfth%
\or thirteenth%
\or fourteenth%
\or fifteenth%
\or sixteenth%
\or seventeenth%
\or eighteenth%
\or nineteenth%
\fi
}
%   \end{macrocode}
% As before, but with the first letter in upper case. The units:
%   \begin{macrocode}
\newcommand*{\@@Unitthstringenglish}[1]{%
\ifcase#1\relax
Zeroth%
\or First%
\or Second%
\or Third%
\or Fourth%
\or Fifth%
\or Sixth%
\or Seventh%
\or Eighth%
\or Ninth%
\fi
}
%    \end{macrocode}
% The tens:
%    \begin{macrocode}
\newcommand*{\@@Tenthstringenglish}[1]{%
\ifcase#1\relax
\or Tenth%
\or Twentieth%
\or Thirtieth%
\or Fortieth%
\or Fiftieth%
\or Sixtieth%
\or Seventieth%
\or Eightieth%
\or Ninetieth%
\fi
}
%    \end{macrocode}
% The teens:
%    \begin{macrocode}
\newcommand*{\@@Teenthstringenglish}[1]{%
\ifcase#1\relax
Tenth%
\or Eleventh%
\or Twelfth%
\or Thirteenth%
\or Fourteenth%
\or Fifteenth%
\or Sixteenth%
\or Seventeenth%
\or Eighteenth%
\or Nineteenth%
\fi
}
%    \end{macrocode}
% Again, as from version 1.09, this has been changed to take two
% arguments, where the second argument is a control sequence.
% The resulting text is stored in the control sequence, and nothing
% is displayed.
%    \begin{macrocode}
\newcommand*{\@@ordinalstringenglish}[2]{%
\@strctr=#1\relax
\ifnum#1>99999
\PackageError{fmtcount}{Out of range}%
{This macro only works for values less than 100000 (value given: \number\@strctr)}%
\else
\ifnum#1<0
\PackageError{fmtcount}{Negative numbers not permitted}%
{This macro does not work for negative numbers, however
you can try typing "minus" first, and then pass the modulus of
this number}%
\fi
\def#2{}%
\fi
\@strctr=#1\relax \divide\@strctr by 1000\relax
\ifnum\@strctr>9\relax
% #1 is greater or equal to 10000
  \divide\@strctr by 10
  \ifnum\@strctr>1\relax
    \let\@@fc@ordstr#2\relax
    \edef#2{\@@fc@ordstr\@tenstring{\@strctr}}%
    \@strctr=#1\relax
    \divide\@strctr by 1000\relax
    \@modulo{\@strctr}{10}%
    \ifnum\@strctr>0\relax
      \let\@@fc@ordstr#2\relax
      \edef#2{\@@fc@ordstr-\@unitstring{\@strctr}}%
    \fi
  \else
    \@strctr=#1\relax \divide\@strctr by 1000\relax
    \@modulo{\@strctr}{10}%
    \let\@@fc@ordstr#2\relax
    \edef#2{\@@fc@ordstr\@teenstring{\@strctr}}%
  \fi
  \@strctr=#1\relax \@modulo{\@strctr}{1000}%
  \ifnum\@strctr=0\relax
    \let\@@fc@ordstr#2\relax
    \edef#2{\@@fc@ordstr\ \@thousandth}%
  \else
    \let\@@fc@ordstr#2\relax
    \edef#2{\@@fc@ordstr\ \@thousand}%
  \fi
\else
  \ifnum\@strctr>0\relax
    \let\@@fc@ordstr#2\relax
    \edef#2{\@@fc@ordstr\@unitstring{\@strctr}}%
    \@strctr=#1\relax \@modulo{\@strctr}{1000}%
    \let\@@fc@ordstr#2\relax
    \ifnum\@strctr=0\relax
      \edef#2{\@@fc@ordstr\ \@thousandth}%
    \else
      \edef#2{\@@fc@ordstr\ \@thousand}%
    \fi
  \fi
\fi
\@strctr=#1\relax \@modulo{\@strctr}{1000}%
\divide\@strctr by 100
\ifnum\@strctr>0\relax
  \ifnum#1>1000\relax
    \let\@@fc@ordstr#2\relax
    \edef#2{\@@fc@ordstr\ }%
  \fi
  \let\@@fc@ordstr#2\relax
  \edef#2{\@@fc@ordstr\@unitstring{\@strctr}}%
  \@strctr=#1\relax \@modulo{\@strctr}{100}%
  \let\@@fc@ordstr#2\relax
  \ifnum\@strctr=0\relax
    \edef#2{\@@fc@ordstr\ \@hundredth}%
  \else
    \edef#2{\@@fc@ordstr\ \@hundred}%
  \fi
\fi
\@strctr=#1\relax \@modulo{\@strctr}{100}%
\ifnum#1>100\relax
  \ifnum\@strctr>0\relax
    \let\@@fc@ordstr#2\relax
    \edef#2{\@@fc@ordstr\ \@andname\ }%
  \fi
\fi
\ifnum\@strctr>19\relax
  \@tmpstrctr=\@strctr
  \divide\@strctr by 10\relax
  \@modulo{\@tmpstrctr}{10}%
  \let\@@fc@ordstr#2\relax
  \ifnum\@tmpstrctr=0\relax
    \edef#2{\@@fc@ordstr\@tenthstring{\@strctr}}%
  \else
    \edef#2{\@@fc@ordstr\@tenstring{\@strctr}}%
  \fi
  \@strctr=#1\relax \@modulo{\@strctr}{10}%
  \ifnum\@strctr>0\relax
    \let\@@fc@ordstr#2\relax
    \edef#2{\@@fc@ordstr-\@unitthstring{\@strctr}}%
  \fi
\else
  \ifnum\@strctr<10\relax
    \ifnum\@strctr=0\relax
      \ifnum#1<100\relax
        \let\@@fc@ordstr#2\relax
        \edef#2{\@@fc@ordstr\@unitthstring{\@strctr}}%
      \fi
    \else
      \let\@@fc@ordstr#2\relax
      \edef#2{\@@fc@ordstr\@unitthstring{\@strctr}}%
    \fi
  \else
    \@modulo{\@strctr}{10}%
    \let\@@fc@ordstr#2\relax
    \edef#2{\@@fc@ordstr\@teenthstring{\@strctr}}%
  \fi
\fi
}
%    \end{macrocode}
% All lower case version. Again, the second argument must be a
% control sequence in which the resulting text is stored.
%    \begin{macrocode}
\DeclareRobustCommand{\@ordinalstringMenglish}[2]{%
\let\@unitthstring=\@@unitthstringenglish 
\let\@teenthstring=\@@teenthstringenglish 
\let\@tenthstring=\@@tenthstringenglish
\let\@unitstring=\@@unitstringenglish 
\let\@teenstring=\@@teenstringenglish
\let\@tenstring=\@@tenstringenglish
\def\@andname{and}%
\def\@hundred{hundred}\def\@thousand{thousand}%
\def\@hundredth{hundredth}\def\@thousandth{thousandth}%
\@@ordinalstringenglish{#1}{#2}}
%    \end{macrocode}
% No gender in English, so make feminine and neuter same as masculine:
%    \begin{macrocode}
\let\@ordinalstringFenglish=\@ordinalstringMenglish
\let\@ordinalstringNenglish=\@ordinalstringMenglish
%    \end{macrocode}
% First letter of each word in upper case:
%    \begin{macrocode}
\DeclareRobustCommand{\@OrdinalstringMenglish}[2]{%
\let\@unitthstring=\@@Unitthstringenglish
\let\@teenthstring=\@@Teenthstringenglish
\let\@tenthstring=\@@Tenthstringenglish
\let\@unitstring=\@@Unitstringenglish
\let\@teenstring=\@@Teenstringenglish
\let\@tenstring=\@@Tenstringenglish
\def\@andname{and}%
\def\@hundred{Hundred}\def\@thousand{Thousand}%
\def\@hundredth{Hundredth}\def\@thousandth{Thousandth}%
\@@ordinalstringenglish{#1}{#2}}
%    \end{macrocode}
% No gender in English, so make feminine and neuter same as masculine:
%    \begin{macrocode}
\let\@OrdinalstringFenglish=\@OrdinalstringMenglish
\let\@OrdinalstringNenglish=\@OrdinalstringMenglish
%    \end{macrocode}
%\iffalse
%    \begin{macrocode}
%</fc-english.def>
%    \end{macrocode}
%\fi
%\iffalse
%    \begin{macrocode}
%<*fc-french.def>
%    \end{macrocode}
%\fi
% \subsection{fc-french.def}
% French definitions
%    \begin{macrocode}
\ProvidesFile{fc-french.def}[2007/05/26]
%    \end{macrocode}
% Define macro that converts a number or count register (first
% argument) to an ordinal, and store the result in the second
% argument, which must be a control sequence. Masculine:
%    \begin{macrocode}
\newcommand*{\@ordinalMfrench}[2]{%
\iffmtord@abbrv
  \edef#2{\number#1\relax\noexpand\fmtord{e}}%
\else
  \ifnum#1=1\relax
    \edef#2{\number#1\relax\noexpand\fmtord{er}}%
  \else
    \edef#2{\number#1\relax\noexpand\fmtord{eme}}%
  \fi
\fi}
%    \end{macrocode}
% Feminine:
%    \begin{macrocode}
\newcommand*{\@ordinalFfrench}[2]{%
\iffmtord@abbrv
  \edef#2{\number#1\relax\noexpand\fmtord{e}}%
\else
  \ifnum#1=1\relax
     \edef#2{\number#1\relax\noexpand\fmtord{ere}}%
  \else
     \edef#2{\number#1\relax\noexpand\fmtord{eme}}%
  \fi
\fi}
%    \end{macrocode}
% Make neuter same as masculine:
%    \begin{macrocode}
\let\@ordinalNfrench\@ordinalMfrench
%    \end{macrocode}
% Textual representation of a number. To make it easier break it
% into units, tens and teens. First the units:
%   \begin{macrocode}
\newcommand*{\@@unitstringfrench}[1]{%
\ifcase#1\relax
zero%
\or un%
\or deux%
\or trois%
\or quatre%
\or cinq%
\or six%
\or sept%
\or huit%
\or neuf%
\fi
}
%    \end{macrocode}
% Feminine only changes for 1:
%    \begin{macrocode}
\newcommand*{\@@unitstringFfrench}[1]{%
\ifnum#1=1\relax
une%
\else\@@unitstringfrench{#1}%
\fi
}
%    \end{macrocode}
% Tens (this includes the Belgian and Swiss variants, special
% cases employed lower down.)
%    \begin{macrocode}
\newcommand*{\@@tenstringfrench}[1]{%
\ifcase#1\relax
\or dix%
\or vingt%
\or trente%
\or quarante%
\or cinquante%
\or soixante%
\or septente%
\or huitante%
\or nonente%
\or cent%
\fi
}
%    \end{macrocode}
% Teens:
%    \begin{macrocode}
\newcommand*{\@@teenstringfrench}[1]{%
\ifcase#1\relax
dix%
\or onze%
\or douze%
\or treize%
\or quatorze%
\or quinze%
\or seize%
\or dix-sept%
\or dix-huit%
\or dix-neuf%
\fi
}
%    \end{macrocode}
% Seventies are a special case, depending on dialect:
%    \begin{macrocode}
\newcommand*{\@@seventiesfrench}[1]{%
\@tenstring{6}%
\ifnum#1=1\relax
\ \@andname\ 
\else
-%
\fi
\@teenstring{#1}%
}
%    \end{macrocode}
% Eighties are a special case, depending on dialect:
%    \begin{macrocode}
\newcommand*{\@@eightiesfrench}[1]{%
\@unitstring{4}-\@tenstring{2}%
\ifnum#1>0
-\@unitstring{#1}%
\else
s%
\fi
}
%    \end{macrocode}
% Nineties are a special case, depending on dialect:
%    \begin{macrocode}
\newcommand*{\@@ninetiesfrench}[1]{%
\@unitstring{4}-\@tenstring{2}-\@teenstring{#1}%
}
%    \end{macrocode}
% Swiss seventies:
%    \begin{macrocode}
\newcommand*{\@@seventiesfrenchswiss}[1]{%
\@tenstring{7}%
\ifnum#1=1\ \@andname\ \fi
\ifnum#1>1-\fi
\ifnum#1>0\@unitstring{#1}\fi
}
%    \end{macrocode}
% Swiss eighties:
%    \begin{macrocode}
\newcommand*{\@@eightiesfrenchswiss}[1]{%
\@tenstring{8}%
\ifnum#1=1\ \@andname\ \fi
\ifnum#1>1-\fi
\ifnum#1>0\@unitstring{#1}\fi
}
%    \end{macrocode}
% Swiss nineties:
%    \begin{macrocode}
\newcommand*{\@@ninetiesfrenchswiss}[1]{%
\@tenstring{9}%
\ifnum#1=1\ \@andname\ \fi
\ifnum#1>1-\fi
\ifnum#1>0\@unitstring{#1}\fi
}
%    \end{macrocode}
% Units with initial letter in upper case:
%    \begin{macrocode}
\newcommand*{\@@Unitstringfrench}[1]{%
\ifcase#1\relax
Zero%
\or Un%
\or Deux%
\or Trois%
\or Quatre%
\or Cinq%
\or Six%
\or Sept%
\or Huit%
\or Neuf%
\fi
}
%    \end{macrocode}
% As above, but feminine:
%    \begin{macrocode}
\newcommand*{\@@UnitstringFfrench}[1]{%
\ifnum#1=1\relax
Une%
\else \@@Unitstringfrench{#1}%
\fi
}
%    \end{macrocode}
% Tens, with initial letter in upper case (includes Swiss and
% Belgian variants):
%    \begin{macrocode}
\newcommand*{\@@Tenstringfrench}[1]{%
\ifcase#1\relax
\or Dix%
\or Vingt%
\or Trente%
\or Quarante%
\or Cinquante%
\or Soixante%
\or Septente%
\or Huitante%
\or Nonente%
\or Cent%
\fi
}
%    \end{macrocode}
% Teens, with initial letter in upper case:
%    \begin{macrocode}
\newcommand*{\@@Teenstringfrench}[1]{%
\ifcase#1\relax
Dix%
\or Onze%
\or Douze%
\or Treize%
\or Quatorze%
\or Quinze%
\or Seize%
\or Dix-Sept%
\or Dix-Huit%
\or Dix-Neuf%
\fi
}
%    \end{macrocode}
% This has changed in version 1.09, so that it now stores the
% result in the second argument, but doesn't display anything.
% Since it only affects internal macros, it shouldn't affect
% documents created with older versions. (These internal macros
% are not defined for use in documents.) Firstly, the Swiss
% version:
%    \begin{macrocode}
\DeclareRobustCommand{\@numberstringMfrenchswiss}[2]{%
\let\@unitstring=\@@unitstringfrench
\let\@teenstring=\@@teenstringfrench
\let\@tenstring=\@@tenstringfrench
\let\@seventies=\@@seventiesfrenchswiss
\let\@eighties=\@@eightiesfrenchswiss
\let\@nineties=\@@ninetiesfrenchswiss
\def\@hundred{cent}\def\@thousand{mille}%
\def\@andname{et}%
\@@numberstringfrench{#1}{#2}}
%    \end{macrocode}
% Same as above, but for French as spoken in France:
%    \begin{macrocode}
\DeclareRobustCommand{\@numberstringMfrenchfrance}[2]{%
\let\@unitstring=\@@unitstringfrench
\let\@teenstring=\@@teenstringfrench
\let\@tenstring=\@@tenstringfrench
\let\@seventies=\@@seventiesfrench
\let\@eighties=\@@eightiesfrench
\let\@nineties=\@@ninetiesfrench
\def\@hundred{cent}\def\@thousand{mille}%
\def\@andname{et}%
\@@numberstringfrench{#1}{#2}}
%    \end{macrocode}
% Same as above, but for Belgian dialect:
%    \begin{macrocode}
\DeclareRobustCommand{\@numberstringMfrenchbelgian}[2]{%
\let\@unitstring=\@@unitstringfrench
\let\@teenstring=\@@teenstringfrench
\let\@tenstring=\@@tenstringfrench
\let\@seventies=\@@seventiesfrenchswiss
\let\@eighties=\@@eightiesfrench
\let\@nineties=\@@ninetiesfrench
\def\@hundred{cent}\def\@thousand{mille}%
\def\@andname{et}%
\@@numberstringfrench{#1}{#2}}
%    \end{macrocode}
% Set default dialect:
%    \begin{macrocode}
\let\@numberstringMfrench=\@numberstringMfrenchfrance
%    \end{macrocode}
% As above, but for feminine version. Swiss:
%    \begin{macrocode}
\DeclareRobustCommand{\@numberstringFfrenchswiss}[2]{%
\let\@unitstring=\@@unitstringFfrench
\let\@teenstring=\@@teenstringfrench
\let\@tenstring=\@@tenstringfrench
\let\@seventies=\@@seventiesfrenchswiss
\let\@eighties=\@@eightiesfrenchswiss
\let\@nineties=\@@ninetiesfrenchswiss
\def\@hundred{cent}\def\@thousand{mille}%
\def\@andname{et}%
\@@numberstringfrench{#1}{#2}}
%    \end{macrocode}
% French:
%    \begin{macrocode}
\DeclareRobustCommand{\@numberstringFfrenchfrance}[2]{%
\let\@unitstring=\@@unitstringFfrench
\let\@teenstring=\@@teenstringfrench
\let\@tenstring=\@@tenstringfrench
\let\@seventies=\@@seventiesfrench
\let\@eighties=\@@eightiesfrench
\let\@nineties=\@@ninetiesfrench
\def\@hundred{cent}\def\@thousand{mille}%
\def\@andname{et}%
\@@numberstringfrench{#1}{#2}}
%    \end{macrocode}
% Belgian:
%    \begin{macrocode}
\DeclareRobustCommand{\@numberstringFfrenchbelgian}[2]{%
\let\@unitstring=\@@unitstringFfrench
\let\@teenstring=\@@teenstringfrench
\let\@tenstring=\@@tenstringfrench
\let\@seventies=\@@seventiesfrenchswiss
\let\@eighties=\@@eightiesfrench
\let\@nineties=\@@ninetiesfrench
\def\@hundred{cent}\def\@thousand{mille}%
\def\@andname{et}%
\@@numberstringfrench{#1}{#2}}
%    \end{macrocode}
% Set default dialect:
%    \begin{macrocode}
\let\@numberstringFfrench=\@numberstringFfrenchfrance
%    \end{macrocode}
% Make neuter same as masculine:
%    \begin{macrocode}
\let\@ordinalstringNfrench\@ordinalstringMfrench
%    \end{macrocode}
% As above, but with initial letter in upper case. Swiss (masculine):
%    \begin{macrocode}
\DeclareRobustCommand{\@NumberstringMfrenchswiss}[2]{%
\let\@unitstring=\@@Unitstringfrench
\let\@teenstring=\@@Teenstringfrench
\let\@tenstring=\@@Tenstringfrench
\let\@seventies=\@@seventiesfrenchswiss
\let\@eighties=\@@eightiesfrenchswiss
\let\@nineties=\@@ninetiesfrenchswiss
\def\@hundred{Cent}\def\@thousand{Mille}%
\def\@andname{et}%
\@@numberstringfrench{#1}{#2}}
%    \end{macrocode}
% French:
%    \begin{macrocode}
\DeclareRobustCommand{\@NumberstringMfrenchfrance}[2]{%
\let\@unitstring=\@@Unitstringfrench
\let\@teenstring=\@@Teenstringfrench
\let\@tenstring=\@@Tenstringfrench
\let\@seventies=\@@seventiesfrench
\let\@eighties=\@@eightiesfrench
\let\@nineties=\@@ninetiesfrench
\def\@hundred{Cent}\def\@thousand{Mille}%
\def\@andname{et}%
\@@numberstringfrench{#1}{#2}}
%    \end{macrocode}
% Belgian:
%    \begin{macrocode}
\DeclareRobustCommand{\@NumberstringMfrenchbelgian}[2]{%
\let\@unitstring=\@@Unitstringfrench
\let\@teenstring=\@@Teenstringfrench
\let\@tenstring=\@@Tenstringfrench
\let\@seventies=\@@seventiesfrenchswiss
\let\@eighties=\@@eightiesfrench
\let\@nineties=\@@ninetiesfrench
\def\@hundred{Cent}\def\@thousand{Mille}%
\def\@andname{et}%
\@@numberstringfrench{#1}{#2}}
%    \end{macrocode}
% Set default dialect:
%    \begin{macrocode}
\let\@NumberstringMfrench=\@NumberstringMfrenchfrance
%    \end{macrocode}
% As above, but feminine. Swiss:
%    \begin{macrocode}
\DeclareRobustCommand{\@NumberstringFfrenchswiss}[2]{%
\let\@unitstring=\@@UnitstringFfrench
\let\@teenstring=\@@Teenstringfrench
\let\@tenstring=\@@Tenstringfrench
\let\@seventies=\@@seventiesfrenchswiss
\let\@eighties=\@@eightiesfrenchswiss
\let\@nineties=\@@ninetiesfrenchswiss
\def\@hundred{Cent}\def\@thousand{Mille}%
\def\@andname{et}%
\@@numberstringfrench{#1}{#2}}
%    \end{macrocode}
% French (feminine):
%    \begin{macrocode}
\DeclareRobustCommand{\@NumberstringFfrenchfrance}[2]{%
\let\@unitstring=\@@UnitstringFfrench
\let\@teenstring=\@@Teenstringfrench
\let\@tenstring=\@@Tenstringfrench
\let\@seventies=\@@seventiesfrench
\let\@eighties=\@@eightiesfrench
\let\@nineties=\@@ninetiesfrench
\def\@hundred{Cent}\def\@thousand{Mille}%
\def\@andname{et}%
\@@numberstringfrench{#1}{#2}}
%    \end{macrocode}
% Belgian (feminine):
%    \begin{macrocode}
\DeclareRobustCommand{\@NumberstringFfrenchbelgian}[2]{%
\let\@unitstring=\@@UnitstringFfrench
\let\@teenstring=\@@Teenstringfrench
\let\@tenstring=\@@Tenstringfrench
\let\@seventies=\@@seventiesfrenchswiss
\let\@eighties=\@@eightiesfrench
\let\@nineties=\@@ninetiesfrench
\def\@hundred{Cent}\def\@thousand{Mille}%
\def\@andname{et}%
\@@numberstringfrench{#1}{#2}}
%    \end{macrocode}
% Set default dialect:
%    \begin{macrocode}
\let\@NumberstringFfrench=\@NumberstringFfrenchfrance
%    \end{macrocode}
% Make neuter same as masculine:
%    \begin{macrocode}
\let\@NumberstringNfrench\@NumberstringMfrench
%    \end{macrocode}
% Again, as from version 1.09, this has been changed to take
% two arguments, where the second argument is a control
% sequence, and nothing is displayed. Store textual representation
% of an ordinal in the given control sequence. Swiss dialect (masculine):
%    \begin{macrocode}
\DeclareRobustCommand{\@ordinalstringMfrenchswiss}[2]{%
\ifnum#1=1\relax
\def#2{premier}%
\else
\let\@unitthstring=\@@unitthstringfrench
\let\@unitstring=\@@unitstringfrench
\let\@teenthstring=\@@teenthstringfrench
\let\@teenstring=\@@teenstringfrench
\let\@tenthstring=\@@tenthstringfrench
\let\@tenstring=\@@tenstringfrench
\let\@seventieths=\@@seventiethsfrenchswiss
\let\@eightieths=\@@eightiethsfrenchswiss
\let\@ninetieths=\@@ninetiethsfrenchswiss
\let\@seventies=\@@seventiesfrenchswiss
\let\@eighties=\@@eightiesfrenchswiss
\let\@nineties=\@@ninetiesfrenchswiss
\def\@hundredth{centi\`eme}\def\@hundred{cent}%
\def\@thousandth{mili\`eme}\def\@thousand{mille}%
\def\@andname{et}%
\@@ordinalstringfrench{#1}{#2}%
\fi}
%    \end{macrocode}
% French (masculine):
%    \begin{macrocode}
\DeclareRobustCommand{\@ordinalstringMfrenchfrance}[2]{%
\ifnum#1=1\relax
\def#2{premier}%
\else
\let\@unitthstring=\@@unitthstringfrench
\let\@unitstring=\@@unitstringfrench
\let\@teenthstring=\@@teenthstringfrench
\let\@teenstring=\@@teenstringfrench
\let\@tenthstring=\@@tenthstringfrench
\let\@tenstring=\@@tenstringfrench
\let\@seventieths=\@@seventiethsfrench
\let\@eightieths=\@@eightiethsfrench
\let\@ninetieths=\@@ninetiethsfrench
\let\@seventies=\@@seventiesfrench
\let\@eighties=\@@eightiesfrench
\let\@nineties=\@@ninetiesfrench
\let\@teenstring=\@@teenstringfrench
\def\@hundredth{centi\`eme}\def\@hundred{cent}%
\def\@thousandth{mili\`eme}\def\@thousand{mille}%
\def\@andname{et}%
\@@ordinalstringfrench{#1}{#2}%
\fi}
%    \end{macrocode}
% Belgian dialect (masculine):
%    \begin{macrocode}
\DeclareRobustCommand{\@ordinalstringMfrenchbelgian}[2]{%
\ifnum#1=1\relax
\def#2{premier}%
\else
\let\@unitthstring=\@@unitthstringfrench
\let\@unitstring=\@@unitstringfrench
\let\@teenthstring=\@@teenthstringfrench
\let\@teenstring=\@@teenstringfrench
\let\@tenthstring=\@@tenthstringfrench
\let\@tenstring=\@@tenstringfrench
\let\@seventieths=\@@seventiethsfrenchswiss
\let\@eightieths=\@@eightiethsfrench
\let\@ninetieths=\@@ninetiethsfrenchswiss
\let\@seventies=\@@seventiesfrench
\let\@eighties=\@@eightiesfrench
\let\@nineties=\@@ninetiesfrench
\let\@teenstring=\@@teenstringfrench
\def\@hundredth{centi\`eme}\def\@hundred{cent}%
\def\@thousandth{mili\`eme}\def\@thousand{mille}%
\def\@andname{et}%
\@@ordinalstringfrench{#1}{#2}%
\fi}
%    \end{macrocode}
% Set up default dialect:
%    \begin{macrocode}
\let\@ordinalstringMfrench=\@ordinalstringMfrenchfrance
%    \end{macrocode}
% As above, but feminine. Swiss:
%    \begin{macrocode}
\DeclareRobustCommand{\@ordinalstringFfrenchswiss}[2]{%
\ifnum#1=1\relax
\def#2{premi\`ere}%
\else
\let\@unitthstring=\@@unitthstringfrench
\let\@unitstring=\@@unitstringFfrench
\let\@teenthstring=\@@teenthstringfrench
\let\@teenstring=\@@teenstringfrench
\let\@tenthstring=\@@tenthstringfrench
\let\@tenstring=\@@tenstringfrench
\let\@seventieths=\@@seventiethsfrenchswiss
\let\@eightieths=\@@eightiethsfrenchswiss
\let\@ninetieths=\@@ninetiethsfrenchswiss
\let\@seventies=\@@seventiesfrenchswiss
\let\@eighties=\@@eightiesfrenchswiss
\let\@nineties=\@@ninetiesfrenchswiss
\def\@hundredth{centi\`eme}\def\@hundred{cent}%
\def\@thousandth{mili\`eme}\def\@thousand{mille}%
\def\@andname{et}%
\@@ordinalstringfrench{#1}{#2}%
\fi}
%    \end{macrocode}
% French (feminine):
%    \begin{macrocode}
\DeclareRobustCommand{\@ordinalstringFfrenchfrance}[2]{%
\ifnum#1=1\relax
\def#2{premi\`ere}%
\else
\let\@unitthstring=\@@unitthstringfrench
\let\@unitstring=\@@unitstringFfrench
\let\@teenthstring=\@@teenthstringfrench
\let\@teenstring=\@@teenstringfrench
\let\@tenthstring=\@@tenthstringfrench
\let\@tenstring=\@@tenstringfrench
\let\@seventieths=\@@seventiethsfrench
\let\@eightieths=\@@eightiethsfrench
\let\@ninetieths=\@@ninetiethsfrench
\let\@seventies=\@@seventiesfrench
\let\@eighties=\@@eightiesfrench
\let\@nineties=\@@ninetiesfrench
\let\@teenstring=\@@teenstringfrench
\def\@hundredth{centi\`eme}\def\@hundred{cent}%
\def\@thousandth{mili\`eme}\def\@thousand{mille}%
\def\@andname{et}%
\@@ordinalstringfrench{#1}{#2}%
\fi}
%    \end{macrocode}
% Belgian (feminine):
%    \begin{macrocode}
\DeclareRobustCommand{\@ordinalstringFfrenchbelgian}[2]{%
\ifnum#1=1\relax
\def#2{premi\`ere}%
\else
\let\@unitthstring=\@@unitthstringfrench
\let\@unitstring=\@@unitstringFfrench
\let\@teenthstring=\@@teenthstringfrench
\let\@teenstring=\@@teenstringfrench
\let\@tenthstring=\@@tenthstringfrench
\let\@tenstring=\@@tenstringfrench
\let\@seventieths=\@@seventiethsfrenchswiss
\let\@eightieths=\@@eightiethsfrench
\let\@ninetieths=\@@ninetiethsfrench
\let\@seventies=\@@seventiesfrench
\let\@eighties=\@@eightiesfrench
\let\@nineties=\@@ninetiesfrench
\let\@teenstring=\@@teenstringfrench
\def\@hundredth{centi\`eme}\def\@hundred{cent}%
\def\@thousandth{mili\`eme}\def\@thousand{mille}%
\def\@andname{et}%
\@@ordinalstringfrench{#1}{#2}%
\fi}
%    \end{macrocode}
% Set up default dialect:
%    \begin{macrocode}
\let\@ordinalstringFfrench=\@ordinalstringFfrenchfrance
%    \end{macrocode}
% Make neuter same as masculine:
%    \begin{macrocode}
\let\@ordinalstringNfrench\@ordinalstringMfrench
%    \end{macrocode}
% As above, but with initial letters in upper case. Swiss (masculine):
%    \begin{macrocode}
\DeclareRobustCommand{\@OrdinalstringMfrenchswiss}[2]{%
\ifnum#1=1\relax
\def#2{Premi\`ere}%
\else
\let\@unitthstring=\@@Unitthstringfrench
\let\@unitstring=\@@Unitstringfrench
\let\@teenthstring=\@@Teenthstringfrench
\let\@teenstring=\@@Teenstringfrench
\let\@tenthstring=\@@Tenthstringfrench
\let\@tenstring=\@@Tenstringfrench
\let\@seventieths=\@@seventiethsfrenchswiss
\let\@eightieths=\@@eightiethsfrenchswiss
\let\@ninetieths=\@@ninetiethsfrenchswiss
\let\@seventies=\@@seventiesfrenchswiss
\let\@eighties=\@@eightiesfrenchswiss
\let\@nineties=\@@ninetiesfrenchswiss
\def\@hundredth{Centi\`eme}\def\@hundred{Cent}%
\def\@thousandth{Mili\`eme}\def\@thousand{Mille}%
\def\@andname{et}%
\@@ordinalstringfrench{#1}{#2}%
\fi}
%    \end{macrocode}
% French (masculine):
%    \begin{macrocode}
\DeclareRobustCommand{\@OrdinalstringMfrenchfrance}[2]{%
\ifnum#1=1\relax
\def#2{Premi\`ere}%
\else
\let\@unitthstring=\@@Unitthstringfrench
\let\@unitstring=\@@Unitstringfrench
\let\@teenthstring=\@@Teenthstringfrench
\let\@teenstring=\@@Teenstringfrench
\let\@tenthstring=\@@Tenthstringfrench
\let\@tenstring=\@@Tenstringfrench
\let\@seventieths=\@@seventiethsfrench
\let\@eightieths=\@@eightiethsfrench
\let\@ninetieths=\@@ninetiethsfrench
\let\@seventies=\@@seventiesfrench
\let\@eighties=\@@eightiesfrench
\let\@nineties=\@@ninetiesfrench
\let\@teenstring=\@@Teenstringfrench
\def\@hundredth{Centi\`eme}\def\@hundred{Cent}%
\def\@thousandth{Mili\`eme}\def\@thousand{Mille}%
\def\@andname{et}%
\@@ordinalstringfrench{#1}{#2}%
\fi}
%    \end{macrocode}
% Belgian (masculine):
%    \begin{macrocode}
\DeclareRobustCommand{\@OrdinalstringMfrenchbelgian}[2]{%
\ifnum#1=1\relax
\def#2{Premi\`ere}%
\else
\let\@unitthstring=\@@Unitthstringfrench
\let\@unitstring=\@@Unitstringfrench
\let\@teenthstring=\@@Teenthstringfrench
\let\@teenstring=\@@Teenstringfrench
\let\@tenthstring=\@@Tenthstringfrench
\let\@tenstring=\@@Tenstringfrench
\let\@seventieths=\@@seventiethsfrenchswiss
\let\@eightieths=\@@eightiethsfrench
\let\@ninetieths=\@@ninetiethsfrench
\let\@seventies=\@@seventiesfrench
\let\@eighties=\@@eightiesfrench
\let\@nineties=\@@ninetiesfrench
\let\@teenstring=\@@Teenstringfrench
\def\@hundredth{Centi\`eme}\def\@hundred{Cent}%
\def\@thousandth{Mili\`eme}\def\@thousand{Mille}%
\def\@andname{et}%
\@@ordinalstringfrench{#1}{#2}%
\fi}
%    \end{macrocode}
% Set up default dialect:
%    \begin{macrocode}
\let\@OrdinalstringMfrench=\@OrdinalstringMfrenchfrance
%    \end{macrocode}
% As above, but feminine form. Swiss:
%    \begin{macrocode}
\DeclareRobustCommand{\@OrdinalstringFfrenchswiss}[2]{%
\ifnum#1=1\relax
\def#2{Premi\`ere}%
\else
\let\@unitthstring=\@@Unitthstringfrench
\let\@unitstring=\@@UnitstringFfrench
\let\@teenthstring=\@@Teenthstringfrench
\let\@teenstring=\@@Teenstringfrench
\let\@tenthstring=\@@Tenthstringfrench
\let\@tenstring=\@@Tenstringfrench
\let\@seventieths=\@@seventiethsfrenchswiss
\let\@eightieths=\@@eightiethsfrenchswiss
\let\@ninetieths=\@@ninetiethsfrenchswiss
\let\@seventies=\@@seventiesfrenchswiss
\let\@eighties=\@@eightiesfrenchswiss
\let\@nineties=\@@ninetiesfrenchswiss
\def\@hundredth{Centi\`eme}\def\@hundred{Cent}%
\def\@thousandth{Mili\`eme}\def\@thousand{Mille}%
\def\@andname{et}%
\@@ordinalstringfrench{#1}{#2}%
\fi}
%    \end{macrocode}
% French (feminine):
%    \begin{macrocode}
\DeclareRobustCommand{\@OrdinalstringFfrenchfrance}[2]{%
\ifnum#1=1\relax
\def#2{Premi\`ere}%
\else
\let\@unitthstring=\@@Unitthstringfrench
\let\@unitstring=\@@UnitstringFfrench
\let\@teenthstring=\@@Teenthstringfrench
\let\@teenstring=\@@Teenstringfrench
\let\@tenthstring=\@@Tenthstringfrench
\let\@tenstring=\@@Tenstringfrench
\let\@seventieths=\@@seventiethsfrench
\let\@eightieths=\@@eightiethsfrench
\let\@ninetieths=\@@ninetiethsfrench
\let\@seventies=\@@seventiesfrench
\let\@eighties=\@@eightiesfrench
\let\@nineties=\@@ninetiesfrench
\let\@teenstring=\@@Teenstringfrench
\def\@hundredth{Centi\`eme}\def\@hundred{Cent}%
\def\@thousandth{Mili\`eme}\def\@thousand{Mille}%
\def\@andname{et}%
\@@ordinalstringfrench{#1}{#2}%
\fi}
%    \end{macrocode}
% Belgian (feminine):
%    \begin{macrocode}
\DeclareRobustCommand{\@OrdinalstringFfrenchbelgian}[2]{%
\ifnum#1=1\relax
\def#2{Premi\`ere}%
\else
\let\@unitthstring=\@@Unitthstringfrench
\let\@unitstring=\@@UnitstringFfrench
\let\@teenthstring=\@@Teenthstringfrench
\let\@teenstring=\@@Teenstringfrench
\let\@tenthstring=\@@Tenthstringfrench
\let\@tenstring=\@@Tenstringfrench
\let\@seventieths=\@@seventiethsfrenchswiss
\let\@eightieths=\@@eightiethsfrench
\let\@ninetieths=\@@ninetiethsfrench
\let\@seventies=\@@seventiesfrench
\let\@eighties=\@@eightiesfrench
\let\@nineties=\@@ninetiesfrench
\let\@teenstring=\@@Teenstringfrench
\def\@hundredth{Centi\`eme}\def\@hundred{Cent}%
\def\@thousandth{Mili\`eme}\def\@thousand{Mille}%
\def\@andname{et}%
\@@ordinalstringfrench{#1}{#2}%
\fi}
%    \end{macrocode}
% Set up default dialect:
%    \begin{macrocode}
\let\@OrdinalstringFfrench=\@OrdinalstringFfrenchfrance
%    \end{macrocode}
% Make neuter same as masculine:
%    \begin{macrocode}
\let\@OrdinalstringNfrench\@OrdinalstringMfrench
%    \end{macrocode}
% In order to convert numbers into textual ordinals, need
% to break it up into units, tens and teens. First the units.
% The argument must be a number or count register between 0
% and 9.
%    \begin{macrocode}
\newcommand*{\@@unitthstringfrench}[1]{%
\ifcase#1\relax
zero%
\or uni\`eme%
\or deuxi\`eme%
\or troisi\`eme%
\or quatri\`eme%
\or cinqui\`eme%
\or sixi\`eme%
\or septi\`eme%
\or huiti\`eme%
\or neuvi\`eme%
\fi
}
%    \end{macrocode}
% Tens (includes Swiss and Belgian variants, special cases are
% dealt with later.)
%    \begin{macrocode}
\newcommand*{\@@tenthstringfrench}[1]{%
\ifcase#1\relax
\or dixi\`eme%
\or vingti\`eme%
\or trentri\`eme%
\or quaranti\`eme%
\or cinquanti\`eme%
\or soixanti\`eme%
\or septenti\`eme%
\or huitanti\`eme%
\or nonenti\`eme%
\fi
}
%    \end{macrocode}
% Teens:
%    \begin{macrocode}
\newcommand*{\@@teenthstringfrench}[1]{%
\ifcase#1\relax
dixi\`eme%
\or onzi\`eme%
\or douzi\`eme%
\or treizi\`eme%
\or quatorzi\`eme%
\or quinzi\`eme%
\or seizi\`eme%
\or dix-septi\`eme%
\or dix-huiti\`eme%
\or dix-neuvi\`eme%
\fi
}
%    \end{macrocode}
% Seventies vary depending on dialect. Swiss:
%    \begin{macrocode}
\newcommand*{\@@seventiethsfrenchswiss}[1]{%
\ifcase#1\relax
\@tenthstring{7}%
\or
\@tenstring{7} \@andname\ \@unitthstring{1}%
\else
\@tenstring{7}-\@unitthstring{#1}%
\fi}
%    \end{macrocode}
% Eighties vary depending on dialect. Swiss:
%    \begin{macrocode}
\newcommand*{\@@eightiethsfrenchswiss}[1]{%
\ifcase#1\relax
\@tenthstring{8}%
\or
\@tenstring{8} \@andname\ \@unitthstring{1}%
\else
\@tenstring{8}-\@unitthstring{#1}%
\fi}
%    \end{macrocode}
% Nineties vary depending on dialect. Swiss:
%    \begin{macrocode}
\newcommand*{\@@ninetiethsfrenchswiss}[1]{%
\ifcase#1\relax
\@tenthstring{9}%
\or
\@tenstring{9} \@andname\ \@unitthstring{1}%
\else
\@tenstring{9}-\@unitthstring{#1}%
\fi}
%    \end{macrocode}
% French (as spoken in France) version:
%    \begin{macrocode}
\newcommand*{\@@seventiethsfrench}[1]{%
\ifnum#1=0\relax
\@tenstring{6}%
-%
\else
\@tenstring{6}%
\ \@andname\ 
\fi
\@teenthstring{#1}%
}
%    \end{macrocode}
% Eighties (as spoken in France):
%    \begin{macrocode}
\newcommand*{\@@eightiethsfrench}[1]{%
\ifnum#1>0\relax
\@unitstring{4}-\@tenstring{2}%
-\@unitthstring{#1}%
\else
\@unitstring{4}-\@tenthstring{2}%
\fi
}
%    \end{macrocode}
% Nineties (as spoken in France):
%    \begin{macrocode}
\newcommand*{\@@ninetiethsfrench}[1]{%
\@unitstring{4}-\@tenstring{2}-\@teenthstring{#1}%
}
%    \end{macrocode}
% As above, but with initial letter in upper case. Units:
%    \begin{macrocode}
\newcommand*{\@@Unitthstringfrench}[1]{%
\ifcase#1\relax
Zero%
\or Uni\`eme%
\or Deuxi\`eme%
\or Troisi\`eme%
\or Quatri\`eme%
\or Cinqui\`eme%
\or Sixi\`eme%
\or Septi\`eme%
\or Huiti\`eme%
\or Neuvi\`eme%
\fi
}
%    \end{macrocode}
% Tens (includes Belgian and Swiss variants):
%    \begin{macrocode}
\newcommand*{\@@Tenthstringfrench}[1]{%
\ifcase#1\relax
\or Dixi\`eme%
\or Vingti\`eme%
\or Trentri\`eme%
\or Quaranti\`eme%
\or Cinquanti\`eme%
\or Soixanti\`eme%
\or Septenti\`eme%
\or Huitanti\`eme%
\or Nonenti\`eme%
\fi
}
%    \end{macrocode}
% Teens:
%    \begin{macrocode}
\newcommand*{\@@Teenthstringfrench}[1]{%
\ifcase#1\relax
Dixi\`eme%
\or Onzi\`eme%
\or Douzi\`eme%
\or Treizi\`eme%
\or Quatorzi\`eme%
\or Quinzi\`eme%
\or Seizi\`eme%
\or Dix-Septi\`eme%
\or Dix-Huiti\`eme%
\or Dix-Neuvi\`eme%
\fi
}
%    \end{macrocode}
% Store textual representation of number (first argument) in given control
% sequence (second argument).
%    \begin{macrocode}
\newcommand*{\@@numberstringfrench}[2]{%
\ifnum#1>99999
\PackageError{fmtcount}{Out of range}%
{This macro only works for values less than 100000}%
\else
\ifnum#1<0
\PackageError{fmtcount}{Negative numbers not permitted}%
{This macro does not work for negative numbers, however
you can try typing "minus" first, and then pass the modulus of
this number}%
\fi
\fi
\def#2{}%
\@strctr=#1\relax \divide\@strctr by 1000\relax
\ifnum\@strctr>9\relax
% #1 is greater or equal to 10000
  \@tmpstrctr=\@strctr
  \divide\@strctr by 10\relax
  \ifnum\@strctr>1\relax
    \ifthenelse{\(\@strctr>6\)\and\(\@strctr<10\)}{%
      \@modulo{\@tmpstrctr}{10}%
      \ifnum\@strctr<8\relax
        \let\@@fc@numstr#2\relax
        \edef#2{\@@fc@numstr\@seventies{\@tmpstrctr}}%
      \else
        \ifnum\@strctr<9\relax
          \let\@@fc@numstr#2\relax
          \edef#2{\@@fc@numstr\@eighties{\@tmpstrctr}}%
        \else
          \ifnum\@strctr<10\relax
             \let\@@fc@numstr#2\relax
             \edef#2{\@@fc@numstr\@nineties{\@tmpstrctr}}%
          \fi
        \fi
      \fi
    }{%
      \let\@@fc@numstr#2\relax
      \edef#2{\@@fc@numstr\@tenstring{\@strctr}}%
      \@strctr=#1\relax
      \divide\@strctr by 1000\relax
      \@modulo{\@strctr}{10}%
      \ifnum\@strctr>0\relax
        \let\@@fc@numstr#2\relax
        \edef#2{\@@fc@numstr\ \@unitstring{\@strctr}}%
      \fi
    }%
  \else
    \@strctr=#1\relax
    \divide\@strctr by 1000
    \@modulo{\@strctr}{10}%
    \let\@@fc@numstr#2\relax
    \edef#2{\@@fc@numstr\@teenstring{\@strctr}}%
  \fi
  \let\@@fc@numstr#2\relax
  \edef#2{\@@fc@numstr\ \@thousand}%
\else
  \ifnum\@strctr>0\relax 
    \ifnum\@strctr>1\relax
      \let\@@fc@numstr#2\relax
      \edef#2{\@@fc@numstr\@unitstring{\@strctr}\ }%
    \fi
    \let\@@fc@numstr#2\relax
    \edef#2{\@@fc@numstr\@thousand}%
  \fi
\fi
\@strctr=#1\relax \@modulo{\@strctr}{1000}%
\divide\@strctr by 100
\ifnum\@strctr>0\relax
  \ifnum#1>1000\relax
    \let\@@fc@numstr#2\relax
    \edef#2{\@@fc@numstr\ }%
  \fi
  \@tmpstrctr=#1\relax
  \@modulo{\@tmpstrctr}{1000}\relax
  \ifnum\@tmpstrctr=100\relax
    \let\@@fc@numstr#2\relax
    \edef#2{\@@fc@numstr\@tenstring{10}}%
  \else
    \ifnum\@strctr>1\relax
      \let\@@fc@numstr#2\relax
      \edef#2{\@@fc@numstr\@unitstring{\@strctr}\ }%
    \fi
    \let\@@fc@numstr#2\relax
    \edef#2{\@@fc@numstr\@hundred}%
  \fi
\fi
\@strctr=#1\relax \@modulo{\@strctr}{100}%
%\@tmpstrctr=#1\relax
%\divide\@tmpstrctr by 100\relax
\ifnum#1>100\relax
  \ifnum\@strctr>0\relax
    \let\@@fc@numstr#2\relax
    \edef#2{\@@fc@numstr\ }%
  \else
    \ifnum\@tmpstrctr>0\relax
       \let\@@fc@numstr#2\relax
       \edef#2{\@@fc@numstr s}%
    \fi%
  \fi
\fi
\ifnum\@strctr>19\relax
  \@tmpstrctr=\@strctr
  \divide\@strctr by 10\relax
  \ifthenelse{\@strctr>6}{%
    \@modulo{\@tmpstrctr}{10}%
    \ifnum\@strctr<8\relax
      \let\@@fc@numstr#2\relax
      \edef#2{\@@fc@numstr\@seventies{\@tmpstrctr}}%
    \else
      \ifnum\@strctr<9\relax
        \let\@@fc@numstr#2\relax
        \edef#2{\@@fc@numstr\@eighties{\@tmpstrctr}}%
      \else
        \let\@@fc@numstr#2\relax
        \edef#2{\@@fc@numstr\@nineties{\@tmpstrctr}}%
      \fi
    \fi
  }{%
    \let\@@fc@numstr#2\relax
    \edef#2{\@@fc@numstr\@tenstring{\@strctr}}%
    \@strctr=#1\relax \@modulo{\@strctr}{10}%
    \ifnum\@strctr>0\relax
      \let\@@fc@numstr#2\relax
      \ifnum\@strctr=1\relax
         \edef#2{\@@fc@numstr\ \@andname\ }%
      \else
         \edef#2{\@@fc@numstr-}%
      \fi
      \let\@@fc@numstr#2\relax
      \edef#2{\@@fc@numstr\@unitstring{\@strctr}}%
    \fi
  }%
\else
  \ifnum\@strctr<10\relax
    \ifnum\@strctr=0\relax
      \ifnum#1<100\relax
        \let\@@fc@numstr#2\relax
        \edef#2{\@@fc@numstr\@unitstring{\@strctr}}%
      \fi
    \else%(>0,<10)
      \let\@@fc@numstr#2\relax
      \edef#2{\@@fc@numstr\@unitstring{\@strctr}}%
    \fi
  \else%>10
    \@modulo{\@strctr}{10}%
    \let\@@fc@numstr#2\relax
    \edef#2{\@@fc@numstr\@teenstring{\@strctr}}%
  \fi
\fi
}
%    \end{macrocode}
% Store textual representation of an ordinal (from number 
% specified in first argument) in given control
% sequence (second argument).
%    \begin{macrocode}
\newcommand*{\@@ordinalstringfrench}[2]{%
\ifnum#1>99999
\PackageError{fmtcount}{Out of range}%
{This macro only works for values less than 100000}%
\else
\ifnum#1<0
\PackageError{fmtcount}{Negative numbers not permitted}%
{This macro does not work for negative numbers, however
you can try typing "minus" first, and then pass the modulus of
this number}%
\fi
\fi
\def#2{}%
\@strctr=#1\relax \divide\@strctr by 1000\relax
\ifnum\@strctr>9
% #1 is greater or equal to 10000
  \@tmpstrctr=\@strctr
  \divide\@strctr by 10\relax
  \ifnum\@strctr>1\relax
    \ifthenelse{\@strctr>6}{%
      \@modulo{\@tmpstrctr}{10}%
      \ifnum\@strctr=7\relax
        \let\@@fc@ordstr#2\relax
        \edef#2{\@@fc@ordstr\@seventies{\@tmpstrctr}}%
      \else
        \ifnum\@strctr=8\relax
          \let\@@fc@ordstr#2\relax
          \edef#2{\@@fc@ordstr\@eighties{\@tmpstrctr}}%
        \else
          \let\@@fc@ordstr#2\relax
          \edef#2{\@@fc@ordstr\@nineties{\@tmpstrctr}}%
        \fi
      \fi
    }{%
      \let\@@fc@ordstr#2\relax
      \edef#2{\@@fc@ordstr\@tenstring{\@strctr}}%
      \@strctr=#1\relax
      \divide\@strctr by 1000\relax
      \@modulo{\@strctr}{10}%
      \ifnum\@strctr=1\relax
         \let\@@fc@ordstr#2\relax
         \edef#2{\@@fc@ordstr\ \@andname}%
      \fi
      \ifnum\@strctr>0\relax
         \let\@@fc@ordstr#2\relax
         \edef#2{\@@fc@ordstr\ \@unitstring{\@strctr}}%
      \fi
    }%
  \else
    \@strctr=#1\relax
    \divide\@strctr by 1000\relax
    \@modulo{\@strctr}{10}%
    \let\@@fc@ordstr#2\relax
    \edef#2{\@@fc@ordstr\@teenstring{\@strctr}}%
  \fi
  \@strctr=#1\relax \@modulo{\@strctr}{1000}%
  \ifnum\@strctr=0\relax
    \let\@@fc@ordstr#2\relax
    \edef#2{\@@fc@ordstr\ \@thousandth}%
  \else
    \let\@@fc@ordstr#2\relax
    \edef#2{\@@fc@ordstr\ \@thousand}%
  \fi
\else
  \ifnum\@strctr>0\relax
    \let\@@fc@ordstr#2\relax
    \edef#2{\@@fc@ordstr\@unitstring{\@strctr}}%
    \@strctr=#1\relax \@modulo{\@strctr}{1000}%
    \ifnum\@strctr=0\relax
      \let\@@fc@ordstr#2\relax
      \edef#2{\@@fc@ordstr\ \@thousandth}%
    \else
      \let\@@fc@ordstr#2\relax
      \edef#2{\@@fc@ordstr\ \@thousand}%
    \fi
  \fi
\fi
\@strctr=#1\relax \@modulo{\@strctr}{1000}%
\divide\@strctr by 100\relax
\ifnum\@strctr>0\relax
  \ifnum#1>1000\relax
    \let\@@fc@ordstr#2\relax
    \edef#2{\@@fc@ordstr\ }%
  \fi
  \let\@@fc@ordstr#2\relax
  \edef#2{\@@fc@ordstr\@unitstring{\@strctr}}%
  \@strctr=#1\relax \@modulo{\@strctr}{100}%
  \let\@@fc@ordstr#2\relax
  \ifnum\@strctr=0\relax
    \edef#2{\@@fc@ordstr\ \@hundredth}%
  \else
    \edef#2{\@@fc@ordstr\ \@hundred}%
  \fi
\fi
\@tmpstrctr=\@strctr
\@strctr=#1\relax \@modulo{\@strctr}{100}%
\ifnum#1>100\relax
  \ifnum\@strctr>0\relax
    \let\@@fc@ordstr#2\relax
    \edef#2{\@@fc@ordstr\ \@andname\ }%
  \fi
\fi
\ifnum\@strctr>19\relax
  \@tmpstrctr=\@strctr
  \divide\@strctr by 10\relax
  \@modulo{\@tmpstrctr}{10}%
  \ifthenelse{\@strctr>6}{%
    \ifnum\@strctr=7\relax
      \let\@@fc@ordstr#2\relax
      \edef#2{\@@fc@ordstr\@seventieths{\@tmpstrctr}}%
    \else
      \ifnum\@strctr=8\relax
        \let\@@fc@ordstr#2\relax
        \edef#2{\@@fc@ordstr\@eightieths{\@tmpstrctr}}%
      \else
        \let\@@fc@ordstr#2\relax
        \edef#2{\@@fc@ordstr\@ninetieths{\@tmpstrctr}}%
      \fi
    \fi
  }{%
    \ifnum\@tmpstrctr=0\relax
      \let\@@fc@ordstr#2\relax
      \edef#2{\@@fc@ordstr\@tenthstring{\@strctr}}%
    \else 
      \let\@@fc@ordstr#2\relax
      \edef#2{\@@fc@ordstr\@tenstring{\@strctr}}%
    \fi
    \@strctr=#1\relax \@modulo{\@strctr}{10}%
    \ifnum\@strctr=1\relax
      \let\@@fc@ordstr#2\relax
      \edef#2{\@@fc@ordstr\ \@andname}%
    \fi
    \ifnum\@strctr>0\relax
      \let\@@fc@ordstr#2\relax
      \edef#2{\@@fc@ordstr\ \@unitthstring{\@strctr}}%
    \fi
  }%
\else
  \ifnum\@strctr<10\relax
    \ifnum\@strctr=0\relax
      \ifnum#1<100\relax
        \let\@@fc@ordstr#2\relax
        \edef#2{\@@fc@ordstr\@unitthstring{\@strctr}}%
      \fi
    \else
      \let\@@fc@ordstr#2\relax
      \edef#2{\@@fc@ordstr\@unitthstring{\@strctr}}%
    \fi
  \else
    \@modulo{\@strctr}{10}%
    \let\@@fc@ordstr#2\relax
    \edef#2{\@@fc@ordstr\@teenthstring{\@strctr}}%
  \fi
\fi
}
%    \end{macrocode}
%\iffalse
%    \begin{macrocode}
%</fc-french.def>
%    \end{macrocode}
%\fi
%\iffalse
%    \begin{macrocode}
%<*fc-german.def>
%    \end{macrocode}
%\fi
% \subsection{fc-german.def}
% German definitions (thank you to K. H. Fricke for supplying
% this information)
%    \begin{macrocode}
\ProvidesFile{fc-german.def}[2007/06/14]
%    \end{macrocode}
% Define macro that converts a number or count register (first
% argument) to an ordinal, and stores the result in the
% second argument, which must be a control sequence.
% Masculine:
%    \begin{macrocode}
\newcommand{\@ordinalMgerman}[2]{%
\edef#2{\number#1\relax.}}
%    \end{macrocode}
% Feminine:
%    \begin{macrocode}
\newcommand{\@ordinalFgerman}[2]{%
\edef#2{\number#1\relax.}}
%    \end{macrocode}
% Neuter:
%    \begin{macrocode}
\newcommand{\@ordinalNgerman}[2]{%
\edef#2{\number#1\relax.}}
%    \end{macrocode}
% Convert a number to text. The easiest way to do this is to
% break it up into units, tens and teens.
% Units (argument must be a number from 0 to 9, 1 on its own (eins)
% is dealt with separately):
%    \begin{macrocode}
\newcommand{\@@unitstringgerman}[1]{%
\ifcase#1%
null%
\or ein%
\or zwei%
\or drei%
\or vier%
\or f\"unf%
\or sechs%
\or sieben%
\or acht%
\or neun%
\fi
}
%    \end{macrocode}
% Tens (argument must go from 1 to 10):
%    \begin{macrocode}
\newcommand{\@@tenstringgerman}[1]{%
\ifcase#1%
\or zehn%
\or zwanzig%
\or drei{\ss}ig%
\or vierzig%
\or f\"unfzig%
\or sechzig%
\or siebzig%
\or achtzig%
\or neunzig%
\or einhundert%
\fi
}
%    \end{macrocode}
% |\einhundert| is set to |einhundert| by default, user can
% redefine this command to just |hundert| if required, similarly
% for |\eintausend|.
%    \begin{macrocode}
\providecommand*{\einhundert}{einhundert}
\providecommand*{\eintausend}{eintausend}
%    \end{macrocode}
% Teens:
%    \begin{macrocode}
\newcommand{\@@teenstringgerman}[1]{%
\ifcase#1%
zehn%
\or elf%
\or zw\"olf%
\or dreizehn%
\or vierzehn%
\or f\"unfzehn%
\or sechzehn%
\or siebzehn%
\or achtzehn%
\or neunzehn%
\fi
}
%    \end{macrocode}
% The results are stored in the second argument, but doesn't 
% display anything.
%    \begin{macrocode}
\DeclareRobustCommand{\@numberstringMgerman}[2]{%
\let\@unitstring=\@@unitstringgerman
\let\@teenstring=\@@teenstringgerman
\let\@tenstring=\@@tenstringgerman
\@@numberstringgerman{#1}{#2}}
%    \end{macrocode}
% Feminine and neuter forms:
%    \begin{macrocode}
\let\@numberstringFgerman=\@numberstringMgerman
\let\@numberstringNgerman=\@numberstringMgerman
%    \end{macrocode}
% As above, but initial letters in upper case:
%    \begin{macrocode}
\DeclareRobustCommand{\@NumberstringMgerman}[2]{%
\@numberstringMgerman{#1}{\@@num@str}%
\edef#2{\noexpand\MakeUppercase\@@num@str}}
%    \end{macrocode}
% Feminine and neuter form:
%    \begin{macrocode}
\let\@NumberstringFgerman=\@NumberstringMgerman
\let\@NumberstringNgerman=\@NumberstringMgerman
%    \end{macrocode}
% As above, but for ordinals.
%    \begin{macrocode}
\DeclareRobustCommand{\@ordinalstringMgerman}[2]{%
\let\@unitthstring=\@@unitthstringMgerman
\let\@teenthstring=\@@teenthstringMgerman
\let\@tenthstring=\@@tenthstringMgerman
\let\@unitstring=\@@unitstringgerman
\let\@teenstring=\@@teenstringgerman
\let\@tenstring=\@@tenstringgerman
\def\@thousandth{tausendster}%
\def\@hundredth{hundertster}%
\@@ordinalstringgerman{#1}{#2}}
%    \end{macrocode}
% Feminine form:
%    \begin{macrocode}
\DeclareRobustCommand{\@ordinalstringFgerman}[2]{%
\let\@unitthstring=\@@unitthstringFgerman
\let\@teenthstring=\@@teenthstringFgerman
\let\@tenthstring=\@@tenthstringFgerman
\let\@unitstring=\@@unitstringgerman
\let\@teenstring=\@@teenstringgerman
\let\@tenstring=\@@tenstringgerman
\def\@thousandth{tausendste}%
\def\@hundredth{hundertste}%
\@@ordinalstringgerman{#1}{#2}}
%    \end{macrocode}
% Neuter form:
%    \begin{macrocode}
\DeclareRobustCommand{\@ordinalstringNgerman}[2]{%
\let\@unitthstring=\@@unitthstringNgerman
\let\@teenthstring=\@@teenthstringNgerman
\let\@tenthstring=\@@tenthstringNgerman
\let\@unitstring=\@@unitstringgerman
\let\@teenstring=\@@teenstringgerman
\let\@tenstring=\@@tenstringgerman
\def\@thousandth{tausendstes}%
\def\@hundredth{hunderstes}%
\@@ordinalstringgerman{#1}{#2}}
%    \end{macrocode}
% As above, but with initial letters in upper case.
%    \begin{macrocode}
\DeclareRobustCommand{\@OrdinalstringMgerman}[2]{%
\@ordinalstringMgerman{#1}{\@@num@str}%
\edef#2{\protect\MakeUppercase\@@num@str}}
%    \end{macrocode}
% Feminine form:
%    \begin{macrocode}
\DeclareRobustCommand{\@OrdinalstringFgerman}[2]{%
\@ordinalstringFgerman{#1}{\@@num@str}%
\edef#2{\protect\MakeUppercase\@@num@str}}
%    \end{macrocode}
% Neuter form:
%    \begin{macrocode}
\DeclareRobustCommand{\@OrdinalstringNgerman}[2]{%
\@ordinalstringNgerman{#1}{\@@num@str}%
\edef#2{\protect\MakeUppercase\@@num@str}}
%    \end{macrocode}
% Code for converting numbers into textual ordinals. As before,
% it is easier to split it into units, tens and teens.
% Units:
%    \begin{macrocode}
\newcommand{\@@unitthstringMgerman}[1]{%
\ifcase#1%
nullter%
\or erster%
\or zweiter%
\or dritter%
\or vierter%
\or f\"unter%
\or sechster%
\or siebter%
\or achter%
\or neunter%
\fi
}
%    \end{macrocode}
% Tens:
%    \begin{macrocode}
\newcommand{\@@tenthstringMgerman}[1]{%
\ifcase#1%
\or zehnter%
\or zwanzigster%
\or drei{\ss}igster%
\or vierzigster%
\or f\"unfzigster%
\or sechzigster%
\or siebzigster%
\or achtzigster%
\or neunzigster%
\fi
}
%    \end{macrocode}
% Teens:
%    \begin{macrocode}
\newcommand{\@@teenthstringMgerman}[1]{%
\ifcase#1%
zehnter%
\or elfter%
\or zw\"olfter%
\or dreizehnter%
\or vierzehnter%
\or f\"unfzehnter%
\or sechzehnter%
\or siebzehnter%
\or achtzehnter%
\or neunzehnter%
\fi
}
%    \end{macrocode}
% Units (feminine):
%    \begin{macrocode}
\newcommand{\@@unitthstringFgerman}[1]{%
\ifcase#1%
nullte%
\or erste%
\or zweite%
\or dritte%
\or vierte%
\or f\"unfte%
\or sechste%
\or siebte%
\or achte%
\or neunte%
\fi
}
%    \end{macrocode}
% Tens (feminine):
%    \begin{macrocode}
\newcommand{\@@tenthstringFgerman}[1]{%
\ifcase#1%
\or zehnte%
\or zwanzigste%
\or drei{\ss}igste%
\or vierzigste%
\or f\"unfzigste%
\or sechzigste%
\or siebzigste%
\or achtzigste%
\or neunzigste%
\fi
}
%    \end{macrocode}
% Teens (feminine)
%    \begin{macrocode}
\newcommand{\@@teenthstringFgerman}[1]{%
\ifcase#1%
zehnte%
\or elfte%
\or zw\"olfte%
\or dreizehnte%
\or vierzehnte%
\or f\"unfzehnte%
\or sechzehnte%
\or siebzehnte%
\or achtzehnte%
\or neunzehnte%
\fi
}
%    \end{macrocode}
% Units (neuter):
%    \begin{macrocode}
\newcommand{\@@unitthstringNgerman}[1]{%
\ifcase#1%
nulltes%
\or erstes%
\or zweites%
\or drittes%
\or viertes%
\or f\"unte%
\or sechstes%
\or siebtes%
\or achtes%
\or neuntes%
\fi
}
%    \end{macrocode}
% Tens (neuter):
%    \begin{macrocode}
\newcommand{\@@tenthstringNgerman}[1]{%
\ifcase#1%
\or zehntes%
\or zwanzigstes%
\or drei{\ss}igstes%
\or vierzigstes%
\or f\"unfzigstes%
\or sechzigstes%
\or siebzigstes%
\or achtzigstes%
\or neunzigstes%
\fi
}
%    \end{macrocode}
% Teens (neuter)
%    \begin{macrocode}
\newcommand{\@@teenthstringNgerman}[1]{%
\ifcase#1%
zehntes%
\or elftes%
\or zw\"olftes%
\or dreizehntes%
\or vierzehntes%
\or f\"unfzehntes%
\or sechzehntes%
\or siebzehntes%
\or achtzehntes%
\or neunzehntes%
\fi
}
%    \end{macrocode}
% This appends the results to |#2| for number |#2| (in range 0 to 100.)
% null and eins are dealt with separately in |\@@numberstringgerman|.
%    \begin{macrocode}
\newcommand{\@@numberunderhundredgerman}[2]{%
\ifnum#1<10\relax
  \ifnum#1>0\relax
    \let\@@fc@numstr#2\relax
    \edef#2{\@@fc@numstr\@unitstring{#1}}%
  \fi
\else
  \@tmpstrctr=#1\relax
  \@modulo{\@tmpstrctr}{10}%
  \ifnum#1<20\relax
    \let\@@fc@numstr#2\relax
    \edef#2{\@@fc@numstr\@teenstring{\@tmpstrctr}}%
  \else
    \ifnum\@tmpstrctr=0\relax
    \else
      \let\@@fc@numstr#2\relax
      \edef#2{\@@fc@numstr\@unitstring{\@tmpstrctr}und}%
    \fi
    \@tmpstrctr=#1\relax
    \divide\@tmpstrctr by 10\relax
    \let\@@fc@numstr#2\relax
    \edef#2{\@@fc@numstr\@tenstring{\@tmpstrctr}}%
  \fi
\fi
}
%    \end{macrocode}
% This stores the results in the second argument 
% (which must be a control
% sequence), but it doesn't display anything.
%    \begin{macrocode}
\newcommand{\@@numberstringgerman}[2]{%
\ifnum#1>99999\relax
  \PackageError{fmtcount}{Out of range}%
  {This macro only works for values less than 100000}%
\else
  \ifnum#1<0\relax
    \PackageError{fmtcount}{Negative numbers not permitted}%
    {This macro does not work for negative numbers, however
    you can try typing "minus" first, and then pass the modulus of
    this number}%
  \fi
\fi
\def#2{}%
\@strctr=#1\relax \divide\@strctr by 1000\relax
\ifnum\@strctr>1\relax
% #1 is >= 2000, \@strctr now contains the number of thousands
\@@numberunderhundredgerman{\@strctr}{#2}%
  \let\@@fc@numstr#2\relax
  \edef#2{\@@fc@numstr tausend}%
\else
% #1 lies in range [1000,1999]
  \ifnum\@strctr=1\relax
    \let\@@fc@numstr#2\relax
    \edef#2{\@@fc@numstr\eintausend}%
  \fi
\fi
\@strctr=#1\relax
\@modulo{\@strctr}{1000}%
\divide\@strctr by 100\relax
\ifnum\@strctr>1\relax
% now dealing with number in range [200,999]
  \let\@@fc@numstr#2\relax
  \edef#2{\@@fc@numstr\@unitstring{\@strctr}hundert}%
\else
   \ifnum\@strctr=1\relax
% dealing with number in range [100,199]
     \ifnum#1>1000\relax
% if orginal number > 1000, use einhundert
        \let\@@fc@numstr#2\relax
        \edef#2{\@@fc@numstr einhundert}%
     \else
% otherwise use \einhundert
        \let\@@fc@numstr#2\relax
        \edef#2{\@@fc@numstr\einhundert}%
      \fi
   \fi
\fi
\@strctr=#1\relax
\@modulo{\@strctr}{100}%
\ifnum#1=0\relax
  \def#2{null}%
\else
  \ifnum\@strctr=1\relax
    \let\@@fc@numstr#2\relax
    \edef#2{\@@fc@numstr eins}%
  \else
    \@@numberunderhundredgerman{\@strctr}{#2}%
  \fi
\fi
}
%    \end{macrocode}
% As above, but for ordinals
%    \begin{macrocode}
\newcommand{\@@numberunderhundredthgerman}[2]{%
\ifnum#1<10\relax
 \let\@@fc@numstr#2\relax
 \edef#2{\@@fc@numstr\@unitthstring{#1}}%
\else
  \@tmpstrctr=#1\relax
  \@modulo{\@tmpstrctr}{10}%
  \ifnum#1<20\relax
    \let\@@fc@numstr#2\relax
    \edef#2{\@@fc@numstr\@teenthstring{\@tmpstrctr}}%
  \else
    \ifnum\@tmpstrctr=0\relax
    \else
      \let\@@fc@numstr#2\relax
      \edef#2{\@@fc@numstr\@unitstring{\@tmpstrctr}und}%
    \fi
    \@tmpstrctr=#1\relax
    \divide\@tmpstrctr by 10\relax
    \let\@@fc@numstr#2\relax
    \edef#2{\@@fc@numstr\@tenthstring{\@tmpstrctr}}%
  \fi
\fi
}
%    \end{macrocode}
%    \begin{macrocode}
\newcommand{\@@ordinalstringgerman}[2]{%
\ifnum#1>99999\relax
  \PackageError{fmtcount}{Out of range}%
  {This macro only works for values less than 100000}%
\else
  \ifnum#1<0\relax
    \PackageError{fmtcount}{Negative numbers not permitted}%
    {This macro does not work for negative numbers, however
    you can try typing "minus" first, and then pass the modulus of
    this number}%
  \fi
\fi
\def#2{}%
\@strctr=#1\relax \divide\@strctr by 1000\relax
\ifnum\@strctr>1\relax
% #1 is >= 2000, \@strctr now contains the number of thousands
\@@numberunderhundredgerman{\@strctr}{#2}%
  \let\@@fc@numstr#2\relax
  % is that it, or is there more?
  \@tmpstrctr=#1\relax \@modulo{\@tmpstrctr}{1000}%
  \ifnum\@tmpstrctr=0\relax
    \edef#2{\@@fc@numstr\@thousandth}%
  \else
    \edef#2{\@@fc@numstr tausend}%
  \fi
\else
% #1 lies in range [1000,1999]
  \ifnum\@strctr=1\relax
    \ifnum#1=1000\relax
      \let\@@fc@numstr#2\relax
      \edef#2{\@@fc@numstr\@thousandth}%
    \else
      \let\@@fc@numstr#2\relax
      \edef#2{\@@fc@numstr\eintausend}%
    \fi
  \fi
\fi
\@strctr=#1\relax
\@modulo{\@strctr}{1000}%
\divide\@strctr by 100\relax
\ifnum\@strctr>1\relax
% now dealing with number in range [200,999]
  \let\@@fc@numstr#2\relax
  % is that it, or is there more?
  \@tmpstrctr=#1\relax \@modulo{\@tmpstrctr}{100}%
  \ifnum\@tmpstrctr=0\relax
     \ifnum\@strctr=1\relax
       \edef#2{\@@fc@numstr\@hundredth}%
     \else
       \edef#2{\@@fc@numstr\@unitstring{\@strctr}\@hundredth}%
     \fi
  \else
     \edef#2{\@@fc@numstr\@unitstring{\@strctr}hundert}%
  \fi
\else
   \ifnum\@strctr=1\relax
% dealing with number in range [100,199]
% is that it, or is there more?
     \@tmpstrctr=#1\relax \@modulo{\@tmpstrctr}{100}%
     \ifnum\@tmpstrctr=0\relax
        \let\@@fc@numstr#2\relax
        \edef#2{\@@fc@numstr\@hundredth}%
     \else
     \ifnum#1>1000\relax
        \let\@@fc@numstr#2\relax
        \edef#2{\@@fc@numstr einhundert}%
     \else
        \let\@@fc@numstr#2\relax
        \edef#2{\@@fc@numstr\einhundert}%
     \fi
     \fi
   \fi
\fi
\@strctr=#1\relax
\@modulo{\@strctr}{100}%
\ifthenelse{\@strctr=0 \and #1>0}{}{%
\@@numberunderhundredthgerman{\@strctr}{#2}%
}%
}
%    \end{macrocode}
% Set |ngerman| to be equivalent to |german|. Is it okay to do
% this? (I don't know the difference between the two.)
%    \begin{macrocode}
\let\@ordinalMngerman=\@ordinalMgerman
\let\@ordinalFngerman=\@ordinalFgerman
\let\@ordinalNngerman=\@ordinalNgerman
\let\@numberstringMngerman=\@numberstringMgerman
\let\@numberstringFngerman=\@numberstringFgerman
\let\@numberstringNngerman=\@numberstringNgerman
\let\@NumberstringMngerman=\@NumberstringMgerman
\let\@NumberstringFngerman=\@NumberstringFgerman
\let\@NumberstringNngerman=\@NumberstringNgerman
\let\@ordinalstringMngerman=\@ordinalstringMgerman
\let\@ordinalstringFngerman=\@ordinalstringFgerman
\let\@ordinalstringNngerman=\@ordinalstringNgerman
\let\@OrdinalstringMngerman=\@OrdinalstringMgerman
\let\@OrdinalstringFngerman=\@OrdinalstringFgerman
\let\@OrdinalstringNngerman=\@OrdinalstringNgerman
%    \end{macrocode}
%\iffalse
%    \begin{macrocode}
%</fc-german.def>
%    \end{macrocode}
%\fi
%\iffalse
%    \begin{macrocode}
%<*fc-portuges.def>
%    \end{macrocode}
%\fi
% \subsection{fc-portuges.def}
% Portuguse definitions
%    \begin{macrocode}
\ProvidesFile{fc-portuges.def}[2007/05/26]
%    \end{macrocode}
% Define macro that converts a number or count register (first
% argument) to an ordinal, and stores the result in the second
% argument, which should be a control sequence. Masculine:
%    \begin{macrocode}
\newcommand*{\@ordinalMportuges}[2]{%
\ifnum#1=0\relax
  \edef#2{\number#1}%
\else
  \edef#2{\number#1\relax\noexpand\fmtord{o}}%
\fi}
%    \end{macrocode}
% Feminine:
%    \begin{macrocode}
\newcommand*{\@ordinalFportuges}[2]{%
\ifnum#1=0\relax
  \edef#2{\number#1}%
\else
  \edef#2{\number#1\relax\noexpand\fmtord{a}}%
\fi}
%    \end{macrocode}
% Make neuter same as masculine:
%    \begin{macrocode}
\let\@ordinalNportuges\@ordinalMportuges
%    \end{macrocode}
% Convert a number to a textual representation. To make it easier,
% split it up into units, tens, teens and hundreds. Units (argument must
% be a number from 0 to 9):
%    \begin{macrocode}
\newcommand*{\@@unitstringportuges}[1]{%
\ifcase#1\relax
zero%
\or um%
\or dois%
\or tr\^es%
\or quatro%
\or cinco%
\or seis%
\or sete%
\or oito%
\or nove%
\fi
}
%   \end{macrocode}
% As above, but for feminine:
%   \begin{macrocode}
\newcommand*{\@@unitstringFportuges}[1]{%
\ifcase#1\relax
zero%
\or uma%
\or duas%
\or tr\^es%
\or quatro%
\or cinco%
\or seis%
\or sete%
\or oito%
\or nove%
\fi
}
%    \end{macrocode}
% Tens (argument must be a number from 0 to 10):
%    \begin{macrocode}
\newcommand*{\@@tenstringportuges}[1]{%
\ifcase#1\relax
\or dez%
\or vinte%
\or trinta%
\or quarenta%
\or cinq\"uenta%
\or sessenta%
\or setenta%
\or oitenta%
\or noventa%
\or cem%
\fi
}
%    \end{macrocode}
% Teens (argument must be a number from 0 to 9):
%    \begin{macrocode}
\newcommand*{\@@teenstringportuges}[1]{%
\ifcase#1\relax
dez%
\or onze%
\or doze%
\or treze%
\or quatorze%
\or quinze%
\or dezesseis%
\or dezessete%
\or dezoito%
\or dezenove%
\fi
}
%    \end{macrocode}
% Hundreds:
%    \begin{macrocode}
\newcommand*{\@@hundredstringportuges}[1]{%
\ifcase#1\relax
\or cento%
\or duzentos%
\or trezentos%
\or quatrocentos%
\or quinhentos%
\or seiscentos%
\or setecentos%
\or oitocentos%
\or novecentos%
\fi}
%    \end{macrocode}
% Hundreds (feminine):
%    \begin{macrocode}
\newcommand*{\@@hundredstringFportuges}[1]{%
\ifcase#1\relax
\or cento%
\or duzentas%
\or trezentas%
\or quatrocentas%
\or quinhentas%
\or seiscentas%
\or setecentas%
\or oitocentas%
\or novecentas%
\fi}
%    \end{macrocode}
% Units (initial letter in upper case):
%    \begin{macrocode}
\newcommand*{\@@Unitstringportuges}[1]{%
\ifcase#1\relax
Zero%
\or Um%
\or Dois%
\or Tr\^es%
\or Quatro%
\or Cinco%
\or Seis%
\or Sete%
\or Oito%
\or Nove%
\fi
}
%    \end{macrocode}
% As above, but feminine:
%    \begin{macrocode}
\newcommand*{\@@UnitstringFportuges}[1]{%
\ifcase#1\relax
Zera%
\or Uma%
\or Duas%
\or Tr\^es%
\or Quatro%
\or Cinco%
\or Seis%
\or Sete%
\or Oito%
\or Nove%
\fi
}
%    \end{macrocode}
% Tens (with initial letter in upper case):
%    \begin{macrocode}
\newcommand*{\@@Tenstringportuges}[1]{%
\ifcase#1\relax
\or Dez%
\or Vinte%
\or Trinta%
\or Quarenta%
\or Cinq\"uenta%
\or Sessenta%
\or Setenta%
\or Oitenta%
\or Noventa%
\or Cem%
\fi
}
%    \end{macrocode}
% Teens (with initial letter in upper case):
%    \begin{macrocode}
\newcommand*{\@@Teenstringportuges}[1]{%
\ifcase#1\relax
Dez%
\or Onze%
\or Doze%
\or Treze%
\or Quatorze%
\or Quinze%
\or Dezesseis%
\or Dezessete%
\or Dezoito%
\or Dezenove%
\fi
}
%    \end{macrocode}
% Hundreds (with initial letter in upper case):
%    \begin{macrocode}
\newcommand*{\@@Hundredstringportuges}[1]{%
\ifcase#1\relax
\or Cento%
\or Duzentos%
\or Trezentos%
\or Quatrocentos%
\or Quinhentos%
\or Seiscentos%
\or Setecentos%
\or Oitocentos%
\or Novecentos%
\fi}
%    \end{macrocode}
% As above, but feminine:
%    \begin{macrocode}
\newcommand*{\@@HundredstringFportuges}[1]{%
\ifcase#1\relax
\or Cento%
\or Duzentas%
\or Trezentas%
\or Quatrocentas%
\or Quinhentas%
\or Seiscentas%
\or Setecentas%
\or Oitocentas%
\or Novecentas%
\fi}
%    \end{macrocode}
% This has changed in version 1.08, so that it now stores
% the result in the second argument, but doesn't display
% anything. Since it only affects internal macros, it shouldn't
% affect documents created with older versions. (These internal
% macros are not meant for use in documents.)
%    \begin{macrocode}
\DeclareRobustCommand{\@numberstringMportuges}[2]{%
\let\@unitstring=\@@unitstringportuges
\let\@teenstring=\@@teenstringportuges
\let\@tenstring=\@@tenstringportuges
\let\@hundredstring=\@@hundredstringportuges
\def\@hundred{cem}\def\@thousand{mil}%
\def\@andname{e}%
\@@numberstringportuges{#1}{#2}}
%    \end{macrocode}
% As above, but feminine form:
%    \begin{macrocode}
\DeclareRobustCommand{\@numberstringFportuges}[2]{%
\let\@unitstring=\@@unitstringFportuges
\let\@teenstring=\@@teenstringportuges
\let\@tenstring=\@@tenstringportuges
\let\@hundredstring=\@@hundredstringFportuges
\def\@hundred{cem}\def\@thousand{mil}%
\def\@andname{e}%
\@@numberstringportuges{#1}{#2}}
%    \end{macrocode}
% Make neuter same as masculine:
%    \begin{macrocode}
\let\@numberstringNportuges\@numberstringMportuges
%    \end{macrocode}
% As above, but initial letters in upper case:
%    \begin{macrocode}
\DeclareRobustCommand{\@NumberstringMportuges}[2]{%
\let\@unitstring=\@@Unitstringportuges
\let\@teenstring=\@@Teenstringportuges
\let\@tenstring=\@@Tenstringportuges
\let\@hundredstring=\@@Hundredstringportuges
\def\@hundred{Cem}\def\@thousand{Mil}%
\def\@andname{e}%
\@@numberstringportuges{#1}{#2}}
%    \end{macrocode}
% As above, but feminine form:
%    \begin{macrocode}
\DeclareRobustCommand{\@NumberstringFportuges}[2]{%
\let\@unitstring=\@@UnitstringFportuges
\let\@teenstring=\@@Teenstringportuges
\let\@tenstring=\@@Tenstringportuges
\let\@hundredstring=\@@HundredstringFportuges
\def\@hundred{Cem}\def\@thousand{Mil}%
\def\@andname{e}%
\@@numberstringportuges{#1}{#2}}
%    \end{macrocode}
% Make neuter same as masculine:
%    \begin{macrocode}
\let\@NumberstringNportuges\@NumberstringMportuges
%    \end{macrocode}
% As above, but for ordinals.
%    \begin{macrocode}
\DeclareRobustCommand{\@ordinalstringMportuges}[2]{%
\let\@unitthstring=\@@unitthstringportuges
\let\@unitstring=\@@unitstringportuges
\let\@teenthstring=\@@teenthstringportuges
\let\@tenthstring=\@@tenthstringportuges
\let\@hundredthstring=\@@hundredthstringportuges
\def\@thousandth{mil\'esimo}%
\@@ordinalstringportuges{#1}{#2}}
%    \end{macrocode}
% Feminine form:
%    \begin{macrocode}
\DeclareRobustCommand{\@ordinalstringFportuges}[2]{%
\let\@unitthstring=\@@unitthstringFportuges
\let\@unitstring=\@@unitstringFportuges
\let\@teenthstring=\@@teenthstringportuges
\let\@tenthstring=\@@tenthstringFportuges
\let\@hundredthstring=\@@hundredthstringFportuges
\def\@thousandth{mil\'esima}%
\@@ordinalstringportuges{#1}{#2}}
%    \end{macrocode}
% Make neuter same as masculine:
%    \begin{macrocode}
\let\@ordinalstringNportuges\@ordinalstringMportuges
%    \end{macrocode}
% As above, but initial letters in upper case (masculine):
%    \begin{macrocode}
\DeclareRobustCommand{\@OrdinalstringMportuges}[2]{%
\let\@unitthstring=\@@Unitthstringportuges
\let\@unitstring=\@@Unitstringportuges
\let\@teenthstring=\@@teenthstringportuges
\let\@tenthstring=\@@Tenthstringportuges
\let\@hundredthstring=\@@Hundredthstringportuges
\def\@thousandth{Mil\'esimo}%
\@@ordinalstringportuges{#1}{#2}}
%    \end{macrocode}
% Feminine form:
%    \begin{macrocode}
\DeclareRobustCommand{\@OrdinalstringFportuges}[2]{%
\let\@unitthstring=\@@UnitthstringFportuges
\let\@unitstring=\@@UnitstringFportuges
\let\@teenthstring=\@@teenthstringportuges
\let\@tenthstring=\@@TenthstringFportuges
\let\@hundredthstring=\@@HundredthstringFportuges
\def\@thousandth{Mil\'esima}%
\@@ordinalstringportuges{#1}{#2}}
%    \end{macrocode}
% Make neuter same as masculine:
%    \begin{macrocode}
\let\@OrdinalstringNportuges\@OrdinalstringMportuges
%    \end{macrocode}
% In order to do the ordinals, split into units, teens, tens
% and hundreds. Units:
%    \begin{macrocode}
\newcommand*{\@@unitthstringportuges}[1]{%
\ifcase#1\relax
zero%
\or primeiro%
\or segundo%
\or terceiro%
\or quarto%
\or quinto%
\or sexto%
\or s\'etimo%
\or oitavo%
\or nono%
\fi
}
%    \end{macrocode}
% Tens:
%    \begin{macrocode}
\newcommand*{\@@tenthstringportuges}[1]{%
\ifcase#1\relax
\or d\'ecimo%
\or vig\'esimo%
\or trig\'esimo%
\or quadrag\'esimo%
\or q\"uinquag\'esimo%
\or sexag\'esimo%
\or setuag\'esimo%
\or octog\'esimo%
\or nonag\'esimo%
\fi
}
%    \end{macrocode}
% Teens:
%    \begin{macrocode}
\newcommand*{\@@teenthstringportuges}[1]{%
\@tenthstring{1}%
\ifnum#1>0\relax
-\@unitthstring{#1}%
\fi}
%    \end{macrocode}
% Hundreds:
%    \begin{macrocode}
\newcommand*{\@@hundredthstringportuges}[1]{%
\ifcase#1\relax
\or cent\'esimo%
\or ducent\'esimo%
\or trecent\'esimo%
\or quadringent\'esimo%
\or q\"uingent\'esimo%
\or seiscent\'esimo%
\or setingent\'esimo%
\or octingent\'esimo%
\or nongent\'esimo%
\fi}
%    \end{macrocode}
% Units (feminine):
%    \begin{macrocode}
\newcommand*{\@@unitthstringFportuges}[1]{%
\ifcase#1\relax
zero%
\or primeira%
\or segunda%
\or terceira%
\or quarta%
\or quinta%
\or sexta%
\or s\'etima%
\or oitava%
\or nona%
\fi
}
%    \end{macrocode}
% Tens (feminine):
%    \begin{macrocode}
\newcommand*{\@@tenthstringFportuges}[1]{%
\ifcase#1\relax
\or d\'ecima%
\or vig\'esima%
\or trig\'esima%
\or quadrag\'esima%
\or q\"uinquag\'esima%
\or sexag\'esima%
\or setuag\'esima%
\or octog\'esima%
\or nonag\'esima%
\fi
}
%    \end{macrocode}
% Hundreds (feminine):
%    \begin{macrocode}
\newcommand*{\@@hundredthstringFportuges}[1]{%
\ifcase#1\relax
\or cent\'esima%
\or ducent\'esima%
\or trecent\'esima%
\or quadringent\'esima%
\or q\"uingent\'esima%
\or seiscent\'esima%
\or setingent\'esima%
\or octingent\'esima%
\or nongent\'esima%
\fi}
%    \end{macrocode}
% As above, but with initial letter in upper case. Units:
%    \begin{macrocode}
\newcommand*{\@@Unitthstringportuges}[1]{%
\ifcase#1\relax
Zero%
\or Primeiro%
\or Segundo%
\or Terceiro%
\or Quarto%
\or Quinto%
\or Sexto%
\or S\'etimo%
\or Oitavo%
\or Nono%
\fi
}
%    \end{macrocode}
% Tens:
%    \begin{macrocode}
\newcommand*{\@@Tenthstringportuges}[1]{%
\ifcase#1\relax
\or D\'ecimo%
\or Vig\'esimo%
\or Trig\'esimo%
\or Quadrag\'esimo%
\or Q\"uinquag\'esimo%
\or Sexag\'esimo%
\or Setuag\'esimo%
\or Octog\'esimo%
\or Nonag\'esimo%
\fi
}
%    \end{macrocode}
% Hundreds:
%    \begin{macrocode}
\newcommand*{\@@Hundredthstringportuges}[1]{%
\ifcase#1\relax
\or Cent\'esimo%
\or Ducent\'esimo%
\or Trecent\'esimo%
\or Quadringent\'esimo%
\or Q\"uingent\'esimo%
\or Seiscent\'esimo%
\or Setingent\'esimo%
\or Octingent\'esimo%
\or Nongent\'esimo%
\fi}
%    \end{macrocode}
% As above, but feminine. Units:
%    \begin{macrocode}
\newcommand*{\@@UnitthstringFportuges}[1]{%
\ifcase#1\relax
Zera%
\or Primeira%
\or Segunda%
\or Terceira%
\or Quarta%
\or Quinta%
\or Sexta%
\or S\'etima%
\or Oitava%
\or Nona%
\fi
}
%    \end{macrocode}
% Tens (feminine);
%    \begin{macrocode}
\newcommand*{\@@TenthstringFportuges}[1]{%
\ifcase#1\relax
\or D\'ecima%
\or Vig\'esima%
\or Trig\'esima%
\or Quadrag\'esima%
\or Q\"uinquag\'esima%
\or Sexag\'esima%
\or Setuag\'esima%
\or Octog\'esima%
\or Nonag\'esima%
\fi
}
%    \end{macrocode}
% Hundreds (feminine):
%    \begin{macrocode}
\newcommand*{\@@HundredthstringFportuges}[1]{%
\ifcase#1\relax
\or Cent\'esima%
\or Ducent\'esima%
\or Trecent\'esima%
\or Quadringent\'esima%
\or Q\"uingent\'esima%
\or Seiscent\'esima%
\or Setingent\'esima%
\or Octingent\'esima%
\or Nongent\'esima%
\fi}
%    \end{macrocode}
% This has changed in version 1.09, so that it now stores
% the result in the second argument (a control sequence), but it
% doesn't display anything. Since it only affects internal macros,
% it shouldn't affect documents created with older versions.
% (These internal macros are not meant for use in documents.)
%    \begin{macrocode}
\newcommand*{\@@numberstringportuges}[2]{%
\ifnum#1>99999
\PackageError{fmtcount}{Out of range}%
{This macro only works for values less than 100000}%
\else
\ifnum#1<0
\PackageError{fmtcount}{Negative numbers not permitted}%
{This macro does not work for negative numbers, however
you can try typing "minus" first, and then pass the modulus of
this number}%
\fi
\fi
\def#2{}%
\@strctr=#1\relax \divide\@strctr by 1000\relax
\ifnum\@strctr>9
% #1 is greater or equal to 10000
  \divide\@strctr by 10
  \ifnum\@strctr>1\relax
    \let\@@fc@numstr#2\relax
    \edef#2{\@@fc@numstr\@tenstring{\@strctr}}%
    \@strctr=#1 \divide\@strctr by 1000\relax
    \@modulo{\@strctr}{10}%
    \ifnum\@strctr>0
      \ifnum\@strctr=1\relax
        \let\@@fc@numstr#2\relax
        \edef#2{\@@fc@numstr\ \@andname}%
      \fi
      \let\@@fc@numstr#2\relax
      \edef#2{\@@fc@numstr\ \@unitstring{\@strctr}}%
    \fi
  \else
    \@strctr=#1\relax
    \divide\@strctr by 1000\relax
    \@modulo{\@strctr}{10}%
    \let\@@fc@numstr#2\relax
    \edef#2{\@@fc@numstr\@teenstring{\@strctr}}%
  \fi
  \let\@@fc@numstr#2\relax
  \edef#2{\@@fc@numstr\ \@thousand}%
\else
  \ifnum\@strctr>0\relax 
    \ifnum\@strctr>1\relax
      \let\@@fc@numstr#2\relax
      \edef#2{\@@fc@numstr\@unitstring{\@strctr}\ }%
    \fi
    \let\@@fc@numstr#2\relax
    \edef#2{\@@fc@numstr\@thousand}%
  \fi
\fi
\@strctr=#1\relax \@modulo{\@strctr}{1000}%
\divide\@strctr by 100\relax
\ifnum\@strctr>0\relax
  \ifnum#1>1000 \relax
    \let\@@fc@numstr#2\relax
    \edef#2{\@@fc@numstr\ }%
  \fi
  \@tmpstrctr=#1\relax
  \@modulo{\@tmpstrctr}{1000}%
  \let\@@fc@numstr#2\relax
  \ifnum\@tmpstrctr=100\relax
    \edef#2{\@@fc@numstr\@tenstring{10}}%
  \else
    \edef#2{\@@fc@numstr\@hundredstring{\@strctr}}%
  \fi%
\fi
\@strctr=#1\relax \@modulo{\@strctr}{100}%
\ifnum#1>100\relax
  \ifnum\@strctr>0\relax
    \let\@@fc@numstr#2\relax
    \edef#2{\@@fc@numstr\ \@andname\ }%
  \fi
\fi
\ifnum\@strctr>19\relax
  \divide\@strctr by 10\relax
  \let\@@fc@numstr#2\relax
  \edef#2{\@@fc@numstr\@tenstring{\@strctr}}%
  \@strctr=#1\relax \@modulo{\@strctr}{10}%
  \ifnum\@strctr>0
    \ifnum\@strctr=1\relax
      \let\@@fc@numstr#2\relax
      \edef#2{\@@fc@numstr\ \@andname}%
    \else
      \ifnum#1>100\relax
        \let\@@fc@numstr#2\relax
        \edef#2{\@@fc@numstr\ \@andname}%
      \fi
    \fi 
    \let\@@fc@numstr#2\relax
    \edef#2{\@@fc@numstr\ \@unitstring{\@strctr}}%
  \fi
\else
  \ifnum\@strctr<10\relax
    \ifnum\@strctr=0\relax
      \ifnum#1<100\relax
        \let\@@fc@numstr#2\relax
        \edef#2{\@@fc@numstr\@unitstring{\@strctr}}%
      \fi
    \else%(>0,<10)
      \let\@@fc@numstr#2\relax
      \edef#2{\@@fc@numstr\@unitstring{\@strctr}}%
    \fi
  \else%>10
    \@modulo{\@strctr}{10}%
    \let\@@fc@numstr#2\relax
    \edef#2{\@@fc@numstr\@teenstring{\@strctr}}%
  \fi
\fi
}
%    \end{macrocode}
% As above, but for ordinals.
%    \begin{macrocode}
\newcommand*{\@@ordinalstringportuges}[2]{%
\@strctr=#1\relax
\ifnum#1>99999
\PackageError{fmtcount}{Out of range}%
{This macro only works for values less than 100000}%
\else
\ifnum#1<0
\PackageError{fmtcount}{Negative numbers not permitted}%
{This macro does not work for negative numbers, however
you can try typing "minus" first, and then pass the modulus of
this number}%
\else
\def#2{}%
\ifnum\@strctr>999\relax
  \divide\@strctr by 1000\relax
  \ifnum\@strctr>1\relax
    \ifnum\@strctr>9\relax
      \@tmpstrctr=\@strctr
      \ifnum\@strctr<20
        \@modulo{\@tmpstrctr}{10}%
        \let\@@fc@ordstr#2\relax
        \edef#2{\@@fc@ordstr\@teenthstring{\@tmpstrctr}}%
      \else
        \divide\@tmpstrctr by 10\relax
        \let\@@fc@ordstr#2\relax
        \edef#2{\@@fc@ordstr\@tenthstring{\@tmpstrctr}}%
        \@tmpstrctr=\@strctr
        \@modulo{\@tmpstrctr}{10}%
        \ifnum\@tmpstrctr>0\relax
          \let\@@fc@ordstr#2\relax
          \edef#2{\@@fc@ordstr\@unitthstring{\@tmpstrctr}}%
        \fi
      \fi
    \else
      \let\@@fc@ordstr#2\relax
      \edef#2{\@@fc@ordstr\@unitstring{\@strctr}}%
    \fi
  \fi
  \let\@@fc@ordstr#2\relax
  \edef#2{\@@fc@ordstr\@thousandth}%
\fi
\@strctr=#1\relax
\@modulo{\@strctr}{1000}%
\ifnum\@strctr>99\relax
  \@tmpstrctr=\@strctr
  \divide\@tmpstrctr by 100\relax
  \ifnum#1>1000\relax
    \let\@@fc@ordstr#2\relax
    \edef#2{\@@fc@ordstr-}%
  \fi
  \let\@@fc@ordstr#2\relax
  \edef#2{\@@fc@ordstr\@hundredthstring{\@tmpstrctr}}%
\fi
\@modulo{\@strctr}{100}%
\ifnum#1>99\relax
  \ifnum\@strctr>0\relax
    \let\@@fc@ordstr#2\relax
    \edef#2{\@@fc@ordstr-}%
  \fi
\fi
\ifnum\@strctr>9\relax
  \@tmpstrctr=\@strctr
  \divide\@tmpstrctr by 10\relax
  \let\@@fc@ordstr#2\relax
  \edef#2{\@@fc@ordstr\@tenthstring{\@tmpstrctr}}%
  \@tmpstrctr=\@strctr
  \@modulo{\@tmpstrctr}{10}%
  \ifnum\@tmpstrctr>0\relax
    \let\@@fc@ordstr#2\relax
    \edef#2{\@@fc@ordstr-\@unitthstring{\@tmpstrctr}}%
  \fi
\else
  \ifnum\@strctr=0\relax
    \ifnum#1=0\relax
      \let\@@fc@ordstr#2\relax
      \edef#2{\@@fc@ordstr\@unitstring{0}}%
    \fi
  \else
    \let\@@fc@ordstr#2\relax
    \edef#2{\@@fc@ordstr\@unitthstring{\@strctr}}%
  \fi
\fi
\fi
\fi
}
%    \end{macrocode}
%\iffalse
%    \begin{macrocode}
%</fc-portuges.def>
%    \end{macrocode}
%\fi
%\iffalse
%    \begin{macrocode}
%<*fc-spanish.def>
%    \end{macrocode}
%\fi
% \subsection{fc-spanish.def}
% Spanish definitions
%    \begin{macrocode}
\ProvidesFile{fc-spanish.def}[2007/05/26]
%    \end{macrocode}
% Define macro that converts a number or count register (first
% argument) to an ordinal, and stores the result in the
% second argument, which must be a control sequence.
% Masculine:
%    \begin{macrocode}
\newcommand{\@ordinalMspanish}[2]{%
\edef#2{\number#1\relax\noexpand\fmtord{o}}}
%    \end{macrocode}
% Feminine:
%    \begin{macrocode}
\newcommand{\@ordinalFspanish}[2]{%
\edef#2{\number#1\relax\noexpand\fmtord{a}}}
%    \end{macrocode}
% Make neuter same as masculine:
%    \begin{macrocode}
\let\@ordinalNspanish\@ordinalMspanish
%    \end{macrocode}
% Convert a number to text. The easiest way to do this is to
% break it up into units, tens, teens, twenties and hundreds.
% Units (argument must be a number from 0 to 9):
%    \begin{macrocode}
\newcommand{\@@unitstringspanish}[1]{%
\ifcase#1\relax
cero%
\or uno%
\or dos%
\or tres%
\or cuatro%
\or cinco%
\or seis%
\or siete%
\or ocho%
\or nueve%
\fi
}
%    \end{macrocode}
% Feminine:
%    \begin{macrocode}
\newcommand{\@@unitstringFspanish}[1]{%
\ifcase#1\relax
cera%
\or una%
\or dos%
\or tres%
\or cuatro%
\or cinco%
\or seis%
\or siete%
\or ocho%
\or nueve%
\fi
}
%    \end{macrocode}
% Tens (argument must go from 1 to 10):
%    \begin{macrocode}
\newcommand{\@@tenstringspanish}[1]{%
\ifcase#1\relax
\or diez%
\or viente%
\or treinta%
\or cuarenta%
\or cincuenta%
\or sesenta%
\or setenta%
\or ochenta%
\or noventa%
\or cien%
\fi
}
%    \end{macrocode}
% Teens:
%    \begin{macrocode}
\newcommand{\@@teenstringspanish}[1]{%
\ifcase#1\relax
diez%
\or once%
\or doce%
\or trece%
\or catorce%
\or quince%
\or diecis\'eis%
\or diecisiete%
\or dieciocho%
\or diecinueve%
\fi
}
%    \end{macrocode}
% Twenties:
%    \begin{macrocode}
\newcommand{\@@twentystringspanish}[1]{%
\ifcase#1\relax
veinte%
\or veintiuno%
\or veintid\'os%
\or veintitr\'es%
\or veinticuatro%
\or veinticinco%
\or veintis\'eis%
\or veintisiete%
\or veintiocho%
\or veintinueve%
\fi}
%    \end{macrocode}
% Feminine form:
%    \begin{macrocode}
\newcommand{\@@twentystringFspanish}[1]{%
\ifcase#1\relax
veinte%
\or veintiuna%
\or veintid\'os%
\or veintitr\'es%
\or veinticuatro%
\or veinticinco%
\or veintis\'eis%
\or veintisiete%
\or veintiocho%
\or veintinueve%
\fi}
%    \end{macrocode}
% Hundreds:
%    \begin{macrocode}
\newcommand{\@@hundredstringspanish}[1]{%
\ifcase#1\relax
\or ciento%
\or doscientos%
\or trescientos%
\or cuatrocientos%
\or quinientos%
\or seiscientos%
\or setecientos%
\or ochocientos%
\or novecientos%
\fi}
%    \end{macrocode}
% Feminine form:
%    \begin{macrocode}
\newcommand{\@@hundredstringFspanish}[1]{%
\ifcase#1\relax
\or cienta%
\or doscientas%
\or trescientas%
\or cuatrocientas%
\or quinientas%
\or seiscientas%
\or setecientas%
\or ochocientas%
\or novecientas%
\fi}
%    \end{macrocode}
% As above, but with initial letter uppercase:
%    \begin{macrocode}
\newcommand{\@@Unitstringspanish}[1]{%
\ifcase#1\relax
Cero%
\or Uno%
\or Dos%
\or Tres%
\or Cuatro%
\or Cinco%
\or Seis%
\or Siete%
\or Ocho%
\or Nueve%
\fi
}
%    \end{macrocode}
% Feminine form:
%    \begin{macrocode}
\newcommand{\@@UnitstringFspanish}[1]{%
\ifcase#1\relax
Cera%
\or Una%
\or Dos%
\or Tres%
\or Cuatro%
\or Cinco%
\or Seis%
\or Siete%
\or Ocho%
\or Nueve%
\fi
}
%    \end{macrocode}
% Tens:
%    \begin{macrocode}
\newcommand{\@@Tenstringspanish}[1]{%
\ifcase#1\relax
\or Diez%
\or Viente%
\or Treinta%
\or Cuarenta%
\or Cincuenta%
\or Sesenta%
\or Setenta%
\or Ochenta%
\or Noventa%
\or Cien%
\fi
}
%    \end{macrocode}
% Teens:
%    \begin{macrocode}
\newcommand{\@@Teenstringspanish}[1]{%
\ifcase#1\relax
Diez%
\or Once%
\or Doce%
\or Trece%
\or Catorce%
\or Quince%
\or Diecis\'eis%
\or Diecisiete%
\or Dieciocho%
\or Diecinueve%
\fi
}
%    \end{macrocode}
% Twenties:
%    \begin{macrocode}
\newcommand{\@@Twentystringspanish}[1]{%
\ifcase#1\relax
Veinte%
\or Veintiuno%
\or Veintid\'os%
\or Veintitr\'es%
\or Veinticuatro%
\or Veinticinco%
\or Veintis\'eis%
\or Veintisiete%
\or Veintiocho%
\or Veintinueve%
\fi}
%    \end{macrocode}
% Feminine form:
%    \begin{macrocode}
\newcommand{\@@TwentystringFspanish}[1]{%
\ifcase#1\relax
Veinte%
\or Veintiuna%
\or Veintid\'os%
\or Veintitr\'es%
\or Veinticuatro%
\or Veinticinco%
\or Veintis\'eis%
\or Veintisiete%
\or Veintiocho%
\or Veintinueve%
\fi}
%    \end{macrocode}
% Hundreds:
%    \begin{macrocode}
\newcommand{\@@Hundredstringspanish}[1]{%
\ifcase#1\relax
\or Ciento%
\or Doscientos%
\or Trescientos%
\or Cuatrocientos%
\or Quinientos%
\or Seiscientos%
\or Setecientos%
\or Ochocientos%
\or Novecientos%
\fi}
%    \end{macrocode}
% Feminine form:
%    \begin{macrocode}
\newcommand{\@@HundredstringFspanish}[1]{%
\ifcase#1\relax
\or Cienta%
\or Doscientas%
\or Trescientas%
\or Cuatrocientas%
\or Quinientas%
\or Seiscientas%
\or Setecientas%
\or Ochocientas%
\or Novecientas%
\fi}
%    \end{macrocode}
% This has changed in version 1.09, so that it now stores the
% result in the second argument, but doesn't display anything.
% Since it only affects internal macros, it shouldn't affect
% documents created with older versions. (These internal macros
% are not meant for use in documents.)
%    \begin{macrocode}
\DeclareRobustCommand{\@numberstringMspanish}[2]{%
\let\@unitstring=\@@unitstringspanish
\let\@teenstring=\@@teenstringspanish
\let\@tenstring=\@@tenstringspanish
\let\@twentystring=\@@twentystringspanish
\let\@hundredstring=\@@hundredstringspanish
\def\@hundred{cien}\def\@thousand{mil}%
\def\@andname{y}%
\@@numberstringspanish{#1}{#2}}
%    \end{macrocode}
% Feminine form:
%    \begin{macrocode}
\DeclareRobustCommand{\@numberstringFspanish}[2]{%
\let\@unitstring=\@@unitstringFspanish
\let\@teenstring=\@@teenstringspanish
\let\@tenstring=\@@tenstringspanish
\let\@twentystring=\@@twentystringFspanish
\let\@hundredstring=\@@hundredstringFspanish
\def\@hundred{cien}\def\@thousand{mil}%
\def\@andname{y}%
\@@numberstringspanish{#1}{#2}}
%    \end{macrocode}
% Make neuter same as masculine:
%    \begin{macrocode}
\let\@numberstringNspanish\@numberstringMspanish
%    \end{macrocode}
% As above, but initial letters in upper case:
%    \begin{macrocode}
\DeclareRobustCommand{\@NumberstringMspanish}[2]{%
\let\@unitstring=\@@Unitstringspanish
\let\@teenstring=\@@Teenstringspanish
\let\@tenstring=\@@Tenstringspanish
\let\@twentystring=\@@Twentystringspanish
\let\@hundredstring=\@@Hundredstringspanish
\def\@andname{y}%
\def\@hundred{Cien}\def\@thousand{Mil}%
\@@numberstringspanish{#1}{#2}}
%    \end{macrocode}
% Feminine form:
%    \begin{macrocode}
\DeclareRobustCommand{\@NumberstringFspanish}[2]{%
\let\@unitstring=\@@UnitstringFspanish
\let\@teenstring=\@@Teenstringspanish
\let\@tenstring=\@@Tenstringspanish
\let\@twentystring=\@@TwentystringFspanish
\let\@hundredstring=\@@HundredstringFspanish
\def\@andname{y}%
\def\@hundred{Cien}\def\@thousand{Mil}%
\@@numberstringspanish{#1}{#2}}
%    \end{macrocode}
% Make neuter same as masculine:
%    \begin{macrocode}
\let\@NumberstringNspanish\@NumberstringMspanish
%    \end{macrocode}
% As above, but for ordinals.
%    \begin{macrocode}
\DeclareRobustCommand{\@ordinalstringMspanish}[2]{%
\let\@unitthstring=\@@unitthstringspanish
\let\@unitstring=\@@unitstringspanish
\let\@teenthstring=\@@teenthstringspanish
\let\@tenthstring=\@@tenthstringspanish
\let\@hundredthstring=\@@hundredthstringspanish
\def\@thousandth{mil\'esimo}%
\@@ordinalstringspanish{#1}{#2}}
%    \end{macrocode}
% Feminine form:
%    \begin{macrocode}
\DeclareRobustCommand{\@ordinalstringFspanish}[2]{%
\let\@unitthstring=\@@unitthstringFspanish
\let\@unitstring=\@@unitstringFspanish
\let\@teenthstring=\@@teenthstringFspanish
\let\@tenthstring=\@@tenthstringFspanish
\let\@hundredthstring=\@@hundredthstringFspanish
\def\@thousandth{mil\'esima}%
\@@ordinalstringspanish{#1}{#2}}
%    \end{macrocode}
% Make neuter same as masculine:
%    \begin{macrocode}
\let\@ordinalstringNspanish\@ordinalstringMspanish
%    \end{macrocode}
% As above, but with initial letters in upper case.
%    \begin{macrocode}
\DeclareRobustCommand{\@OrdinalstringMspanish}[2]{%
\let\@unitthstring=\@@Unitthstringspanish
\let\@unitstring=\@@Unitstringspanish
\let\@teenthstring=\@@Teenthstringspanish
\let\@tenthstring=\@@Tenthstringspanish
\let\@hundredthstring=\@@Hundredthstringspanish
\def\@thousandth{Mil\'esimo}%
\@@ordinalstringspanish{#1}{#2}}
%    \end{macrocode}
% Feminine form:
%    \begin{macrocode}
\DeclareRobustCommand{\@OrdinalstringFspanish}[2]{%
\let\@unitthstring=\@@UnitthstringFspanish
\let\@unitstring=\@@UnitstringFspanish
\let\@teenthstring=\@@TeenthstringFspanish
\let\@tenthstring=\@@TenthstringFspanish
\let\@hundredthstring=\@@HundredthstringFspanish
\def\@thousandth{Mil\'esima}%
\@@ordinalstringspanish{#1}{#2}}
%    \end{macrocode}
% Make neuter same as masculine:
%    \begin{macrocode}
\let\@OrdinalstringNspanish\@OrdinalstringMspanish
%    \end{macrocode}
% Code for convert numbers into textual ordinals. As before,
% it is easier to split it into units, tens, teens and hundreds.
% Units:
%    \begin{macrocode}
\newcommand{\@@unitthstringspanish}[1]{%
\ifcase#1\relax
cero%
\or primero%
\or segundo%
\or tercero%
\or cuarto%
\or quinto%
\or sexto%
\or s\'eptimo%
\or octavo%
\or noveno%
\fi
}
%    \end{macrocode}
% Tens:
%    \begin{macrocode}
\newcommand{\@@tenthstringspanish}[1]{%
\ifcase#1\relax
\or d\'ecimo%
\or vig\'esimo%
\or trig\'esimo%
\or cuadrag\'esimo%
\or quincuag\'esimo%
\or sexag\'esimo%
\or septuag\'esimo%
\or octog\'esimo%
\or nonag\'esimo%
\fi
}
%    \end{macrocode}
% Teens:
%    \begin{macrocode}
\newcommand{\@@teenthstringspanish}[1]{%
\ifcase#1\relax
d\'ecimo%
\or und\'ecimo%
\or duod\'ecimo%
\or decimotercero%
\or decimocuarto%
\or decimoquinto%
\or decimosexto%
\or decimos\'eptimo%
\or decimoctavo%
\or decimonoveno%
\fi
}
%    \end{macrocode}
% Hundreds:
%    \begin{macrocode}
\newcommand{\@@hundredthstringspanish}[1]{%
\ifcase#1\relax
\or cent\'esimo%
\or ducent\'esimo%
\or tricent\'esimo%
\or cuadringent\'esimo%
\or quingent\'esimo%
\or sexcent\'esimo%
\or septing\'esimo%
\or octingent\'esimo%
\or noningent\'esimo%
\fi}
%    \end{macrocode}
% Units (feminine):
%    \begin{macrocode}
\newcommand{\@@unitthstringFspanish}[1]{%
\ifcase#1\relax
cera%
\or primera%
\or segunda%
\or tercera%
\or cuarta%
\or quinta%
\or sexta%
\or s\'eptima%
\or octava%
\or novena%
\fi
}
%    \end{macrocode}
% Tens (feminine):
%    \begin{macrocode}
\newcommand{\@@tenthstringFspanish}[1]{%
\ifcase#1\relax
\or d\'ecima%
\or vig\'esima%
\or trig\'esima%
\or cuadrag\'esima%
\or quincuag\'esima%
\or sexag\'esima%
\or septuag\'esima%
\or octog\'esima%
\or nonag\'esima%
\fi
}
%    \end{macrocode}
% Teens (feminine)
%    \begin{macrocode}
\newcommand{\@@teenthstringFspanish}[1]{%
\ifcase#1\relax
d\'ecima%
\or und\'ecima%
\or duod\'ecima%
\or decimotercera%
\or decimocuarta%
\or decimoquinta%
\or decimosexta%
\or decimos\'eptima%
\or decimoctava%
\or decimonovena%
\fi
}
%    \end{macrocode}
% Hundreds (feminine)
%    \begin{macrocode}
\newcommand{\@@hundredthstringFspanish}[1]{%
\ifcase#1\relax
\or cent\'esima%
\or ducent\'esima%
\or tricent\'esima%
\or cuadringent\'esima%
\or quingent\'esima%
\or sexcent\'esima%
\or septing\'esima%
\or octingent\'esima%
\or noningent\'esima%
\fi}
%    \end{macrocode}
% As above, but with initial letters in upper case
%    \begin{macrocode}
\newcommand{\@@Unitthstringspanish}[1]{%
\ifcase#1\relax
Cero%
\or Primero%
\or Segundo%
\or Tercero%
\or Cuarto%
\or Quinto%
\or Sexto%
\or S\'eptimo%
\or Octavo%
\or Noveno%
\fi
}
%    \end{macrocode}
% Tens:
%    \begin{macrocode}
\newcommand{\@@Tenthstringspanish}[1]{%
\ifcase#1\relax
\or D\'ecimo%
\or Vig\'esimo%
\or Trig\'esimo%
\or Cuadrag\'esimo%
\or Quincuag\'esimo%
\or Sexag\'esimo%
\or Septuag\'esimo%
\or Octog\'esimo%
\or Nonag\'esimo%
\fi
}
%    \end{macrocode}
% Teens:
%    \begin{macrocode}
\newcommand{\@@Teenthstringspanish}[1]{%
\ifcase#1\relax
D\'ecimo%
\or Und\'ecimo%
\or Duod\'ecimo%
\or Decimotercero%
\or Decimocuarto%
\or Decimoquinto%
\or Decimosexto%
\or Decimos\'eptimo%
\or Decimoctavo%
\or Decimonoveno%
\fi
}
%    \end{macrocode}
% Hundreds
%    \begin{macrocode}
\newcommand{\@@Hundredthstringspanish}[1]{%
\ifcase#1\relax
\or Cent\'esimo%
\or Ducent\'esimo%
\or Tricent\'esimo%
\or Cuadringent\'esimo%
\or Quingent\'esimo%
\or Sexcent\'esimo%
\or Septing\'esimo%
\or Octingent\'esimo%
\or Noningent\'esimo%
\fi}
%    \end{macrocode}
% As above, but feminine.
%    \begin{macrocode}
\newcommand{\@@UnitthstringFspanish}[1]{%
\ifcase#1\relax
Cera%
\or Primera%
\or Segunda%
\or Tercera%
\or Cuarta%
\or Quinta%
\or Sexta%
\or S\'eptima%
\or Octava%
\or Novena%
\fi
}
%    \end{macrocode}
% Tens (feminine)
%    \begin{macrocode}
\newcommand{\@@TenthstringFspanish}[1]{%
\ifcase#1\relax
\or D\'ecima%
\or Vig\'esima%
\or Trig\'esima%
\or Cuadrag\'esima%
\or Quincuag\'esima%
\or Sexag\'esima%
\or Septuag\'esima%
\or Octog\'esima%
\or Nonag\'esima%
\fi
}
%    \end{macrocode}
% Teens (feminine):
%    \begin{macrocode}
\newcommand{\@@TeenthstringFspanish}[1]{%
\ifcase#1\relax
D\'ecima%
\or Und\'ecima%
\or Duod\'ecima%
\or Decimotercera%
\or Decimocuarta%
\or Decimoquinta%
\or Decimosexta%
\or Decimos\'eptima%
\or Decimoctava%
\or Decimonovena%
\fi
}
%    \end{macrocode}
% Hundreds (feminine):
%    \begin{macrocode}
\newcommand{\@@HundredthstringFspanish}[1]{%
\ifcase#1\relax
\or Cent\'esima%
\or Ducent\'esima%
\or Tricent\'esima%
\or Cuadringent\'esima%
\or Quingent\'esima%
\or Sexcent\'esima%
\or Septing\'esima%
\or Octingent\'esima%
\or Noningent\'esima%
\fi}

%    \end{macrocode}
% This has changed in version 1.09, so that it now stores the
% results in the second argument (which must be a control
% sequence), but it doesn't display anything. Since it only
% affects internal macros, it shouldn't affect documnets created
% with older versions. (These internal macros are not meant for
% use in documents.)
%    \begin{macrocode}
\newcommand{\@@numberstringspanish}[2]{%
\ifnum#1>99999
\PackageError{fmtcount}{Out of range}%
{This macro only works for values less than 100000}%
\else
\ifnum#1<0
\PackageError{fmtcount}{Negative numbers not permitted}%
{This macro does not work for negative numbers, however
you can try typing "minus" first, and then pass the modulus of
this number}%
\fi
\fi
\def#2{}%
\@strctr=#1\relax \divide\@strctr by 1000\relax
\ifnum\@strctr>9
% #1 is greater or equal to 10000
  \divide\@strctr by 10
  \ifnum\@strctr>1
    \let\@@fc@numstr#2\relax
    \edef#2{\@@fc@numstr\@tenstring{\@strctr}}%
    \@strctr=#1 \divide\@strctr by 1000\relax
    \@modulo{\@strctr}{10}%
    \ifnum\@strctr>0\relax
       \let\@@fc@numstr#2\relax
       \edef#2{\@@fc@numstr\ \@andname\ \@unitstring{\@strctr}}%
    \fi
  \else
    \@strctr=#1\relax
    \divide\@strctr by 1000\relax
    \@modulo{\@strctr}{10}%
    \let\@@fc@numstr#2\relax
    \edef#2{\@@fc@numstr\@teenstring{\@strctr}}%
  \fi
  \let\@@fc@numstr#2\relax
  \edef#2{\@@fc@numstr\ \@thousand}%
\else
  \ifnum\@strctr>0\relax 
    \ifnum\@strctr>1\relax
       \let\@@fc@numstr#2\relax
       \edef#2{\@@fc@numstr\@unitstring{\@strctr}\ }%
    \fi
    \let\@@fc@numstr#2\relax
    \edef#2{\@@fc@numstr\@thousand}%
  \fi
\fi
\@strctr=#1\relax \@modulo{\@strctr}{1000}%
\divide\@strctr by 100\relax
\ifnum\@strctr>0\relax
  \ifnum#1>1000\relax
    \let\@@fc@numstr#2\relax
    \edef#2{\@@fc@numstr\ }%
  \fi
  \@tmpstrctr=#1\relax
  \@modulo{\@tmpstrctr}{1000}%
  \ifnum\@tmpstrctr=100\relax
    \let\@@fc@numstr#2\relax
    \edef#2{\@@fc@numstr\@tenstring{10}}%
  \else
    \let\@@fc@numstr#2\relax
    \edef#2{\@@fc@numstr\@hundredstring{\@strctr}}%
  \fi
\fi
\@strctr=#1\relax \@modulo{\@strctr}{100}%
\ifnum#1>100\relax
  \ifnum\@strctr>0\relax
    \let\@@fc@numstr#2\relax
    \edef#2{\@@fc@numstr\ \@andname\ }%
  \fi
\fi
\ifnum\@strctr>29\relax
  \divide\@strctr by 10\relax
  \let\@@fc@numstr#2\relax
  \edef#2{\@@fc@numstr\@tenstring{\@strctr}}%
  \@strctr=#1\relax \@modulo{\@strctr}{10}%
  \ifnum\@strctr>0\relax
    \let\@@fc@numstr#2\relax
    \edef#2{\@@fc@numstr\ \@andname\ \@unitstring{\@strctr}}%
  \fi
\else
  \ifnum\@strctr<10\relax
    \ifnum\@strctr=0\relax
      \ifnum#1<100\relax
        \let\@@fc@numstr#2\relax
        \edef#2{\@@fc@numstr\@unitstring{\@strctr}}%
      \fi
    \else
      \let\@@fc@numstr#2\relax
      \edef#2{\@@fc@numstr\@unitstring{\@strctr}}%
    \fi
  \else
    \ifnum\@strctr>19\relax
      \@modulo{\@strctr}{10}%
      \let\@@fc@numstr#2\relax
      \edef#2{\@@fc@numstr\@twentystring{\@strctr}}%
    \else
      \@modulo{\@strctr}{10}%
      \let\@@fc@numstr#2\relax
      \edef#2{\@@fc@numstr\@teenstring{\@strctr}}%
    \fi
  \fi
\fi
}
%    \end{macrocode}
% As above, but for ordinals
%    \begin{macrocode}
\newcommand{\@@ordinalstringspanish}[2]{%
\@strctr=#1\relax
\ifnum#1>99999
\PackageError{fmtcount}{Out of range}%
{This macro only works for values less than 100000}%
\else
\ifnum#1<0
\PackageError{fmtcount}{Negative numbers not permitted}%
{This macro does not work for negative numbers, however
you can try typing "minus" first, and then pass the modulus of
this number}%
\else
\def#2{}%
\ifnum\@strctr>999\relax
  \divide\@strctr by 1000\relax
  \ifnum\@strctr>1\relax
    \ifnum\@strctr>9\relax
      \@tmpstrctr=\@strctr
      \ifnum\@strctr<20
        \@modulo{\@tmpstrctr}{10}%
        \let\@@fc@ordstr#2\relax
        \edef#2{\@@fc@ordstr\@teenthstring{\@tmpstrctr}}%
      \else
        \divide\@tmpstrctr by 10\relax
        \let\@@fc@ordstr#2\relax
        \edef#2{\@@fc@ordstr\@tenthstring{\@tmpstrctr}}%
        \@tmpstrctr=\@strctr
        \@modulo{\@tmpstrctr}{10}%
        \ifnum\@tmpstrctr>0\relax
          \let\@@fc@ordstr#2\relax
          \edef#2{\@@fc@ordstr\@unitthstring{\@tmpstrctr}}%
        \fi
      \fi
    \else
       \let\@@fc@ordstr#2\relax
       \edef#2{\@@fc@ordstr\@unitstring{\@strctr}}%
    \fi
  \fi
  \let\@@fc@ordstr#2\relax
  \edef#2{\@@fc@ordstr\@thousandth}%
\fi
\@strctr=#1\relax
\@modulo{\@strctr}{1000}%
\ifnum\@strctr>99\relax
  \@tmpstrctr=\@strctr
  \divide\@tmpstrctr by 100\relax
  \ifnum#1>1000\relax
    \let\@@fc@ordstr#2\relax
    \edef#2{\@@fc@ordstr\ }%
  \fi
  \let\@@fc@ordstr#2\relax
  \edef#2{\@@fc@ordstr\@hundredthstring{\@tmpstrctr}}%
\fi
\@modulo{\@strctr}{100}%
\ifnum#1>99\relax
  \ifnum\@strctr>0\relax
    \let\@@fc@ordstr#2\relax
    \edef#2{\@@fc@ordstr\ }%
  \fi
\fi
\ifnum\@strctr>19\relax
  \@tmpstrctr=\@strctr
  \divide\@tmpstrctr by 10\relax
  \let\@@fc@ordstr#2\relax
  \edef#2{\@@fc@ordstr\@tenthstring{\@tmpstrctr}}%
  \@tmpstrctr=\@strctr
  \@modulo{\@tmpstrctr}{10}%
  \ifnum\@tmpstrctr>0\relax
    \let\@@fc@ordstr#2\relax
    \edef#2{\@@fc@ordstr\ \@unitthstring{\@tmpstrctr}}%
  \fi
\else
  \ifnum\@strctr>9\relax
    \@modulo{\@strctr}{10}%
    \let\@@fc@ordstr#2\relax
    \edef#2{\@@fc@ordstr\@teenthstring{\@strctr}}%
  \else
    \ifnum\@strctr=0\relax
      \ifnum#1=0\relax
        \let\@@fc@ordstr#2\relax
        \edef#2{\@@fc@ordstr\@unitstring{0}}%
      \fi
    \else
      \let\@@fc@ordstr#2\relax
      \edef#2{\@@fc@ordstr\@unitthstring{\@strctr}}%
    \fi
  \fi
\fi
\fi
\fi
}
%    \end{macrocode}
%\iffalse
%    \begin{macrocode}
%</fc-spanish.def>
%    \end{macrocode}
%\fi
%\iffalse
%    \begin{macrocode}
%<*fc-UKenglish.def>
%    \end{macrocode}
%\fi
% \subsection{fc-UKenglish.def}
% UK English definitions
%    \begin{macrocode}
\ProvidesFile{fc-UKenglish}[2007/06/14]
%    \end{macrocode}
% Check that fc-english.def has been loaded
%    \begin{macrocode}
\@ifundefined{@ordinalMenglish}{\input{fc-english.def}}{}
%    \end{macrocode}
% These are all just synonyms for the commands provided by
% fc-english.def.
%    \begin{macrocode}
\let\@ordinalMUKenglish\@ordinalMenglish
\let\@ordinalFUKenglish\@ordinalMenglish
\let\@ordinalNUKenglish\@ordinalMenglish
\let\@numberstringMUKenglish\@numberstringMenglish
\let\@numberstringFUKenglish\@numberstringMenglish
\let\@numberstringNUKenglish\@numberstringMenglish
\let\@NumberstringMUKenglish\@NumberstringMenglish
\let\@NumberstringFUKenglish\@NumberstringMenglish
\let\@NumberstringNUKenglish\@NumberstringMenglish
\let\@ordinalstringMUKenglish\@ordinalstringMenglish
\let\@ordinalstringFUKenglish\@ordinalstringMenglish
\let\@ordinalstringNUKenglish\@ordinalstringMenglish
\let\@OrdinalstringMUKenglish\@OrdinalstringMenglish
\let\@OrdinalstringFUKenglish\@OrdinalstringMenglish
\let\@OrdinalstringNUKenglish\@OrdinalstringMenglish
%    \end{macrocode}
%\iffalse
%    \begin{macrocode}
%</fc-UKenglish.def>
%    \end{macrocode}
%\fi
%\iffalse
%    \begin{macrocode}
%<*fc-USenglish.def>
%    \end{macrocode}
%\fi
% \subsection{fc-USenglish.def}
% US English definitions
%    \begin{macrocode}
\ProvidesFile{fc-USenglish}[2007/06/14]
%    \end{macrocode}
% Check that fc-english.def has been loaded
%    \begin{macrocode}
\@ifundefined{@ordinalMenglish}{\input{fc-english.def}}{}
%    \end{macrocode}
% These are all just synonyms for the commands provided by
% fc-english.def.
%    \begin{macrocode}
\let\@ordinalMUSenglish\@ordinalMenglish
\let\@ordinalFUSenglish\@ordinalMenglish
\let\@ordinalNUSenglish\@ordinalMenglish
\let\@numberstringMUSenglish\@numberstringMenglish
\let\@numberstringFUSenglish\@numberstringMenglish
\let\@numberstringNUSenglish\@numberstringMenglish
\let\@NumberstringMUSenglish\@NumberstringMenglish
\let\@NumberstringFUSenglish\@NumberstringMenglish
\let\@NumberstringNUSenglish\@NumberstringMenglish
\let\@ordinalstringMUSenglish\@ordinalstringMenglish
\let\@ordinalstringFUSenglish\@ordinalstringMenglish
\let\@ordinalstringNUSenglish\@ordinalstringMenglish
\let\@OrdinalstringMUSenglish\@OrdinalstringMenglish
\let\@OrdinalstringFUSenglish\@OrdinalstringMenglish
\let\@OrdinalstringNUSenglish\@OrdinalstringMenglish
%    \end{macrocode}
%\iffalse
%    \begin{macrocode}
%</fc-USenglish.def>
%    \end{macrocode}
%\fi
%\iffalse
%    \begin{macrocode}
%<*fmtcount.sty>
%    \end{macrocode}
%\fi
%\subsection{fmtcount.sty}
% This section deals with the code for |fmtcount.sty|
%    \begin{macrocode}
\NeedsTeXFormat{LaTeX2e}
\ProvidesPackage{fmtcount}[2007/06/22 v1.2]
\RequirePackage{ifthen}
\RequirePackage{keyval}
%    \end{macrocode}
% As from version 1.2, now load xspace package:
%    \begin{macrocode}
\RequirePackage{xspace}
%    \end{macrocode}
% These commands need to be defined before the
% configuration file is loaded.
%
% Define the macro to format the |st|, |nd|, |rd| or |th| of an 
% ordinal.
%    \begin{macrocode}
\providecommand{\fmtord}[1]{\textsuperscript{#1}}
%    \end{macrocode}
% Define |\padzeroes| to specify how many digits should be 
% displayed.
%    \begin{macrocode}
\newcount\c@padzeroesN
\c@padzeroesN=1\relax
\providecommand{\padzeroes}[1][17]{\c@padzeroesN=#1}
%    \end{macrocode}
% Load appropriate language definition files (I don't
% know if there is a standard way of detecting which
% languages are defined, so I'm just going to check
% if \verb"\date"\meta{language} is defined):
%\changes{v1.1}{14 June 2007}{added check for UKenglish,
% british and USenglish babel settings}
%    \begin{macrocode}
\@ifundefined{dateenglish}{}{\input{fc-english.def}}
\@ifundefined{l@UKenglish}{}{\input{fc-UKenglish.def}}
\@ifundefined{l@british}{}{\input{fc-british.def}}
\@ifundefined{l@USenglish}{}{\input{fc-USenglish.def}}
\@ifundefined{datespanish}{}{\input{fc-spanish.def}}
\@ifundefined{dateportuges}{}{\input{fc-portuges.def}}
\@ifundefined{datefrench}{}{\input{fc-french.def}}
\@ifundefined{dategerman}{%
\@ifundefined{datengerman}{}{\input{fc-german.def}}}{%
\input{fc-german.def}}
%    \end{macrocode}
% Define keys for use with |\fmtcountsetoptions|.
% Key to switch French dialects (Does babel store
%this kind of information?)
%    \begin{macrocode}
\def\fmtcount@french{france}
\define@key{fmtcount}{french}[france]{%
\@ifundefined{datefrench}{%
\PackageError{fmtcount}{Language `french' not defined}{You need
to load babel before loading fmtcount}}{
\ifthenelse{\equal{#1}{france}
         \or\equal{#1}{swiss}
         \or\equal{#1}{belgian}}{%
         \def\fmtcount@french{#1}}{%
\PackageError{fmtcount}{Invalid value `#1' to french key}
{Option `french' can only take the values `france', 
`belgian' or `swiss'}}
}}
%    \end{macrocode}
% Key to determine how to display the ordinal
%    \begin{macrocode}
\define@key{fmtcount}{fmtord}{%
\ifthenelse{\equal{#1}{level}
          \or\equal{#1}{raise}
          \or\equal{#1}{user}}{
          \def\fmtcount@fmtord{#1}}{%
\PackageError{fmtcount}{Invalid value `#1' to fmtord key}
{Option `fmtord' can only take the values `level', `raise' 
or `user'}}}
%    \end{macrocode}
% Key to determine whether the ordinal should be abbreviated
% (language dependent, currently only affects French ordinals.)
%    \begin{macrocode}
\newif\iffmtord@abbrv
\fmtord@abbrvfalse
\define@key{fmtcount}{abbrv}[true]{%
\ifthenelse{\equal{#1}{true}\or\equal{#1}{false}}{
          \csname fmtord@abbrv#1\endcsname}{%
\PackageError{fmtcount}{Invalid value `#1' to fmtord key}
{Option `fmtord' can only take the values `true' or
`false'}}}
%    \end{macrocode}
% Define command to set options.
%    \begin{macrocode}
\newcommand{\fmtcountsetoptions}[1]{%
\def\fmtcount@fmtord{}%
\setkeys{fmtcount}{#1}%
\@ifundefined{datefrench}{}{%
\edef\@ordinalstringMfrench{\noexpand
\csname @ordinalstringMfrench\fmtcount@french\noexpand\endcsname}%
\edef\@ordinalstringFfrench{\noexpand
\csname @ordinalstringFfrench\fmtcount@french\noexpand\endcsname}%
\edef\@OrdinalstringMfrench{\noexpand
\csname @OrdinalstringMfrench\fmtcount@french\noexpand\endcsname}%
\edef\@OrdinalstringFfrench{\noexpand
\csname @OrdinalstringFfrench\fmtcount@french\noexpand\endcsname}%
\edef\@numberstringMfrench{\noexpand
\csname @numberstringMfrench\fmtcount@french\noexpand\endcsname}%
\edef\@numberstringFfrench{\noexpand
\csname @numberstringFfrench\fmtcount@french\noexpand\endcsname}%
\edef\@NumberstringMfrench{\noexpand
\csname @NumberstringMfrench\fmtcount@french\noexpand\endcsname}%
\edef\@NumberstringFfrench{\noexpand
\csname @NumberstringFfrench\fmtcount@french\noexpand\endcsname}%
}%
%
\ifthenelse{\equal{\fmtcount@fmtord}{level}}{%
\renewcommand{\fmtord}[1]{##1}}{%
\ifthenelse{\equal{\fmtcount@fmtord}{raise}}{%
\renewcommand{\fmtord}[1]{\textsuperscript{##1}}}{%
}}
}
%    \end{macrocode}
% Load confguration file if it exists.  This needs to be done
% before the package options, to allow the user to override
% the settings in the configuration file.
%    \begin{macrocode}
\InputIfFileExists{fmtcount.cfg}{%
\typeout{Using configuration file fmtcount.cfg}}{%
\typeout{No configuration file fmtcount.cfg found.}}
%    \end{macrocode}
%Declare options
%    \begin{macrocode}
\DeclareOption{level}{\def\fmtcount@fmtord{level}%
\def\fmtord#1{#1}}
\DeclareOption{raise}{\def\fmtcount@fmtord{raise}%
\def\fmtord#1{\textsuperscript{#1}}}
%    \end{macrocode}
% Process package options
%    \begin{macrocode}
\ProcessOptions
%    \end{macrocode}
% Define macro that performs modulo arthmetic. This is used for the
% date, time, ordinal and numberstring commands. (The fmtcount
% package was originally part of the datetime package.)
%    \begin{macrocode}
\newcount\@DT@modctr
\def\@modulo#1#2{%
\@DT@modctr=#1\relax
\divide \@DT@modctr by #2\relax
\multiply \@DT@modctr by #2\relax
\advance #1 by -\@DT@modctr}
%    \end{macrocode}
% The following registers are needed by |\@ordinal| etc
%    \begin{macrocode}
\newcount\@ordinalctr
\newcount\@orgargctr
\newcount\@strctr
\newcount\@tmpstrctr
%    \end{macrocode}
%Define commands that display numbers in different bases.
% Define counters and conditionals needed.
%    \begin{macrocode}
\newif\if@DT@padzeroes
\newcount\@DT@loopN
\newcount\@DT@X
%    \end{macrocode}
% Binary
%    \begin{macrocode}
\newcommand{\@binary}[1]{%
\@DT@padzeroestrue
\@DT@loopN=17\relax
\@strctr=\@DT@loopN
\whiledo{\@strctr<\c@padzeroesN}{0\advance\@strctr by 1}%
\@strctr=65536\relax
\@DT@X=#1\relax
\loop
\@DT@modctr=\@DT@X
\divide\@DT@modctr by \@strctr
\ifthenelse{\boolean{@DT@padzeroes} \and \(\@DT@modctr=0\) \and \(\@DT@loopN>\c@padzeroesN\)}{}{\the\@DT@modctr}%
\ifnum\@DT@modctr=0\else\@DT@padzeroesfalse\fi
\multiply\@DT@modctr by \@strctr
\advance\@DT@X by -\@DT@modctr
\divide\@strctr by 2\relax
\advance\@DT@loopN by -1\relax
\ifnum\@strctr>1
\repeat
\the\@DT@X}

\let\binarynum=\@binary
%    \end{macrocode}
% Octal
%    \begin{macrocode}
\newcommand{\@octal}[1]{%
\ifnum#1>32768
\PackageError{fmtcount}{Value of counter too large for \protect\@octal}{Maximum value 32768}
\else
\@DT@padzeroestrue
\@DT@loopN=6\relax
\@strctr=\@DT@loopN
\whiledo{\@strctr<\c@padzeroesN}{0\advance\@strctr by 1}%
\@strctr=32768\relax
\@DT@X=#1\relax
\loop
\@DT@modctr=\@DT@X
\divide\@DT@modctr by \@strctr
\ifthenelse{\boolean{@DT@padzeroes} \and \(\@DT@modctr=0\) \and \(\@DT@loopN>\c@padzeroesN\)}{}{\the\@DT@modctr}%
\ifnum\@DT@modctr=0\else\@DT@padzeroesfalse\fi
\multiply\@DT@modctr by \@strctr
\advance\@DT@X by -\@DT@modctr
\divide\@strctr by 8\relax
\advance\@DT@loopN by -1\relax
\ifnum\@strctr>1
\repeat
\the\@DT@X
\fi}
\let\octalnum=\@octal
%    \end{macrocode}
% Lowercase hexadecimal
%    \begin{macrocode}
\newcommand{\@@hexadecimal}[1]{\ifcase#10\or1\or2\or3\or4\or5\or6\or7\or8\or9\or a\or b\or c\or d\or e\or f\fi}

\newcommand{\@hexadecimal}[1]{%
\@DT@padzeroestrue
\@DT@loopN=5\relax
\@strctr=\@DT@loopN
\whiledo{\@strctr<\c@padzeroesN}{0\advance\@strctr by 1}%
\@strctr=65536\relax
\@DT@X=#1\relax
\loop
\@DT@modctr=\@DT@X
\divide\@DT@modctr by \@strctr
\ifthenelse{\boolean{@DT@padzeroes} \and \(\@DT@modctr=0\) \and \(\@DT@loopN>\c@padzeroesN\)}{}{\@@hexadecimal\@DT@modctr}%
\ifnum\@DT@modctr=0\else\@DT@padzeroesfalse\fi
\multiply\@DT@modctr by \@strctr
\advance\@DT@X by -\@DT@modctr
\divide\@strctr by 16\relax
\advance\@DT@loopN by -1\relax
\ifnum\@strctr>1
\repeat
\@@hexadecimal\@DT@X}

\let\hexadecimalnum=\@hexadecimal
%    \end{macrocode}
% Uppercase hexadecimal
%    \begin{macrocode}
\newcommand{\@@Hexadecimal}[1]{\ifcase#10\or1\or2\or3\or4\or5\or6\or
7\or8\or9\or A\or B\or C\or D\or E\or F\fi}

\newcommand{\@Hexadecimal}[1]{%
\@DT@padzeroestrue
\@DT@loopN=5\relax
\@strctr=\@DT@loopN
\whiledo{\@strctr<\c@padzeroesN}{0\advance\@strctr by 1}%
\@strctr=65536\relax
\@DT@X=#1\relax
\loop
\@DT@modctr=\@DT@X
\divide\@DT@modctr by \@strctr
\ifthenelse{\boolean{@DT@padzeroes} \and \(\@DT@modctr=0\) \and \(\@DT@loopN>\c@padzeroesN\)}{}{\@@Hexadecimal\@DT@modctr}%
\ifnum\@DT@modctr=0\else\@DT@padzeroesfalse\fi
\multiply\@DT@modctr by \@strctr
\advance\@DT@X by -\@DT@modctr
\divide\@strctr by 16\relax
\advance\@DT@loopN by -1\relax
\ifnum\@strctr>1
\repeat
\@@Hexadecimal\@DT@X}

\let\Hexadecimalnum=\@Hexadecimal
%    \end{macrocode}
% Uppercase alphabetical representation (a \ldots\ z aa \ldots\ zz)
%    \begin{macrocode}
\newcommand{\@aaalph}[1]{%
\@DT@loopN=#1\relax
\advance\@DT@loopN by -1\relax
\divide\@DT@loopN by 26\relax
\@DT@modctr=\@DT@loopN
\multiply\@DT@modctr by 26\relax
\@DT@X=#1\relax
\advance\@DT@X by -1\relax
\advance\@DT@X by -\@DT@modctr
\advance\@DT@loopN by 1\relax
\advance\@DT@X by 1\relax
\loop
\@alph\@DT@X
\advance\@DT@loopN by -1\relax
\ifnum\@DT@loopN>0
\repeat
}

\let\aaalphnum=\@aaalph
%    \end{macrocode}
% Uppercase alphabetical representation (a \ldots\ z aa \ldots\ zz)
%    \begin{macrocode}
\newcommand{\@AAAlph}[1]{%
\@DT@loopN=#1\relax
\advance\@DT@loopN by -1\relax
\divide\@DT@loopN by 26\relax
\@DT@modctr=\@DT@loopN
\multiply\@DT@modctr by 26\relax
\@DT@X=#1\relax
\advance\@DT@X by -1\relax
\advance\@DT@X by -\@DT@modctr
\advance\@DT@loopN by 1\relax
\advance\@DT@X by 1\relax
\loop
\@Alph\@DT@X
\advance\@DT@loopN by -1\relax
\ifnum\@DT@loopN>0
\repeat
}

\let\AAAlphnum=\@AAAlph
%    \end{macrocode}
% Lowercase alphabetical representation
%    \begin{macrocode}
\newcommand{\@abalph}[1]{%
\ifnum#1>17576
\PackageError{fmtcount}{Value of counter too large for \protect\@abalph}{Maximum value 17576}
\else
\@DT@padzeroestrue
\@strctr=17576\relax
\@DT@X=#1\relax
\advance\@DT@X by -1\relax
\loop
\@DT@modctr=\@DT@X
\divide\@DT@modctr by \@strctr
\ifthenelse{\boolean{@DT@padzeroes} \and \(\@DT@modctr=1\)}{}{\@alph\@DT@modctr}%
\ifnum\@DT@modctr=1\else\@DT@padzeroesfalse\fi
\multiply\@DT@modctr by \@strctr
\advance\@DT@X by -\@DT@modctr
\divide\@strctr by 26\relax
\ifnum\@strctr>1
\repeat
\advance\@DT@X by 1\relax
\@alph\@DT@X
\fi}

\let\abalphnum=\@abalph
%    \end{macrocode}
% Uppercase alphabetical representation
%    \begin{macrocode}
\newcommand{\@ABAlph}[1]{%
\ifnum#1>17576
\PackageError{fmtcount}{Value of counter too large for \protect\@ABAlph}{Maximum value 17576}
\else
\@DT@padzeroestrue
\@strctr=17576\relax
\@DT@X=#1\relax
\advance\@DT@X by -1\relax
\loop
\@DT@modctr=\@DT@X
\divide\@DT@modctr by \@strctr
\ifthenelse{\boolean{@DT@padzeroes} \and \(\@DT@modctr=1\)}{}{\@Alph\@DT@modctr}%
\ifnum\@DT@modctr=1\else\@DT@padzeroesfalse\fi
\multiply\@DT@modctr by \@strctr
\advance\@DT@X by -\@DT@modctr
\divide\@strctr by 26\relax
\ifnum\@strctr>1
\repeat
\advance\@DT@X by 1\relax
\@Alph\@DT@X
\fi}

\let\ABAlphnum=\@ABAlph
%    \end{macrocode}
% Recursive command to count number of characters in argument.
% |\@strctr| should be set to zero before calling it.
%    \begin{macrocode}
\def\@fmtc@count#1#2\relax{%
\if\relax#1
\else
\advance\@strctr by 1\relax
\@fmtc@count#2\relax
\fi}
%    \end{macrocode}
% Internal decimal macro:
%    \begin{macrocode}
\newcommand{\@decimal}[1]{%
\@strctr=0\relax
\expandafter\@fmtc@count\number#1\relax
\@DT@loopN=\c@padzeroesN
\advance\@DT@loopN by -\@strctr
\ifnum\@DT@loopN>0\relax
\@strctr=0\relax
\whiledo{\@strctr < \@DT@loopN}{0\advance\@strctr by 1}%
\fi
\number#1\relax
}

\let\decimalnum=\@decimal
%    \end{macrocode}
% This is a bit cumbersome.  Previously \verb"\@ordinal"
% was defined in a similar way to \verb"\abalph" etc.
% This ensured that the actual value of the counter was
% written in the new label stuff in the .aux file. However
% adding in an optional argument to determine the gender
% for multilingual compatibility messed things up somewhat.
% This was the only work around I could get to keep the
% the cross-referencing stuff working, which is why
% the optional argument comes \emph{after} the compulsory
% argument, instead of the usual manner of placing it before.
% Version 1.04 changed \verb"\ordinal" to \verb"\FCordinal"
% to prevent it clashing with the memoir class. 
%    \begin{macrocode}
\newcommand{\FCordinal}[1]{%
\expandafter\protect\expandafter\ordinalnum{%
\expandafter\the\csname c@#1\endcsname}}
%    \end{macrocode}
% If \verb"\ordinal" isn't defined make \verb"\ordinal" a synonym
% for \verb"\FCordinal" to maintain compatibility with previous
% versions.
%    \begin{macrocode}
\@ifundefined{ordinal}{\let\ordinal\FCordinal}{%
\PackageWarning{fmtcount}{\string\ordinal
\space already defined use \string\FCordinal \space instead.}}
%    \end{macrocode}
% Display ordinal where value is given as a number or 
% count register instead of a counter:
%    \begin{macrocode}
\newcommand{\ordinalnum}[1]{\@ifnextchar[{\@ordinalnum{#1}}{%
\@ordinalnum{#1}[m]}}
%    \end{macrocode}
% Display ordinal according to gender (neuter added in v1.1,
% \cmdname{xspace} added in v1.2):
%    \begin{macrocode}
\def\@ordinalnum#1[#2]{{%
\ifthenelse{\equal{#2}{f}}{%
\protect\@ordinalF{#1}{\@fc@ordstr}}{%
\ifthenelse{\equal{#2}{n}}{%
\protect\@ordinalN{#1}{\@fc@ordstr}}{%
\ifthenelse{\equal{#2}{m}}{}{%
\PackageError{fmtcount}{Invalid gender option `#2'}{%
Available options are m, f or n}}%
\protect\@ordinalM{#1}{\@fc@ordstr}}}\@fc@ordstr}\xspace}
%    \end{macrocode}
% Store the ordinal (first argument
% is identifying name, second argument is a counter.)
%    \begin{macrocode}
\newcommand*{\storeordinal}[2]{%
\expandafter\protect\expandafter\storeordinalnum{#1}{%
\expandafter\the\csname c@#2\endcsname}}
%    \end{macrocode}
% Store ordinal (first argument
% is identifying name, second argument is a number or
% count register.)
%    \begin{macrocode}
\newcommand*{\storeordinalnum}[2]{%
\@ifnextchar[{\@storeordinalnum{#1}{#2}}{%
\@storeordinalnum{#1}{#2}[m]}}
%    \end{macrocode}
% Store ordinal according to gender:
%    \begin{macrocode}
\def\@storeordinalnum#1#2[#3]{%
\ifthenelse{\equal{#3}{f}}{%
\protect\@ordinalF{#2}{\@fc@ord}}{%
\ifthenelse{\equal{#3}{n}}{%
\protect\@ordinalN{#2}{\@fc@ord}}{%
\ifthenelse{\equal{#3}{m}}{}{%
\PackageError{fmtcount}{Invalid gender option `#3'}{%
Available options are m or f}}%
\protect\@ordinalM{#2}{\@fc@ord}}}%
\expandafter\let\csname @fcs@#1\endcsname\@fc@ord}
%    \end{macrocode}
% Get stored information:
%    \begin{macrocode}
\newcommand*{\FMCuse}[1]{\csname @fcs@#1\endcsname}
%    \end{macrocode}
% Display ordinal as a string (argument is a counter)
%    \begin{macrocode}
\newcommand{\ordinalstring}[1]{%
\expandafter\protect\expandafter\ordinalstringnum{%
\expandafter\the\csname c@#1\endcsname}}
%    \end{macrocode}
% Display ordinal as a string (argument is a count register or
% number.)
%    \begin{macrocode}
\newcommand{\ordinalstringnum}[1]{%
\@ifnextchar[{\@ordinal@string{#1}}{\@ordinal@string{#1}[m]}}
%    \end{macrocode}
% Display ordinal as a string according to gender (\cmdname{xspace}
% added in version 1.2).
%    \begin{macrocode}
\def\@ordinal@string#1[#2]{{%
\ifthenelse{\equal{#2}{f}}{%
\protect\@ordinalstringF{#1}{\@fc@ordstr}}{%
\ifthenelse{\equal{#2}{n}}{%
\protect\@ordinalstringN{#1}{\@fc@ordstr}}{%
\ifthenelse{\equal{#2}{m}}{}{%
\PackageError{fmtcount}{Invalid gender option `#2' to 
\string\ordinalstring}{Available options are m, f or f}}%
\protect\@ordinalstringM{#1}{\@fc@ordstr}}}\@fc@ordstr}\xspace}
%    \end{macrocode}
% Store textual representation of number. First argument is 
% identifying name, second argument is the counter set to the 
% required number.
%    \begin{macrocode}
\newcommand{\storeordinalstring}[2]{%
\expandafter\protect\expandafter\storeordinalstringnum{#1}{%
\expandafter\the\csname c@#2\endcsname}}
%    \end{macrocode}
% Store textual representation of number. First argument is 
% identifying name, second argument is a count register or number.
%    \begin{macrocode}
\newcommand{\storeordinalstringnum}[2]{%
\@ifnextchar[{\@store@ordinal@string{#1}{#2}}{%
\@store@ordinal@string{#1}{#2}[m]}}
%    \end{macrocode}
% Store textual representation of number according to gender.
%    \begin{macrocode}
\def\@store@ordinal@string#1#2[#3]{%
\ifthenelse{\equal{#3}{f}}{%
\protect\@ordinalstringF{#2}{\@fc@ordstr}}{%
\ifthenelse{\equal{#3}{n}}{%
\protect\@ordinalstringN{#2}{\@fc@ordstr}}{%
\ifthenelse{\equal{#3}{m}}{}{%
\PackageError{fmtcount}{Invalid gender option `#3' to 
\string\ordinalstring}{Available options are m, f or n}}%
\protect\@ordinalstringM{#2}{\@fc@ordstr}}}%
\expandafter\let\csname @fcs@#1\endcsname\@fc@ordstr}
%    \end{macrocode}
% Display ordinal as a string with initial letters in upper case
% (argument is a counter)
%    \begin{macrocode}
\newcommand{\Ordinalstring}[1]{%
\expandafter\protect\expandafter\Ordinalstringnum{%
\expandafter\the\csname c@#1\endcsname}}
%    \end{macrocode}
% Display ordinal as a string with initial letters in upper case
% (argument is a number or count register)
%    \begin{macrocode}
\newcommand{\Ordinalstringnum}[1]{%
\@ifnextchar[{\@Ordinal@string{#1}}{\@Ordinal@string{#1}[m]}}
%    \end{macrocode}
% Display ordinal as a string with initial letters in upper case
% according to gender
%    \begin{macrocode}
\def\@Ordinal@string#1[#2]{{%
\ifthenelse{\equal{#2}{f}}{%
\protect\@OrdinalstringF{#1}{\@fc@ordstr}}{%
\ifthenelse{\equal{#2}{n}}{%
\protect\@OrdinalstringN{#1}{\@fc@ordstr}}{%
\ifthenelse{\equal{#2}{m}}{}{%
\PackageError{fmtcount}{Invalid gender option `#2'}{%
Available options are m, f or n}}%
\protect\@OrdinalstringM{#1}{\@fc@ordstr}}}\@fc@ordstr}\xspace}
%    \end{macrocode}
% Store textual representation of number, with initial letters in 
% upper case. First argument is identifying name, second argument 
% is the counter set to the 
% required number.
%    \begin{macrocode}
\newcommand{\storeOrdinalstring}[2]{%
\expandafter\protect\expandafter\storeOrdinalstringnum{#1}{%
\expandafter\the\csname c@#2\endcsname}}
%    \end{macrocode}
% Store textual representation of number, with initial letters in 
% upper case. First argument is identifying name, second argument 
% is a count register or number.
%    \begin{macrocode}
\newcommand{\storeOrdinalstringnum}[2]{%
\@ifnextchar[{\@store@Ordinal@string{#1}{#2}}{%
\@store@Ordinal@string{#1}{#2}[m]}}
%    \end{macrocode}
% Store textual representation of number according to gender, 
% with initial letters in upper case.
%    \begin{macrocode}
\def\@store@Ordinal@string#1#2[#3]{%
\ifthenelse{\equal{#3}{f}}{%
\protect\@OrdinalstringF{#2}{\@fc@ordstr}}{%
\ifthenelse{\equal{#3}{n}}{%
\protect\@OrdinalstringN{#2}{\@fc@ordstr}}{%
\ifthenelse{\equal{#3}{m}}{}{%
\PackageError{fmtcount}{Invalid gender option `#3'}{%
Available options are m or f}}%
\protect\@OrdinalstringM{#2}{\@fc@ordstr}}}%
\expandafter\let\csname @fcs@#1\endcsname\@fc@ordstr}
%    \end{macrocode}
% Store upper case textual representation of ordinal. The first 
% argument is identifying name, the second argument is a counter.
%    \begin{macrocode}
\newcommand{\storeORDINALstring}[2]{%
\expandafter\protect\expandafter\storeORDINALstringnum{#1}{%
\expandafter\the\csname c@#2\endcsname}}
%    \end{macrocode}
% As above, but the second argument is a count register or a
% number.
%    \begin{macrocode}
\newcommand{\storeORDINALstringnum}[2]{%
\@ifnextchar[{\@store@ORDINAL@string{#1}{#2}}{%
\@store@ORDINAL@string{#1}{#2}[m]}}
%    \end{macrocode}
% Gender is specified as an optional argument at the end.
%    \begin{macrocode}
\def\@store@ORDINAL@string#1#2[#3]{%
\ifthenelse{\equal{#3}{f}}{%
\protect\@ordinalstringF{#2}{\@fc@ordstr}}{%
\ifthenelse{\equal{#3}{n}}{%
\protect\@ordinalstringN{#2}{\@fc@ordstr}}{%
\ifthenelse{\equal{#3}{m}}{}{%
\PackageError{fmtcount}{Invalid gender option `#3'}{%
Available options are m or f}}%
\protect\@ordinalstringM{#2}{\@fc@ordstr}}}%
\expandafter\edef\csname @fcs@#1\endcsname{%
\noexpand\MakeUppercase{\@fc@ordstr}}}
%    \end{macrocode}
% Display upper case textual representation of an ordinal. The
% argument must be a counter.
%    \begin{macrocode}
\newcommand{\ORDINALstring}[1]{%
\expandafter\protect\expandafter\ORDINALstringnum{%
\expandafter\the\csname c@#1\endcsname}}
%    \end{macrocode}
% As above, but the argument is a count register or a number.
%    \begin{macrocode}
\newcommand{\ORDINALstringnum}[1]{%
\@ifnextchar[{\@ORDINAL@string{#1}}{\@ORDINAL@string{#1}[m]}}
%    \end{macrocode}
% Gender is specified as an optional argument at the end.
%    \begin{macrocode}
\def\@ORDINAL@string#1[#2]{{%
\ifthenelse{\equal{#2}{f}}{%
\protect\@ordinalstringF{#1}{\@fc@ordstr}}{%
\ifthenelse{\equal{#2}{n}}{%
\protect\@ordinalstringN{#1}{\@fc@ordstr}}{%
\ifthenelse{\equal{#2}{m}}{}{%
\PackageError{fmtcount}{Invalid gender option `#2'}{%
Available options are m, f or n}}%
\protect\@ordinalstringM{#1}{\@fc@ordstr}}}%
\MakeUppercase{\@fc@ordstr}}\xspace}
%    \end{macrocode}
% Convert number to textual respresentation, and store. First 
% argument is the identifying name, second argument is a counter 
% containing the number.
%    \begin{macrocode}
\newcommand{\storenumberstring}[2]{%
\expandafter\protect\expandafter\storenumberstringnum{#1}{%
\expandafter\the\csname c@#2\endcsname}}
%    \end{macrocode}
% As above, but second argument is a number or count register.
%    \begin{macrocode}
\newcommand{\storenumberstringnum}[2]{%
\@ifnextchar[{\@store@number@string{#1}{#2}}{%
\@store@number@string{#1}{#2}[m]}}
%    \end{macrocode}
% Gender is given as optional argument, \emph{at the end}.
%    \begin{macrocode}
\def\@store@number@string#1#2[#3]{%
\ifthenelse{\equal{#3}{f}}{%
\protect\@numberstringF{#2}{\@fc@numstr}}{%
\ifthenelse{\equal{#3}{n}}{%
\protect\@numberstringN{#2}{\@fc@numstr}}{%
\ifthenelse{\equal{#3}{m}}{}{%
\PackageError{fmtcount}{Invalid gender option `#3'}{%
Available options are m, f or n}}%
\protect\@numberstringM{#2}{\@fc@numstr}}}%
\expandafter\let\csname @fcs@#1\endcsname\@fc@numstr}
%    \end{macrocode}
% Display textual representation of a number. The argument
% must be a counter.
%    \begin{macrocode}
\newcommand{\numberstring}[1]{%
\expandafter\protect\expandafter\numberstringnum{%
\expandafter\the\csname c@#1\endcsname}}
%    \end{macrocode}
% As above, but the argument is a count register or a number.
%    \begin{macrocode}
\newcommand{\numberstringnum}[1]{%
\@ifnextchar[{\@number@string{#1}}{\@number@string{#1}[m]}}
%    \end{macrocode}
% Gender is specified as an optional argument \emph{at the end}.
%    \begin{macrocode}
\def\@number@string#1[#2]{{%
\ifthenelse{\equal{#2}{f}}{%
\protect\@numberstringF{#1}{\@fc@numstr}}{%
\ifthenelse{\equal{#2}{n}}{%
\protect\@numberstringN{#1}{\@fc@numstr}}{%
\ifthenelse{\equal{#2}{m}}{}{%
\PackageError{fmtcount}{Invalid gender option `#2'}{%
Available options are m, f or n}}%
\protect\@numberstringM{#1}{\@fc@numstr}}}\@fc@numstr}\xspace}
%    \end{macrocode}
% Store textual representation of number. First argument is 
% identifying name, second argument is a counter.
%    \begin{macrocode}
\newcommand{\storeNumberstring}[2]{%
\expandafter\protect\expandafter\storeNumberstringnum{#1}{%
\expandafter\the\csname c@#2\endcsname}}
%    \end{macrocode}
% As above, but second argument is a count register or number.
%    \begin{macrocode}
\newcommand{\storeNumberstringnum}[2]{%
\@ifnextchar[{\@store@Number@string{#1}{#2}}{%
\@store@Number@string{#1}{#2}[m]}}
%    \end{macrocode}
% Gender is specified as an optional argument \emph{at the end}:
%    \begin{macrocode}
\def\@store@Number@string#1#2[#3]{%
\ifthenelse{\equal{#3}{f}}{%
\protect\@NumberstringF{#2}{\@fc@numstr}}{%
\ifthenelse{\equal{#3}{n}}{%
\protect\@NumberstringN{#2}{\@fc@numstr}}{%
\ifthenelse{\equal{#3}{m}}{}{%
\PackageError{fmtcount}{Invalid gender option `#3'}{%
Available options are m, f or n}}%
\protect\@NumberstringM{#2}{\@fc@numstr}}}%
\expandafter\let\csname @fcs@#1\endcsname\@fc@numstr}
%    \end{macrocode}
% Display textual representation of number. The argument must be
% a counter. 
%    \begin{macrocode}
\newcommand{\Numberstring}[1]{%
\expandafter\protect\expandafter\Numberstringnum{%
\expandafter\the\csname c@#1\endcsname}}
%    \end{macrocode}
% As above, but the argument is a count register or number.
%    \begin{macrocode}
\newcommand{\Numberstringnum}[1]{%
\@ifnextchar[{\@Number@string{#1}}{\@Number@string{#1}[m]}}
%    \end{macrocode}
% Gender is specified as an optional argument at the end.
%    \begin{macrocode}
\def\@Number@string#1[#2]{{%
\ifthenelse{\equal{#2}{f}}{%
\protect\@NumberstringF{#1}{\@fc@numstr}}{%
\ifthenelse{\equal{#2}{n}}{%
\protect\@NumberstringN{#1}{\@fc@numstr}}{%
\ifthenelse{\equal{#2}{m}}{}{%
\PackageError{fmtcount}{Invalid gender option `#2'}{%
Available options are m, f or n}}%
\protect\@NumberstringM{#1}{\@fc@numstr}}}\@fc@numstr}\xspace}
%    \end{macrocode}
% Store upper case textual representation of number. The first 
% argument is identifying name, the second argument is a counter.
%    \begin{macrocode}
\newcommand{\storeNUMBERstring}[2]{%
\expandafter\protect\expandafter\storeNUMBERstringnum{#1}{%
\expandafter\the\csname c@#2\endcsname}}
%    \end{macrocode}
% As above, but the second argument is a count register or a
% number.
%    \begin{macrocode}
\newcommand{\storeNUMBERstringnum}[2]{%
\@ifnextchar[{\@store@NUMBER@string{#1}{#2}}{%
\@store@NUMBER@string{#1}{#2}[m]}}
%    \end{macrocode}
% Gender is specified as an optional argument at the end.
%    \begin{macrocode}
\def\@store@NUMBER@string#1#2[#3]{%
\ifthenelse{\equal{#3}{f}}{%
\protect\@numberstringF{#2}{\@fc@numstr}}{%
\ifthenelse{\equal{#3}{n}}{%
\protect\@numberstringN{#2}{\@fc@numstr}}{%
\ifthenelse{\equal{#3}{m}}{}{%
\PackageError{fmtcount}{Invalid gender option `#3'}{%
Available options are m or f}}%
\protect\@numberstringM{#2}{\@fc@numstr}}}%
\expandafter\edef\csname @fcs@#1\endcsname{%
\noexpand\MakeUppercase{\@fc@numstr}}}
%    \end{macrocode}
% Display upper case textual representation of a number. The
% argument must be a counter.
%    \begin{macrocode}
\newcommand{\NUMBERstring}[1]{%
\expandafter\protect\expandafter\NUMBERstringnum{%
\expandafter\the\csname c@#1\endcsname}}
%    \end{macrocode}
% As above, but the argument is a count register or a number.
%    \begin{macrocode}
\newcommand{\NUMBERstringnum}[1]{%
\@ifnextchar[{\@NUMBER@string{#1}}{\@NUMBER@string{#1}[m]}}
%    \end{macrocode}
% Gender is specified as an optional argument at the end.
%    \begin{macrocode}
\def\@NUMBER@string#1[#2]{{%
\ifthenelse{\equal{#2}{f}}{%
\protect\@numberstringF{#1}{\@fc@numstr}}{%
\ifthenelse{\equal{#2}{n}}{%
\protect\@numberstringN{#1}{\@fc@numstr}}{%
\ifthenelse{\equal{#2}{m}}{}{%
\PackageError{fmtcount}{Invalid gender option `#2'}{%
Available options are m, f or n}}%
\protect\@numberstringM{#1}{\@fc@numstr}}}%
\MakeUppercase{\@fc@numstr}}\xspace}
%    \end{macrocode}
% Number representations in other bases. Binary:
%    \begin{macrocode}
\providecommand{\binary}[1]{%
\expandafter\protect\expandafter\@binary{%
\expandafter\the\csname c@#1\endcsname}}
%    \end{macrocode}
% Like \verb"\alph", but goes beyond 26. (a \ldots\ z aa \ldots zz \ldots)
%    \begin{macrocode}
\providecommand{\aaalph}[1]{%
\expandafter\protect\expandafter\@aaalph{%
\expandafter\the\csname c@#1\endcsname}}
%    \end{macrocode}
% As before, but upper case.
%    \begin{macrocode}
\providecommand{\AAAlph}[1]{%
\expandafter\protect\expandafter\@AAAlph{%
\expandafter\the\csname c@#1\endcsname}}
%    \end{macrocode}
% Like \verb"\alph", but goes beyond 26. (a \ldots\ z ab \ldots az \ldots)
%    \begin{macrocode}
\providecommand{\abalph}[1]{%
\expandafter\protect\expandafter\@abalph{%
\expandafter\the\csname c@#1\endcsname}}
%    \end{macrocode}
% As above, but upper case.
%    \begin{macrocode}
\providecommand{\ABAlph}[1]{%
\expandafter\protect\expandafter\@ABAlph{%
\expandafter\the\csname c@#1\endcsname}}
%    \end{macrocode}
% Hexadecimal:
%    \begin{macrocode}
\providecommand{\hexadecimal}[1]{%
\expandafter\protect\expandafter\@hexadecimal{%
\expandafter\the\csname c@#1\endcsname}}
%    \end{macrocode}
% As above, but in upper case.
%    \begin{macrocode}
\providecommand{\Hexadecimal}[1]{%
\expandafter\protect\expandafter\@Hexadecimal{%
\expandafter\the\csname c@#1\endcsname}}
%    \end{macrocode}
% Octal:
%    \begin{macrocode}
\providecommand{\octal}[1]{%
\expandafter\protect\expandafter\@octal{%
\expandafter\the\csname c@#1\endcsname}}
%    \end{macrocode}
% Decimal:
%    \begin{macrocode}
\providecommand{\decimal}[1]{%
\expandafter\protect\expandafter\@decimal{%
\expandafter\the\csname c@#1\endcsname}}
%    \end{macrocode}
%\subsubsection{Multilinguage Definitions}
% If multilingual support is provided, make \verb"\@numberstring" 
% etc use the correct language (if defined).
% Otherwise use English definitions. "\@setdef@ultfmtcount"
% sets the macros to use English.
%    \begin{macrocode}
\def\@setdef@ultfmtcount{
\@ifundefined{@ordinalMenglish}{\input{fc-english.def}}{}
\def\@ordinalstringM{\@ordinalstringMenglish}
\let\@ordinalstringF=\@ordinalstringMenglish
\let\@ordinalstringN=\@ordinalstringMenglish
\def\@OrdinalstringM{\@OrdinalstringMenglish}
\let\@OrdinalstringF=\@OrdinalstringMenglish
\let\@OrdinalstringN=\@OrdinalstringMenglish
\def\@numberstringM{\@numberstringMenglish}
\let\@numberstringF=\@numberstringMenglish
\let\@numberstringN=\@numberstringMenglish
\def\@NumberstringM{\@NumberstringMenglish}
\let\@NumberstringF=\@NumberstringMenglish
\let\@NumberstringN=\@NumberstringMenglish
\def\@ordinalM{\@ordinalMenglish}
\let\@ordinalF=\@ordinalM
\let\@ordinalN=\@ordinalM
}
%    \end{macrocode}
% Define a command to set macros to use "languagename":
%    \begin{macrocode}
\def\@set@mulitling@fmtcount{%
%
\def\@numberstringM{\@ifundefined{@numberstringM\languagename}{%
\PackageError{fmtcount}{No support for language '\languagename'}{%
The fmtcount package currently does not support language 
'\languagename' for command \string\@numberstringM}}{%
\csname @numberstringM\languagename\endcsname}}%
%
\def\@numberstringF{\@ifundefined{@numberstringF\languagename}{%
\PackageError{fmtcount}{No support for language '\languagename'}{%
The fmtcount package currently does not support language 
'\languagename' for command \string\@numberstringF}}{%
\csname @numberstringF\languagename\endcsname}}%
%
\def\@numberstringN{\@ifundefined{@numberstringN\languagename}{%
\PackageError{fmtcount}{No support for language '\languagename'}{%
The fmtcount package currently does not support language 
'\languagename' for command \string\@numberstringN}}{%
\csname @numberstringN\languagename\endcsname}}%
%
\def\@NumberstringM{\@ifundefined{@NumberstringM\languagename}{%
\PackageError{fmtcount}{No support for language '\languagename'}{%
The fmtcount package currently does not support language 
'\languagename' for command \string\@NumberstringM}}{%
\csname @NumberstringM\languagename\endcsname}}%
%
\def\@NumberstringF{\@ifundefined{@NumberstringF\languagename}{%
\PackageError{fmtcount}{No support for language '\languagename'}{%
The fmtcount package currently does not support language 
'\languagename' for command \string\@NumberstringF}}{%
\csname @NumberstringF\languagename\endcsname}}%
%
\def\@NumberstringN{\@ifundefined{@NumberstringN\languagename}{%
\PackageError{fmtcount}{No support for language '\languagename'}{%
The fmtcount package currently does not support language 
'\languagename' for command \string\@NumberstringN}}{%
\csname @NumberstringN\languagename\endcsname}}%
%
\def\@ordinalM{\@ifundefined{@ordinalM\languagename}{%
\PackageError{fmtcount}{No support for language '\languagename'}{%
The fmtcount package currently does not support language 
'\languagename' for command \string\@ordinalM}}{%
\csname @ordinalM\languagename\endcsname}}%
%
\def\@ordinalF{\@ifundefined{@ordinalF\languagename}{%
\PackageError{fmtcount}{No support for language '\languagename'}{%
The fmtcount package currently does not support language 
'\languagename' for command \string\@ordinalF}}{%
\csname @ordinalF\languagename\endcsname}}%
%
\def\@ordinalN{\@ifundefined{@ordinalN\languagename}{%
\PackageError{fmtcount}{No support for language '\languagename'}{%
The fmtcount package currently does not support language 
'\languagename' for command \string\@ordinalN}}{%
\csname @ordinalN\languagename\endcsname}}%
%
\def\@ordinalstringM{\@ifundefined{@ordinalstringM\languagename}{%
\PackageError{fmtcount}{No support for language '\languagename'}{%
The fmtcount package currently does not support language 
'\languagename' for command \string\@ordinalstringM}}{%
\csname @ordinalstringM\languagename\endcsname}}%
%
\def\@ordinalstringF{\@ifundefined{@ordinalstringF\languagename}{%
\PackageError{fmtcount}{No support for language '\languagename'}{%
The fmtcount package currently does not support language 
'\languagename' for command \string\@ordinalstringF}}{%
\csname @ordinalstringF\languagename\endcsname}}%
%
\def\@ordinalstringN{\@ifundefined{@ordinalstringN\languagename}{%
\PackageError{fmtcount}{No support for language '\languagename'}{%
The fmtcount package currently does not support language 
'\languagename' for command \string\@ordinalstringN}}{%
\csname @ordinalstringN\languagename\endcsname}}%
%
\def\@OrdinalstringM{\@ifundefined{@OrdinalstringM\languagename}{%
\PackageError{fmtcount}{No support for language '\languagename'}{%
The fmtcount package currently does not support language 
'\languagename' for command \string\@OrdinalstringM}}{%
\csname @OrdinalstringM\languagename\endcsname}}%
%
\def\@OrdinalstringF{\@ifundefined{@OrdinalstringF\languagename}{%
\PackageError{fmtcount}{No support for language '\languagename'}{%
The fmtcount package currently does not support language 
'\languagename' for command \string\@OrdinalstringF}}{%
\csname @OrdinalstringF\languagename\endcsname}}%
%
\def\@OrdinalstringN{\@ifundefined{@OrdinalstringN\languagename}{%
\PackageError{fmtcount}{No support for language '\languagename'}{%
The fmtcount package currently does not support language 
'\languagename' for command \string\@OrdinalstringN}}{%
\csname @OrdinalstringN\languagename\endcsname}}
}
%    \end{macrocode}
% Check to see if babel or ngerman packages have been loaded.
%    \begin{macrocode}
\@ifpackageloaded{babel}{%
\ifthenelse{\equal{\languagename}{nohyphenation}\or
\equal{languagename}{english}}{\@setdef@ultfmtcount}{%
\@set@mulitling@fmtcount}
}{%
\@ifpackageloaded{ngerman}{%
\@ifundefined{@numberstringMgerman}{%
\input{fc-german.def}}{}\@set@mulitling@fmtcount}{%
\@setdef@ultfmtcount}}
%    \end{macrocode}
% Backwards compatibility:
%    \begin{macrocode}
\let\@ordinal=\@ordinalM
\let\@ordinalstring=\@ordinalstringM
\let\@Ordinalstring=\@OrdinalstringM
\let\@numberstring=\@numberstringM
\let\@Numberstring=\@NumberstringM
%    \end{macrocode}
%\iffalse
%    \begin{macrocode}
%</fmtcount.sty>
%    \end{macrocode}
%\fi
%\Finale
\endinput
}
%\end{verbatim}
%This, I agree, is an unpleasant cludge.
%
%\end{itemize}
%
%\section{Acknowledgements}
%
%I would like to thank my mother for the French and Portuguese
%support and my Spanish dictionary for the Spanish support.
%Thank you to K. H. Fricke for providing me with the German
%translations.
%
%\section{Troubleshooting}
%
%There is a FAQ available at: \url{http://theoval.cmp.uea.ac.uk/~nlct/latex/packages/faq/}.
%
% \section{Contact Details}
% Dr Nicola Talbot\\
% School of Computing Sciences\\
% University of East Anglia\\
% Norwich.  NR4 7TJ.\\
% United Kingdom.\\
% \url{http://theoval.cmp.uea.ac.uk/~nlct/}
%
%
%\StopEventually{}
%\section{The Code}
%\iffalse
%    \begin{macrocode}
%<*fc-british.def>
%    \end{macrocode}
%\fi
% \subsection{fc-british.def}
% British definitions
%    \begin{macrocode}
\ProvidesFile{fc-british}[2007/06/14]
%    \end{macrocode}
% Check that fc-english.def has been loaded
%    \begin{macrocode}
\@ifundefined{@ordinalMenglish}{\input{fc-english.def}}{}
%    \end{macrocode}
% These are all just synonyms for the commands provided by
% fc-english.def.
%    \begin{macrocode}
\let\@ordinalMbritish\@ordinalMenglish
\let\@ordinalFbritish\@ordinalMenglish
\let\@ordinalNbritish\@ordinalMenglish
\let\@numberstringMbritish\@numberstringMenglish
\let\@numberstringFbritish\@numberstringMenglish
\let\@numberstringNbritish\@numberstringMenglish
\let\@NumberstringMbritish\@NumberstringMenglish
\let\@NumberstringFbritish\@NumberstringMenglish
\let\@NumberstringNbritish\@NumberstringMenglish
\let\@ordinalstringMbritish\@ordinalstringMenglish
\let\@ordinalstringFbritish\@ordinalstringMenglish
\let\@ordinalstringNbritish\@ordinalstringMenglish
\let\@OrdinalstringMbritish\@OrdinalstringMenglish
\let\@OrdinalstringFbritish\@OrdinalstringMenglish
\let\@OrdinalstringNbritish\@OrdinalstringMenglish
%    \end{macrocode}
%\iffalse
%    \begin{macrocode}
%</fc-british.def>
%    \end{macrocode}
%\fi
%\iffalse
%    \begin{macrocode}
%<*fc-english.def>
%    \end{macrocode}
%\fi
% \subsection{fc-english.def}
% English definitions
%    \begin{macrocode}
\ProvidesFile{fc-english}[2007/05/26]
%    \end{macrocode}
% Define macro that converts a number or count register (first 
% argument) to an ordinal, and stores the result in the 
% second argument, which should be a control sequence.
%    \begin{macrocode}
\newcommand*{\@ordinalMenglish}[2]{%
\def\@fc@ord{}%
\@orgargctr=#1\relax
\@ordinalctr=#1%
\@modulo{\@ordinalctr}{100}%
\ifnum\@ordinalctr=11\relax
  \def\@fc@ord{th}%
\else
  \ifnum\@ordinalctr=12\relax
    \def\@fc@ord{th}%
  \else
    \ifnum\@ordinalctr=13\relax
      \def\@fc@ord{th}%
    \else
      \@modulo{\@ordinalctr}{10}%
      \ifcase\@ordinalctr
        \def\@fc@ord{th}%      case 0
        \or \def\@fc@ord{st}%  case 1
        \or \def\@fc@ord{nd}%  case 2
        \or \def\@fc@ord{rd}%  case 3
      \else
        \def\@fc@ord{th}%      default case
      \fi
    \fi
  \fi
\fi
\edef#2{\number#1\relax\noexpand\fmtord{\@fc@ord}}%
}
%    \end{macrocode}
% There is no gender difference in English, so make feminine and
% neuter the same as the masculine.
%    \begin{macrocode}
\let\@ordinalFenglish=\@ordinalMenglish
\let\@ordinalNenglish=\@ordinalMenglish
%    \end{macrocode}
% Define the macro that prints the value of a \TeX\ count register
% as text. To make it easier, break it up into units, teens and
% tens. First, the units: the argument should be between 0 and 9
% inclusive.
%    \begin{macrocode}
\newcommand*{\@@unitstringenglish}[1]{%
\ifcase#1\relax
zero%
\or one%
\or two%
\or three%
\or four%
\or five%
\or six%
\or seven%
\or eight%
\or nine%
\fi
}
%    \end{macrocode}
% Next the tens, again the argument should be between 0 and 9
% inclusive.
%    \begin{macrocode}
\newcommand*{\@@tenstringenglish}[1]{%
\ifcase#1\relax
\or ten%
\or twenty%
\or thirty%
\or forty%
\or fifty%
\or sixty%
\or seventy%
\or eighty%
\or ninety%
\fi
}
%    \end{macrocode}
% Finally the teens, again the argument should be between 0 and 9
% inclusive.
%    \begin{macrocode}
\newcommand*{\@@teenstringenglish}[1]{%
\ifcase#1\relax
ten%
\or eleven%
\or twelve%
\or thirteen%
\or fourteen%
\or fifteen%
\or sixteen%
\or seventeen%
\or eighteen%
\or nineteen%
\fi
}
%    \end{macrocode}
% As above, but with the initial letter in uppercase. The units:
%    \begin{macrocode}
\newcommand*{\@@Unitstringenglish}[1]{%
\ifcase#1\relax
Zero%
\or One%
\or Two%
\or Three%
\or Four%
\or Five%
\or Six%
\or Seven%
\or Eight%
\or Nine%
\fi
}
%    \end{macrocode}
% The tens:
%    \begin{macrocode}
\newcommand*{\@@Tenstringenglish}[1]{%
\ifcase#1\relax
\or Ten%
\or Twenty%
\or Thirty%
\or Forty%
\or Fifty%
\or Sixty%
\or Seventy%
\or Eighty%
\or Ninety%
\fi
}
%    \end{macrocode}
% The teens:
%    \begin{macrocode}
\newcommand*{\@@Teenstringenglish}[1]{%
\ifcase#1\relax
Ten%
\or Eleven%
\or Twelve%
\or Thirteen%
\or Fourteen%
\or Fifteen%
\or Sixteen%
\or Seventeen%
\or Eighteen%
\or Nineteen%
\fi
}
%    \end{macrocode}
% This has changed in version 1.09, so that it now stores
% the result in the second argument, but doesn't display anything.
% Since it only affects internal macros, it shouldn't affect
% documents created with older versions. (These internal macros are
% not meant for use in documents.)
%    \begin{macrocode}
\newcommand*{\@@numberstringenglish}[2]{%
\ifnum#1>99999
\PackageError{fmtcount}{Out of range}%
{This macro only works for values less than 100000}%
\else
\ifnum#1<0
\PackageError{fmtcount}{Negative numbers not permitted}%
{This macro does not work for negative numbers, however
you can try typing "minus" first, and then pass the modulus of
this number}%
\fi
\fi
\def#2{}%
\@strctr=#1\relax \divide\@strctr by 1000\relax
\ifnum\@strctr>9
% #1 is greater or equal to 10000
  \divide\@strctr by 10
  \ifnum\@strctr>1\relax
    \let\@@fc@numstr#2\relax
    \edef#2{\@@fc@numstr\@tenstring{\@strctr}}%
    \@strctr=#1 \divide\@strctr by 1000\relax
    \@modulo{\@strctr}{10}%
    \ifnum\@strctr>0\relax
      \let\@@fc@numstr#2\relax
      \edef#2{\@@fc@numstr-\@unitstring{\@strctr}}%
    \fi
  \else
    \@strctr=#1\relax
    \divide\@strctr by 1000\relax
    \@modulo{\@strctr}{10}%
    \let\@@fc@numstr#2\relax
    \edef#2{\@@fc@numstr\@teenstring{\@strctr}}%
  \fi
  \let\@@fc@numstr#2\relax
  \edef#2{\@@fc@numstr\ \@thousand}%
\else
  \ifnum\@strctr>0\relax
    \let\@@fc@numstr#2\relax
    \edef#2{\@@fc@numstr\@unitstring{\@strctr}\ \@thousand}%
  \fi
\fi
\@strctr=#1\relax \@modulo{\@strctr}{1000}%
\divide\@strctr by 100
\ifnum\@strctr>0\relax
   \ifnum#1>1000\relax
      \let\@@fc@numstr#2\relax
      \edef#2{\@@fc@numstr\ }%
   \fi
   \let\@@fc@numstr#2\relax
   \edef#2{\@@fc@numstr\@unitstring{\@strctr}\ \@hundred}%
\fi
\@strctr=#1\relax \@modulo{\@strctr}{100}%
\ifnum#1>100\relax
  \ifnum\@strctr>0\relax
    \let\@@fc@numstr#2\relax
    \edef#2{\@@fc@numstr\ \@andname\ }%
  \fi
\fi
\ifnum\@strctr>19\relax
  \divide\@strctr by 10\relax
  \let\@@fc@numstr#2\relax
  \edef#2{\@@fc@numstr\@tenstring{\@strctr}}%
  \@strctr=#1\relax \@modulo{\@strctr}{10}%
  \ifnum\@strctr>0\relax
    \let\@@fc@numstr#2\relax
    \edef#2{\@@fc@numstr-\@unitstring{\@strctr}}%
  \fi
\else
  \ifnum\@strctr<10\relax
    \ifnum\@strctr=0\relax
       \ifnum#1<100\relax
          \let\@@fc@numstr#2\relax
          \edef#2{\@@fc@numstr\@unitstring{\@strctr}}%
       \fi
    \else
      \let\@@fc@numstr#2\relax
      \edef#2{\@@fc@numstr\@unitstring{\@strctr}}%
    \fi
  \else
    \@modulo{\@strctr}{10}%
    \let\@@fc@numstr#2\relax
    \edef#2{\@@fc@numstr\@teenstring{\@strctr}}%
  \fi
\fi
}
%    \end{macrocode}
% All lower case version, the second argument must be a 
% control sequence.
%    \begin{macrocode}
\DeclareRobustCommand{\@numberstringMenglish}[2]{%
\let\@unitstring=\@@unitstringenglish 
\let\@teenstring=\@@teenstringenglish 
\let\@tenstring=\@@tenstringenglish
\def\@hundred{hundred}\def\@thousand{thousand}%
\def\@andname{and}%
\@@numberstringenglish{#1}{#2}%
}
%    \end{macrocode}
% There is no gender in English, so make feminine and neuter the same
% as the masculine.
%    \begin{macrocode}
\let\@numberstringFenglish=\@numberstringMenglish
\let\@numberstringNenglish=\@numberstringMenglish
%    \end{macrocode}
% This version makes the first letter of each word an uppercase
% character (except ``and''). The second argument must be a control 
% sequence.
%    \begin{macrocode}
\newcommand*{\@NumberstringMenglish}[2]{%
\let\@unitstring=\@@Unitstringenglish 
\let\@teenstring=\@@Teenstringenglish 
\let\@tenstring=\@@Tenstringenglish
\def\@hundred{Hundred}\def\@thousand{Thousand}%
\def\@andname{and}%
\@@numberstringenglish{#1}{#2}}
%    \end{macrocode}
% There is no gender in English, so make feminine and neuter the same
% as the masculine.
%    \begin{macrocode}
\let\@NumberstringFenglish=\@NumberstringMenglish
\let\@NumberstringNenglish=\@NumberstringMenglish
%    \end{macrocode}
% Define a macro that produces an ordinal as a string. Again, break
% it up into units, teens and tens. First the units:
%    \begin{macrocode}
\newcommand*{\@@unitthstringenglish}[1]{%
\ifcase#1\relax
zeroth%
\or first%
\or second%
\or third%
\or fourth%
\or fifth%
\or sixth%
\or seventh%
\or eighth%
\or ninth%
\fi
}
%    \end{macrocode}
% Next the tens:
%    \begin{macrocode}
\newcommand*{\@@tenthstringenglish}[1]{%
\ifcase#1\relax
\or tenth%
\or twentieth%
\or thirtieth%
\or fortieth%
\or fiftieth%
\or sixtieth%
\or seventieth%
\or eightieth%
\or ninetieth%
\fi
}
%   \end{macrocode}
% The teens:
%   \begin{macrocode}
\newcommand*{\@@teenthstringenglish}[1]{%
\ifcase#1\relax
tenth%
\or eleventh%
\or twelfth%
\or thirteenth%
\or fourteenth%
\or fifteenth%
\or sixteenth%
\or seventeenth%
\or eighteenth%
\or nineteenth%
\fi
}
%   \end{macrocode}
% As before, but with the first letter in upper case. The units:
%   \begin{macrocode}
\newcommand*{\@@Unitthstringenglish}[1]{%
\ifcase#1\relax
Zeroth%
\or First%
\or Second%
\or Third%
\or Fourth%
\or Fifth%
\or Sixth%
\or Seventh%
\or Eighth%
\or Ninth%
\fi
}
%    \end{macrocode}
% The tens:
%    \begin{macrocode}
\newcommand*{\@@Tenthstringenglish}[1]{%
\ifcase#1\relax
\or Tenth%
\or Twentieth%
\or Thirtieth%
\or Fortieth%
\or Fiftieth%
\or Sixtieth%
\or Seventieth%
\or Eightieth%
\or Ninetieth%
\fi
}
%    \end{macrocode}
% The teens:
%    \begin{macrocode}
\newcommand*{\@@Teenthstringenglish}[1]{%
\ifcase#1\relax
Tenth%
\or Eleventh%
\or Twelfth%
\or Thirteenth%
\or Fourteenth%
\or Fifteenth%
\or Sixteenth%
\or Seventeenth%
\or Eighteenth%
\or Nineteenth%
\fi
}
%    \end{macrocode}
% Again, as from version 1.09, this has been changed to take two
% arguments, where the second argument is a control sequence.
% The resulting text is stored in the control sequence, and nothing
% is displayed.
%    \begin{macrocode}
\newcommand*{\@@ordinalstringenglish}[2]{%
\@strctr=#1\relax
\ifnum#1>99999
\PackageError{fmtcount}{Out of range}%
{This macro only works for values less than 100000 (value given: \number\@strctr)}%
\else
\ifnum#1<0
\PackageError{fmtcount}{Negative numbers not permitted}%
{This macro does not work for negative numbers, however
you can try typing "minus" first, and then pass the modulus of
this number}%
\fi
\def#2{}%
\fi
\@strctr=#1\relax \divide\@strctr by 1000\relax
\ifnum\@strctr>9\relax
% #1 is greater or equal to 10000
  \divide\@strctr by 10
  \ifnum\@strctr>1\relax
    \let\@@fc@ordstr#2\relax
    \edef#2{\@@fc@ordstr\@tenstring{\@strctr}}%
    \@strctr=#1\relax
    \divide\@strctr by 1000\relax
    \@modulo{\@strctr}{10}%
    \ifnum\@strctr>0\relax
      \let\@@fc@ordstr#2\relax
      \edef#2{\@@fc@ordstr-\@unitstring{\@strctr}}%
    \fi
  \else
    \@strctr=#1\relax \divide\@strctr by 1000\relax
    \@modulo{\@strctr}{10}%
    \let\@@fc@ordstr#2\relax
    \edef#2{\@@fc@ordstr\@teenstring{\@strctr}}%
  \fi
  \@strctr=#1\relax \@modulo{\@strctr}{1000}%
  \ifnum\@strctr=0\relax
    \let\@@fc@ordstr#2\relax
    \edef#2{\@@fc@ordstr\ \@thousandth}%
  \else
    \let\@@fc@ordstr#2\relax
    \edef#2{\@@fc@ordstr\ \@thousand}%
  \fi
\else
  \ifnum\@strctr>0\relax
    \let\@@fc@ordstr#2\relax
    \edef#2{\@@fc@ordstr\@unitstring{\@strctr}}%
    \@strctr=#1\relax \@modulo{\@strctr}{1000}%
    \let\@@fc@ordstr#2\relax
    \ifnum\@strctr=0\relax
      \edef#2{\@@fc@ordstr\ \@thousandth}%
    \else
      \edef#2{\@@fc@ordstr\ \@thousand}%
    \fi
  \fi
\fi
\@strctr=#1\relax \@modulo{\@strctr}{1000}%
\divide\@strctr by 100
\ifnum\@strctr>0\relax
  \ifnum#1>1000\relax
    \let\@@fc@ordstr#2\relax
    \edef#2{\@@fc@ordstr\ }%
  \fi
  \let\@@fc@ordstr#2\relax
  \edef#2{\@@fc@ordstr\@unitstring{\@strctr}}%
  \@strctr=#1\relax \@modulo{\@strctr}{100}%
  \let\@@fc@ordstr#2\relax
  \ifnum\@strctr=0\relax
    \edef#2{\@@fc@ordstr\ \@hundredth}%
  \else
    \edef#2{\@@fc@ordstr\ \@hundred}%
  \fi
\fi
\@strctr=#1\relax \@modulo{\@strctr}{100}%
\ifnum#1>100\relax
  \ifnum\@strctr>0\relax
    \let\@@fc@ordstr#2\relax
    \edef#2{\@@fc@ordstr\ \@andname\ }%
  \fi
\fi
\ifnum\@strctr>19\relax
  \@tmpstrctr=\@strctr
  \divide\@strctr by 10\relax
  \@modulo{\@tmpstrctr}{10}%
  \let\@@fc@ordstr#2\relax
  \ifnum\@tmpstrctr=0\relax
    \edef#2{\@@fc@ordstr\@tenthstring{\@strctr}}%
  \else
    \edef#2{\@@fc@ordstr\@tenstring{\@strctr}}%
  \fi
  \@strctr=#1\relax \@modulo{\@strctr}{10}%
  \ifnum\@strctr>0\relax
    \let\@@fc@ordstr#2\relax
    \edef#2{\@@fc@ordstr-\@unitthstring{\@strctr}}%
  \fi
\else
  \ifnum\@strctr<10\relax
    \ifnum\@strctr=0\relax
      \ifnum#1<100\relax
        \let\@@fc@ordstr#2\relax
        \edef#2{\@@fc@ordstr\@unitthstring{\@strctr}}%
      \fi
    \else
      \let\@@fc@ordstr#2\relax
      \edef#2{\@@fc@ordstr\@unitthstring{\@strctr}}%
    \fi
  \else
    \@modulo{\@strctr}{10}%
    \let\@@fc@ordstr#2\relax
    \edef#2{\@@fc@ordstr\@teenthstring{\@strctr}}%
  \fi
\fi
}
%    \end{macrocode}
% All lower case version. Again, the second argument must be a
% control sequence in which the resulting text is stored.
%    \begin{macrocode}
\DeclareRobustCommand{\@ordinalstringMenglish}[2]{%
\let\@unitthstring=\@@unitthstringenglish 
\let\@teenthstring=\@@teenthstringenglish 
\let\@tenthstring=\@@tenthstringenglish
\let\@unitstring=\@@unitstringenglish 
\let\@teenstring=\@@teenstringenglish
\let\@tenstring=\@@tenstringenglish
\def\@andname{and}%
\def\@hundred{hundred}\def\@thousand{thousand}%
\def\@hundredth{hundredth}\def\@thousandth{thousandth}%
\@@ordinalstringenglish{#1}{#2}}
%    \end{macrocode}
% No gender in English, so make feminine and neuter same as masculine:
%    \begin{macrocode}
\let\@ordinalstringFenglish=\@ordinalstringMenglish
\let\@ordinalstringNenglish=\@ordinalstringMenglish
%    \end{macrocode}
% First letter of each word in upper case:
%    \begin{macrocode}
\DeclareRobustCommand{\@OrdinalstringMenglish}[2]{%
\let\@unitthstring=\@@Unitthstringenglish
\let\@teenthstring=\@@Teenthstringenglish
\let\@tenthstring=\@@Tenthstringenglish
\let\@unitstring=\@@Unitstringenglish
\let\@teenstring=\@@Teenstringenglish
\let\@tenstring=\@@Tenstringenglish
\def\@andname{and}%
\def\@hundred{Hundred}\def\@thousand{Thousand}%
\def\@hundredth{Hundredth}\def\@thousandth{Thousandth}%
\@@ordinalstringenglish{#1}{#2}}
%    \end{macrocode}
% No gender in English, so make feminine and neuter same as masculine:
%    \begin{macrocode}
\let\@OrdinalstringFenglish=\@OrdinalstringMenglish
\let\@OrdinalstringNenglish=\@OrdinalstringMenglish
%    \end{macrocode}
%\iffalse
%    \begin{macrocode}
%</fc-english.def>
%    \end{macrocode}
%\fi
%\iffalse
%    \begin{macrocode}
%<*fc-french.def>
%    \end{macrocode}
%\fi
% \subsection{fc-french.def}
% French definitions
%    \begin{macrocode}
\ProvidesFile{fc-french.def}[2007/05/26]
%    \end{macrocode}
% Define macro that converts a number or count register (first
% argument) to an ordinal, and store the result in the second
% argument, which must be a control sequence. Masculine:
%    \begin{macrocode}
\newcommand*{\@ordinalMfrench}[2]{%
\iffmtord@abbrv
  \edef#2{\number#1\relax\noexpand\fmtord{e}}%
\else
  \ifnum#1=1\relax
    \edef#2{\number#1\relax\noexpand\fmtord{er}}%
  \else
    \edef#2{\number#1\relax\noexpand\fmtord{eme}}%
  \fi
\fi}
%    \end{macrocode}
% Feminine:
%    \begin{macrocode}
\newcommand*{\@ordinalFfrench}[2]{%
\iffmtord@abbrv
  \edef#2{\number#1\relax\noexpand\fmtord{e}}%
\else
  \ifnum#1=1\relax
     \edef#2{\number#1\relax\noexpand\fmtord{ere}}%
  \else
     \edef#2{\number#1\relax\noexpand\fmtord{eme}}%
  \fi
\fi}
%    \end{macrocode}
% Make neuter same as masculine:
%    \begin{macrocode}
\let\@ordinalNfrench\@ordinalMfrench
%    \end{macrocode}
% Textual representation of a number. To make it easier break it
% into units, tens and teens. First the units:
%   \begin{macrocode}
\newcommand*{\@@unitstringfrench}[1]{%
\ifcase#1\relax
zero%
\or un%
\or deux%
\or trois%
\or quatre%
\or cinq%
\or six%
\or sept%
\or huit%
\or neuf%
\fi
}
%    \end{macrocode}
% Feminine only changes for 1:
%    \begin{macrocode}
\newcommand*{\@@unitstringFfrench}[1]{%
\ifnum#1=1\relax
une%
\else\@@unitstringfrench{#1}%
\fi
}
%    \end{macrocode}
% Tens (this includes the Belgian and Swiss variants, special
% cases employed lower down.)
%    \begin{macrocode}
\newcommand*{\@@tenstringfrench}[1]{%
\ifcase#1\relax
\or dix%
\or vingt%
\or trente%
\or quarante%
\or cinquante%
\or soixante%
\or septente%
\or huitante%
\or nonente%
\or cent%
\fi
}
%    \end{macrocode}
% Teens:
%    \begin{macrocode}
\newcommand*{\@@teenstringfrench}[1]{%
\ifcase#1\relax
dix%
\or onze%
\or douze%
\or treize%
\or quatorze%
\or quinze%
\or seize%
\or dix-sept%
\or dix-huit%
\or dix-neuf%
\fi
}
%    \end{macrocode}
% Seventies are a special case, depending on dialect:
%    \begin{macrocode}
\newcommand*{\@@seventiesfrench}[1]{%
\@tenstring{6}%
\ifnum#1=1\relax
\ \@andname\ 
\else
-%
\fi
\@teenstring{#1}%
}
%    \end{macrocode}
% Eighties are a special case, depending on dialect:
%    \begin{macrocode}
\newcommand*{\@@eightiesfrench}[1]{%
\@unitstring{4}-\@tenstring{2}%
\ifnum#1>0
-\@unitstring{#1}%
\else
s%
\fi
}
%    \end{macrocode}
% Nineties are a special case, depending on dialect:
%    \begin{macrocode}
\newcommand*{\@@ninetiesfrench}[1]{%
\@unitstring{4}-\@tenstring{2}-\@teenstring{#1}%
}
%    \end{macrocode}
% Swiss seventies:
%    \begin{macrocode}
\newcommand*{\@@seventiesfrenchswiss}[1]{%
\@tenstring{7}%
\ifnum#1=1\ \@andname\ \fi
\ifnum#1>1-\fi
\ifnum#1>0\@unitstring{#1}\fi
}
%    \end{macrocode}
% Swiss eighties:
%    \begin{macrocode}
\newcommand*{\@@eightiesfrenchswiss}[1]{%
\@tenstring{8}%
\ifnum#1=1\ \@andname\ \fi
\ifnum#1>1-\fi
\ifnum#1>0\@unitstring{#1}\fi
}
%    \end{macrocode}
% Swiss nineties:
%    \begin{macrocode}
\newcommand*{\@@ninetiesfrenchswiss}[1]{%
\@tenstring{9}%
\ifnum#1=1\ \@andname\ \fi
\ifnum#1>1-\fi
\ifnum#1>0\@unitstring{#1}\fi
}
%    \end{macrocode}
% Units with initial letter in upper case:
%    \begin{macrocode}
\newcommand*{\@@Unitstringfrench}[1]{%
\ifcase#1\relax
Zero%
\or Un%
\or Deux%
\or Trois%
\or Quatre%
\or Cinq%
\or Six%
\or Sept%
\or Huit%
\or Neuf%
\fi
}
%    \end{macrocode}
% As above, but feminine:
%    \begin{macrocode}
\newcommand*{\@@UnitstringFfrench}[1]{%
\ifnum#1=1\relax
Une%
\else \@@Unitstringfrench{#1}%
\fi
}
%    \end{macrocode}
% Tens, with initial letter in upper case (includes Swiss and
% Belgian variants):
%    \begin{macrocode}
\newcommand*{\@@Tenstringfrench}[1]{%
\ifcase#1\relax
\or Dix%
\or Vingt%
\or Trente%
\or Quarante%
\or Cinquante%
\or Soixante%
\or Septente%
\or Huitante%
\or Nonente%
\or Cent%
\fi
}
%    \end{macrocode}
% Teens, with initial letter in upper case:
%    \begin{macrocode}
\newcommand*{\@@Teenstringfrench}[1]{%
\ifcase#1\relax
Dix%
\or Onze%
\or Douze%
\or Treize%
\or Quatorze%
\or Quinze%
\or Seize%
\or Dix-Sept%
\or Dix-Huit%
\or Dix-Neuf%
\fi
}
%    \end{macrocode}
% This has changed in version 1.09, so that it now stores the
% result in the second argument, but doesn't display anything.
% Since it only affects internal macros, it shouldn't affect
% documents created with older versions. (These internal macros
% are not defined for use in documents.) Firstly, the Swiss
% version:
%    \begin{macrocode}
\DeclareRobustCommand{\@numberstringMfrenchswiss}[2]{%
\let\@unitstring=\@@unitstringfrench
\let\@teenstring=\@@teenstringfrench
\let\@tenstring=\@@tenstringfrench
\let\@seventies=\@@seventiesfrenchswiss
\let\@eighties=\@@eightiesfrenchswiss
\let\@nineties=\@@ninetiesfrenchswiss
\def\@hundred{cent}\def\@thousand{mille}%
\def\@andname{et}%
\@@numberstringfrench{#1}{#2}}
%    \end{macrocode}
% Same as above, but for French as spoken in France:
%    \begin{macrocode}
\DeclareRobustCommand{\@numberstringMfrenchfrance}[2]{%
\let\@unitstring=\@@unitstringfrench
\let\@teenstring=\@@teenstringfrench
\let\@tenstring=\@@tenstringfrench
\let\@seventies=\@@seventiesfrench
\let\@eighties=\@@eightiesfrench
\let\@nineties=\@@ninetiesfrench
\def\@hundred{cent}\def\@thousand{mille}%
\def\@andname{et}%
\@@numberstringfrench{#1}{#2}}
%    \end{macrocode}
% Same as above, but for Belgian dialect:
%    \begin{macrocode}
\DeclareRobustCommand{\@numberstringMfrenchbelgian}[2]{%
\let\@unitstring=\@@unitstringfrench
\let\@teenstring=\@@teenstringfrench
\let\@tenstring=\@@tenstringfrench
\let\@seventies=\@@seventiesfrenchswiss
\let\@eighties=\@@eightiesfrench
\let\@nineties=\@@ninetiesfrench
\def\@hundred{cent}\def\@thousand{mille}%
\def\@andname{et}%
\@@numberstringfrench{#1}{#2}}
%    \end{macrocode}
% Set default dialect:
%    \begin{macrocode}
\let\@numberstringMfrench=\@numberstringMfrenchfrance
%    \end{macrocode}
% As above, but for feminine version. Swiss:
%    \begin{macrocode}
\DeclareRobustCommand{\@numberstringFfrenchswiss}[2]{%
\let\@unitstring=\@@unitstringFfrench
\let\@teenstring=\@@teenstringfrench
\let\@tenstring=\@@tenstringfrench
\let\@seventies=\@@seventiesfrenchswiss
\let\@eighties=\@@eightiesfrenchswiss
\let\@nineties=\@@ninetiesfrenchswiss
\def\@hundred{cent}\def\@thousand{mille}%
\def\@andname{et}%
\@@numberstringfrench{#1}{#2}}
%    \end{macrocode}
% French:
%    \begin{macrocode}
\DeclareRobustCommand{\@numberstringFfrenchfrance}[2]{%
\let\@unitstring=\@@unitstringFfrench
\let\@teenstring=\@@teenstringfrench
\let\@tenstring=\@@tenstringfrench
\let\@seventies=\@@seventiesfrench
\let\@eighties=\@@eightiesfrench
\let\@nineties=\@@ninetiesfrench
\def\@hundred{cent}\def\@thousand{mille}%
\def\@andname{et}%
\@@numberstringfrench{#1}{#2}}
%    \end{macrocode}
% Belgian:
%    \begin{macrocode}
\DeclareRobustCommand{\@numberstringFfrenchbelgian}[2]{%
\let\@unitstring=\@@unitstringFfrench
\let\@teenstring=\@@teenstringfrench
\let\@tenstring=\@@tenstringfrench
\let\@seventies=\@@seventiesfrenchswiss
\let\@eighties=\@@eightiesfrench
\let\@nineties=\@@ninetiesfrench
\def\@hundred{cent}\def\@thousand{mille}%
\def\@andname{et}%
\@@numberstringfrench{#1}{#2}}
%    \end{macrocode}
% Set default dialect:
%    \begin{macrocode}
\let\@numberstringFfrench=\@numberstringFfrenchfrance
%    \end{macrocode}
% Make neuter same as masculine:
%    \begin{macrocode}
\let\@ordinalstringNfrench\@ordinalstringMfrench
%    \end{macrocode}
% As above, but with initial letter in upper case. Swiss (masculine):
%    \begin{macrocode}
\DeclareRobustCommand{\@NumberstringMfrenchswiss}[2]{%
\let\@unitstring=\@@Unitstringfrench
\let\@teenstring=\@@Teenstringfrench
\let\@tenstring=\@@Tenstringfrench
\let\@seventies=\@@seventiesfrenchswiss
\let\@eighties=\@@eightiesfrenchswiss
\let\@nineties=\@@ninetiesfrenchswiss
\def\@hundred{Cent}\def\@thousand{Mille}%
\def\@andname{et}%
\@@numberstringfrench{#1}{#2}}
%    \end{macrocode}
% French:
%    \begin{macrocode}
\DeclareRobustCommand{\@NumberstringMfrenchfrance}[2]{%
\let\@unitstring=\@@Unitstringfrench
\let\@teenstring=\@@Teenstringfrench
\let\@tenstring=\@@Tenstringfrench
\let\@seventies=\@@seventiesfrench
\let\@eighties=\@@eightiesfrench
\let\@nineties=\@@ninetiesfrench
\def\@hundred{Cent}\def\@thousand{Mille}%
\def\@andname{et}%
\@@numberstringfrench{#1}{#2}}
%    \end{macrocode}
% Belgian:
%    \begin{macrocode}
\DeclareRobustCommand{\@NumberstringMfrenchbelgian}[2]{%
\let\@unitstring=\@@Unitstringfrench
\let\@teenstring=\@@Teenstringfrench
\let\@tenstring=\@@Tenstringfrench
\let\@seventies=\@@seventiesfrenchswiss
\let\@eighties=\@@eightiesfrench
\let\@nineties=\@@ninetiesfrench
\def\@hundred{Cent}\def\@thousand{Mille}%
\def\@andname{et}%
\@@numberstringfrench{#1}{#2}}
%    \end{macrocode}
% Set default dialect:
%    \begin{macrocode}
\let\@NumberstringMfrench=\@NumberstringMfrenchfrance
%    \end{macrocode}
% As above, but feminine. Swiss:
%    \begin{macrocode}
\DeclareRobustCommand{\@NumberstringFfrenchswiss}[2]{%
\let\@unitstring=\@@UnitstringFfrench
\let\@teenstring=\@@Teenstringfrench
\let\@tenstring=\@@Tenstringfrench
\let\@seventies=\@@seventiesfrenchswiss
\let\@eighties=\@@eightiesfrenchswiss
\let\@nineties=\@@ninetiesfrenchswiss
\def\@hundred{Cent}\def\@thousand{Mille}%
\def\@andname{et}%
\@@numberstringfrench{#1}{#2}}
%    \end{macrocode}
% French (feminine):
%    \begin{macrocode}
\DeclareRobustCommand{\@NumberstringFfrenchfrance}[2]{%
\let\@unitstring=\@@UnitstringFfrench
\let\@teenstring=\@@Teenstringfrench
\let\@tenstring=\@@Tenstringfrench
\let\@seventies=\@@seventiesfrench
\let\@eighties=\@@eightiesfrench
\let\@nineties=\@@ninetiesfrench
\def\@hundred{Cent}\def\@thousand{Mille}%
\def\@andname{et}%
\@@numberstringfrench{#1}{#2}}
%    \end{macrocode}
% Belgian (feminine):
%    \begin{macrocode}
\DeclareRobustCommand{\@NumberstringFfrenchbelgian}[2]{%
\let\@unitstring=\@@UnitstringFfrench
\let\@teenstring=\@@Teenstringfrench
\let\@tenstring=\@@Tenstringfrench
\let\@seventies=\@@seventiesfrenchswiss
\let\@eighties=\@@eightiesfrench
\let\@nineties=\@@ninetiesfrench
\def\@hundred{Cent}\def\@thousand{Mille}%
\def\@andname{et}%
\@@numberstringfrench{#1}{#2}}
%    \end{macrocode}
% Set default dialect:
%    \begin{macrocode}
\let\@NumberstringFfrench=\@NumberstringFfrenchfrance
%    \end{macrocode}
% Make neuter same as masculine:
%    \begin{macrocode}
\let\@NumberstringNfrench\@NumberstringMfrench
%    \end{macrocode}
% Again, as from version 1.09, this has been changed to take
% two arguments, where the second argument is a control
% sequence, and nothing is displayed. Store textual representation
% of an ordinal in the given control sequence. Swiss dialect (masculine):
%    \begin{macrocode}
\DeclareRobustCommand{\@ordinalstringMfrenchswiss}[2]{%
\ifnum#1=1\relax
\def#2{premier}%
\else
\let\@unitthstring=\@@unitthstringfrench
\let\@unitstring=\@@unitstringfrench
\let\@teenthstring=\@@teenthstringfrench
\let\@teenstring=\@@teenstringfrench
\let\@tenthstring=\@@tenthstringfrench
\let\@tenstring=\@@tenstringfrench
\let\@seventieths=\@@seventiethsfrenchswiss
\let\@eightieths=\@@eightiethsfrenchswiss
\let\@ninetieths=\@@ninetiethsfrenchswiss
\let\@seventies=\@@seventiesfrenchswiss
\let\@eighties=\@@eightiesfrenchswiss
\let\@nineties=\@@ninetiesfrenchswiss
\def\@hundredth{centi\`eme}\def\@hundred{cent}%
\def\@thousandth{mili\`eme}\def\@thousand{mille}%
\def\@andname{et}%
\@@ordinalstringfrench{#1}{#2}%
\fi}
%    \end{macrocode}
% French (masculine):
%    \begin{macrocode}
\DeclareRobustCommand{\@ordinalstringMfrenchfrance}[2]{%
\ifnum#1=1\relax
\def#2{premier}%
\else
\let\@unitthstring=\@@unitthstringfrench
\let\@unitstring=\@@unitstringfrench
\let\@teenthstring=\@@teenthstringfrench
\let\@teenstring=\@@teenstringfrench
\let\@tenthstring=\@@tenthstringfrench
\let\@tenstring=\@@tenstringfrench
\let\@seventieths=\@@seventiethsfrench
\let\@eightieths=\@@eightiethsfrench
\let\@ninetieths=\@@ninetiethsfrench
\let\@seventies=\@@seventiesfrench
\let\@eighties=\@@eightiesfrench
\let\@nineties=\@@ninetiesfrench
\let\@teenstring=\@@teenstringfrench
\def\@hundredth{centi\`eme}\def\@hundred{cent}%
\def\@thousandth{mili\`eme}\def\@thousand{mille}%
\def\@andname{et}%
\@@ordinalstringfrench{#1}{#2}%
\fi}
%    \end{macrocode}
% Belgian dialect (masculine):
%    \begin{macrocode}
\DeclareRobustCommand{\@ordinalstringMfrenchbelgian}[2]{%
\ifnum#1=1\relax
\def#2{premier}%
\else
\let\@unitthstring=\@@unitthstringfrench
\let\@unitstring=\@@unitstringfrench
\let\@teenthstring=\@@teenthstringfrench
\let\@teenstring=\@@teenstringfrench
\let\@tenthstring=\@@tenthstringfrench
\let\@tenstring=\@@tenstringfrench
\let\@seventieths=\@@seventiethsfrenchswiss
\let\@eightieths=\@@eightiethsfrench
\let\@ninetieths=\@@ninetiethsfrenchswiss
\let\@seventies=\@@seventiesfrench
\let\@eighties=\@@eightiesfrench
\let\@nineties=\@@ninetiesfrench
\let\@teenstring=\@@teenstringfrench
\def\@hundredth{centi\`eme}\def\@hundred{cent}%
\def\@thousandth{mili\`eme}\def\@thousand{mille}%
\def\@andname{et}%
\@@ordinalstringfrench{#1}{#2}%
\fi}
%    \end{macrocode}
% Set up default dialect:
%    \begin{macrocode}
\let\@ordinalstringMfrench=\@ordinalstringMfrenchfrance
%    \end{macrocode}
% As above, but feminine. Swiss:
%    \begin{macrocode}
\DeclareRobustCommand{\@ordinalstringFfrenchswiss}[2]{%
\ifnum#1=1\relax
\def#2{premi\`ere}%
\else
\let\@unitthstring=\@@unitthstringfrench
\let\@unitstring=\@@unitstringFfrench
\let\@teenthstring=\@@teenthstringfrench
\let\@teenstring=\@@teenstringfrench
\let\@tenthstring=\@@tenthstringfrench
\let\@tenstring=\@@tenstringfrench
\let\@seventieths=\@@seventiethsfrenchswiss
\let\@eightieths=\@@eightiethsfrenchswiss
\let\@ninetieths=\@@ninetiethsfrenchswiss
\let\@seventies=\@@seventiesfrenchswiss
\let\@eighties=\@@eightiesfrenchswiss
\let\@nineties=\@@ninetiesfrenchswiss
\def\@hundredth{centi\`eme}\def\@hundred{cent}%
\def\@thousandth{mili\`eme}\def\@thousand{mille}%
\def\@andname{et}%
\@@ordinalstringfrench{#1}{#2}%
\fi}
%    \end{macrocode}
% French (feminine):
%    \begin{macrocode}
\DeclareRobustCommand{\@ordinalstringFfrenchfrance}[2]{%
\ifnum#1=1\relax
\def#2{premi\`ere}%
\else
\let\@unitthstring=\@@unitthstringfrench
\let\@unitstring=\@@unitstringFfrench
\let\@teenthstring=\@@teenthstringfrench
\let\@teenstring=\@@teenstringfrench
\let\@tenthstring=\@@tenthstringfrench
\let\@tenstring=\@@tenstringfrench
\let\@seventieths=\@@seventiethsfrench
\let\@eightieths=\@@eightiethsfrench
\let\@ninetieths=\@@ninetiethsfrench
\let\@seventies=\@@seventiesfrench
\let\@eighties=\@@eightiesfrench
\let\@nineties=\@@ninetiesfrench
\let\@teenstring=\@@teenstringfrench
\def\@hundredth{centi\`eme}\def\@hundred{cent}%
\def\@thousandth{mili\`eme}\def\@thousand{mille}%
\def\@andname{et}%
\@@ordinalstringfrench{#1}{#2}%
\fi}
%    \end{macrocode}
% Belgian (feminine):
%    \begin{macrocode}
\DeclareRobustCommand{\@ordinalstringFfrenchbelgian}[2]{%
\ifnum#1=1\relax
\def#2{premi\`ere}%
\else
\let\@unitthstring=\@@unitthstringfrench
\let\@unitstring=\@@unitstringFfrench
\let\@teenthstring=\@@teenthstringfrench
\let\@teenstring=\@@teenstringfrench
\let\@tenthstring=\@@tenthstringfrench
\let\@tenstring=\@@tenstringfrench
\let\@seventieths=\@@seventiethsfrenchswiss
\let\@eightieths=\@@eightiethsfrench
\let\@ninetieths=\@@ninetiethsfrench
\let\@seventies=\@@seventiesfrench
\let\@eighties=\@@eightiesfrench
\let\@nineties=\@@ninetiesfrench
\let\@teenstring=\@@teenstringfrench
\def\@hundredth{centi\`eme}\def\@hundred{cent}%
\def\@thousandth{mili\`eme}\def\@thousand{mille}%
\def\@andname{et}%
\@@ordinalstringfrench{#1}{#2}%
\fi}
%    \end{macrocode}
% Set up default dialect:
%    \begin{macrocode}
\let\@ordinalstringFfrench=\@ordinalstringFfrenchfrance
%    \end{macrocode}
% Make neuter same as masculine:
%    \begin{macrocode}
\let\@ordinalstringNfrench\@ordinalstringMfrench
%    \end{macrocode}
% As above, but with initial letters in upper case. Swiss (masculine):
%    \begin{macrocode}
\DeclareRobustCommand{\@OrdinalstringMfrenchswiss}[2]{%
\ifnum#1=1\relax
\def#2{Premi\`ere}%
\else
\let\@unitthstring=\@@Unitthstringfrench
\let\@unitstring=\@@Unitstringfrench
\let\@teenthstring=\@@Teenthstringfrench
\let\@teenstring=\@@Teenstringfrench
\let\@tenthstring=\@@Tenthstringfrench
\let\@tenstring=\@@Tenstringfrench
\let\@seventieths=\@@seventiethsfrenchswiss
\let\@eightieths=\@@eightiethsfrenchswiss
\let\@ninetieths=\@@ninetiethsfrenchswiss
\let\@seventies=\@@seventiesfrenchswiss
\let\@eighties=\@@eightiesfrenchswiss
\let\@nineties=\@@ninetiesfrenchswiss
\def\@hundredth{Centi\`eme}\def\@hundred{Cent}%
\def\@thousandth{Mili\`eme}\def\@thousand{Mille}%
\def\@andname{et}%
\@@ordinalstringfrench{#1}{#2}%
\fi}
%    \end{macrocode}
% French (masculine):
%    \begin{macrocode}
\DeclareRobustCommand{\@OrdinalstringMfrenchfrance}[2]{%
\ifnum#1=1\relax
\def#2{Premi\`ere}%
\else
\let\@unitthstring=\@@Unitthstringfrench
\let\@unitstring=\@@Unitstringfrench
\let\@teenthstring=\@@Teenthstringfrench
\let\@teenstring=\@@Teenstringfrench
\let\@tenthstring=\@@Tenthstringfrench
\let\@tenstring=\@@Tenstringfrench
\let\@seventieths=\@@seventiethsfrench
\let\@eightieths=\@@eightiethsfrench
\let\@ninetieths=\@@ninetiethsfrench
\let\@seventies=\@@seventiesfrench
\let\@eighties=\@@eightiesfrench
\let\@nineties=\@@ninetiesfrench
\let\@teenstring=\@@Teenstringfrench
\def\@hundredth{Centi\`eme}\def\@hundred{Cent}%
\def\@thousandth{Mili\`eme}\def\@thousand{Mille}%
\def\@andname{et}%
\@@ordinalstringfrench{#1}{#2}%
\fi}
%    \end{macrocode}
% Belgian (masculine):
%    \begin{macrocode}
\DeclareRobustCommand{\@OrdinalstringMfrenchbelgian}[2]{%
\ifnum#1=1\relax
\def#2{Premi\`ere}%
\else
\let\@unitthstring=\@@Unitthstringfrench
\let\@unitstring=\@@Unitstringfrench
\let\@teenthstring=\@@Teenthstringfrench
\let\@teenstring=\@@Teenstringfrench
\let\@tenthstring=\@@Tenthstringfrench
\let\@tenstring=\@@Tenstringfrench
\let\@seventieths=\@@seventiethsfrenchswiss
\let\@eightieths=\@@eightiethsfrench
\let\@ninetieths=\@@ninetiethsfrench
\let\@seventies=\@@seventiesfrench
\let\@eighties=\@@eightiesfrench
\let\@nineties=\@@ninetiesfrench
\let\@teenstring=\@@Teenstringfrench
\def\@hundredth{Centi\`eme}\def\@hundred{Cent}%
\def\@thousandth{Mili\`eme}\def\@thousand{Mille}%
\def\@andname{et}%
\@@ordinalstringfrench{#1}{#2}%
\fi}
%    \end{macrocode}
% Set up default dialect:
%    \begin{macrocode}
\let\@OrdinalstringMfrench=\@OrdinalstringMfrenchfrance
%    \end{macrocode}
% As above, but feminine form. Swiss:
%    \begin{macrocode}
\DeclareRobustCommand{\@OrdinalstringFfrenchswiss}[2]{%
\ifnum#1=1\relax
\def#2{Premi\`ere}%
\else
\let\@unitthstring=\@@Unitthstringfrench
\let\@unitstring=\@@UnitstringFfrench
\let\@teenthstring=\@@Teenthstringfrench
\let\@teenstring=\@@Teenstringfrench
\let\@tenthstring=\@@Tenthstringfrench
\let\@tenstring=\@@Tenstringfrench
\let\@seventieths=\@@seventiethsfrenchswiss
\let\@eightieths=\@@eightiethsfrenchswiss
\let\@ninetieths=\@@ninetiethsfrenchswiss
\let\@seventies=\@@seventiesfrenchswiss
\let\@eighties=\@@eightiesfrenchswiss
\let\@nineties=\@@ninetiesfrenchswiss
\def\@hundredth{Centi\`eme}\def\@hundred{Cent}%
\def\@thousandth{Mili\`eme}\def\@thousand{Mille}%
\def\@andname{et}%
\@@ordinalstringfrench{#1}{#2}%
\fi}
%    \end{macrocode}
% French (feminine):
%    \begin{macrocode}
\DeclareRobustCommand{\@OrdinalstringFfrenchfrance}[2]{%
\ifnum#1=1\relax
\def#2{Premi\`ere}%
\else
\let\@unitthstring=\@@Unitthstringfrench
\let\@unitstring=\@@UnitstringFfrench
\let\@teenthstring=\@@Teenthstringfrench
\let\@teenstring=\@@Teenstringfrench
\let\@tenthstring=\@@Tenthstringfrench
\let\@tenstring=\@@Tenstringfrench
\let\@seventieths=\@@seventiethsfrench
\let\@eightieths=\@@eightiethsfrench
\let\@ninetieths=\@@ninetiethsfrench
\let\@seventies=\@@seventiesfrench
\let\@eighties=\@@eightiesfrench
\let\@nineties=\@@ninetiesfrench
\let\@teenstring=\@@Teenstringfrench
\def\@hundredth{Centi\`eme}\def\@hundred{Cent}%
\def\@thousandth{Mili\`eme}\def\@thousand{Mille}%
\def\@andname{et}%
\@@ordinalstringfrench{#1}{#2}%
\fi}
%    \end{macrocode}
% Belgian (feminine):
%    \begin{macrocode}
\DeclareRobustCommand{\@OrdinalstringFfrenchbelgian}[2]{%
\ifnum#1=1\relax
\def#2{Premi\`ere}%
\else
\let\@unitthstring=\@@Unitthstringfrench
\let\@unitstring=\@@UnitstringFfrench
\let\@teenthstring=\@@Teenthstringfrench
\let\@teenstring=\@@Teenstringfrench
\let\@tenthstring=\@@Tenthstringfrench
\let\@tenstring=\@@Tenstringfrench
\let\@seventieths=\@@seventiethsfrenchswiss
\let\@eightieths=\@@eightiethsfrench
\let\@ninetieths=\@@ninetiethsfrench
\let\@seventies=\@@seventiesfrench
\let\@eighties=\@@eightiesfrench
\let\@nineties=\@@ninetiesfrench
\let\@teenstring=\@@Teenstringfrench
\def\@hundredth{Centi\`eme}\def\@hundred{Cent}%
\def\@thousandth{Mili\`eme}\def\@thousand{Mille}%
\def\@andname{et}%
\@@ordinalstringfrench{#1}{#2}%
\fi}
%    \end{macrocode}
% Set up default dialect:
%    \begin{macrocode}
\let\@OrdinalstringFfrench=\@OrdinalstringFfrenchfrance
%    \end{macrocode}
% Make neuter same as masculine:
%    \begin{macrocode}
\let\@OrdinalstringNfrench\@OrdinalstringMfrench
%    \end{macrocode}
% In order to convert numbers into textual ordinals, need
% to break it up into units, tens and teens. First the units.
% The argument must be a number or count register between 0
% and 9.
%    \begin{macrocode}
\newcommand*{\@@unitthstringfrench}[1]{%
\ifcase#1\relax
zero%
\or uni\`eme%
\or deuxi\`eme%
\or troisi\`eme%
\or quatri\`eme%
\or cinqui\`eme%
\or sixi\`eme%
\or septi\`eme%
\or huiti\`eme%
\or neuvi\`eme%
\fi
}
%    \end{macrocode}
% Tens (includes Swiss and Belgian variants, special cases are
% dealt with later.)
%    \begin{macrocode}
\newcommand*{\@@tenthstringfrench}[1]{%
\ifcase#1\relax
\or dixi\`eme%
\or vingti\`eme%
\or trentri\`eme%
\or quaranti\`eme%
\or cinquanti\`eme%
\or soixanti\`eme%
\or septenti\`eme%
\or huitanti\`eme%
\or nonenti\`eme%
\fi
}
%    \end{macrocode}
% Teens:
%    \begin{macrocode}
\newcommand*{\@@teenthstringfrench}[1]{%
\ifcase#1\relax
dixi\`eme%
\or onzi\`eme%
\or douzi\`eme%
\or treizi\`eme%
\or quatorzi\`eme%
\or quinzi\`eme%
\or seizi\`eme%
\or dix-septi\`eme%
\or dix-huiti\`eme%
\or dix-neuvi\`eme%
\fi
}
%    \end{macrocode}
% Seventies vary depending on dialect. Swiss:
%    \begin{macrocode}
\newcommand*{\@@seventiethsfrenchswiss}[1]{%
\ifcase#1\relax
\@tenthstring{7}%
\or
\@tenstring{7} \@andname\ \@unitthstring{1}%
\else
\@tenstring{7}-\@unitthstring{#1}%
\fi}
%    \end{macrocode}
% Eighties vary depending on dialect. Swiss:
%    \begin{macrocode}
\newcommand*{\@@eightiethsfrenchswiss}[1]{%
\ifcase#1\relax
\@tenthstring{8}%
\or
\@tenstring{8} \@andname\ \@unitthstring{1}%
\else
\@tenstring{8}-\@unitthstring{#1}%
\fi}
%    \end{macrocode}
% Nineties vary depending on dialect. Swiss:
%    \begin{macrocode}
\newcommand*{\@@ninetiethsfrenchswiss}[1]{%
\ifcase#1\relax
\@tenthstring{9}%
\or
\@tenstring{9} \@andname\ \@unitthstring{1}%
\else
\@tenstring{9}-\@unitthstring{#1}%
\fi}
%    \end{macrocode}
% French (as spoken in France) version:
%    \begin{macrocode}
\newcommand*{\@@seventiethsfrench}[1]{%
\ifnum#1=0\relax
\@tenstring{6}%
-%
\else
\@tenstring{6}%
\ \@andname\ 
\fi
\@teenthstring{#1}%
}
%    \end{macrocode}
% Eighties (as spoken in France):
%    \begin{macrocode}
\newcommand*{\@@eightiethsfrench}[1]{%
\ifnum#1>0\relax
\@unitstring{4}-\@tenstring{2}%
-\@unitthstring{#1}%
\else
\@unitstring{4}-\@tenthstring{2}%
\fi
}
%    \end{macrocode}
% Nineties (as spoken in France):
%    \begin{macrocode}
\newcommand*{\@@ninetiethsfrench}[1]{%
\@unitstring{4}-\@tenstring{2}-\@teenthstring{#1}%
}
%    \end{macrocode}
% As above, but with initial letter in upper case. Units:
%    \begin{macrocode}
\newcommand*{\@@Unitthstringfrench}[1]{%
\ifcase#1\relax
Zero%
\or Uni\`eme%
\or Deuxi\`eme%
\or Troisi\`eme%
\or Quatri\`eme%
\or Cinqui\`eme%
\or Sixi\`eme%
\or Septi\`eme%
\or Huiti\`eme%
\or Neuvi\`eme%
\fi
}
%    \end{macrocode}
% Tens (includes Belgian and Swiss variants):
%    \begin{macrocode}
\newcommand*{\@@Tenthstringfrench}[1]{%
\ifcase#1\relax
\or Dixi\`eme%
\or Vingti\`eme%
\or Trentri\`eme%
\or Quaranti\`eme%
\or Cinquanti\`eme%
\or Soixanti\`eme%
\or Septenti\`eme%
\or Huitanti\`eme%
\or Nonenti\`eme%
\fi
}
%    \end{macrocode}
% Teens:
%    \begin{macrocode}
\newcommand*{\@@Teenthstringfrench}[1]{%
\ifcase#1\relax
Dixi\`eme%
\or Onzi\`eme%
\or Douzi\`eme%
\or Treizi\`eme%
\or Quatorzi\`eme%
\or Quinzi\`eme%
\or Seizi\`eme%
\or Dix-Septi\`eme%
\or Dix-Huiti\`eme%
\or Dix-Neuvi\`eme%
\fi
}
%    \end{macrocode}
% Store textual representation of number (first argument) in given control
% sequence (second argument).
%    \begin{macrocode}
\newcommand*{\@@numberstringfrench}[2]{%
\ifnum#1>99999
\PackageError{fmtcount}{Out of range}%
{This macro only works for values less than 100000}%
\else
\ifnum#1<0
\PackageError{fmtcount}{Negative numbers not permitted}%
{This macro does not work for negative numbers, however
you can try typing "minus" first, and then pass the modulus of
this number}%
\fi
\fi
\def#2{}%
\@strctr=#1\relax \divide\@strctr by 1000\relax
\ifnum\@strctr>9\relax
% #1 is greater or equal to 10000
  \@tmpstrctr=\@strctr
  \divide\@strctr by 10\relax
  \ifnum\@strctr>1\relax
    \ifthenelse{\(\@strctr>6\)\and\(\@strctr<10\)}{%
      \@modulo{\@tmpstrctr}{10}%
      \ifnum\@strctr<8\relax
        \let\@@fc@numstr#2\relax
        \edef#2{\@@fc@numstr\@seventies{\@tmpstrctr}}%
      \else
        \ifnum\@strctr<9\relax
          \let\@@fc@numstr#2\relax
          \edef#2{\@@fc@numstr\@eighties{\@tmpstrctr}}%
        \else
          \ifnum\@strctr<10\relax
             \let\@@fc@numstr#2\relax
             \edef#2{\@@fc@numstr\@nineties{\@tmpstrctr}}%
          \fi
        \fi
      \fi
    }{%
      \let\@@fc@numstr#2\relax
      \edef#2{\@@fc@numstr\@tenstring{\@strctr}}%
      \@strctr=#1\relax
      \divide\@strctr by 1000\relax
      \@modulo{\@strctr}{10}%
      \ifnum\@strctr>0\relax
        \let\@@fc@numstr#2\relax
        \edef#2{\@@fc@numstr\ \@unitstring{\@strctr}}%
      \fi
    }%
  \else
    \@strctr=#1\relax
    \divide\@strctr by 1000
    \@modulo{\@strctr}{10}%
    \let\@@fc@numstr#2\relax
    \edef#2{\@@fc@numstr\@teenstring{\@strctr}}%
  \fi
  \let\@@fc@numstr#2\relax
  \edef#2{\@@fc@numstr\ \@thousand}%
\else
  \ifnum\@strctr>0\relax 
    \ifnum\@strctr>1\relax
      \let\@@fc@numstr#2\relax
      \edef#2{\@@fc@numstr\@unitstring{\@strctr}\ }%
    \fi
    \let\@@fc@numstr#2\relax
    \edef#2{\@@fc@numstr\@thousand}%
  \fi
\fi
\@strctr=#1\relax \@modulo{\@strctr}{1000}%
\divide\@strctr by 100
\ifnum\@strctr>0\relax
  \ifnum#1>1000\relax
    \let\@@fc@numstr#2\relax
    \edef#2{\@@fc@numstr\ }%
  \fi
  \@tmpstrctr=#1\relax
  \@modulo{\@tmpstrctr}{1000}\relax
  \ifnum\@tmpstrctr=100\relax
    \let\@@fc@numstr#2\relax
    \edef#2{\@@fc@numstr\@tenstring{10}}%
  \else
    \ifnum\@strctr>1\relax
      \let\@@fc@numstr#2\relax
      \edef#2{\@@fc@numstr\@unitstring{\@strctr}\ }%
    \fi
    \let\@@fc@numstr#2\relax
    \edef#2{\@@fc@numstr\@hundred}%
  \fi
\fi
\@strctr=#1\relax \@modulo{\@strctr}{100}%
%\@tmpstrctr=#1\relax
%\divide\@tmpstrctr by 100\relax
\ifnum#1>100\relax
  \ifnum\@strctr>0\relax
    \let\@@fc@numstr#2\relax
    \edef#2{\@@fc@numstr\ }%
  \else
    \ifnum\@tmpstrctr>0\relax
       \let\@@fc@numstr#2\relax
       \edef#2{\@@fc@numstr s}%
    \fi%
  \fi
\fi
\ifnum\@strctr>19\relax
  \@tmpstrctr=\@strctr
  \divide\@strctr by 10\relax
  \ifthenelse{\@strctr>6}{%
    \@modulo{\@tmpstrctr}{10}%
    \ifnum\@strctr<8\relax
      \let\@@fc@numstr#2\relax
      \edef#2{\@@fc@numstr\@seventies{\@tmpstrctr}}%
    \else
      \ifnum\@strctr<9\relax
        \let\@@fc@numstr#2\relax
        \edef#2{\@@fc@numstr\@eighties{\@tmpstrctr}}%
      \else
        \let\@@fc@numstr#2\relax
        \edef#2{\@@fc@numstr\@nineties{\@tmpstrctr}}%
      \fi
    \fi
  }{%
    \let\@@fc@numstr#2\relax
    \edef#2{\@@fc@numstr\@tenstring{\@strctr}}%
    \@strctr=#1\relax \@modulo{\@strctr}{10}%
    \ifnum\@strctr>0\relax
      \let\@@fc@numstr#2\relax
      \ifnum\@strctr=1\relax
         \edef#2{\@@fc@numstr\ \@andname\ }%
      \else
         \edef#2{\@@fc@numstr-}%
      \fi
      \let\@@fc@numstr#2\relax
      \edef#2{\@@fc@numstr\@unitstring{\@strctr}}%
    \fi
  }%
\else
  \ifnum\@strctr<10\relax
    \ifnum\@strctr=0\relax
      \ifnum#1<100\relax
        \let\@@fc@numstr#2\relax
        \edef#2{\@@fc@numstr\@unitstring{\@strctr}}%
      \fi
    \else%(>0,<10)
      \let\@@fc@numstr#2\relax
      \edef#2{\@@fc@numstr\@unitstring{\@strctr}}%
    \fi
  \else%>10
    \@modulo{\@strctr}{10}%
    \let\@@fc@numstr#2\relax
    \edef#2{\@@fc@numstr\@teenstring{\@strctr}}%
  \fi
\fi
}
%    \end{macrocode}
% Store textual representation of an ordinal (from number 
% specified in first argument) in given control
% sequence (second argument).
%    \begin{macrocode}
\newcommand*{\@@ordinalstringfrench}[2]{%
\ifnum#1>99999
\PackageError{fmtcount}{Out of range}%
{This macro only works for values less than 100000}%
\else
\ifnum#1<0
\PackageError{fmtcount}{Negative numbers not permitted}%
{This macro does not work for negative numbers, however
you can try typing "minus" first, and then pass the modulus of
this number}%
\fi
\fi
\def#2{}%
\@strctr=#1\relax \divide\@strctr by 1000\relax
\ifnum\@strctr>9
% #1 is greater or equal to 10000
  \@tmpstrctr=\@strctr
  \divide\@strctr by 10\relax
  \ifnum\@strctr>1\relax
    \ifthenelse{\@strctr>6}{%
      \@modulo{\@tmpstrctr}{10}%
      \ifnum\@strctr=7\relax
        \let\@@fc@ordstr#2\relax
        \edef#2{\@@fc@ordstr\@seventies{\@tmpstrctr}}%
      \else
        \ifnum\@strctr=8\relax
          \let\@@fc@ordstr#2\relax
          \edef#2{\@@fc@ordstr\@eighties{\@tmpstrctr}}%
        \else
          \let\@@fc@ordstr#2\relax
          \edef#2{\@@fc@ordstr\@nineties{\@tmpstrctr}}%
        \fi
      \fi
    }{%
      \let\@@fc@ordstr#2\relax
      \edef#2{\@@fc@ordstr\@tenstring{\@strctr}}%
      \@strctr=#1\relax
      \divide\@strctr by 1000\relax
      \@modulo{\@strctr}{10}%
      \ifnum\@strctr=1\relax
         \let\@@fc@ordstr#2\relax
         \edef#2{\@@fc@ordstr\ \@andname}%
      \fi
      \ifnum\@strctr>0\relax
         \let\@@fc@ordstr#2\relax
         \edef#2{\@@fc@ordstr\ \@unitstring{\@strctr}}%
      \fi
    }%
  \else
    \@strctr=#1\relax
    \divide\@strctr by 1000\relax
    \@modulo{\@strctr}{10}%
    \let\@@fc@ordstr#2\relax
    \edef#2{\@@fc@ordstr\@teenstring{\@strctr}}%
  \fi
  \@strctr=#1\relax \@modulo{\@strctr}{1000}%
  \ifnum\@strctr=0\relax
    \let\@@fc@ordstr#2\relax
    \edef#2{\@@fc@ordstr\ \@thousandth}%
  \else
    \let\@@fc@ordstr#2\relax
    \edef#2{\@@fc@ordstr\ \@thousand}%
  \fi
\else
  \ifnum\@strctr>0\relax
    \let\@@fc@ordstr#2\relax
    \edef#2{\@@fc@ordstr\@unitstring{\@strctr}}%
    \@strctr=#1\relax \@modulo{\@strctr}{1000}%
    \ifnum\@strctr=0\relax
      \let\@@fc@ordstr#2\relax
      \edef#2{\@@fc@ordstr\ \@thousandth}%
    \else
      \let\@@fc@ordstr#2\relax
      \edef#2{\@@fc@ordstr\ \@thousand}%
    \fi
  \fi
\fi
\@strctr=#1\relax \@modulo{\@strctr}{1000}%
\divide\@strctr by 100\relax
\ifnum\@strctr>0\relax
  \ifnum#1>1000\relax
    \let\@@fc@ordstr#2\relax
    \edef#2{\@@fc@ordstr\ }%
  \fi
  \let\@@fc@ordstr#2\relax
  \edef#2{\@@fc@ordstr\@unitstring{\@strctr}}%
  \@strctr=#1\relax \@modulo{\@strctr}{100}%
  \let\@@fc@ordstr#2\relax
  \ifnum\@strctr=0\relax
    \edef#2{\@@fc@ordstr\ \@hundredth}%
  \else
    \edef#2{\@@fc@ordstr\ \@hundred}%
  \fi
\fi
\@tmpstrctr=\@strctr
\@strctr=#1\relax \@modulo{\@strctr}{100}%
\ifnum#1>100\relax
  \ifnum\@strctr>0\relax
    \let\@@fc@ordstr#2\relax
    \edef#2{\@@fc@ordstr\ \@andname\ }%
  \fi
\fi
\ifnum\@strctr>19\relax
  \@tmpstrctr=\@strctr
  \divide\@strctr by 10\relax
  \@modulo{\@tmpstrctr}{10}%
  \ifthenelse{\@strctr>6}{%
    \ifnum\@strctr=7\relax
      \let\@@fc@ordstr#2\relax
      \edef#2{\@@fc@ordstr\@seventieths{\@tmpstrctr}}%
    \else
      \ifnum\@strctr=8\relax
        \let\@@fc@ordstr#2\relax
        \edef#2{\@@fc@ordstr\@eightieths{\@tmpstrctr}}%
      \else
        \let\@@fc@ordstr#2\relax
        \edef#2{\@@fc@ordstr\@ninetieths{\@tmpstrctr}}%
      \fi
    \fi
  }{%
    \ifnum\@tmpstrctr=0\relax
      \let\@@fc@ordstr#2\relax
      \edef#2{\@@fc@ordstr\@tenthstring{\@strctr}}%
    \else 
      \let\@@fc@ordstr#2\relax
      \edef#2{\@@fc@ordstr\@tenstring{\@strctr}}%
    \fi
    \@strctr=#1\relax \@modulo{\@strctr}{10}%
    \ifnum\@strctr=1\relax
      \let\@@fc@ordstr#2\relax
      \edef#2{\@@fc@ordstr\ \@andname}%
    \fi
    \ifnum\@strctr>0\relax
      \let\@@fc@ordstr#2\relax
      \edef#2{\@@fc@ordstr\ \@unitthstring{\@strctr}}%
    \fi
  }%
\else
  \ifnum\@strctr<10\relax
    \ifnum\@strctr=0\relax
      \ifnum#1<100\relax
        \let\@@fc@ordstr#2\relax
        \edef#2{\@@fc@ordstr\@unitthstring{\@strctr}}%
      \fi
    \else
      \let\@@fc@ordstr#2\relax
      \edef#2{\@@fc@ordstr\@unitthstring{\@strctr}}%
    \fi
  \else
    \@modulo{\@strctr}{10}%
    \let\@@fc@ordstr#2\relax
    \edef#2{\@@fc@ordstr\@teenthstring{\@strctr}}%
  \fi
\fi
}
%    \end{macrocode}
%\iffalse
%    \begin{macrocode}
%</fc-french.def>
%    \end{macrocode}
%\fi
%\iffalse
%    \begin{macrocode}
%<*fc-german.def>
%    \end{macrocode}
%\fi
% \subsection{fc-german.def}
% German definitions (thank you to K. H. Fricke for supplying
% this information)
%    \begin{macrocode}
\ProvidesFile{fc-german.def}[2007/06/14]
%    \end{macrocode}
% Define macro that converts a number or count register (first
% argument) to an ordinal, and stores the result in the
% second argument, which must be a control sequence.
% Masculine:
%    \begin{macrocode}
\newcommand{\@ordinalMgerman}[2]{%
\edef#2{\number#1\relax.}}
%    \end{macrocode}
% Feminine:
%    \begin{macrocode}
\newcommand{\@ordinalFgerman}[2]{%
\edef#2{\number#1\relax.}}
%    \end{macrocode}
% Neuter:
%    \begin{macrocode}
\newcommand{\@ordinalNgerman}[2]{%
\edef#2{\number#1\relax.}}
%    \end{macrocode}
% Convert a number to text. The easiest way to do this is to
% break it up into units, tens and teens.
% Units (argument must be a number from 0 to 9, 1 on its own (eins)
% is dealt with separately):
%    \begin{macrocode}
\newcommand{\@@unitstringgerman}[1]{%
\ifcase#1%
null%
\or ein%
\or zwei%
\or drei%
\or vier%
\or f\"unf%
\or sechs%
\or sieben%
\or acht%
\or neun%
\fi
}
%    \end{macrocode}
% Tens (argument must go from 1 to 10):
%    \begin{macrocode}
\newcommand{\@@tenstringgerman}[1]{%
\ifcase#1%
\or zehn%
\or zwanzig%
\or drei{\ss}ig%
\or vierzig%
\or f\"unfzig%
\or sechzig%
\or siebzig%
\or achtzig%
\or neunzig%
\or einhundert%
\fi
}
%    \end{macrocode}
% |\einhundert| is set to |einhundert| by default, user can
% redefine this command to just |hundert| if required, similarly
% for |\eintausend|.
%    \begin{macrocode}
\providecommand*{\einhundert}{einhundert}
\providecommand*{\eintausend}{eintausend}
%    \end{macrocode}
% Teens:
%    \begin{macrocode}
\newcommand{\@@teenstringgerman}[1]{%
\ifcase#1%
zehn%
\or elf%
\or zw\"olf%
\or dreizehn%
\or vierzehn%
\or f\"unfzehn%
\or sechzehn%
\or siebzehn%
\or achtzehn%
\or neunzehn%
\fi
}
%    \end{macrocode}
% The results are stored in the second argument, but doesn't 
% display anything.
%    \begin{macrocode}
\DeclareRobustCommand{\@numberstringMgerman}[2]{%
\let\@unitstring=\@@unitstringgerman
\let\@teenstring=\@@teenstringgerman
\let\@tenstring=\@@tenstringgerman
\@@numberstringgerman{#1}{#2}}
%    \end{macrocode}
% Feminine and neuter forms:
%    \begin{macrocode}
\let\@numberstringFgerman=\@numberstringMgerman
\let\@numberstringNgerman=\@numberstringMgerman
%    \end{macrocode}
% As above, but initial letters in upper case:
%    \begin{macrocode}
\DeclareRobustCommand{\@NumberstringMgerman}[2]{%
\@numberstringMgerman{#1}{\@@num@str}%
\edef#2{\noexpand\MakeUppercase\@@num@str}}
%    \end{macrocode}
% Feminine and neuter form:
%    \begin{macrocode}
\let\@NumberstringFgerman=\@NumberstringMgerman
\let\@NumberstringNgerman=\@NumberstringMgerman
%    \end{macrocode}
% As above, but for ordinals.
%    \begin{macrocode}
\DeclareRobustCommand{\@ordinalstringMgerman}[2]{%
\let\@unitthstring=\@@unitthstringMgerman
\let\@teenthstring=\@@teenthstringMgerman
\let\@tenthstring=\@@tenthstringMgerman
\let\@unitstring=\@@unitstringgerman
\let\@teenstring=\@@teenstringgerman
\let\@tenstring=\@@tenstringgerman
\def\@thousandth{tausendster}%
\def\@hundredth{hundertster}%
\@@ordinalstringgerman{#1}{#2}}
%    \end{macrocode}
% Feminine form:
%    \begin{macrocode}
\DeclareRobustCommand{\@ordinalstringFgerman}[2]{%
\let\@unitthstring=\@@unitthstringFgerman
\let\@teenthstring=\@@teenthstringFgerman
\let\@tenthstring=\@@tenthstringFgerman
\let\@unitstring=\@@unitstringgerman
\let\@teenstring=\@@teenstringgerman
\let\@tenstring=\@@tenstringgerman
\def\@thousandth{tausendste}%
\def\@hundredth{hundertste}%
\@@ordinalstringgerman{#1}{#2}}
%    \end{macrocode}
% Neuter form:
%    \begin{macrocode}
\DeclareRobustCommand{\@ordinalstringNgerman}[2]{%
\let\@unitthstring=\@@unitthstringNgerman
\let\@teenthstring=\@@teenthstringNgerman
\let\@tenthstring=\@@tenthstringNgerman
\let\@unitstring=\@@unitstringgerman
\let\@teenstring=\@@teenstringgerman
\let\@tenstring=\@@tenstringgerman
\def\@thousandth{tausendstes}%
\def\@hundredth{hunderstes}%
\@@ordinalstringgerman{#1}{#2}}
%    \end{macrocode}
% As above, but with initial letters in upper case.
%    \begin{macrocode}
\DeclareRobustCommand{\@OrdinalstringMgerman}[2]{%
\@ordinalstringMgerman{#1}{\@@num@str}%
\edef#2{\protect\MakeUppercase\@@num@str}}
%    \end{macrocode}
% Feminine form:
%    \begin{macrocode}
\DeclareRobustCommand{\@OrdinalstringFgerman}[2]{%
\@ordinalstringFgerman{#1}{\@@num@str}%
\edef#2{\protect\MakeUppercase\@@num@str}}
%    \end{macrocode}
% Neuter form:
%    \begin{macrocode}
\DeclareRobustCommand{\@OrdinalstringNgerman}[2]{%
\@ordinalstringNgerman{#1}{\@@num@str}%
\edef#2{\protect\MakeUppercase\@@num@str}}
%    \end{macrocode}
% Code for converting numbers into textual ordinals. As before,
% it is easier to split it into units, tens and teens.
% Units:
%    \begin{macrocode}
\newcommand{\@@unitthstringMgerman}[1]{%
\ifcase#1%
nullter%
\or erster%
\or zweiter%
\or dritter%
\or vierter%
\or f\"unter%
\or sechster%
\or siebter%
\or achter%
\or neunter%
\fi
}
%    \end{macrocode}
% Tens:
%    \begin{macrocode}
\newcommand{\@@tenthstringMgerman}[1]{%
\ifcase#1%
\or zehnter%
\or zwanzigster%
\or drei{\ss}igster%
\or vierzigster%
\or f\"unfzigster%
\or sechzigster%
\or siebzigster%
\or achtzigster%
\or neunzigster%
\fi
}
%    \end{macrocode}
% Teens:
%    \begin{macrocode}
\newcommand{\@@teenthstringMgerman}[1]{%
\ifcase#1%
zehnter%
\or elfter%
\or zw\"olfter%
\or dreizehnter%
\or vierzehnter%
\or f\"unfzehnter%
\or sechzehnter%
\or siebzehnter%
\or achtzehnter%
\or neunzehnter%
\fi
}
%    \end{macrocode}
% Units (feminine):
%    \begin{macrocode}
\newcommand{\@@unitthstringFgerman}[1]{%
\ifcase#1%
nullte%
\or erste%
\or zweite%
\or dritte%
\or vierte%
\or f\"unfte%
\or sechste%
\or siebte%
\or achte%
\or neunte%
\fi
}
%    \end{macrocode}
% Tens (feminine):
%    \begin{macrocode}
\newcommand{\@@tenthstringFgerman}[1]{%
\ifcase#1%
\or zehnte%
\or zwanzigste%
\or drei{\ss}igste%
\or vierzigste%
\or f\"unfzigste%
\or sechzigste%
\or siebzigste%
\or achtzigste%
\or neunzigste%
\fi
}
%    \end{macrocode}
% Teens (feminine)
%    \begin{macrocode}
\newcommand{\@@teenthstringFgerman}[1]{%
\ifcase#1%
zehnte%
\or elfte%
\or zw\"olfte%
\or dreizehnte%
\or vierzehnte%
\or f\"unfzehnte%
\or sechzehnte%
\or siebzehnte%
\or achtzehnte%
\or neunzehnte%
\fi
}
%    \end{macrocode}
% Units (neuter):
%    \begin{macrocode}
\newcommand{\@@unitthstringNgerman}[1]{%
\ifcase#1%
nulltes%
\or erstes%
\or zweites%
\or drittes%
\or viertes%
\or f\"unte%
\or sechstes%
\or siebtes%
\or achtes%
\or neuntes%
\fi
}
%    \end{macrocode}
% Tens (neuter):
%    \begin{macrocode}
\newcommand{\@@tenthstringNgerman}[1]{%
\ifcase#1%
\or zehntes%
\or zwanzigstes%
\or drei{\ss}igstes%
\or vierzigstes%
\or f\"unfzigstes%
\or sechzigstes%
\or siebzigstes%
\or achtzigstes%
\or neunzigstes%
\fi
}
%    \end{macrocode}
% Teens (neuter)
%    \begin{macrocode}
\newcommand{\@@teenthstringNgerman}[1]{%
\ifcase#1%
zehntes%
\or elftes%
\or zw\"olftes%
\or dreizehntes%
\or vierzehntes%
\or f\"unfzehntes%
\or sechzehntes%
\or siebzehntes%
\or achtzehntes%
\or neunzehntes%
\fi
}
%    \end{macrocode}
% This appends the results to |#2| for number |#2| (in range 0 to 100.)
% null and eins are dealt with separately in |\@@numberstringgerman|.
%    \begin{macrocode}
\newcommand{\@@numberunderhundredgerman}[2]{%
\ifnum#1<10\relax
  \ifnum#1>0\relax
    \let\@@fc@numstr#2\relax
    \edef#2{\@@fc@numstr\@unitstring{#1}}%
  \fi
\else
  \@tmpstrctr=#1\relax
  \@modulo{\@tmpstrctr}{10}%
  \ifnum#1<20\relax
    \let\@@fc@numstr#2\relax
    \edef#2{\@@fc@numstr\@teenstring{\@tmpstrctr}}%
  \else
    \ifnum\@tmpstrctr=0\relax
    \else
      \let\@@fc@numstr#2\relax
      \edef#2{\@@fc@numstr\@unitstring{\@tmpstrctr}und}%
    \fi
    \@tmpstrctr=#1\relax
    \divide\@tmpstrctr by 10\relax
    \let\@@fc@numstr#2\relax
    \edef#2{\@@fc@numstr\@tenstring{\@tmpstrctr}}%
  \fi
\fi
}
%    \end{macrocode}
% This stores the results in the second argument 
% (which must be a control
% sequence), but it doesn't display anything.
%    \begin{macrocode}
\newcommand{\@@numberstringgerman}[2]{%
\ifnum#1>99999\relax
  \PackageError{fmtcount}{Out of range}%
  {This macro only works for values less than 100000}%
\else
  \ifnum#1<0\relax
    \PackageError{fmtcount}{Negative numbers not permitted}%
    {This macro does not work for negative numbers, however
    you can try typing "minus" first, and then pass the modulus of
    this number}%
  \fi
\fi
\def#2{}%
\@strctr=#1\relax \divide\@strctr by 1000\relax
\ifnum\@strctr>1\relax
% #1 is >= 2000, \@strctr now contains the number of thousands
\@@numberunderhundredgerman{\@strctr}{#2}%
  \let\@@fc@numstr#2\relax
  \edef#2{\@@fc@numstr tausend}%
\else
% #1 lies in range [1000,1999]
  \ifnum\@strctr=1\relax
    \let\@@fc@numstr#2\relax
    \edef#2{\@@fc@numstr\eintausend}%
  \fi
\fi
\@strctr=#1\relax
\@modulo{\@strctr}{1000}%
\divide\@strctr by 100\relax
\ifnum\@strctr>1\relax
% now dealing with number in range [200,999]
  \let\@@fc@numstr#2\relax
  \edef#2{\@@fc@numstr\@unitstring{\@strctr}hundert}%
\else
   \ifnum\@strctr=1\relax
% dealing with number in range [100,199]
     \ifnum#1>1000\relax
% if orginal number > 1000, use einhundert
        \let\@@fc@numstr#2\relax
        \edef#2{\@@fc@numstr einhundert}%
     \else
% otherwise use \einhundert
        \let\@@fc@numstr#2\relax
        \edef#2{\@@fc@numstr\einhundert}%
      \fi
   \fi
\fi
\@strctr=#1\relax
\@modulo{\@strctr}{100}%
\ifnum#1=0\relax
  \def#2{null}%
\else
  \ifnum\@strctr=1\relax
    \let\@@fc@numstr#2\relax
    \edef#2{\@@fc@numstr eins}%
  \else
    \@@numberunderhundredgerman{\@strctr}{#2}%
  \fi
\fi
}
%    \end{macrocode}
% As above, but for ordinals
%    \begin{macrocode}
\newcommand{\@@numberunderhundredthgerman}[2]{%
\ifnum#1<10\relax
 \let\@@fc@numstr#2\relax
 \edef#2{\@@fc@numstr\@unitthstring{#1}}%
\else
  \@tmpstrctr=#1\relax
  \@modulo{\@tmpstrctr}{10}%
  \ifnum#1<20\relax
    \let\@@fc@numstr#2\relax
    \edef#2{\@@fc@numstr\@teenthstring{\@tmpstrctr}}%
  \else
    \ifnum\@tmpstrctr=0\relax
    \else
      \let\@@fc@numstr#2\relax
      \edef#2{\@@fc@numstr\@unitstring{\@tmpstrctr}und}%
    \fi
    \@tmpstrctr=#1\relax
    \divide\@tmpstrctr by 10\relax
    \let\@@fc@numstr#2\relax
    \edef#2{\@@fc@numstr\@tenthstring{\@tmpstrctr}}%
  \fi
\fi
}
%    \end{macrocode}
%    \begin{macrocode}
\newcommand{\@@ordinalstringgerman}[2]{%
\ifnum#1>99999\relax
  \PackageError{fmtcount}{Out of range}%
  {This macro only works for values less than 100000}%
\else
  \ifnum#1<0\relax
    \PackageError{fmtcount}{Negative numbers not permitted}%
    {This macro does not work for negative numbers, however
    you can try typing "minus" first, and then pass the modulus of
    this number}%
  \fi
\fi
\def#2{}%
\@strctr=#1\relax \divide\@strctr by 1000\relax
\ifnum\@strctr>1\relax
% #1 is >= 2000, \@strctr now contains the number of thousands
\@@numberunderhundredgerman{\@strctr}{#2}%
  \let\@@fc@numstr#2\relax
  % is that it, or is there more?
  \@tmpstrctr=#1\relax \@modulo{\@tmpstrctr}{1000}%
  \ifnum\@tmpstrctr=0\relax
    \edef#2{\@@fc@numstr\@thousandth}%
  \else
    \edef#2{\@@fc@numstr tausend}%
  \fi
\else
% #1 lies in range [1000,1999]
  \ifnum\@strctr=1\relax
    \ifnum#1=1000\relax
      \let\@@fc@numstr#2\relax
      \edef#2{\@@fc@numstr\@thousandth}%
    \else
      \let\@@fc@numstr#2\relax
      \edef#2{\@@fc@numstr\eintausend}%
    \fi
  \fi
\fi
\@strctr=#1\relax
\@modulo{\@strctr}{1000}%
\divide\@strctr by 100\relax
\ifnum\@strctr>1\relax
% now dealing with number in range [200,999]
  \let\@@fc@numstr#2\relax
  % is that it, or is there more?
  \@tmpstrctr=#1\relax \@modulo{\@tmpstrctr}{100}%
  \ifnum\@tmpstrctr=0\relax
     \ifnum\@strctr=1\relax
       \edef#2{\@@fc@numstr\@hundredth}%
     \else
       \edef#2{\@@fc@numstr\@unitstring{\@strctr}\@hundredth}%
     \fi
  \else
     \edef#2{\@@fc@numstr\@unitstring{\@strctr}hundert}%
  \fi
\else
   \ifnum\@strctr=1\relax
% dealing with number in range [100,199]
% is that it, or is there more?
     \@tmpstrctr=#1\relax \@modulo{\@tmpstrctr}{100}%
     \ifnum\@tmpstrctr=0\relax
        \let\@@fc@numstr#2\relax
        \edef#2{\@@fc@numstr\@hundredth}%
     \else
     \ifnum#1>1000\relax
        \let\@@fc@numstr#2\relax
        \edef#2{\@@fc@numstr einhundert}%
     \else
        \let\@@fc@numstr#2\relax
        \edef#2{\@@fc@numstr\einhundert}%
     \fi
     \fi
   \fi
\fi
\@strctr=#1\relax
\@modulo{\@strctr}{100}%
\ifthenelse{\@strctr=0 \and #1>0}{}{%
\@@numberunderhundredthgerman{\@strctr}{#2}%
}%
}
%    \end{macrocode}
% Set |ngerman| to be equivalent to |german|. Is it okay to do
% this? (I don't know the difference between the two.)
%    \begin{macrocode}
\let\@ordinalMngerman=\@ordinalMgerman
\let\@ordinalFngerman=\@ordinalFgerman
\let\@ordinalNngerman=\@ordinalNgerman
\let\@numberstringMngerman=\@numberstringMgerman
\let\@numberstringFngerman=\@numberstringFgerman
\let\@numberstringNngerman=\@numberstringNgerman
\let\@NumberstringMngerman=\@NumberstringMgerman
\let\@NumberstringFngerman=\@NumberstringFgerman
\let\@NumberstringNngerman=\@NumberstringNgerman
\let\@ordinalstringMngerman=\@ordinalstringMgerman
\let\@ordinalstringFngerman=\@ordinalstringFgerman
\let\@ordinalstringNngerman=\@ordinalstringNgerman
\let\@OrdinalstringMngerman=\@OrdinalstringMgerman
\let\@OrdinalstringFngerman=\@OrdinalstringFgerman
\let\@OrdinalstringNngerman=\@OrdinalstringNgerman
%    \end{macrocode}
%\iffalse
%    \begin{macrocode}
%</fc-german.def>
%    \end{macrocode}
%\fi
%\iffalse
%    \begin{macrocode}
%<*fc-portuges.def>
%    \end{macrocode}
%\fi
% \subsection{fc-portuges.def}
% Portuguse definitions
%    \begin{macrocode}
\ProvidesFile{fc-portuges.def}[2007/05/26]
%    \end{macrocode}
% Define macro that converts a number or count register (first
% argument) to an ordinal, and stores the result in the second
% argument, which should be a control sequence. Masculine:
%    \begin{macrocode}
\newcommand*{\@ordinalMportuges}[2]{%
\ifnum#1=0\relax
  \edef#2{\number#1}%
\else
  \edef#2{\number#1\relax\noexpand\fmtord{o}}%
\fi}
%    \end{macrocode}
% Feminine:
%    \begin{macrocode}
\newcommand*{\@ordinalFportuges}[2]{%
\ifnum#1=0\relax
  \edef#2{\number#1}%
\else
  \edef#2{\number#1\relax\noexpand\fmtord{a}}%
\fi}
%    \end{macrocode}
% Make neuter same as masculine:
%    \begin{macrocode}
\let\@ordinalNportuges\@ordinalMportuges
%    \end{macrocode}
% Convert a number to a textual representation. To make it easier,
% split it up into units, tens, teens and hundreds. Units (argument must
% be a number from 0 to 9):
%    \begin{macrocode}
\newcommand*{\@@unitstringportuges}[1]{%
\ifcase#1\relax
zero%
\or um%
\or dois%
\or tr\^es%
\or quatro%
\or cinco%
\or seis%
\or sete%
\or oito%
\or nove%
\fi
}
%   \end{macrocode}
% As above, but for feminine:
%   \begin{macrocode}
\newcommand*{\@@unitstringFportuges}[1]{%
\ifcase#1\relax
zero%
\or uma%
\or duas%
\or tr\^es%
\or quatro%
\or cinco%
\or seis%
\or sete%
\or oito%
\or nove%
\fi
}
%    \end{macrocode}
% Tens (argument must be a number from 0 to 10):
%    \begin{macrocode}
\newcommand*{\@@tenstringportuges}[1]{%
\ifcase#1\relax
\or dez%
\or vinte%
\or trinta%
\or quarenta%
\or cinq\"uenta%
\or sessenta%
\or setenta%
\or oitenta%
\or noventa%
\or cem%
\fi
}
%    \end{macrocode}
% Teens (argument must be a number from 0 to 9):
%    \begin{macrocode}
\newcommand*{\@@teenstringportuges}[1]{%
\ifcase#1\relax
dez%
\or onze%
\or doze%
\or treze%
\or quatorze%
\or quinze%
\or dezesseis%
\or dezessete%
\or dezoito%
\or dezenove%
\fi
}
%    \end{macrocode}
% Hundreds:
%    \begin{macrocode}
\newcommand*{\@@hundredstringportuges}[1]{%
\ifcase#1\relax
\or cento%
\or duzentos%
\or trezentos%
\or quatrocentos%
\or quinhentos%
\or seiscentos%
\or setecentos%
\or oitocentos%
\or novecentos%
\fi}
%    \end{macrocode}
% Hundreds (feminine):
%    \begin{macrocode}
\newcommand*{\@@hundredstringFportuges}[1]{%
\ifcase#1\relax
\or cento%
\or duzentas%
\or trezentas%
\or quatrocentas%
\or quinhentas%
\or seiscentas%
\or setecentas%
\or oitocentas%
\or novecentas%
\fi}
%    \end{macrocode}
% Units (initial letter in upper case):
%    \begin{macrocode}
\newcommand*{\@@Unitstringportuges}[1]{%
\ifcase#1\relax
Zero%
\or Um%
\or Dois%
\or Tr\^es%
\or Quatro%
\or Cinco%
\or Seis%
\or Sete%
\or Oito%
\or Nove%
\fi
}
%    \end{macrocode}
% As above, but feminine:
%    \begin{macrocode}
\newcommand*{\@@UnitstringFportuges}[1]{%
\ifcase#1\relax
Zera%
\or Uma%
\or Duas%
\or Tr\^es%
\or Quatro%
\or Cinco%
\or Seis%
\or Sete%
\or Oito%
\or Nove%
\fi
}
%    \end{macrocode}
% Tens (with initial letter in upper case):
%    \begin{macrocode}
\newcommand*{\@@Tenstringportuges}[1]{%
\ifcase#1\relax
\or Dez%
\or Vinte%
\or Trinta%
\or Quarenta%
\or Cinq\"uenta%
\or Sessenta%
\or Setenta%
\or Oitenta%
\or Noventa%
\or Cem%
\fi
}
%    \end{macrocode}
% Teens (with initial letter in upper case):
%    \begin{macrocode}
\newcommand*{\@@Teenstringportuges}[1]{%
\ifcase#1\relax
Dez%
\or Onze%
\or Doze%
\or Treze%
\or Quatorze%
\or Quinze%
\or Dezesseis%
\or Dezessete%
\or Dezoito%
\or Dezenove%
\fi
}
%    \end{macrocode}
% Hundreds (with initial letter in upper case):
%    \begin{macrocode}
\newcommand*{\@@Hundredstringportuges}[1]{%
\ifcase#1\relax
\or Cento%
\or Duzentos%
\or Trezentos%
\or Quatrocentos%
\or Quinhentos%
\or Seiscentos%
\or Setecentos%
\or Oitocentos%
\or Novecentos%
\fi}
%    \end{macrocode}
% As above, but feminine:
%    \begin{macrocode}
\newcommand*{\@@HundredstringFportuges}[1]{%
\ifcase#1\relax
\or Cento%
\or Duzentas%
\or Trezentas%
\or Quatrocentas%
\or Quinhentas%
\or Seiscentas%
\or Setecentas%
\or Oitocentas%
\or Novecentas%
\fi}
%    \end{macrocode}
% This has changed in version 1.08, so that it now stores
% the result in the second argument, but doesn't display
% anything. Since it only affects internal macros, it shouldn't
% affect documents created with older versions. (These internal
% macros are not meant for use in documents.)
%    \begin{macrocode}
\DeclareRobustCommand{\@numberstringMportuges}[2]{%
\let\@unitstring=\@@unitstringportuges
\let\@teenstring=\@@teenstringportuges
\let\@tenstring=\@@tenstringportuges
\let\@hundredstring=\@@hundredstringportuges
\def\@hundred{cem}\def\@thousand{mil}%
\def\@andname{e}%
\@@numberstringportuges{#1}{#2}}
%    \end{macrocode}
% As above, but feminine form:
%    \begin{macrocode}
\DeclareRobustCommand{\@numberstringFportuges}[2]{%
\let\@unitstring=\@@unitstringFportuges
\let\@teenstring=\@@teenstringportuges
\let\@tenstring=\@@tenstringportuges
\let\@hundredstring=\@@hundredstringFportuges
\def\@hundred{cem}\def\@thousand{mil}%
\def\@andname{e}%
\@@numberstringportuges{#1}{#2}}
%    \end{macrocode}
% Make neuter same as masculine:
%    \begin{macrocode}
\let\@numberstringNportuges\@numberstringMportuges
%    \end{macrocode}
% As above, but initial letters in upper case:
%    \begin{macrocode}
\DeclareRobustCommand{\@NumberstringMportuges}[2]{%
\let\@unitstring=\@@Unitstringportuges
\let\@teenstring=\@@Teenstringportuges
\let\@tenstring=\@@Tenstringportuges
\let\@hundredstring=\@@Hundredstringportuges
\def\@hundred{Cem}\def\@thousand{Mil}%
\def\@andname{e}%
\@@numberstringportuges{#1}{#2}}
%    \end{macrocode}
% As above, but feminine form:
%    \begin{macrocode}
\DeclareRobustCommand{\@NumberstringFportuges}[2]{%
\let\@unitstring=\@@UnitstringFportuges
\let\@teenstring=\@@Teenstringportuges
\let\@tenstring=\@@Tenstringportuges
\let\@hundredstring=\@@HundredstringFportuges
\def\@hundred{Cem}\def\@thousand{Mil}%
\def\@andname{e}%
\@@numberstringportuges{#1}{#2}}
%    \end{macrocode}
% Make neuter same as masculine:
%    \begin{macrocode}
\let\@NumberstringNportuges\@NumberstringMportuges
%    \end{macrocode}
% As above, but for ordinals.
%    \begin{macrocode}
\DeclareRobustCommand{\@ordinalstringMportuges}[2]{%
\let\@unitthstring=\@@unitthstringportuges
\let\@unitstring=\@@unitstringportuges
\let\@teenthstring=\@@teenthstringportuges
\let\@tenthstring=\@@tenthstringportuges
\let\@hundredthstring=\@@hundredthstringportuges
\def\@thousandth{mil\'esimo}%
\@@ordinalstringportuges{#1}{#2}}
%    \end{macrocode}
% Feminine form:
%    \begin{macrocode}
\DeclareRobustCommand{\@ordinalstringFportuges}[2]{%
\let\@unitthstring=\@@unitthstringFportuges
\let\@unitstring=\@@unitstringFportuges
\let\@teenthstring=\@@teenthstringportuges
\let\@tenthstring=\@@tenthstringFportuges
\let\@hundredthstring=\@@hundredthstringFportuges
\def\@thousandth{mil\'esima}%
\@@ordinalstringportuges{#1}{#2}}
%    \end{macrocode}
% Make neuter same as masculine:
%    \begin{macrocode}
\let\@ordinalstringNportuges\@ordinalstringMportuges
%    \end{macrocode}
% As above, but initial letters in upper case (masculine):
%    \begin{macrocode}
\DeclareRobustCommand{\@OrdinalstringMportuges}[2]{%
\let\@unitthstring=\@@Unitthstringportuges
\let\@unitstring=\@@Unitstringportuges
\let\@teenthstring=\@@teenthstringportuges
\let\@tenthstring=\@@Tenthstringportuges
\let\@hundredthstring=\@@Hundredthstringportuges
\def\@thousandth{Mil\'esimo}%
\@@ordinalstringportuges{#1}{#2}}
%    \end{macrocode}
% Feminine form:
%    \begin{macrocode}
\DeclareRobustCommand{\@OrdinalstringFportuges}[2]{%
\let\@unitthstring=\@@UnitthstringFportuges
\let\@unitstring=\@@UnitstringFportuges
\let\@teenthstring=\@@teenthstringportuges
\let\@tenthstring=\@@TenthstringFportuges
\let\@hundredthstring=\@@HundredthstringFportuges
\def\@thousandth{Mil\'esima}%
\@@ordinalstringportuges{#1}{#2}}
%    \end{macrocode}
% Make neuter same as masculine:
%    \begin{macrocode}
\let\@OrdinalstringNportuges\@OrdinalstringMportuges
%    \end{macrocode}
% In order to do the ordinals, split into units, teens, tens
% and hundreds. Units:
%    \begin{macrocode}
\newcommand*{\@@unitthstringportuges}[1]{%
\ifcase#1\relax
zero%
\or primeiro%
\or segundo%
\or terceiro%
\or quarto%
\or quinto%
\or sexto%
\or s\'etimo%
\or oitavo%
\or nono%
\fi
}
%    \end{macrocode}
% Tens:
%    \begin{macrocode}
\newcommand*{\@@tenthstringportuges}[1]{%
\ifcase#1\relax
\or d\'ecimo%
\or vig\'esimo%
\or trig\'esimo%
\or quadrag\'esimo%
\or q\"uinquag\'esimo%
\or sexag\'esimo%
\or setuag\'esimo%
\or octog\'esimo%
\or nonag\'esimo%
\fi
}
%    \end{macrocode}
% Teens:
%    \begin{macrocode}
\newcommand*{\@@teenthstringportuges}[1]{%
\@tenthstring{1}%
\ifnum#1>0\relax
-\@unitthstring{#1}%
\fi}
%    \end{macrocode}
% Hundreds:
%    \begin{macrocode}
\newcommand*{\@@hundredthstringportuges}[1]{%
\ifcase#1\relax
\or cent\'esimo%
\or ducent\'esimo%
\or trecent\'esimo%
\or quadringent\'esimo%
\or q\"uingent\'esimo%
\or seiscent\'esimo%
\or setingent\'esimo%
\or octingent\'esimo%
\or nongent\'esimo%
\fi}
%    \end{macrocode}
% Units (feminine):
%    \begin{macrocode}
\newcommand*{\@@unitthstringFportuges}[1]{%
\ifcase#1\relax
zero%
\or primeira%
\or segunda%
\or terceira%
\or quarta%
\or quinta%
\or sexta%
\or s\'etima%
\or oitava%
\or nona%
\fi
}
%    \end{macrocode}
% Tens (feminine):
%    \begin{macrocode}
\newcommand*{\@@tenthstringFportuges}[1]{%
\ifcase#1\relax
\or d\'ecima%
\or vig\'esima%
\or trig\'esima%
\or quadrag\'esima%
\or q\"uinquag\'esima%
\or sexag\'esima%
\or setuag\'esima%
\or octog\'esima%
\or nonag\'esima%
\fi
}
%    \end{macrocode}
% Hundreds (feminine):
%    \begin{macrocode}
\newcommand*{\@@hundredthstringFportuges}[1]{%
\ifcase#1\relax
\or cent\'esima%
\or ducent\'esima%
\or trecent\'esima%
\or quadringent\'esima%
\or q\"uingent\'esima%
\or seiscent\'esima%
\or setingent\'esima%
\or octingent\'esima%
\or nongent\'esima%
\fi}
%    \end{macrocode}
% As above, but with initial letter in upper case. Units:
%    \begin{macrocode}
\newcommand*{\@@Unitthstringportuges}[1]{%
\ifcase#1\relax
Zero%
\or Primeiro%
\or Segundo%
\or Terceiro%
\or Quarto%
\or Quinto%
\or Sexto%
\or S\'etimo%
\or Oitavo%
\or Nono%
\fi
}
%    \end{macrocode}
% Tens:
%    \begin{macrocode}
\newcommand*{\@@Tenthstringportuges}[1]{%
\ifcase#1\relax
\or D\'ecimo%
\or Vig\'esimo%
\or Trig\'esimo%
\or Quadrag\'esimo%
\or Q\"uinquag\'esimo%
\or Sexag\'esimo%
\or Setuag\'esimo%
\or Octog\'esimo%
\or Nonag\'esimo%
\fi
}
%    \end{macrocode}
% Hundreds:
%    \begin{macrocode}
\newcommand*{\@@Hundredthstringportuges}[1]{%
\ifcase#1\relax
\or Cent\'esimo%
\or Ducent\'esimo%
\or Trecent\'esimo%
\or Quadringent\'esimo%
\or Q\"uingent\'esimo%
\or Seiscent\'esimo%
\or Setingent\'esimo%
\or Octingent\'esimo%
\or Nongent\'esimo%
\fi}
%    \end{macrocode}
% As above, but feminine. Units:
%    \begin{macrocode}
\newcommand*{\@@UnitthstringFportuges}[1]{%
\ifcase#1\relax
Zera%
\or Primeira%
\or Segunda%
\or Terceira%
\or Quarta%
\or Quinta%
\or Sexta%
\or S\'etima%
\or Oitava%
\or Nona%
\fi
}
%    \end{macrocode}
% Tens (feminine);
%    \begin{macrocode}
\newcommand*{\@@TenthstringFportuges}[1]{%
\ifcase#1\relax
\or D\'ecima%
\or Vig\'esima%
\or Trig\'esima%
\or Quadrag\'esima%
\or Q\"uinquag\'esima%
\or Sexag\'esima%
\or Setuag\'esima%
\or Octog\'esima%
\or Nonag\'esima%
\fi
}
%    \end{macrocode}
% Hundreds (feminine):
%    \begin{macrocode}
\newcommand*{\@@HundredthstringFportuges}[1]{%
\ifcase#1\relax
\or Cent\'esima%
\or Ducent\'esima%
\or Trecent\'esima%
\or Quadringent\'esima%
\or Q\"uingent\'esima%
\or Seiscent\'esima%
\or Setingent\'esima%
\or Octingent\'esima%
\or Nongent\'esima%
\fi}
%    \end{macrocode}
% This has changed in version 1.09, so that it now stores
% the result in the second argument (a control sequence), but it
% doesn't display anything. Since it only affects internal macros,
% it shouldn't affect documents created with older versions.
% (These internal macros are not meant for use in documents.)
%    \begin{macrocode}
\newcommand*{\@@numberstringportuges}[2]{%
\ifnum#1>99999
\PackageError{fmtcount}{Out of range}%
{This macro only works for values less than 100000}%
\else
\ifnum#1<0
\PackageError{fmtcount}{Negative numbers not permitted}%
{This macro does not work for negative numbers, however
you can try typing "minus" first, and then pass the modulus of
this number}%
\fi
\fi
\def#2{}%
\@strctr=#1\relax \divide\@strctr by 1000\relax
\ifnum\@strctr>9
% #1 is greater or equal to 10000
  \divide\@strctr by 10
  \ifnum\@strctr>1\relax
    \let\@@fc@numstr#2\relax
    \edef#2{\@@fc@numstr\@tenstring{\@strctr}}%
    \@strctr=#1 \divide\@strctr by 1000\relax
    \@modulo{\@strctr}{10}%
    \ifnum\@strctr>0
      \ifnum\@strctr=1\relax
        \let\@@fc@numstr#2\relax
        \edef#2{\@@fc@numstr\ \@andname}%
      \fi
      \let\@@fc@numstr#2\relax
      \edef#2{\@@fc@numstr\ \@unitstring{\@strctr}}%
    \fi
  \else
    \@strctr=#1\relax
    \divide\@strctr by 1000\relax
    \@modulo{\@strctr}{10}%
    \let\@@fc@numstr#2\relax
    \edef#2{\@@fc@numstr\@teenstring{\@strctr}}%
  \fi
  \let\@@fc@numstr#2\relax
  \edef#2{\@@fc@numstr\ \@thousand}%
\else
  \ifnum\@strctr>0\relax 
    \ifnum\@strctr>1\relax
      \let\@@fc@numstr#2\relax
      \edef#2{\@@fc@numstr\@unitstring{\@strctr}\ }%
    \fi
    \let\@@fc@numstr#2\relax
    \edef#2{\@@fc@numstr\@thousand}%
  \fi
\fi
\@strctr=#1\relax \@modulo{\@strctr}{1000}%
\divide\@strctr by 100\relax
\ifnum\@strctr>0\relax
  \ifnum#1>1000 \relax
    \let\@@fc@numstr#2\relax
    \edef#2{\@@fc@numstr\ }%
  \fi
  \@tmpstrctr=#1\relax
  \@modulo{\@tmpstrctr}{1000}%
  \let\@@fc@numstr#2\relax
  \ifnum\@tmpstrctr=100\relax
    \edef#2{\@@fc@numstr\@tenstring{10}}%
  \else
    \edef#2{\@@fc@numstr\@hundredstring{\@strctr}}%
  \fi%
\fi
\@strctr=#1\relax \@modulo{\@strctr}{100}%
\ifnum#1>100\relax
  \ifnum\@strctr>0\relax
    \let\@@fc@numstr#2\relax
    \edef#2{\@@fc@numstr\ \@andname\ }%
  \fi
\fi
\ifnum\@strctr>19\relax
  \divide\@strctr by 10\relax
  \let\@@fc@numstr#2\relax
  \edef#2{\@@fc@numstr\@tenstring{\@strctr}}%
  \@strctr=#1\relax \@modulo{\@strctr}{10}%
  \ifnum\@strctr>0
    \ifnum\@strctr=1\relax
      \let\@@fc@numstr#2\relax
      \edef#2{\@@fc@numstr\ \@andname}%
    \else
      \ifnum#1>100\relax
        \let\@@fc@numstr#2\relax
        \edef#2{\@@fc@numstr\ \@andname}%
      \fi
    \fi 
    \let\@@fc@numstr#2\relax
    \edef#2{\@@fc@numstr\ \@unitstring{\@strctr}}%
  \fi
\else
  \ifnum\@strctr<10\relax
    \ifnum\@strctr=0\relax
      \ifnum#1<100\relax
        \let\@@fc@numstr#2\relax
        \edef#2{\@@fc@numstr\@unitstring{\@strctr}}%
      \fi
    \else%(>0,<10)
      \let\@@fc@numstr#2\relax
      \edef#2{\@@fc@numstr\@unitstring{\@strctr}}%
    \fi
  \else%>10
    \@modulo{\@strctr}{10}%
    \let\@@fc@numstr#2\relax
    \edef#2{\@@fc@numstr\@teenstring{\@strctr}}%
  \fi
\fi
}
%    \end{macrocode}
% As above, but for ordinals.
%    \begin{macrocode}
\newcommand*{\@@ordinalstringportuges}[2]{%
\@strctr=#1\relax
\ifnum#1>99999
\PackageError{fmtcount}{Out of range}%
{This macro only works for values less than 100000}%
\else
\ifnum#1<0
\PackageError{fmtcount}{Negative numbers not permitted}%
{This macro does not work for negative numbers, however
you can try typing "minus" first, and then pass the modulus of
this number}%
\else
\def#2{}%
\ifnum\@strctr>999\relax
  \divide\@strctr by 1000\relax
  \ifnum\@strctr>1\relax
    \ifnum\@strctr>9\relax
      \@tmpstrctr=\@strctr
      \ifnum\@strctr<20
        \@modulo{\@tmpstrctr}{10}%
        \let\@@fc@ordstr#2\relax
        \edef#2{\@@fc@ordstr\@teenthstring{\@tmpstrctr}}%
      \else
        \divide\@tmpstrctr by 10\relax
        \let\@@fc@ordstr#2\relax
        \edef#2{\@@fc@ordstr\@tenthstring{\@tmpstrctr}}%
        \@tmpstrctr=\@strctr
        \@modulo{\@tmpstrctr}{10}%
        \ifnum\@tmpstrctr>0\relax
          \let\@@fc@ordstr#2\relax
          \edef#2{\@@fc@ordstr\@unitthstring{\@tmpstrctr}}%
        \fi
      \fi
    \else
      \let\@@fc@ordstr#2\relax
      \edef#2{\@@fc@ordstr\@unitstring{\@strctr}}%
    \fi
  \fi
  \let\@@fc@ordstr#2\relax
  \edef#2{\@@fc@ordstr\@thousandth}%
\fi
\@strctr=#1\relax
\@modulo{\@strctr}{1000}%
\ifnum\@strctr>99\relax
  \@tmpstrctr=\@strctr
  \divide\@tmpstrctr by 100\relax
  \ifnum#1>1000\relax
    \let\@@fc@ordstr#2\relax
    \edef#2{\@@fc@ordstr-}%
  \fi
  \let\@@fc@ordstr#2\relax
  \edef#2{\@@fc@ordstr\@hundredthstring{\@tmpstrctr}}%
\fi
\@modulo{\@strctr}{100}%
\ifnum#1>99\relax
  \ifnum\@strctr>0\relax
    \let\@@fc@ordstr#2\relax
    \edef#2{\@@fc@ordstr-}%
  \fi
\fi
\ifnum\@strctr>9\relax
  \@tmpstrctr=\@strctr
  \divide\@tmpstrctr by 10\relax
  \let\@@fc@ordstr#2\relax
  \edef#2{\@@fc@ordstr\@tenthstring{\@tmpstrctr}}%
  \@tmpstrctr=\@strctr
  \@modulo{\@tmpstrctr}{10}%
  \ifnum\@tmpstrctr>0\relax
    \let\@@fc@ordstr#2\relax
    \edef#2{\@@fc@ordstr-\@unitthstring{\@tmpstrctr}}%
  \fi
\else
  \ifnum\@strctr=0\relax
    \ifnum#1=0\relax
      \let\@@fc@ordstr#2\relax
      \edef#2{\@@fc@ordstr\@unitstring{0}}%
    \fi
  \else
    \let\@@fc@ordstr#2\relax
    \edef#2{\@@fc@ordstr\@unitthstring{\@strctr}}%
  \fi
\fi
\fi
\fi
}
%    \end{macrocode}
%\iffalse
%    \begin{macrocode}
%</fc-portuges.def>
%    \end{macrocode}
%\fi
%\iffalse
%    \begin{macrocode}
%<*fc-spanish.def>
%    \end{macrocode}
%\fi
% \subsection{fc-spanish.def}
% Spanish definitions
%    \begin{macrocode}
\ProvidesFile{fc-spanish.def}[2007/05/26]
%    \end{macrocode}
% Define macro that converts a number or count register (first
% argument) to an ordinal, and stores the result in the
% second argument, which must be a control sequence.
% Masculine:
%    \begin{macrocode}
\newcommand{\@ordinalMspanish}[2]{%
\edef#2{\number#1\relax\noexpand\fmtord{o}}}
%    \end{macrocode}
% Feminine:
%    \begin{macrocode}
\newcommand{\@ordinalFspanish}[2]{%
\edef#2{\number#1\relax\noexpand\fmtord{a}}}
%    \end{macrocode}
% Make neuter same as masculine:
%    \begin{macrocode}
\let\@ordinalNspanish\@ordinalMspanish
%    \end{macrocode}
% Convert a number to text. The easiest way to do this is to
% break it up into units, tens, teens, twenties and hundreds.
% Units (argument must be a number from 0 to 9):
%    \begin{macrocode}
\newcommand{\@@unitstringspanish}[1]{%
\ifcase#1\relax
cero%
\or uno%
\or dos%
\or tres%
\or cuatro%
\or cinco%
\or seis%
\or siete%
\or ocho%
\or nueve%
\fi
}
%    \end{macrocode}
% Feminine:
%    \begin{macrocode}
\newcommand{\@@unitstringFspanish}[1]{%
\ifcase#1\relax
cera%
\or una%
\or dos%
\or tres%
\or cuatro%
\or cinco%
\or seis%
\or siete%
\or ocho%
\or nueve%
\fi
}
%    \end{macrocode}
% Tens (argument must go from 1 to 10):
%    \begin{macrocode}
\newcommand{\@@tenstringspanish}[1]{%
\ifcase#1\relax
\or diez%
\or viente%
\or treinta%
\or cuarenta%
\or cincuenta%
\or sesenta%
\or setenta%
\or ochenta%
\or noventa%
\or cien%
\fi
}
%    \end{macrocode}
% Teens:
%    \begin{macrocode}
\newcommand{\@@teenstringspanish}[1]{%
\ifcase#1\relax
diez%
\or once%
\or doce%
\or trece%
\or catorce%
\or quince%
\or diecis\'eis%
\or diecisiete%
\or dieciocho%
\or diecinueve%
\fi
}
%    \end{macrocode}
% Twenties:
%    \begin{macrocode}
\newcommand{\@@twentystringspanish}[1]{%
\ifcase#1\relax
veinte%
\or veintiuno%
\or veintid\'os%
\or veintitr\'es%
\or veinticuatro%
\or veinticinco%
\or veintis\'eis%
\or veintisiete%
\or veintiocho%
\or veintinueve%
\fi}
%    \end{macrocode}
% Feminine form:
%    \begin{macrocode}
\newcommand{\@@twentystringFspanish}[1]{%
\ifcase#1\relax
veinte%
\or veintiuna%
\or veintid\'os%
\or veintitr\'es%
\or veinticuatro%
\or veinticinco%
\or veintis\'eis%
\or veintisiete%
\or veintiocho%
\or veintinueve%
\fi}
%    \end{macrocode}
% Hundreds:
%    \begin{macrocode}
\newcommand{\@@hundredstringspanish}[1]{%
\ifcase#1\relax
\or ciento%
\or doscientos%
\or trescientos%
\or cuatrocientos%
\or quinientos%
\or seiscientos%
\or setecientos%
\or ochocientos%
\or novecientos%
\fi}
%    \end{macrocode}
% Feminine form:
%    \begin{macrocode}
\newcommand{\@@hundredstringFspanish}[1]{%
\ifcase#1\relax
\or cienta%
\or doscientas%
\or trescientas%
\or cuatrocientas%
\or quinientas%
\or seiscientas%
\or setecientas%
\or ochocientas%
\or novecientas%
\fi}
%    \end{macrocode}
% As above, but with initial letter uppercase:
%    \begin{macrocode}
\newcommand{\@@Unitstringspanish}[1]{%
\ifcase#1\relax
Cero%
\or Uno%
\or Dos%
\or Tres%
\or Cuatro%
\or Cinco%
\or Seis%
\or Siete%
\or Ocho%
\or Nueve%
\fi
}
%    \end{macrocode}
% Feminine form:
%    \begin{macrocode}
\newcommand{\@@UnitstringFspanish}[1]{%
\ifcase#1\relax
Cera%
\or Una%
\or Dos%
\or Tres%
\or Cuatro%
\or Cinco%
\or Seis%
\or Siete%
\or Ocho%
\or Nueve%
\fi
}
%    \end{macrocode}
% Tens:
%    \begin{macrocode}
\newcommand{\@@Tenstringspanish}[1]{%
\ifcase#1\relax
\or Diez%
\or Viente%
\or Treinta%
\or Cuarenta%
\or Cincuenta%
\or Sesenta%
\or Setenta%
\or Ochenta%
\or Noventa%
\or Cien%
\fi
}
%    \end{macrocode}
% Teens:
%    \begin{macrocode}
\newcommand{\@@Teenstringspanish}[1]{%
\ifcase#1\relax
Diez%
\or Once%
\or Doce%
\or Trece%
\or Catorce%
\or Quince%
\or Diecis\'eis%
\or Diecisiete%
\or Dieciocho%
\or Diecinueve%
\fi
}
%    \end{macrocode}
% Twenties:
%    \begin{macrocode}
\newcommand{\@@Twentystringspanish}[1]{%
\ifcase#1\relax
Veinte%
\or Veintiuno%
\or Veintid\'os%
\or Veintitr\'es%
\or Veinticuatro%
\or Veinticinco%
\or Veintis\'eis%
\or Veintisiete%
\or Veintiocho%
\or Veintinueve%
\fi}
%    \end{macrocode}
% Feminine form:
%    \begin{macrocode}
\newcommand{\@@TwentystringFspanish}[1]{%
\ifcase#1\relax
Veinte%
\or Veintiuna%
\or Veintid\'os%
\or Veintitr\'es%
\or Veinticuatro%
\or Veinticinco%
\or Veintis\'eis%
\or Veintisiete%
\or Veintiocho%
\or Veintinueve%
\fi}
%    \end{macrocode}
% Hundreds:
%    \begin{macrocode}
\newcommand{\@@Hundredstringspanish}[1]{%
\ifcase#1\relax
\or Ciento%
\or Doscientos%
\or Trescientos%
\or Cuatrocientos%
\or Quinientos%
\or Seiscientos%
\or Setecientos%
\or Ochocientos%
\or Novecientos%
\fi}
%    \end{macrocode}
% Feminine form:
%    \begin{macrocode}
\newcommand{\@@HundredstringFspanish}[1]{%
\ifcase#1\relax
\or Cienta%
\or Doscientas%
\or Trescientas%
\or Cuatrocientas%
\or Quinientas%
\or Seiscientas%
\or Setecientas%
\or Ochocientas%
\or Novecientas%
\fi}
%    \end{macrocode}
% This has changed in version 1.09, so that it now stores the
% result in the second argument, but doesn't display anything.
% Since it only affects internal macros, it shouldn't affect
% documents created with older versions. (These internal macros
% are not meant for use in documents.)
%    \begin{macrocode}
\DeclareRobustCommand{\@numberstringMspanish}[2]{%
\let\@unitstring=\@@unitstringspanish
\let\@teenstring=\@@teenstringspanish
\let\@tenstring=\@@tenstringspanish
\let\@twentystring=\@@twentystringspanish
\let\@hundredstring=\@@hundredstringspanish
\def\@hundred{cien}\def\@thousand{mil}%
\def\@andname{y}%
\@@numberstringspanish{#1}{#2}}
%    \end{macrocode}
% Feminine form:
%    \begin{macrocode}
\DeclareRobustCommand{\@numberstringFspanish}[2]{%
\let\@unitstring=\@@unitstringFspanish
\let\@teenstring=\@@teenstringspanish
\let\@tenstring=\@@tenstringspanish
\let\@twentystring=\@@twentystringFspanish
\let\@hundredstring=\@@hundredstringFspanish
\def\@hundred{cien}\def\@thousand{mil}%
\def\@andname{y}%
\@@numberstringspanish{#1}{#2}}
%    \end{macrocode}
% Make neuter same as masculine:
%    \begin{macrocode}
\let\@numberstringNspanish\@numberstringMspanish
%    \end{macrocode}
% As above, but initial letters in upper case:
%    \begin{macrocode}
\DeclareRobustCommand{\@NumberstringMspanish}[2]{%
\let\@unitstring=\@@Unitstringspanish
\let\@teenstring=\@@Teenstringspanish
\let\@tenstring=\@@Tenstringspanish
\let\@twentystring=\@@Twentystringspanish
\let\@hundredstring=\@@Hundredstringspanish
\def\@andname{y}%
\def\@hundred{Cien}\def\@thousand{Mil}%
\@@numberstringspanish{#1}{#2}}
%    \end{macrocode}
% Feminine form:
%    \begin{macrocode}
\DeclareRobustCommand{\@NumberstringFspanish}[2]{%
\let\@unitstring=\@@UnitstringFspanish
\let\@teenstring=\@@Teenstringspanish
\let\@tenstring=\@@Tenstringspanish
\let\@twentystring=\@@TwentystringFspanish
\let\@hundredstring=\@@HundredstringFspanish
\def\@andname{y}%
\def\@hundred{Cien}\def\@thousand{Mil}%
\@@numberstringspanish{#1}{#2}}
%    \end{macrocode}
% Make neuter same as masculine:
%    \begin{macrocode}
\let\@NumberstringNspanish\@NumberstringMspanish
%    \end{macrocode}
% As above, but for ordinals.
%    \begin{macrocode}
\DeclareRobustCommand{\@ordinalstringMspanish}[2]{%
\let\@unitthstring=\@@unitthstringspanish
\let\@unitstring=\@@unitstringspanish
\let\@teenthstring=\@@teenthstringspanish
\let\@tenthstring=\@@tenthstringspanish
\let\@hundredthstring=\@@hundredthstringspanish
\def\@thousandth{mil\'esimo}%
\@@ordinalstringspanish{#1}{#2}}
%    \end{macrocode}
% Feminine form:
%    \begin{macrocode}
\DeclareRobustCommand{\@ordinalstringFspanish}[2]{%
\let\@unitthstring=\@@unitthstringFspanish
\let\@unitstring=\@@unitstringFspanish
\let\@teenthstring=\@@teenthstringFspanish
\let\@tenthstring=\@@tenthstringFspanish
\let\@hundredthstring=\@@hundredthstringFspanish
\def\@thousandth{mil\'esima}%
\@@ordinalstringspanish{#1}{#2}}
%    \end{macrocode}
% Make neuter same as masculine:
%    \begin{macrocode}
\let\@ordinalstringNspanish\@ordinalstringMspanish
%    \end{macrocode}
% As above, but with initial letters in upper case.
%    \begin{macrocode}
\DeclareRobustCommand{\@OrdinalstringMspanish}[2]{%
\let\@unitthstring=\@@Unitthstringspanish
\let\@unitstring=\@@Unitstringspanish
\let\@teenthstring=\@@Teenthstringspanish
\let\@tenthstring=\@@Tenthstringspanish
\let\@hundredthstring=\@@Hundredthstringspanish
\def\@thousandth{Mil\'esimo}%
\@@ordinalstringspanish{#1}{#2}}
%    \end{macrocode}
% Feminine form:
%    \begin{macrocode}
\DeclareRobustCommand{\@OrdinalstringFspanish}[2]{%
\let\@unitthstring=\@@UnitthstringFspanish
\let\@unitstring=\@@UnitstringFspanish
\let\@teenthstring=\@@TeenthstringFspanish
\let\@tenthstring=\@@TenthstringFspanish
\let\@hundredthstring=\@@HundredthstringFspanish
\def\@thousandth{Mil\'esima}%
\@@ordinalstringspanish{#1}{#2}}
%    \end{macrocode}
% Make neuter same as masculine:
%    \begin{macrocode}
\let\@OrdinalstringNspanish\@OrdinalstringMspanish
%    \end{macrocode}
% Code for convert numbers into textual ordinals. As before,
% it is easier to split it into units, tens, teens and hundreds.
% Units:
%    \begin{macrocode}
\newcommand{\@@unitthstringspanish}[1]{%
\ifcase#1\relax
cero%
\or primero%
\or segundo%
\or tercero%
\or cuarto%
\or quinto%
\or sexto%
\or s\'eptimo%
\or octavo%
\or noveno%
\fi
}
%    \end{macrocode}
% Tens:
%    \begin{macrocode}
\newcommand{\@@tenthstringspanish}[1]{%
\ifcase#1\relax
\or d\'ecimo%
\or vig\'esimo%
\or trig\'esimo%
\or cuadrag\'esimo%
\or quincuag\'esimo%
\or sexag\'esimo%
\or septuag\'esimo%
\or octog\'esimo%
\or nonag\'esimo%
\fi
}
%    \end{macrocode}
% Teens:
%    \begin{macrocode}
\newcommand{\@@teenthstringspanish}[1]{%
\ifcase#1\relax
d\'ecimo%
\or und\'ecimo%
\or duod\'ecimo%
\or decimotercero%
\or decimocuarto%
\or decimoquinto%
\or decimosexto%
\or decimos\'eptimo%
\or decimoctavo%
\or decimonoveno%
\fi
}
%    \end{macrocode}
% Hundreds:
%    \begin{macrocode}
\newcommand{\@@hundredthstringspanish}[1]{%
\ifcase#1\relax
\or cent\'esimo%
\or ducent\'esimo%
\or tricent\'esimo%
\or cuadringent\'esimo%
\or quingent\'esimo%
\or sexcent\'esimo%
\or septing\'esimo%
\or octingent\'esimo%
\or noningent\'esimo%
\fi}
%    \end{macrocode}
% Units (feminine):
%    \begin{macrocode}
\newcommand{\@@unitthstringFspanish}[1]{%
\ifcase#1\relax
cera%
\or primera%
\or segunda%
\or tercera%
\or cuarta%
\or quinta%
\or sexta%
\or s\'eptima%
\or octava%
\or novena%
\fi
}
%    \end{macrocode}
% Tens (feminine):
%    \begin{macrocode}
\newcommand{\@@tenthstringFspanish}[1]{%
\ifcase#1\relax
\or d\'ecima%
\or vig\'esima%
\or trig\'esima%
\or cuadrag\'esima%
\or quincuag\'esima%
\or sexag\'esima%
\or septuag\'esima%
\or octog\'esima%
\or nonag\'esima%
\fi
}
%    \end{macrocode}
% Teens (feminine)
%    \begin{macrocode}
\newcommand{\@@teenthstringFspanish}[1]{%
\ifcase#1\relax
d\'ecima%
\or und\'ecima%
\or duod\'ecima%
\or decimotercera%
\or decimocuarta%
\or decimoquinta%
\or decimosexta%
\or decimos\'eptima%
\or decimoctava%
\or decimonovena%
\fi
}
%    \end{macrocode}
% Hundreds (feminine)
%    \begin{macrocode}
\newcommand{\@@hundredthstringFspanish}[1]{%
\ifcase#1\relax
\or cent\'esima%
\or ducent\'esima%
\or tricent\'esima%
\or cuadringent\'esima%
\or quingent\'esima%
\or sexcent\'esima%
\or septing\'esima%
\or octingent\'esima%
\or noningent\'esima%
\fi}
%    \end{macrocode}
% As above, but with initial letters in upper case
%    \begin{macrocode}
\newcommand{\@@Unitthstringspanish}[1]{%
\ifcase#1\relax
Cero%
\or Primero%
\or Segundo%
\or Tercero%
\or Cuarto%
\or Quinto%
\or Sexto%
\or S\'eptimo%
\or Octavo%
\or Noveno%
\fi
}
%    \end{macrocode}
% Tens:
%    \begin{macrocode}
\newcommand{\@@Tenthstringspanish}[1]{%
\ifcase#1\relax
\or D\'ecimo%
\or Vig\'esimo%
\or Trig\'esimo%
\or Cuadrag\'esimo%
\or Quincuag\'esimo%
\or Sexag\'esimo%
\or Septuag\'esimo%
\or Octog\'esimo%
\or Nonag\'esimo%
\fi
}
%    \end{macrocode}
% Teens:
%    \begin{macrocode}
\newcommand{\@@Teenthstringspanish}[1]{%
\ifcase#1\relax
D\'ecimo%
\or Und\'ecimo%
\or Duod\'ecimo%
\or Decimotercero%
\or Decimocuarto%
\or Decimoquinto%
\or Decimosexto%
\or Decimos\'eptimo%
\or Decimoctavo%
\or Decimonoveno%
\fi
}
%    \end{macrocode}
% Hundreds
%    \begin{macrocode}
\newcommand{\@@Hundredthstringspanish}[1]{%
\ifcase#1\relax
\or Cent\'esimo%
\or Ducent\'esimo%
\or Tricent\'esimo%
\or Cuadringent\'esimo%
\or Quingent\'esimo%
\or Sexcent\'esimo%
\or Septing\'esimo%
\or Octingent\'esimo%
\or Noningent\'esimo%
\fi}
%    \end{macrocode}
% As above, but feminine.
%    \begin{macrocode}
\newcommand{\@@UnitthstringFspanish}[1]{%
\ifcase#1\relax
Cera%
\or Primera%
\or Segunda%
\or Tercera%
\or Cuarta%
\or Quinta%
\or Sexta%
\or S\'eptima%
\or Octava%
\or Novena%
\fi
}
%    \end{macrocode}
% Tens (feminine)
%    \begin{macrocode}
\newcommand{\@@TenthstringFspanish}[1]{%
\ifcase#1\relax
\or D\'ecima%
\or Vig\'esima%
\or Trig\'esima%
\or Cuadrag\'esima%
\or Quincuag\'esima%
\or Sexag\'esima%
\or Septuag\'esima%
\or Octog\'esima%
\or Nonag\'esima%
\fi
}
%    \end{macrocode}
% Teens (feminine):
%    \begin{macrocode}
\newcommand{\@@TeenthstringFspanish}[1]{%
\ifcase#1\relax
D\'ecima%
\or Und\'ecima%
\or Duod\'ecima%
\or Decimotercera%
\or Decimocuarta%
\or Decimoquinta%
\or Decimosexta%
\or Decimos\'eptima%
\or Decimoctava%
\or Decimonovena%
\fi
}
%    \end{macrocode}
% Hundreds (feminine):
%    \begin{macrocode}
\newcommand{\@@HundredthstringFspanish}[1]{%
\ifcase#1\relax
\or Cent\'esima%
\or Ducent\'esima%
\or Tricent\'esima%
\or Cuadringent\'esima%
\or Quingent\'esima%
\or Sexcent\'esima%
\or Septing\'esima%
\or Octingent\'esima%
\or Noningent\'esima%
\fi}

%    \end{macrocode}
% This has changed in version 1.09, so that it now stores the
% results in the second argument (which must be a control
% sequence), but it doesn't display anything. Since it only
% affects internal macros, it shouldn't affect documnets created
% with older versions. (These internal macros are not meant for
% use in documents.)
%    \begin{macrocode}
\newcommand{\@@numberstringspanish}[2]{%
\ifnum#1>99999
\PackageError{fmtcount}{Out of range}%
{This macro only works for values less than 100000}%
\else
\ifnum#1<0
\PackageError{fmtcount}{Negative numbers not permitted}%
{This macro does not work for negative numbers, however
you can try typing "minus" first, and then pass the modulus of
this number}%
\fi
\fi
\def#2{}%
\@strctr=#1\relax \divide\@strctr by 1000\relax
\ifnum\@strctr>9
% #1 is greater or equal to 10000
  \divide\@strctr by 10
  \ifnum\@strctr>1
    \let\@@fc@numstr#2\relax
    \edef#2{\@@fc@numstr\@tenstring{\@strctr}}%
    \@strctr=#1 \divide\@strctr by 1000\relax
    \@modulo{\@strctr}{10}%
    \ifnum\@strctr>0\relax
       \let\@@fc@numstr#2\relax
       \edef#2{\@@fc@numstr\ \@andname\ \@unitstring{\@strctr}}%
    \fi
  \else
    \@strctr=#1\relax
    \divide\@strctr by 1000\relax
    \@modulo{\@strctr}{10}%
    \let\@@fc@numstr#2\relax
    \edef#2{\@@fc@numstr\@teenstring{\@strctr}}%
  \fi
  \let\@@fc@numstr#2\relax
  \edef#2{\@@fc@numstr\ \@thousand}%
\else
  \ifnum\@strctr>0\relax 
    \ifnum\@strctr>1\relax
       \let\@@fc@numstr#2\relax
       \edef#2{\@@fc@numstr\@unitstring{\@strctr}\ }%
    \fi
    \let\@@fc@numstr#2\relax
    \edef#2{\@@fc@numstr\@thousand}%
  \fi
\fi
\@strctr=#1\relax \@modulo{\@strctr}{1000}%
\divide\@strctr by 100\relax
\ifnum\@strctr>0\relax
  \ifnum#1>1000\relax
    \let\@@fc@numstr#2\relax
    \edef#2{\@@fc@numstr\ }%
  \fi
  \@tmpstrctr=#1\relax
  \@modulo{\@tmpstrctr}{1000}%
  \ifnum\@tmpstrctr=100\relax
    \let\@@fc@numstr#2\relax
    \edef#2{\@@fc@numstr\@tenstring{10}}%
  \else
    \let\@@fc@numstr#2\relax
    \edef#2{\@@fc@numstr\@hundredstring{\@strctr}}%
  \fi
\fi
\@strctr=#1\relax \@modulo{\@strctr}{100}%
\ifnum#1>100\relax
  \ifnum\@strctr>0\relax
    \let\@@fc@numstr#2\relax
    \edef#2{\@@fc@numstr\ \@andname\ }%
  \fi
\fi
\ifnum\@strctr>29\relax
  \divide\@strctr by 10\relax
  \let\@@fc@numstr#2\relax
  \edef#2{\@@fc@numstr\@tenstring{\@strctr}}%
  \@strctr=#1\relax \@modulo{\@strctr}{10}%
  \ifnum\@strctr>0\relax
    \let\@@fc@numstr#2\relax
    \edef#2{\@@fc@numstr\ \@andname\ \@unitstring{\@strctr}}%
  \fi
\else
  \ifnum\@strctr<10\relax
    \ifnum\@strctr=0\relax
      \ifnum#1<100\relax
        \let\@@fc@numstr#2\relax
        \edef#2{\@@fc@numstr\@unitstring{\@strctr}}%
      \fi
    \else
      \let\@@fc@numstr#2\relax
      \edef#2{\@@fc@numstr\@unitstring{\@strctr}}%
    \fi
  \else
    \ifnum\@strctr>19\relax
      \@modulo{\@strctr}{10}%
      \let\@@fc@numstr#2\relax
      \edef#2{\@@fc@numstr\@twentystring{\@strctr}}%
    \else
      \@modulo{\@strctr}{10}%
      \let\@@fc@numstr#2\relax
      \edef#2{\@@fc@numstr\@teenstring{\@strctr}}%
    \fi
  \fi
\fi
}
%    \end{macrocode}
% As above, but for ordinals
%    \begin{macrocode}
\newcommand{\@@ordinalstringspanish}[2]{%
\@strctr=#1\relax
\ifnum#1>99999
\PackageError{fmtcount}{Out of range}%
{This macro only works for values less than 100000}%
\else
\ifnum#1<0
\PackageError{fmtcount}{Negative numbers not permitted}%
{This macro does not work for negative numbers, however
you can try typing "minus" first, and then pass the modulus of
this number}%
\else
\def#2{}%
\ifnum\@strctr>999\relax
  \divide\@strctr by 1000\relax
  \ifnum\@strctr>1\relax
    \ifnum\@strctr>9\relax
      \@tmpstrctr=\@strctr
      \ifnum\@strctr<20
        \@modulo{\@tmpstrctr}{10}%
        \let\@@fc@ordstr#2\relax
        \edef#2{\@@fc@ordstr\@teenthstring{\@tmpstrctr}}%
      \else
        \divide\@tmpstrctr by 10\relax
        \let\@@fc@ordstr#2\relax
        \edef#2{\@@fc@ordstr\@tenthstring{\@tmpstrctr}}%
        \@tmpstrctr=\@strctr
        \@modulo{\@tmpstrctr}{10}%
        \ifnum\@tmpstrctr>0\relax
          \let\@@fc@ordstr#2\relax
          \edef#2{\@@fc@ordstr\@unitthstring{\@tmpstrctr}}%
        \fi
      \fi
    \else
       \let\@@fc@ordstr#2\relax
       \edef#2{\@@fc@ordstr\@unitstring{\@strctr}}%
    \fi
  \fi
  \let\@@fc@ordstr#2\relax
  \edef#2{\@@fc@ordstr\@thousandth}%
\fi
\@strctr=#1\relax
\@modulo{\@strctr}{1000}%
\ifnum\@strctr>99\relax
  \@tmpstrctr=\@strctr
  \divide\@tmpstrctr by 100\relax
  \ifnum#1>1000\relax
    \let\@@fc@ordstr#2\relax
    \edef#2{\@@fc@ordstr\ }%
  \fi
  \let\@@fc@ordstr#2\relax
  \edef#2{\@@fc@ordstr\@hundredthstring{\@tmpstrctr}}%
\fi
\@modulo{\@strctr}{100}%
\ifnum#1>99\relax
  \ifnum\@strctr>0\relax
    \let\@@fc@ordstr#2\relax
    \edef#2{\@@fc@ordstr\ }%
  \fi
\fi
\ifnum\@strctr>19\relax
  \@tmpstrctr=\@strctr
  \divide\@tmpstrctr by 10\relax
  \let\@@fc@ordstr#2\relax
  \edef#2{\@@fc@ordstr\@tenthstring{\@tmpstrctr}}%
  \@tmpstrctr=\@strctr
  \@modulo{\@tmpstrctr}{10}%
  \ifnum\@tmpstrctr>0\relax
    \let\@@fc@ordstr#2\relax
    \edef#2{\@@fc@ordstr\ \@unitthstring{\@tmpstrctr}}%
  \fi
\else
  \ifnum\@strctr>9\relax
    \@modulo{\@strctr}{10}%
    \let\@@fc@ordstr#2\relax
    \edef#2{\@@fc@ordstr\@teenthstring{\@strctr}}%
  \else
    \ifnum\@strctr=0\relax
      \ifnum#1=0\relax
        \let\@@fc@ordstr#2\relax
        \edef#2{\@@fc@ordstr\@unitstring{0}}%
      \fi
    \else
      \let\@@fc@ordstr#2\relax
      \edef#2{\@@fc@ordstr\@unitthstring{\@strctr}}%
    \fi
  \fi
\fi
\fi
\fi
}
%    \end{macrocode}
%\iffalse
%    \begin{macrocode}
%</fc-spanish.def>
%    \end{macrocode}
%\fi
%\iffalse
%    \begin{macrocode}
%<*fc-UKenglish.def>
%    \end{macrocode}
%\fi
% \subsection{fc-UKenglish.def}
% UK English definitions
%    \begin{macrocode}
\ProvidesFile{fc-UKenglish}[2007/06/14]
%    \end{macrocode}
% Check that fc-english.def has been loaded
%    \begin{macrocode}
\@ifundefined{@ordinalMenglish}{\input{fc-english.def}}{}
%    \end{macrocode}
% These are all just synonyms for the commands provided by
% fc-english.def.
%    \begin{macrocode}
\let\@ordinalMUKenglish\@ordinalMenglish
\let\@ordinalFUKenglish\@ordinalMenglish
\let\@ordinalNUKenglish\@ordinalMenglish
\let\@numberstringMUKenglish\@numberstringMenglish
\let\@numberstringFUKenglish\@numberstringMenglish
\let\@numberstringNUKenglish\@numberstringMenglish
\let\@NumberstringMUKenglish\@NumberstringMenglish
\let\@NumberstringFUKenglish\@NumberstringMenglish
\let\@NumberstringNUKenglish\@NumberstringMenglish
\let\@ordinalstringMUKenglish\@ordinalstringMenglish
\let\@ordinalstringFUKenglish\@ordinalstringMenglish
\let\@ordinalstringNUKenglish\@ordinalstringMenglish
\let\@OrdinalstringMUKenglish\@OrdinalstringMenglish
\let\@OrdinalstringFUKenglish\@OrdinalstringMenglish
\let\@OrdinalstringNUKenglish\@OrdinalstringMenglish
%    \end{macrocode}
%\iffalse
%    \begin{macrocode}
%</fc-UKenglish.def>
%    \end{macrocode}
%\fi
%\iffalse
%    \begin{macrocode}
%<*fc-USenglish.def>
%    \end{macrocode}
%\fi
% \subsection{fc-USenglish.def}
% US English definitions
%    \begin{macrocode}
\ProvidesFile{fc-USenglish}[2007/06/14]
%    \end{macrocode}
% Check that fc-english.def has been loaded
%    \begin{macrocode}
\@ifundefined{@ordinalMenglish}{\input{fc-english.def}}{}
%    \end{macrocode}
% These are all just synonyms for the commands provided by
% fc-english.def.
%    \begin{macrocode}
\let\@ordinalMUSenglish\@ordinalMenglish
\let\@ordinalFUSenglish\@ordinalMenglish
\let\@ordinalNUSenglish\@ordinalMenglish
\let\@numberstringMUSenglish\@numberstringMenglish
\let\@numberstringFUSenglish\@numberstringMenglish
\let\@numberstringNUSenglish\@numberstringMenglish
\let\@NumberstringMUSenglish\@NumberstringMenglish
\let\@NumberstringFUSenglish\@NumberstringMenglish
\let\@NumberstringNUSenglish\@NumberstringMenglish
\let\@ordinalstringMUSenglish\@ordinalstringMenglish
\let\@ordinalstringFUSenglish\@ordinalstringMenglish
\let\@ordinalstringNUSenglish\@ordinalstringMenglish
\let\@OrdinalstringMUSenglish\@OrdinalstringMenglish
\let\@OrdinalstringFUSenglish\@OrdinalstringMenglish
\let\@OrdinalstringNUSenglish\@OrdinalstringMenglish
%    \end{macrocode}
%\iffalse
%    \begin{macrocode}
%</fc-USenglish.def>
%    \end{macrocode}
%\fi
%\iffalse
%    \begin{macrocode}
%<*fmtcount.sty>
%    \end{macrocode}
%\fi
%\subsection{fmtcount.sty}
% This section deals with the code for |fmtcount.sty|
%    \begin{macrocode}
\NeedsTeXFormat{LaTeX2e}
\ProvidesPackage{fmtcount}[2007/06/22 v1.2]
\RequirePackage{ifthen}
\RequirePackage{keyval}
%    \end{macrocode}
% As from version 1.2, now load xspace package:
%    \begin{macrocode}
\RequirePackage{xspace}
%    \end{macrocode}
% These commands need to be defined before the
% configuration file is loaded.
%
% Define the macro to format the |st|, |nd|, |rd| or |th| of an 
% ordinal.
%    \begin{macrocode}
\providecommand{\fmtord}[1]{\textsuperscript{#1}}
%    \end{macrocode}
% Define |\padzeroes| to specify how many digits should be 
% displayed.
%    \begin{macrocode}
\newcount\c@padzeroesN
\c@padzeroesN=1\relax
\providecommand{\padzeroes}[1][17]{\c@padzeroesN=#1}
%    \end{macrocode}
% Load appropriate language definition files (I don't
% know if there is a standard way of detecting which
% languages are defined, so I'm just going to check
% if \verb"\date"\meta{language} is defined):
%\changes{v1.1}{14 June 2007}{added check for UKenglish,
% british and USenglish babel settings}
%    \begin{macrocode}
\@ifundefined{dateenglish}{}{\input{fc-english.def}}
\@ifundefined{l@UKenglish}{}{\input{fc-UKenglish.def}}
\@ifundefined{l@british}{}{\input{fc-british.def}}
\@ifundefined{l@USenglish}{}{\input{fc-USenglish.def}}
\@ifundefined{datespanish}{}{\input{fc-spanish.def}}
\@ifundefined{dateportuges}{}{\input{fc-portuges.def}}
\@ifundefined{datefrench}{}{\input{fc-french.def}}
\@ifundefined{dategerman}{%
\@ifundefined{datengerman}{}{\input{fc-german.def}}}{%
\input{fc-german.def}}
%    \end{macrocode}
% Define keys for use with |\fmtcountsetoptions|.
% Key to switch French dialects (Does babel store
%this kind of information?)
%    \begin{macrocode}
\def\fmtcount@french{france}
\define@key{fmtcount}{french}[france]{%
\@ifundefined{datefrench}{%
\PackageError{fmtcount}{Language `french' not defined}{You need
to load babel before loading fmtcount}}{
\ifthenelse{\equal{#1}{france}
         \or\equal{#1}{swiss}
         \or\equal{#1}{belgian}}{%
         \def\fmtcount@french{#1}}{%
\PackageError{fmtcount}{Invalid value `#1' to french key}
{Option `french' can only take the values `france', 
`belgian' or `swiss'}}
}}
%    \end{macrocode}
% Key to determine how to display the ordinal
%    \begin{macrocode}
\define@key{fmtcount}{fmtord}{%
\ifthenelse{\equal{#1}{level}
          \or\equal{#1}{raise}
          \or\equal{#1}{user}}{
          \def\fmtcount@fmtord{#1}}{%
\PackageError{fmtcount}{Invalid value `#1' to fmtord key}
{Option `fmtord' can only take the values `level', `raise' 
or `user'}}}
%    \end{macrocode}
% Key to determine whether the ordinal should be abbreviated
% (language dependent, currently only affects French ordinals.)
%    \begin{macrocode}
\newif\iffmtord@abbrv
\fmtord@abbrvfalse
\define@key{fmtcount}{abbrv}[true]{%
\ifthenelse{\equal{#1}{true}\or\equal{#1}{false}}{
          \csname fmtord@abbrv#1\endcsname}{%
\PackageError{fmtcount}{Invalid value `#1' to fmtord key}
{Option `fmtord' can only take the values `true' or
`false'}}}
%    \end{macrocode}
% Define command to set options.
%    \begin{macrocode}
\newcommand{\fmtcountsetoptions}[1]{%
\def\fmtcount@fmtord{}%
\setkeys{fmtcount}{#1}%
\@ifundefined{datefrench}{}{%
\edef\@ordinalstringMfrench{\noexpand
\csname @ordinalstringMfrench\fmtcount@french\noexpand\endcsname}%
\edef\@ordinalstringFfrench{\noexpand
\csname @ordinalstringFfrench\fmtcount@french\noexpand\endcsname}%
\edef\@OrdinalstringMfrench{\noexpand
\csname @OrdinalstringMfrench\fmtcount@french\noexpand\endcsname}%
\edef\@OrdinalstringFfrench{\noexpand
\csname @OrdinalstringFfrench\fmtcount@french\noexpand\endcsname}%
\edef\@numberstringMfrench{\noexpand
\csname @numberstringMfrench\fmtcount@french\noexpand\endcsname}%
\edef\@numberstringFfrench{\noexpand
\csname @numberstringFfrench\fmtcount@french\noexpand\endcsname}%
\edef\@NumberstringMfrench{\noexpand
\csname @NumberstringMfrench\fmtcount@french\noexpand\endcsname}%
\edef\@NumberstringFfrench{\noexpand
\csname @NumberstringFfrench\fmtcount@french\noexpand\endcsname}%
}%
%
\ifthenelse{\equal{\fmtcount@fmtord}{level}}{%
\renewcommand{\fmtord}[1]{##1}}{%
\ifthenelse{\equal{\fmtcount@fmtord}{raise}}{%
\renewcommand{\fmtord}[1]{\textsuperscript{##1}}}{%
}}
}
%    \end{macrocode}
% Load confguration file if it exists.  This needs to be done
% before the package options, to allow the user to override
% the settings in the configuration file.
%    \begin{macrocode}
\InputIfFileExists{fmtcount.cfg}{%
\typeout{Using configuration file fmtcount.cfg}}{%
\typeout{No configuration file fmtcount.cfg found.}}
%    \end{macrocode}
%Declare options
%    \begin{macrocode}
\DeclareOption{level}{\def\fmtcount@fmtord{level}%
\def\fmtord#1{#1}}
\DeclareOption{raise}{\def\fmtcount@fmtord{raise}%
\def\fmtord#1{\textsuperscript{#1}}}
%    \end{macrocode}
% Process package options
%    \begin{macrocode}
\ProcessOptions
%    \end{macrocode}
% Define macro that performs modulo arthmetic. This is used for the
% date, time, ordinal and numberstring commands. (The fmtcount
% package was originally part of the datetime package.)
%    \begin{macrocode}
\newcount\@DT@modctr
\def\@modulo#1#2{%
\@DT@modctr=#1\relax
\divide \@DT@modctr by #2\relax
\multiply \@DT@modctr by #2\relax
\advance #1 by -\@DT@modctr}
%    \end{macrocode}
% The following registers are needed by |\@ordinal| etc
%    \begin{macrocode}
\newcount\@ordinalctr
\newcount\@orgargctr
\newcount\@strctr
\newcount\@tmpstrctr
%    \end{macrocode}
%Define commands that display numbers in different bases.
% Define counters and conditionals needed.
%    \begin{macrocode}
\newif\if@DT@padzeroes
\newcount\@DT@loopN
\newcount\@DT@X
%    \end{macrocode}
% Binary
%    \begin{macrocode}
\newcommand{\@binary}[1]{%
\@DT@padzeroestrue
\@DT@loopN=17\relax
\@strctr=\@DT@loopN
\whiledo{\@strctr<\c@padzeroesN}{0\advance\@strctr by 1}%
\@strctr=65536\relax
\@DT@X=#1\relax
\loop
\@DT@modctr=\@DT@X
\divide\@DT@modctr by \@strctr
\ifthenelse{\boolean{@DT@padzeroes} \and \(\@DT@modctr=0\) \and \(\@DT@loopN>\c@padzeroesN\)}{}{\the\@DT@modctr}%
\ifnum\@DT@modctr=0\else\@DT@padzeroesfalse\fi
\multiply\@DT@modctr by \@strctr
\advance\@DT@X by -\@DT@modctr
\divide\@strctr by 2\relax
\advance\@DT@loopN by -1\relax
\ifnum\@strctr>1
\repeat
\the\@DT@X}

\let\binarynum=\@binary
%    \end{macrocode}
% Octal
%    \begin{macrocode}
\newcommand{\@octal}[1]{%
\ifnum#1>32768
\PackageError{fmtcount}{Value of counter too large for \protect\@octal}{Maximum value 32768}
\else
\@DT@padzeroestrue
\@DT@loopN=6\relax
\@strctr=\@DT@loopN
\whiledo{\@strctr<\c@padzeroesN}{0\advance\@strctr by 1}%
\@strctr=32768\relax
\@DT@X=#1\relax
\loop
\@DT@modctr=\@DT@X
\divide\@DT@modctr by \@strctr
\ifthenelse{\boolean{@DT@padzeroes} \and \(\@DT@modctr=0\) \and \(\@DT@loopN>\c@padzeroesN\)}{}{\the\@DT@modctr}%
\ifnum\@DT@modctr=0\else\@DT@padzeroesfalse\fi
\multiply\@DT@modctr by \@strctr
\advance\@DT@X by -\@DT@modctr
\divide\@strctr by 8\relax
\advance\@DT@loopN by -1\relax
\ifnum\@strctr>1
\repeat
\the\@DT@X
\fi}
\let\octalnum=\@octal
%    \end{macrocode}
% Lowercase hexadecimal
%    \begin{macrocode}
\newcommand{\@@hexadecimal}[1]{\ifcase#10\or1\or2\or3\or4\or5\or6\or7\or8\or9\or a\or b\or c\or d\or e\or f\fi}

\newcommand{\@hexadecimal}[1]{%
\@DT@padzeroestrue
\@DT@loopN=5\relax
\@strctr=\@DT@loopN
\whiledo{\@strctr<\c@padzeroesN}{0\advance\@strctr by 1}%
\@strctr=65536\relax
\@DT@X=#1\relax
\loop
\@DT@modctr=\@DT@X
\divide\@DT@modctr by \@strctr
\ifthenelse{\boolean{@DT@padzeroes} \and \(\@DT@modctr=0\) \and \(\@DT@loopN>\c@padzeroesN\)}{}{\@@hexadecimal\@DT@modctr}%
\ifnum\@DT@modctr=0\else\@DT@padzeroesfalse\fi
\multiply\@DT@modctr by \@strctr
\advance\@DT@X by -\@DT@modctr
\divide\@strctr by 16\relax
\advance\@DT@loopN by -1\relax
\ifnum\@strctr>1
\repeat
\@@hexadecimal\@DT@X}

\let\hexadecimalnum=\@hexadecimal
%    \end{macrocode}
% Uppercase hexadecimal
%    \begin{macrocode}
\newcommand{\@@Hexadecimal}[1]{\ifcase#10\or1\or2\or3\or4\or5\or6\or
7\or8\or9\or A\or B\or C\or D\or E\or F\fi}

\newcommand{\@Hexadecimal}[1]{%
\@DT@padzeroestrue
\@DT@loopN=5\relax
\@strctr=\@DT@loopN
\whiledo{\@strctr<\c@padzeroesN}{0\advance\@strctr by 1}%
\@strctr=65536\relax
\@DT@X=#1\relax
\loop
\@DT@modctr=\@DT@X
\divide\@DT@modctr by \@strctr
\ifthenelse{\boolean{@DT@padzeroes} \and \(\@DT@modctr=0\) \and \(\@DT@loopN>\c@padzeroesN\)}{}{\@@Hexadecimal\@DT@modctr}%
\ifnum\@DT@modctr=0\else\@DT@padzeroesfalse\fi
\multiply\@DT@modctr by \@strctr
\advance\@DT@X by -\@DT@modctr
\divide\@strctr by 16\relax
\advance\@DT@loopN by -1\relax
\ifnum\@strctr>1
\repeat
\@@Hexadecimal\@DT@X}

\let\Hexadecimalnum=\@Hexadecimal
%    \end{macrocode}
% Uppercase alphabetical representation (a \ldots\ z aa \ldots\ zz)
%    \begin{macrocode}
\newcommand{\@aaalph}[1]{%
\@DT@loopN=#1\relax
\advance\@DT@loopN by -1\relax
\divide\@DT@loopN by 26\relax
\@DT@modctr=\@DT@loopN
\multiply\@DT@modctr by 26\relax
\@DT@X=#1\relax
\advance\@DT@X by -1\relax
\advance\@DT@X by -\@DT@modctr
\advance\@DT@loopN by 1\relax
\advance\@DT@X by 1\relax
\loop
\@alph\@DT@X
\advance\@DT@loopN by -1\relax
\ifnum\@DT@loopN>0
\repeat
}

\let\aaalphnum=\@aaalph
%    \end{macrocode}
% Uppercase alphabetical representation (a \ldots\ z aa \ldots\ zz)
%    \begin{macrocode}
\newcommand{\@AAAlph}[1]{%
\@DT@loopN=#1\relax
\advance\@DT@loopN by -1\relax
\divide\@DT@loopN by 26\relax
\@DT@modctr=\@DT@loopN
\multiply\@DT@modctr by 26\relax
\@DT@X=#1\relax
\advance\@DT@X by -1\relax
\advance\@DT@X by -\@DT@modctr
\advance\@DT@loopN by 1\relax
\advance\@DT@X by 1\relax
\loop
\@Alph\@DT@X
\advance\@DT@loopN by -1\relax
\ifnum\@DT@loopN>0
\repeat
}

\let\AAAlphnum=\@AAAlph
%    \end{macrocode}
% Lowercase alphabetical representation
%    \begin{macrocode}
\newcommand{\@abalph}[1]{%
\ifnum#1>17576
\PackageError{fmtcount}{Value of counter too large for \protect\@abalph}{Maximum value 17576}
\else
\@DT@padzeroestrue
\@strctr=17576\relax
\@DT@X=#1\relax
\advance\@DT@X by -1\relax
\loop
\@DT@modctr=\@DT@X
\divide\@DT@modctr by \@strctr
\ifthenelse{\boolean{@DT@padzeroes} \and \(\@DT@modctr=1\)}{}{\@alph\@DT@modctr}%
\ifnum\@DT@modctr=1\else\@DT@padzeroesfalse\fi
\multiply\@DT@modctr by \@strctr
\advance\@DT@X by -\@DT@modctr
\divide\@strctr by 26\relax
\ifnum\@strctr>1
\repeat
\advance\@DT@X by 1\relax
\@alph\@DT@X
\fi}

\let\abalphnum=\@abalph
%    \end{macrocode}
% Uppercase alphabetical representation
%    \begin{macrocode}
\newcommand{\@ABAlph}[1]{%
\ifnum#1>17576
\PackageError{fmtcount}{Value of counter too large for \protect\@ABAlph}{Maximum value 17576}
\else
\@DT@padzeroestrue
\@strctr=17576\relax
\@DT@X=#1\relax
\advance\@DT@X by -1\relax
\loop
\@DT@modctr=\@DT@X
\divide\@DT@modctr by \@strctr
\ifthenelse{\boolean{@DT@padzeroes} \and \(\@DT@modctr=1\)}{}{\@Alph\@DT@modctr}%
\ifnum\@DT@modctr=1\else\@DT@padzeroesfalse\fi
\multiply\@DT@modctr by \@strctr
\advance\@DT@X by -\@DT@modctr
\divide\@strctr by 26\relax
\ifnum\@strctr>1
\repeat
\advance\@DT@X by 1\relax
\@Alph\@DT@X
\fi}

\let\ABAlphnum=\@ABAlph
%    \end{macrocode}
% Recursive command to count number of characters in argument.
% |\@strctr| should be set to zero before calling it.
%    \begin{macrocode}
\def\@fmtc@count#1#2\relax{%
\if\relax#1
\else
\advance\@strctr by 1\relax
\@fmtc@count#2\relax
\fi}
%    \end{macrocode}
% Internal decimal macro:
%    \begin{macrocode}
\newcommand{\@decimal}[1]{%
\@strctr=0\relax
\expandafter\@fmtc@count\number#1\relax
\@DT@loopN=\c@padzeroesN
\advance\@DT@loopN by -\@strctr
\ifnum\@DT@loopN>0\relax
\@strctr=0\relax
\whiledo{\@strctr < \@DT@loopN}{0\advance\@strctr by 1}%
\fi
\number#1\relax
}

\let\decimalnum=\@decimal
%    \end{macrocode}
% This is a bit cumbersome.  Previously \verb"\@ordinal"
% was defined in a similar way to \verb"\abalph" etc.
% This ensured that the actual value of the counter was
% written in the new label stuff in the .aux file. However
% adding in an optional argument to determine the gender
% for multilingual compatibility messed things up somewhat.
% This was the only work around I could get to keep the
% the cross-referencing stuff working, which is why
% the optional argument comes \emph{after} the compulsory
% argument, instead of the usual manner of placing it before.
% Version 1.04 changed \verb"\ordinal" to \verb"\FCordinal"
% to prevent it clashing with the memoir class. 
%    \begin{macrocode}
\newcommand{\FCordinal}[1]{%
\expandafter\protect\expandafter\ordinalnum{%
\expandafter\the\csname c@#1\endcsname}}
%    \end{macrocode}
% If \verb"\ordinal" isn't defined make \verb"\ordinal" a synonym
% for \verb"\FCordinal" to maintain compatibility with previous
% versions.
%    \begin{macrocode}
\@ifundefined{ordinal}{\let\ordinal\FCordinal}{%
\PackageWarning{fmtcount}{\string\ordinal
\space already defined use \string\FCordinal \space instead.}}
%    \end{macrocode}
% Display ordinal where value is given as a number or 
% count register instead of a counter:
%    \begin{macrocode}
\newcommand{\ordinalnum}[1]{\@ifnextchar[{\@ordinalnum{#1}}{%
\@ordinalnum{#1}[m]}}
%    \end{macrocode}
% Display ordinal according to gender (neuter added in v1.1,
% \cmdname{xspace} added in v1.2):
%    \begin{macrocode}
\def\@ordinalnum#1[#2]{{%
\ifthenelse{\equal{#2}{f}}{%
\protect\@ordinalF{#1}{\@fc@ordstr}}{%
\ifthenelse{\equal{#2}{n}}{%
\protect\@ordinalN{#1}{\@fc@ordstr}}{%
\ifthenelse{\equal{#2}{m}}{}{%
\PackageError{fmtcount}{Invalid gender option `#2'}{%
Available options are m, f or n}}%
\protect\@ordinalM{#1}{\@fc@ordstr}}}\@fc@ordstr}\xspace}
%    \end{macrocode}
% Store the ordinal (first argument
% is identifying name, second argument is a counter.)
%    \begin{macrocode}
\newcommand*{\storeordinal}[2]{%
\expandafter\protect\expandafter\storeordinalnum{#1}{%
\expandafter\the\csname c@#2\endcsname}}
%    \end{macrocode}
% Store ordinal (first argument
% is identifying name, second argument is a number or
% count register.)
%    \begin{macrocode}
\newcommand*{\storeordinalnum}[2]{%
\@ifnextchar[{\@storeordinalnum{#1}{#2}}{%
\@storeordinalnum{#1}{#2}[m]}}
%    \end{macrocode}
% Store ordinal according to gender:
%    \begin{macrocode}
\def\@storeordinalnum#1#2[#3]{%
\ifthenelse{\equal{#3}{f}}{%
\protect\@ordinalF{#2}{\@fc@ord}}{%
\ifthenelse{\equal{#3}{n}}{%
\protect\@ordinalN{#2}{\@fc@ord}}{%
\ifthenelse{\equal{#3}{m}}{}{%
\PackageError{fmtcount}{Invalid gender option `#3'}{%
Available options are m or f}}%
\protect\@ordinalM{#2}{\@fc@ord}}}%
\expandafter\let\csname @fcs@#1\endcsname\@fc@ord}
%    \end{macrocode}
% Get stored information:
%    \begin{macrocode}
\newcommand*{\FMCuse}[1]{\csname @fcs@#1\endcsname}
%    \end{macrocode}
% Display ordinal as a string (argument is a counter)
%    \begin{macrocode}
\newcommand{\ordinalstring}[1]{%
\expandafter\protect\expandafter\ordinalstringnum{%
\expandafter\the\csname c@#1\endcsname}}
%    \end{macrocode}
% Display ordinal as a string (argument is a count register or
% number.)
%    \begin{macrocode}
\newcommand{\ordinalstringnum}[1]{%
\@ifnextchar[{\@ordinal@string{#1}}{\@ordinal@string{#1}[m]}}
%    \end{macrocode}
% Display ordinal as a string according to gender (\cmdname{xspace}
% added in version 1.2).
%    \begin{macrocode}
\def\@ordinal@string#1[#2]{{%
\ifthenelse{\equal{#2}{f}}{%
\protect\@ordinalstringF{#1}{\@fc@ordstr}}{%
\ifthenelse{\equal{#2}{n}}{%
\protect\@ordinalstringN{#1}{\@fc@ordstr}}{%
\ifthenelse{\equal{#2}{m}}{}{%
\PackageError{fmtcount}{Invalid gender option `#2' to 
\string\ordinalstring}{Available options are m, f or f}}%
\protect\@ordinalstringM{#1}{\@fc@ordstr}}}\@fc@ordstr}\xspace}
%    \end{macrocode}
% Store textual representation of number. First argument is 
% identifying name, second argument is the counter set to the 
% required number.
%    \begin{macrocode}
\newcommand{\storeordinalstring}[2]{%
\expandafter\protect\expandafter\storeordinalstringnum{#1}{%
\expandafter\the\csname c@#2\endcsname}}
%    \end{macrocode}
% Store textual representation of number. First argument is 
% identifying name, second argument is a count register or number.
%    \begin{macrocode}
\newcommand{\storeordinalstringnum}[2]{%
\@ifnextchar[{\@store@ordinal@string{#1}{#2}}{%
\@store@ordinal@string{#1}{#2}[m]}}
%    \end{macrocode}
% Store textual representation of number according to gender.
%    \begin{macrocode}
\def\@store@ordinal@string#1#2[#3]{%
\ifthenelse{\equal{#3}{f}}{%
\protect\@ordinalstringF{#2}{\@fc@ordstr}}{%
\ifthenelse{\equal{#3}{n}}{%
\protect\@ordinalstringN{#2}{\@fc@ordstr}}{%
\ifthenelse{\equal{#3}{m}}{}{%
\PackageError{fmtcount}{Invalid gender option `#3' to 
\string\ordinalstring}{Available options are m, f or n}}%
\protect\@ordinalstringM{#2}{\@fc@ordstr}}}%
\expandafter\let\csname @fcs@#1\endcsname\@fc@ordstr}
%    \end{macrocode}
% Display ordinal as a string with initial letters in upper case
% (argument is a counter)
%    \begin{macrocode}
\newcommand{\Ordinalstring}[1]{%
\expandafter\protect\expandafter\Ordinalstringnum{%
\expandafter\the\csname c@#1\endcsname}}
%    \end{macrocode}
% Display ordinal as a string with initial letters in upper case
% (argument is a number or count register)
%    \begin{macrocode}
\newcommand{\Ordinalstringnum}[1]{%
\@ifnextchar[{\@Ordinal@string{#1}}{\@Ordinal@string{#1}[m]}}
%    \end{macrocode}
% Display ordinal as a string with initial letters in upper case
% according to gender
%    \begin{macrocode}
\def\@Ordinal@string#1[#2]{{%
\ifthenelse{\equal{#2}{f}}{%
\protect\@OrdinalstringF{#1}{\@fc@ordstr}}{%
\ifthenelse{\equal{#2}{n}}{%
\protect\@OrdinalstringN{#1}{\@fc@ordstr}}{%
\ifthenelse{\equal{#2}{m}}{}{%
\PackageError{fmtcount}{Invalid gender option `#2'}{%
Available options are m, f or n}}%
\protect\@OrdinalstringM{#1}{\@fc@ordstr}}}\@fc@ordstr}\xspace}
%    \end{macrocode}
% Store textual representation of number, with initial letters in 
% upper case. First argument is identifying name, second argument 
% is the counter set to the 
% required number.
%    \begin{macrocode}
\newcommand{\storeOrdinalstring}[2]{%
\expandafter\protect\expandafter\storeOrdinalstringnum{#1}{%
\expandafter\the\csname c@#2\endcsname}}
%    \end{macrocode}
% Store textual representation of number, with initial letters in 
% upper case. First argument is identifying name, second argument 
% is a count register or number.
%    \begin{macrocode}
\newcommand{\storeOrdinalstringnum}[2]{%
\@ifnextchar[{\@store@Ordinal@string{#1}{#2}}{%
\@store@Ordinal@string{#1}{#2}[m]}}
%    \end{macrocode}
% Store textual representation of number according to gender, 
% with initial letters in upper case.
%    \begin{macrocode}
\def\@store@Ordinal@string#1#2[#3]{%
\ifthenelse{\equal{#3}{f}}{%
\protect\@OrdinalstringF{#2}{\@fc@ordstr}}{%
\ifthenelse{\equal{#3}{n}}{%
\protect\@OrdinalstringN{#2}{\@fc@ordstr}}{%
\ifthenelse{\equal{#3}{m}}{}{%
\PackageError{fmtcount}{Invalid gender option `#3'}{%
Available options are m or f}}%
\protect\@OrdinalstringM{#2}{\@fc@ordstr}}}%
\expandafter\let\csname @fcs@#1\endcsname\@fc@ordstr}
%    \end{macrocode}
% Store upper case textual representation of ordinal. The first 
% argument is identifying name, the second argument is a counter.
%    \begin{macrocode}
\newcommand{\storeORDINALstring}[2]{%
\expandafter\protect\expandafter\storeORDINALstringnum{#1}{%
\expandafter\the\csname c@#2\endcsname}}
%    \end{macrocode}
% As above, but the second argument is a count register or a
% number.
%    \begin{macrocode}
\newcommand{\storeORDINALstringnum}[2]{%
\@ifnextchar[{\@store@ORDINAL@string{#1}{#2}}{%
\@store@ORDINAL@string{#1}{#2}[m]}}
%    \end{macrocode}
% Gender is specified as an optional argument at the end.
%    \begin{macrocode}
\def\@store@ORDINAL@string#1#2[#3]{%
\ifthenelse{\equal{#3}{f}}{%
\protect\@ordinalstringF{#2}{\@fc@ordstr}}{%
\ifthenelse{\equal{#3}{n}}{%
\protect\@ordinalstringN{#2}{\@fc@ordstr}}{%
\ifthenelse{\equal{#3}{m}}{}{%
\PackageError{fmtcount}{Invalid gender option `#3'}{%
Available options are m or f}}%
\protect\@ordinalstringM{#2}{\@fc@ordstr}}}%
\expandafter\edef\csname @fcs@#1\endcsname{%
\noexpand\MakeUppercase{\@fc@ordstr}}}
%    \end{macrocode}
% Display upper case textual representation of an ordinal. The
% argument must be a counter.
%    \begin{macrocode}
\newcommand{\ORDINALstring}[1]{%
\expandafter\protect\expandafter\ORDINALstringnum{%
\expandafter\the\csname c@#1\endcsname}}
%    \end{macrocode}
% As above, but the argument is a count register or a number.
%    \begin{macrocode}
\newcommand{\ORDINALstringnum}[1]{%
\@ifnextchar[{\@ORDINAL@string{#1}}{\@ORDINAL@string{#1}[m]}}
%    \end{macrocode}
% Gender is specified as an optional argument at the end.
%    \begin{macrocode}
\def\@ORDINAL@string#1[#2]{{%
\ifthenelse{\equal{#2}{f}}{%
\protect\@ordinalstringF{#1}{\@fc@ordstr}}{%
\ifthenelse{\equal{#2}{n}}{%
\protect\@ordinalstringN{#1}{\@fc@ordstr}}{%
\ifthenelse{\equal{#2}{m}}{}{%
\PackageError{fmtcount}{Invalid gender option `#2'}{%
Available options are m, f or n}}%
\protect\@ordinalstringM{#1}{\@fc@ordstr}}}%
\MakeUppercase{\@fc@ordstr}}\xspace}
%    \end{macrocode}
% Convert number to textual respresentation, and store. First 
% argument is the identifying name, second argument is a counter 
% containing the number.
%    \begin{macrocode}
\newcommand{\storenumberstring}[2]{%
\expandafter\protect\expandafter\storenumberstringnum{#1}{%
\expandafter\the\csname c@#2\endcsname}}
%    \end{macrocode}
% As above, but second argument is a number or count register.
%    \begin{macrocode}
\newcommand{\storenumberstringnum}[2]{%
\@ifnextchar[{\@store@number@string{#1}{#2}}{%
\@store@number@string{#1}{#2}[m]}}
%    \end{macrocode}
% Gender is given as optional argument, \emph{at the end}.
%    \begin{macrocode}
\def\@store@number@string#1#2[#3]{%
\ifthenelse{\equal{#3}{f}}{%
\protect\@numberstringF{#2}{\@fc@numstr}}{%
\ifthenelse{\equal{#3}{n}}{%
\protect\@numberstringN{#2}{\@fc@numstr}}{%
\ifthenelse{\equal{#3}{m}}{}{%
\PackageError{fmtcount}{Invalid gender option `#3'}{%
Available options are m, f or n}}%
\protect\@numberstringM{#2}{\@fc@numstr}}}%
\expandafter\let\csname @fcs@#1\endcsname\@fc@numstr}
%    \end{macrocode}
% Display textual representation of a number. The argument
% must be a counter.
%    \begin{macrocode}
\newcommand{\numberstring}[1]{%
\expandafter\protect\expandafter\numberstringnum{%
\expandafter\the\csname c@#1\endcsname}}
%    \end{macrocode}
% As above, but the argument is a count register or a number.
%    \begin{macrocode}
\newcommand{\numberstringnum}[1]{%
\@ifnextchar[{\@number@string{#1}}{\@number@string{#1}[m]}}
%    \end{macrocode}
% Gender is specified as an optional argument \emph{at the end}.
%    \begin{macrocode}
\def\@number@string#1[#2]{{%
\ifthenelse{\equal{#2}{f}}{%
\protect\@numberstringF{#1}{\@fc@numstr}}{%
\ifthenelse{\equal{#2}{n}}{%
\protect\@numberstringN{#1}{\@fc@numstr}}{%
\ifthenelse{\equal{#2}{m}}{}{%
\PackageError{fmtcount}{Invalid gender option `#2'}{%
Available options are m, f or n}}%
\protect\@numberstringM{#1}{\@fc@numstr}}}\@fc@numstr}\xspace}
%    \end{macrocode}
% Store textual representation of number. First argument is 
% identifying name, second argument is a counter.
%    \begin{macrocode}
\newcommand{\storeNumberstring}[2]{%
\expandafter\protect\expandafter\storeNumberstringnum{#1}{%
\expandafter\the\csname c@#2\endcsname}}
%    \end{macrocode}
% As above, but second argument is a count register or number.
%    \begin{macrocode}
\newcommand{\storeNumberstringnum}[2]{%
\@ifnextchar[{\@store@Number@string{#1}{#2}}{%
\@store@Number@string{#1}{#2}[m]}}
%    \end{macrocode}
% Gender is specified as an optional argument \emph{at the end}:
%    \begin{macrocode}
\def\@store@Number@string#1#2[#3]{%
\ifthenelse{\equal{#3}{f}}{%
\protect\@NumberstringF{#2}{\@fc@numstr}}{%
\ifthenelse{\equal{#3}{n}}{%
\protect\@NumberstringN{#2}{\@fc@numstr}}{%
\ifthenelse{\equal{#3}{m}}{}{%
\PackageError{fmtcount}{Invalid gender option `#3'}{%
Available options are m, f or n}}%
\protect\@NumberstringM{#2}{\@fc@numstr}}}%
\expandafter\let\csname @fcs@#1\endcsname\@fc@numstr}
%    \end{macrocode}
% Display textual representation of number. The argument must be
% a counter. 
%    \begin{macrocode}
\newcommand{\Numberstring}[1]{%
\expandafter\protect\expandafter\Numberstringnum{%
\expandafter\the\csname c@#1\endcsname}}
%    \end{macrocode}
% As above, but the argument is a count register or number.
%    \begin{macrocode}
\newcommand{\Numberstringnum}[1]{%
\@ifnextchar[{\@Number@string{#1}}{\@Number@string{#1}[m]}}
%    \end{macrocode}
% Gender is specified as an optional argument at the end.
%    \begin{macrocode}
\def\@Number@string#1[#2]{{%
\ifthenelse{\equal{#2}{f}}{%
\protect\@NumberstringF{#1}{\@fc@numstr}}{%
\ifthenelse{\equal{#2}{n}}{%
\protect\@NumberstringN{#1}{\@fc@numstr}}{%
\ifthenelse{\equal{#2}{m}}{}{%
\PackageError{fmtcount}{Invalid gender option `#2'}{%
Available options are m, f or n}}%
\protect\@NumberstringM{#1}{\@fc@numstr}}}\@fc@numstr}\xspace}
%    \end{macrocode}
% Store upper case textual representation of number. The first 
% argument is identifying name, the second argument is a counter.
%    \begin{macrocode}
\newcommand{\storeNUMBERstring}[2]{%
\expandafter\protect\expandafter\storeNUMBERstringnum{#1}{%
\expandafter\the\csname c@#2\endcsname}}
%    \end{macrocode}
% As above, but the second argument is a count register or a
% number.
%    \begin{macrocode}
\newcommand{\storeNUMBERstringnum}[2]{%
\@ifnextchar[{\@store@NUMBER@string{#1}{#2}}{%
\@store@NUMBER@string{#1}{#2}[m]}}
%    \end{macrocode}
% Gender is specified as an optional argument at the end.
%    \begin{macrocode}
\def\@store@NUMBER@string#1#2[#3]{%
\ifthenelse{\equal{#3}{f}}{%
\protect\@numberstringF{#2}{\@fc@numstr}}{%
\ifthenelse{\equal{#3}{n}}{%
\protect\@numberstringN{#2}{\@fc@numstr}}{%
\ifthenelse{\equal{#3}{m}}{}{%
\PackageError{fmtcount}{Invalid gender option `#3'}{%
Available options are m or f}}%
\protect\@numberstringM{#2}{\@fc@numstr}}}%
\expandafter\edef\csname @fcs@#1\endcsname{%
\noexpand\MakeUppercase{\@fc@numstr}}}
%    \end{macrocode}
% Display upper case textual representation of a number. The
% argument must be a counter.
%    \begin{macrocode}
\newcommand{\NUMBERstring}[1]{%
\expandafter\protect\expandafter\NUMBERstringnum{%
\expandafter\the\csname c@#1\endcsname}}
%    \end{macrocode}
% As above, but the argument is a count register or a number.
%    \begin{macrocode}
\newcommand{\NUMBERstringnum}[1]{%
\@ifnextchar[{\@NUMBER@string{#1}}{\@NUMBER@string{#1}[m]}}
%    \end{macrocode}
% Gender is specified as an optional argument at the end.
%    \begin{macrocode}
\def\@NUMBER@string#1[#2]{{%
\ifthenelse{\equal{#2}{f}}{%
\protect\@numberstringF{#1}{\@fc@numstr}}{%
\ifthenelse{\equal{#2}{n}}{%
\protect\@numberstringN{#1}{\@fc@numstr}}{%
\ifthenelse{\equal{#2}{m}}{}{%
\PackageError{fmtcount}{Invalid gender option `#2'}{%
Available options are m, f or n}}%
\protect\@numberstringM{#1}{\@fc@numstr}}}%
\MakeUppercase{\@fc@numstr}}\xspace}
%    \end{macrocode}
% Number representations in other bases. Binary:
%    \begin{macrocode}
\providecommand{\binary}[1]{%
\expandafter\protect\expandafter\@binary{%
\expandafter\the\csname c@#1\endcsname}}
%    \end{macrocode}
% Like \verb"\alph", but goes beyond 26. (a \ldots\ z aa \ldots zz \ldots)
%    \begin{macrocode}
\providecommand{\aaalph}[1]{%
\expandafter\protect\expandafter\@aaalph{%
\expandafter\the\csname c@#1\endcsname}}
%    \end{macrocode}
% As before, but upper case.
%    \begin{macrocode}
\providecommand{\AAAlph}[1]{%
\expandafter\protect\expandafter\@AAAlph{%
\expandafter\the\csname c@#1\endcsname}}
%    \end{macrocode}
% Like \verb"\alph", but goes beyond 26. (a \ldots\ z ab \ldots az \ldots)
%    \begin{macrocode}
\providecommand{\abalph}[1]{%
\expandafter\protect\expandafter\@abalph{%
\expandafter\the\csname c@#1\endcsname}}
%    \end{macrocode}
% As above, but upper case.
%    \begin{macrocode}
\providecommand{\ABAlph}[1]{%
\expandafter\protect\expandafter\@ABAlph{%
\expandafter\the\csname c@#1\endcsname}}
%    \end{macrocode}
% Hexadecimal:
%    \begin{macrocode}
\providecommand{\hexadecimal}[1]{%
\expandafter\protect\expandafter\@hexadecimal{%
\expandafter\the\csname c@#1\endcsname}}
%    \end{macrocode}
% As above, but in upper case.
%    \begin{macrocode}
\providecommand{\Hexadecimal}[1]{%
\expandafter\protect\expandafter\@Hexadecimal{%
\expandafter\the\csname c@#1\endcsname}}
%    \end{macrocode}
% Octal:
%    \begin{macrocode}
\providecommand{\octal}[1]{%
\expandafter\protect\expandafter\@octal{%
\expandafter\the\csname c@#1\endcsname}}
%    \end{macrocode}
% Decimal:
%    \begin{macrocode}
\providecommand{\decimal}[1]{%
\expandafter\protect\expandafter\@decimal{%
\expandafter\the\csname c@#1\endcsname}}
%    \end{macrocode}
%\subsubsection{Multilinguage Definitions}
% If multilingual support is provided, make \verb"\@numberstring" 
% etc use the correct language (if defined).
% Otherwise use English definitions. "\@setdef@ultfmtcount"
% sets the macros to use English.
%    \begin{macrocode}
\def\@setdef@ultfmtcount{
\@ifundefined{@ordinalMenglish}{\input{fc-english.def}}{}
\def\@ordinalstringM{\@ordinalstringMenglish}
\let\@ordinalstringF=\@ordinalstringMenglish
\let\@ordinalstringN=\@ordinalstringMenglish
\def\@OrdinalstringM{\@OrdinalstringMenglish}
\let\@OrdinalstringF=\@OrdinalstringMenglish
\let\@OrdinalstringN=\@OrdinalstringMenglish
\def\@numberstringM{\@numberstringMenglish}
\let\@numberstringF=\@numberstringMenglish
\let\@numberstringN=\@numberstringMenglish
\def\@NumberstringM{\@NumberstringMenglish}
\let\@NumberstringF=\@NumberstringMenglish
\let\@NumberstringN=\@NumberstringMenglish
\def\@ordinalM{\@ordinalMenglish}
\let\@ordinalF=\@ordinalM
\let\@ordinalN=\@ordinalM
}
%    \end{macrocode}
% Define a command to set macros to use "languagename":
%    \begin{macrocode}
\def\@set@mulitling@fmtcount{%
%
\def\@numberstringM{\@ifundefined{@numberstringM\languagename}{%
\PackageError{fmtcount}{No support for language '\languagename'}{%
The fmtcount package currently does not support language 
'\languagename' for command \string\@numberstringM}}{%
\csname @numberstringM\languagename\endcsname}}%
%
\def\@numberstringF{\@ifundefined{@numberstringF\languagename}{%
\PackageError{fmtcount}{No support for language '\languagename'}{%
The fmtcount package currently does not support language 
'\languagename' for command \string\@numberstringF}}{%
\csname @numberstringF\languagename\endcsname}}%
%
\def\@numberstringN{\@ifundefined{@numberstringN\languagename}{%
\PackageError{fmtcount}{No support for language '\languagename'}{%
The fmtcount package currently does not support language 
'\languagename' for command \string\@numberstringN}}{%
\csname @numberstringN\languagename\endcsname}}%
%
\def\@NumberstringM{\@ifundefined{@NumberstringM\languagename}{%
\PackageError{fmtcount}{No support for language '\languagename'}{%
The fmtcount package currently does not support language 
'\languagename' for command \string\@NumberstringM}}{%
\csname @NumberstringM\languagename\endcsname}}%
%
\def\@NumberstringF{\@ifundefined{@NumberstringF\languagename}{%
\PackageError{fmtcount}{No support for language '\languagename'}{%
The fmtcount package currently does not support language 
'\languagename' for command \string\@NumberstringF}}{%
\csname @NumberstringF\languagename\endcsname}}%
%
\def\@NumberstringN{\@ifundefined{@NumberstringN\languagename}{%
\PackageError{fmtcount}{No support for language '\languagename'}{%
The fmtcount package currently does not support language 
'\languagename' for command \string\@NumberstringN}}{%
\csname @NumberstringN\languagename\endcsname}}%
%
\def\@ordinalM{\@ifundefined{@ordinalM\languagename}{%
\PackageError{fmtcount}{No support for language '\languagename'}{%
The fmtcount package currently does not support language 
'\languagename' for command \string\@ordinalM}}{%
\csname @ordinalM\languagename\endcsname}}%
%
\def\@ordinalF{\@ifundefined{@ordinalF\languagename}{%
\PackageError{fmtcount}{No support for language '\languagename'}{%
The fmtcount package currently does not support language 
'\languagename' for command \string\@ordinalF}}{%
\csname @ordinalF\languagename\endcsname}}%
%
\def\@ordinalN{\@ifundefined{@ordinalN\languagename}{%
\PackageError{fmtcount}{No support for language '\languagename'}{%
The fmtcount package currently does not support language 
'\languagename' for command \string\@ordinalN}}{%
\csname @ordinalN\languagename\endcsname}}%
%
\def\@ordinalstringM{\@ifundefined{@ordinalstringM\languagename}{%
\PackageError{fmtcount}{No support for language '\languagename'}{%
The fmtcount package currently does not support language 
'\languagename' for command \string\@ordinalstringM}}{%
\csname @ordinalstringM\languagename\endcsname}}%
%
\def\@ordinalstringF{\@ifundefined{@ordinalstringF\languagename}{%
\PackageError{fmtcount}{No support for language '\languagename'}{%
The fmtcount package currently does not support language 
'\languagename' for command \string\@ordinalstringF}}{%
\csname @ordinalstringF\languagename\endcsname}}%
%
\def\@ordinalstringN{\@ifundefined{@ordinalstringN\languagename}{%
\PackageError{fmtcount}{No support for language '\languagename'}{%
The fmtcount package currently does not support language 
'\languagename' for command \string\@ordinalstringN}}{%
\csname @ordinalstringN\languagename\endcsname}}%
%
\def\@OrdinalstringM{\@ifundefined{@OrdinalstringM\languagename}{%
\PackageError{fmtcount}{No support for language '\languagename'}{%
The fmtcount package currently does not support language 
'\languagename' for command \string\@OrdinalstringM}}{%
\csname @OrdinalstringM\languagename\endcsname}}%
%
\def\@OrdinalstringF{\@ifundefined{@OrdinalstringF\languagename}{%
\PackageError{fmtcount}{No support for language '\languagename'}{%
The fmtcount package currently does not support language 
'\languagename' for command \string\@OrdinalstringF}}{%
\csname @OrdinalstringF\languagename\endcsname}}%
%
\def\@OrdinalstringN{\@ifundefined{@OrdinalstringN\languagename}{%
\PackageError{fmtcount}{No support for language '\languagename'}{%
The fmtcount package currently does not support language 
'\languagename' for command \string\@OrdinalstringN}}{%
\csname @OrdinalstringN\languagename\endcsname}}
}
%    \end{macrocode}
% Check to see if babel or ngerman packages have been loaded.
%    \begin{macrocode}
\@ifpackageloaded{babel}{%
\ifthenelse{\equal{\languagename}{nohyphenation}\or
\equal{languagename}{english}}{\@setdef@ultfmtcount}{%
\@set@mulitling@fmtcount}
}{%
\@ifpackageloaded{ngerman}{%
\@ifundefined{@numberstringMgerman}{%
\input{fc-german.def}}{}\@set@mulitling@fmtcount}{%
\@setdef@ultfmtcount}}
%    \end{macrocode}
% Backwards compatibility:
%    \begin{macrocode}
\let\@ordinal=\@ordinalM
\let\@ordinalstring=\@ordinalstringM
\let\@Ordinalstring=\@OrdinalstringM
\let\@numberstring=\@numberstringM
\let\@Numberstring=\@NumberstringM
%    \end{macrocode}
%\iffalse
%    \begin{macrocode}
%</fmtcount.sty>
%    \end{macrocode}
%\fi
%\Finale
\endinput
