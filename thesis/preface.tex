\setchapterpreamble[o]{%
  \dictum{\textit{Virtualization  is a  concept that  one cannot  think  away from
    computer science anymore.}}}

\chapter{Preface}
\thispagestyle{empty}

In the last few years the remote execution of applications has gained more
and more on  importance. This is due to the fact  that paradigms like grid
computing have been developed. The computation power that is required by a
scientist to execute a complex simulation need not to be locally available
anymore.   Instead  he  can  submit   the  simulation  to  a  remote  high
performance  cluster  that  is   provided  by  an  organization  that  has
specialized in providing computation power.

The current developments in the  area of hardware virtualization show that
current  desktop computer  systems  are powerful  enough  to execute  many
different   operating  systems   in  parallel.   The  usage   of  hardware
virtualization imposes only a little overhead.

This  work aims  at sticking  both concepts  together to  provide  a novel
execution environment.  This environment is going to provide secure remote
execution  of   applications  in  user-supplied   virtual  machines.   The
execution will  behave like a batch  job --- send the  execution and input
data away and get the results back.

\vfill

In  den  letzten  Jahren  hat  die Ausf\"{u}hrung  von  Applikationen  auf
entfernten Systemen mehr und mehr  an Bedeutung gewonnen. Das liegt daran,
dass Paradigmen wie zum Beispiel das Grid Computing entwickelt wurden. Die
Rechenleistung,  die  ein   Wissenschaftler  f\"{u}r  eine  aufw\"{a}ndige
Simulation ben\"{o}tigt,  muss nicht mehr lokal  zur Verf\"{u}gung stehen.
Vielmehr   kann   der   Wissenschaftler   seine   Berechnung   auf   einem
High-Performance Cluster durchf\"{u}ren lassen, der von einer Organisation
bereitgestellt  wird,  die   auf  die  Bereitstellung  von  Rechenleistung
spezialisiert ist.

Die  aktuellen  Entwicklungen   im  Bereich  der  Hardware-Virtualisierung
zeigen, dass  aktuelle Prozessoren  leistungsstark genug sind,  um mehrere
verschiedene Betriebssysteme gleichzeitig  ausf\"{u}ren zu k\"{o}nnen. Die
Benutzung    von   Hardware-Virtualisierung    verursacht    nur   geringe
zus\"{a}tzliche Kosten.

Meine Arbeit zielt  darauf ab, diese beiden Verfahren  zu einer neuartigen
Ausf\"{u}hrungsumgebung zu  kombinieren.  Diese Umgebung  wird die sichere
Ausf\"{u}rung   von   benutzerdefinierten   Applikationen  in   virtuellen
Maschinen auf  einem entfernten System  bereitstellen.  Die Ausf\"{u}hrung
wird sich  wie die Ausf\"{u}hrung  eines Batch-Jobs verhalten  --- schicke
den  Auftrag  samt seiner  Eingabedaten  weg  und  bekomme die  Ergebnisse
zur\"{u}ck.


%  THE recent development of  virtualization support in wide-spread processor
% architectures show  that current desktop computers are  powerful enough to
% execute more than one operating  system at a time.

% It can be
% provided by a

% This work aims to provide an execution environment 

% This work  is about  the execution of  arbitrary applications on  a remote
% system exploiting virtual machines. It addresses the problem that multiple
% potentially  broken applications  are typically  executed on  remote hosts
% side by  side ---  if one application  behaves ``wrong''  (e.~g.~CPU cycle
% consumption,  memory  leakage, etc.)  it  may  involve other  applications
% running on the same host.

% \textbf{Motto:} \emph{Secure execution by separation with virtual machines.}
% \vfill

% chapter overview
% \begin{description}
% \item[Chapter \ref{cha:intro}]  The first  chapter introduces you  to past
%   and current  virtualization technologies.  This chapter  also covers the
%   problem  description  and  the  goals of  the  proposed  \emph{Xen-based
%     Execution Environment}.

% \item[Chapter \ref{cha:requirements}] This  chapter outlines and describes
%   the basic  requirements that the  \emph{Xen-based Execution Environment}
%   has to fulfill.
  
% \item[Chapter   \ref{cha:fundamentals}]   The   third  chapter   discusses
%   fundamental technologies that  I have used in this  work.  It covers the
%   \emph{Job  Submission  Description   Language},  the  job-model  of  the
%   \emph{Basic Execution Service} and  the securing of transmitted messages
%   using \emph{Message Layer Security}.
  
% \item[Chapter  \ref{cha:design}] This  chapter deals  with the  design and
%   implementation  of the  \emph{Xen-based Execution  Environment}  and its
%   components. It covers the necessary steps and the involved communication
%   protocols to execute a job on a virtual machine.
  
% \item[Chapter \ref{cha:results}]  In this  chapter the performance  of the
%   \emph{Xen-based Execution Environment} is  analyzed. The impact of using
%   compressed  images and exploitation  of caching  on the  total execution
%   time is analyzed as well.
  
% \item[Chapter \ref{cha:conclusions}]  The final chapter  draws conclusions
%   about the  work that has  been done and  provides some ideas  for future
%   development.
% \end{description}

\clearpage

%%% Local Variables: 
%%% mode: latex
%%% TeX-master: "main"
%%% End: 
