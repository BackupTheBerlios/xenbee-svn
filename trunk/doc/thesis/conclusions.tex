
\chapter{Conclusions and Future Work}
\label{cha:conclusions}

This  chapter analyzes  the work  that  has been  done in  respect of  the
initial goals  that have been claimed.  It also provides  hints for future
developments.

\section*{Conclusions}

The target of  this work was to design and  implement a complete execution
environment based  on virtual machines. The  first step in this was  to define the
execution semantics that have to be supported 

\begin{itemize}
\item job semantics: batch, server deployment
\item access to required data: image, kernel, initrd, input files
\item caching and compression of files
\item uniform way to access files (URI)
\item standardized job description
\item standardized job model
\item reservations
\item hooks to modify input and output files (encryption)
\item secure end-to-end communication xml
\item task security with chroot environment
\item user has full control over the vm
\item results have shown that the implementation is able to execute batch
  jobs and server deployment
\item  also: starting/stopping  of  virtual machines  is  fast enough  for
  on-demand server deployment
\item 
\end{itemize}

\section*{Future work}

The current  implementation of the \gls{glo:XenBEE} is  already usable for
real world  problems as  it has  been shown in  the previous  chapter. But
there are still many aspects that could be implemented and analyzed in the
future. The following sections provide a few ideas for future works.

\subsubsection{Integration into Calana}

The most crucial future development step is the actual integration into an
existing  grid  environment.   The  execution  environment  understands  a
commonly used language for the  job submission and a formalized job-model.
The  basic  requirements  for   the  integration  into  Calana  have  been
implemented,  as well.   But  the glue  between  the \gls{glo:XenBEE}  and
Calana (or some  other grid middleware) --- the  Calana-agent --- is still
missing.

\subsubsection{Unattended Updates and Support for Work flows}

Completely out  of the scope  of this work  was the administration  of the
used virtual  machines images.   It could for  instance be possible  to do
automated  updates of stored  images.  These  updates should  be performed
regularly and without interaction.  The  update process would also need to
verify the updated applications, \ie a test suite must be executed on that
image.   The execution environment  could therefore  be enhanced  with the
support of a complete work flow description language.

\subsubsection{Cache hierarchies}

A cluster of machines that are used for the \gls{glo:XenBEE} could use one
or more shared  caches. If an user wants to execute  a particular job many
times or  on several machines  at the same  time, he could load  the image
into  the shared  cache first.   Each involved  execution host  could then
retrieve the image into its local cache.

\subsubsection{Advanced file system support}

The current implementation  makes the assumption that only  a single image
file is  involved.  But  it could  also be possible  to compose  a virtual
machine out of  several images. A basic image  that contains the operating
system and application installation and a second image that can be used to
store input and output data.  Actually, the additional image could also be
a network file system.

\subsubsection{tools}
\label{sec:tools}



%%% Local Variables: 
%%% mode: latex
%%% TeX-master: "main"
%%% End: 
