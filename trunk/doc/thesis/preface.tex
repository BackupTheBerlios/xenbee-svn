\setchapterpreamble[o]{%
  \dictum{\textit{Virtualization  is a  concept that  one cannot  think  away from
    computer science anymore.}}}

\chapter{Preface}
\thispagestyle{empty}

This work  is about  the execution of  arbitrary applications on  a remote
system exploiting virtual machines. It addresses the problem that multiple
potentially  broken applications  are typically  executed on  remote hosts
side by  side ---  if one application  behaves ``wrong''  (e.~g.~CPU cycle
consumption,  memory  leakage, etc.)  it  may  involve other  applications
running on the same host.

\vfill

% chapter overview
\begin{description}
\item[Chapter   \ref{cha:intro}]   The   first   chapter   contains   some
  introductory material that you should  read if you are not familiar with
  virtualization  technologies   and  especially  the  \emph{\gls{glo:XEN}
    hypervisor}.
  
\item[Chapter \ref{cha:requirements}] In  this chapter the requirements of
  the execution environment are discussed.
  
\item[Chapter \ref{cha:design}] This chapter  deals with the design of the
  \emph{\gls{glo:XEN}-based Execution Environment} and its components.
  
\item[Chapter  \ref{cha:comm-prot}]  The   fourth  chapter  discusses  the
  protocols  used for  communication between  the different  parts  of the
  system.
  
\item[Chapter  \ref{cha:secur-cons}]  In  this  chapter  security  related
  topics are discussed, such as  securing the messages sent between client
  and server  using \emph{Message Layer Security}  (\gls{glo:MLS}) and how
  the executing  of user-supplied  scripts are secured  using \emph{chroot
    environment}.
  
\item[Chapter \ref{cha:results}] This chapter  shows some results of using
  the  \emph{\gls{glo:XEN}-based Execution  Environment}  and compares  it
  against running the same programs on a stand-alone machine.
  
\item[Chapter   \ref{cha:conclusions}]  The   final  chapter   draws  some
  conclusions about  the work  that has been  done and provides  ideas for
  future development.
  
\end{description}

\clearpage

%%% Local Variables: 
%%% mode: latex
%%% TeX-master: "main"
%%% End: 
