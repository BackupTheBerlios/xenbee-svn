\storeglosentry{glo:XEN}{
  name={Xen},
  description={Xen virtual machine monitor \cite{xen}}
}

\storeglosentry{glo:VM}{
  name={VM},
  description={Virtual Machine},
}

\storeglosentry{glo:VMM}{
  name={VMM},
  description={Virtual Machine Monitor}
}

\storeglosentry{glo:MLS}{
  name={MLS},
  description={Message Layer Security}
}

\storeglosentry{glo:TLS}{
  name={TLS},
  description={Transport Layer Security. The protocol allows client/server
    applications to communicate in a way that is designed to prevent
    eavesdropping, tampering, or message forgery (taken from abstract of
    the RFC 4346).
  }
}

\storeglosentry{glo:virtmem}{
  name={virtual memory},
  description={Virtual memory is an addressing scheme implemented in
    hardware and software that allows non-contiguous memory to be
    addressed as if it is contiguous.},
  sort={M}
}

\storeglosentry{glo:multi-programming}{
  name={multi--programming},
  description={
    In multiprogramming systems, the running task keeps running until it
    performs an operation that requires waiting for an external event
    (e.g. reading from a tape) or until the computer's scheduler forcibly
    swaps the running task out of the CPU. Multiprogramming systems are
    designed to maximize CPU usage.},
}

\storeglosentry{glo:time-sharing}{
  name={time--sharing},
  description={time sharing refers to sharing a given computing resource
    among different users by multitasking.}
}

\storeglosentry{glo:multi-tasking}{
  name={multitasking},
  description={Multitasking is a method by which multiple tasks share
    common processing resources.},
}

\storeglosentry{glo:extracode}{
  name={extracode},
  description={
    The term ``extracode'' was used  in the Atlas computer system to allow
    new  instructions being  added in  software (that  what is  now called
    firmware).
  },
}

\storeglosentry{glo:supervisor}{
  name={supervisor},
  description={
    A supervisory program or supervisor is a computer program, usually
    part of an operating system, that controls the execution of other
    routines and regulates work scheduling, input/output operations, error
    actions, and similar functions and regulates the flow of work in a
    data processing system.
  }
}

\storeglosentry{glo:virtualization}{
  name={virtualization},
  description={
    Virtualization in  computing describes a technique for hiding the
    physical characteristics of computing resources.
  }
}

\storeglosentry{glo:BLAST}{
  name={BLAST},
  description={\textbf{B}asic       \textbf{L}ocal      \textbf{A}lignment
    \textbf{S}earch \textbf{T}ool
  }
}

\storeglosentry{glo:image}{
  name={image},
  description={The  term  \emph{``image''}  relates  in  this  work  to  a
    file-system image  (i.e.~a regular file, that for  example contains an
    \texttt{ext2} partition).  Those image-files  must be mountable by the
    standard UNIX-command \texttt{mount}.
 }
}

\storeglosentry{glo:JSDL}{
  name={JSDL},
  description={The    \emph{Job     Submission    Description    Language}
    \cite{jsdl-spec},  an   upcoming  standard  for   the  description  of
    computational jobs.  It is designed to  be used in but  not limited to
    grid environments.
  }
}

\storeglosentry{glo:POSIX}{
  name={POSIX},
  description={The \emph{Portable Operating System Interface} \cite{posix}
    is  a   collective  name  for  a  family   of  specifications.   These
    specifications  for  instance  describe  standard  library  functions,
    system behaviour and many more,  so that applications, which make only
    use of  library functions defined in the  specifications, are supposed
    to run on all POSIX-compliant systems.
  }
}

\storeglosentry{glo:BES}{    name={BES},    description={The   \emph{Basic
      Execution  Service}  \cite{ogsa-bes} describes  the  semantics of  a
    Web-Service,  which   is  responsible   for  the  execution   of  some
    \emph{activity}.   The  specification  does  not consider  the  actual
    execution  of any  activity but  it defines  the operations  and their
    semantics that such a service must implement.} }

\storeglosentry{glo:OGSA}{
  name={OGSA},
  description={The \emph{Open Grid Service Architecture}, see \cite{ogsa}
    for more information.}
}

\storeglosentry{glo:TCP}{
  name={TCP},
  description={The \emph{Transmission Control Protocol} is a
    connection-oriented stream-based protocol which guarantees an ordered,
    loss-free and correct (i.e.~checksummed) transport of packets in
    packet-switched computer networks.
    For detailed information have a look at RFC 793.
  }
}

\storeglosentry{glo:MQS}{
  name={MQS},
  description={A \emph{Message Queue Server} is a server which provides
    access for message-oriented services. This kind of server is mainly
    used in \emph{message-oriented middlewares}.}
}

\storeglosentry{glo:MOM}{
  name={MOM},
  description={\emph{Message Oriented Middleware}}
}

\storeglosentry{glo:NAT}{
  name={NAT},
  description={\emph{Network Address Translation} or \emph{Network Address
    Translator} (RFC 1631) is used to multiplex one official IP address
  to a whole network of private network addresses (contact the site of the
  IANA for a list of IP addresses that are reserved for private use only).}
}

\storeglosentry{glo:URI}{
  name={URI},
  description={A \emph{Uniform Resource Identifier} is a compact string of
  characters for identifying an abstract or physical resource (taken from
  RFC 2396).}
}

\storeglosentry{glo:SSH}{
  name={SSH},
  description={The \emph{Secure SHell} can be used to gain access to a
    remote computer in a secure manner (i.e~encrypted).}
}

\storeglosentry{glo:XenBEE}{
  name={XenBEE},
  description={\emph{Xen-based Execution Environment}}
}

\storeglosentry{glo:X509}{
  name={X.509},
  description={X.509 is an ITU-T standard for public-key infrastructures
    (PKI). Among other things, X.509 defines the format of public-key
    certificates and a certificate validation path.}
}

\storeglosentry{glo:PKI}{
  name={PKI},
  description={\emph{Public Key Infrastructure}. Certificates and a
    Certificate Authority can be used to build up a PKI. Certificates,
    which have  been  signed by a trusted authority are trusted, too.}
}

\storeglosentry{glo:CA}{
  name={CA},
  description={\emph{Certificate Authority}. A CA issues certificates to
  users it trusts (e.g.~after verifying their particulars). Users can then
  rely on the validity of a certificate if it has been issued (i.e.~signed)
  by a trusted CA.}
}

\storeglosentry{glo:XML}{
  name={XML},
  description={\emph{eXtensible Markup Language}}
}



%%% Local Variables: 
%%% mode: latex
%%% TeX-master: "main"
%%% End: 
