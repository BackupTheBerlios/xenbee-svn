%% Title page

\title{{\Huge Diploma Thesis} \\
    {\large Department for computer science} \\
  \vspace{2cm}
  \emph{\huge Design and Implementation of a Xen-based execution environment} \\
  \vspace{1.5cm}
  {\large
    Supervisor: Mathias Dalheimer, AG Distributed Algorithms\\
    University of Kaiserslautern} \\
}
\author{Alexander Petry \footnote{E-Mail: $<$petry@itwm.\@fhg.\@de$>$, Matrikelnummer: 345466}}
\date{\today}
\pagestyle{empty}
\maketitle
\thispagestyle{empty}

%% Eidesstattliche Erklärung
\newpage

\thispagestyle{empty}
\begin{center}  
{\Large Erklärung an Eides Statt}
\end{center}

\vspace{1cm}

\begin{center}
\fbox{
  \begin{minipage}{0.9\textwidth}
    Ich versichere hiermit, dass ich die vorliegende Projektarbeit mit dem
    Thema ,,Eventbasierte Simulation von Middleware Plattformen''
    selbständig verfasst und keine anderen als die angegebenen Hilfsmittel
    benutzt habe.  Die Stellen, die anderen Werken dem Wortlaut oder dem
    Sinn nach entnommen wurden, habe ich durch die Angabe der Quelle, auch
    der benutzten Sekundärliteratur, als Entlehnung kenntlich gemacht.

    \vspace{1.5cm}

    \begin{tabular*}{.9\textwidth}%
      {ll}%
      \rule{0.45\textwidth}{.5pt} & \rule{0.45\textwidth}{.5pt} \\
      (Ort, Datum) & (Name, Unterschrift)
    \end{tabular*}

\end{minipage}
}
\end{center}

\newpage

%% Summary
\pagestyle{fancy}
\begin{abstract}
%%  In dieser  Arbeit stelle ich eine  eventbasierte Simulationsumgebung für
%%  \emph{Calana}  vor. \emph{Calana}  ist eine  Grid-Scheduler Architektur,
%%  die mittels Auktionen die Aufträge an die einzelnen Ressourcen verteilt.
%%  Dabei werden  die Anbieter  der Ressourcen durch  Agenten repräsentiert,
%%  die während einer  Auktion auf die Aufträge bieten  können.  Ziel dieses
%%  Schedulers ist  es, die \emph{Wünsche der Benutzer}  zu berücksichtigen. 
%%  Die  Wünsche werden durch  Präferenzen realisiert  --- zum  Beispiel die
%%  Zeit-Präferenz,  die besagt, dass  der Auftrag  \emph{möglichst} schnell
%%  ausgeführt werden soll.
%%  
%%  Als erstes werde ich einen  kurzen Überblick über das Grid-Umfeld geben. 
%%  Dabei spreche ich verschiedene Scheduling-Verfahren (sowohl für Cluster,
%%  als auch  für Grids)  an und werde  die verwendeten  Präferenzen genauer
%%  erläutern.   In dieser  Arbeit gehe  ich von  reinen \emph{computational
%%    Grids} aus  --- in späteren Erweiterungen des  Simulator können jedoch
%%  zusätzliche  Komponenten  (Netzwerke,  Datenbanken,  etc.)   hinzugefügt
%%  werden.
%%  
%%  Als    nächstes   werden    das   \emph{Calana}-Protokoll    und   seine
%%  Implementierung   in  Java   ausführlich   beschrieben.   Mit   diversen
%%  Experimenten, die  sowohl eine  kleine Zahl an  Anbietern, als  auch ein
%%  Grid mit 50 Anbietern umfassen,  soll gezeigt werden, dass der Scheduler
%%  das  gestellte  Ziel (die  Benutzer-Präferenzen  zu erfüllen)  erreichen
%%  kann.   Es zeigte sich  dabei, dass  die Average  Response Time  für die
%%  Benutzer, die  schnell ihre Ergebnisse wollen,  \emph{unabhängig von der
%%    Load}  ist.   Die Average  Response  Time  ist  in dieser  Arbeit  die
%%  maßgebliche Metrik, um Simulationsläufe zu vergleichen.
%%  
%%  Zuletzt stelle ich anhand  der \emph{Koallokation} eine Möglichkeit vor,
%%  das  System zu  erweitern ohne  bestehende  Teile verändern  zu müssen.  
%%  Dabei zeigt sich  der Vorteil des agentenbasierten Ansatzes,  da nur ein
%%  neuer (jedoch  spezieller) Agent dem  System hinzugefügt werden  musste. 
%%  Experimente,  die  mit   verschiedenen  Anteilen  an  Koallokation  (dem
%%  gleichzeitigen  Reservieren verschiedener  Ressourcen) in  den Workloads
%%  durchgeführt wurden, zeigten,  dass auch hier die ART  für die Benutzer,
%%  die schnell ihre Ergebnisse benötigen, \emph{unabhängig} war.
%%
%%  

%  \begin{comment}
%    Das Schlagwort \emph{Grid Computing} ist mittlerweile in aller Munde
%    und dies nicht zu Unrecht.  Grids dienen der gemeinsamen Nutzung von
%    verteilten Ressourcen.  Sie bieten allen Teilnehmern (sowohl den
%    Benutzern als auch den Anbietern) ganz neue Möglichkeiten Ressourcen
%    zu teilen.  Beispielsweise können Wetterdaten von gemeinsam genutzten
%    Sensoren auf einem High-Performance Cluster aufbereitet und analysiert
%    werden. Anschließend können die Ergebnisse in einer Datenbank für
%    viele Benutzer zur Verfügung gestellt werden.
%    
%    \medskip
%    \noindent
%    Die vorliegende Arbeit beschäftigt sich mit der Implementierung einer
%    Simulationsumgebung für Grid-Umgebungen. Dabei wurde der agentenbasierte
%    Scheduler \emph{Calana} (\cite{dalheimer05agentbased}) implementiert und
%    mit verschiedenen Workloads getestet.
%  \end{comment}

\end{abstract}

\tableofcontents
\listoffigures
%\listoftables
%\listofalgorithms
\newpage


%%% Local Variables: 
%%% mode: latex
%%% TeX-master: "main"
%%% End: 
