\setchapterpreamble[o]{%
  \dictum{\textit{Virtualization  is a  concept that  one cannot  think  away from
    computer science anymore.}}}

\chapter{Preface}
\thispagestyle{empty}

This work  is about  the execution of  arbitrary applications on  a remote
system exploiting virtual machines. It addresses the problem that multiple
potentially  broken applications  are typically  executed on  remote hosts
side by  side ---  if one application  behaves ``wrong''  (e.~g.~CPU cycle
consumption,  memory  leakage, etc.)  it  may  involve other  applications
running on the same host.

\textbf{Motto:} \emph{Secure execution by separation with virtual machines.}
\vfill

% chapter overview
\begin{description}
\item[Chapter \ref{cha:intro}]  The first  chapter introduces you  to past
  and current  virtualization technologies.  This chapter  also covers the
  problem  description  and  the  goals of  the  proposed  \emph{Xen-based
    Execution Environment}.

\item[Chapter \ref{cha:requirements}] This  chapter outlines and describes
  the basic  requirements that the  \emph{Xen-based Execution Environment}
  has to fulfill.
  
\item[Chapter   \ref{cha:fundamentals}]   The   third  chapter   discusses
  fundamental technologies that  I have used in this  work.  It covers the
  \emph{Job  Submission  Description   Language},  the  job-model  of  the
  \emph{Basic Execution Service} and  the securing of transmitted messages
  using \emph{Message Layer Security}.
  
\item[Chapter  \ref{cha:design}] This  chapter deals  with the  design and
  implementation  of the  \emph{Xen-based Execution  Environment}  and its
  components. It covers the necessary steps and the involved communication
  protocols to execute a job on a virtual machine.
  
\item[Chapter \ref{cha:results}]  In this  chapter the performance  of the
  \emph{Xen-based Execution Environment} is  analyzed. The impact of using
  compressed  images and exploitation  of caching  on the  total execution
  time is analyzed as well.
  
\item[Chapter \ref{cha:conclusions}]  The final chapter  draws conclusions
  about the  work that has  been done and  provides some ideas  for future
  development.
\end{description}

\clearpage

%%% Local Variables: 
%%% mode: latex
%%% TeX-master: "main"
%%% End: 
