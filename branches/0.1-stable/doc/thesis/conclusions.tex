
\chapter{Conclusions and Future Work}
\label{cha:conclusions}

This  chapter analyzes  the work  that  has been  done in  respect of  the
initial  goals that  have  been set  and  provides some  hints for  future
developments.

\section*{Conclusions}

This work designed and implemented  an execution environment that is based
on  virtual machines.  The  XenBEE  supports two  different  types of  job
execution:  a  batch-job  semantic  and  an  on-demand  server  deployment
semantic.

In addition to other products such as Amazon EC2 \cite{amazon-ec2} and The
XenoServer  Open Platform  \cite{kotsovinos05}, this  work  adds batch-job
execution capabilities to virtual machine environments.

To execute such a batch-job, the user submits a virtual machine image that
contains the  application he wants to  execute.  The XenBEE  creates a new
virtual machine that  uses this supplied image and  executes the specified
application  within the virtual  machine. Along  with the  submission, the
user can also specify stage-in and stage-out operations that transfer data
into   the  virtual   machine  or   from  the   virtual  machine   to  the
user. Currently only  HTTP and FTP are supported  for these operations but
the XenBEE can easily be extended to support more protocols.

As the results in Chapter~\ref{cha:results} show, the implementation works
and performs well.

With  the  use  of  virtual  machine  images  a  common  problem  of  grid
environments  has been  solved: It  is not  always clear  if and  where an
application is installed on a  given target system.  This problem has been
avoided   by    letting   the    user   actually   provide    the   target
environment.  Additionally, a  virtual  machine is  never  shared so  that
security for the job execution is provided.

\medskip

The other execution semantic is  on-demand server deployment. In this case
the  user  supplies  a  virtual  machine  image  that  contains  a  server
application such  as a game server or  a web server. The  XenBEE creates a
new virtual machine and the server is deployed. Files can be staged-in and
staged-out as well, \eg to provide initial content for web server.

The results have shown that the  deployment of a web server works and took
about $60\ s$.

\medskip

The communication  model that is used  in the XenBEE  is message-based and
supported by  a message-queue  server. The use  of a  message-queue server
makes  it possible  that  the client  can  be behind  a  NAT gateway  (all
connections are outbound).

To  provide authentication,  authorization and  secure  communication, the
XenBEE uses  Public Key Infrastructure  (PKI). That means each  client and
the server is in possess  of a certificate.  These certificates are signed
by  a trusted  Certificate Authority  which  means client  and server  can
verify each other's identity. For  authorization the server uses a list of
authorized certificates.

To provide a secure communication  between client and server Message Layer
Security  is  used. This  prevents  eavesdropping,  tampering and  message
forgery,  \ie confidential  data  can be  transmitted  between client  and
server.

Security is currently  only assured between client and  server but neither
for data transfers, nor between the server and a virtual machine. Both can
be implemented  easily. For the  data transfer, the client  would transfer
credentials  over the  secured connection  to  the server.   For a  secure
communication between  the server and  a virtual machine a  pre-shared key
can be chosen by the server prior virtual machine creation.

\medskip

The   XenBEE   supports   the   use    of   a   local   cache   and   file
compression/decompression.   This can  be  used to  speed  up the  virtual
machine  creation, because  the image  need not  to be  transfered  over a
(slow)  network connection. In  Chapter~\ref{cha:results} the  benefits of
caching files has been shown.

\section*{Future work}

The current  implementation of the \gls{glo:XenBEE} is  already usable for
real world  problems as  it has  been shown in  the previous  chapter. But
there are still many aspects that could be implemented and analyzed in the
future. The following sections provide a few ideas for future works.

\subsubsection{Integration into Calana}

The most crucial future development step is the actual integration into an
existing  grid  environment.   The  execution  environment  understands  a
commonly used language for the  job submission and a formalized job-model.
The  basic  requirements  for   the  integration  into  Calana  have  been
implemented,  as well.   But  the glue  between  the \gls{glo:XenBEE}  and
Calana (or some  other grid middleware) --- the  Calana-agent --- is still
missing.

\subsubsection{Support for Workflows}

The  current implementation  supports only  a  very basic  workflow,  \ie
stage-in of input  data, job execution and stage-out  of generated data. A
sophisticated  work  flow  description  language supports  constructs  for
looping and conditional branches.

The  support  workflows  could  be   implemented  either  on  top  of  the
\gls{glo:XenBEE}, or  directly into the  \gls{glo:XenBEE}.  An independent
``on top'' approach only uses  the execution semantics already provided by
the \gls{glo:XenBEE}. This means it must always wait for the results of an
execution to be  staged out before the  next step of the work  flow can be
submitted. This approach can make use of existing technologies.

An optimized  implementation that is integrated  into the \gls{glo:XenBEE}
could  prepare the  next virtual  machine while  the current  execution is
still  running.   The results  of  the  current  execution could  then  be
directly staged into the new virtual machine.

\subsubsection{Enhanced up-\ and download mechanisms}

The current  implementation of the \gls{glo:XenBEE} does  only support the
HTTP and  FTP protocols to  perform upload and download  operations. Other
mechanism such as  SCP, rsync or GridFTP should be  supported as well. The
access to the different storage  areas could be granted to the \emph{xbed}
by user-supplied certificates.

\subsubsection{Cache hierarchies}

A cluster of machines that are used for the \gls{glo:XenBEE} could use one
or more shared  caches. If a user wants to execute  a particular job many
times or  on several machines  at the same  time, he could load  the image
into  the shared  cache first.   Each involved  execution host  could then
retrieve the image into its local cache.

\subsubsection{Advanced file system support}

The current implementation  makes the assumption that only  a single image
file is  involved. This  image contains all  required data.  But  it could
also be possible to provide access  to network file systems such as NFS or
GridFS and so on.

\subsubsection{Generation of virtual machine images}

The images that  were used in the performed  experiments have been created
by   a  toolchain   that   was  included   in   the  Ubuntu   distribution
(\texttt{xen-tools}).  These  tools provide  everything that is  needed to
create  a  virtual  machine  image,  but the  created  images  are  rather
large. To create  really small images one has to  modify the created image
by hand afterwards.

\subsubsection{User-friendly front-ends}

The \emph{xbe} command line tool has mainly been implemented as a proof of
concept and  to be  able to actually  execute example jobs.  Currently the
\emph{xbe}  does  only support  the  submission  of  already existing  job
description documents.

A graphical  user interface could  show a list  of available images  to a
user.  In  the case  of on-demand server  deployment a user  could simply
double-click on one of the available servers to start it. This can also be
coupled with the support for work flows, \ie the user models the work flow
graphically and submits it to the \gls{glo:XenBEE}.

%%% Local Variables: 
%%% mode: latex
%%% TeX-master: "main"
%%% End: 
